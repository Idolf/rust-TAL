\section{Decidable Equality}

It is a common occurrence in Agda to want to prove decidable equality for a
specific type $T$. Decidable equality is a lemma of the following form:

\begin{lemma}[Decidable Equality]
  Let $v_1, v_2 \in T$. We can now derive a proof of either $v_1 \equiv v_2$ or
  $v_1 \not\equiv v_2$.
\end{lemma}

This lemma is in Agda often written as a dependent function with the signature:

$$\cdot ≟ \cdot\;\;:\;\;(v_1 : T) \to (v_2 : T) \to \mathtt{Dec}(v_1 \equiv v_2)$$

An example from Agda standard library (with slight modifications):

\begin{code}%
\>\<%
\\
\>\AgdaFunction{\_≟\_} \AgdaSymbol{:} \AgdaSymbol{(}\AgdaBound{v₁} \AgdaSymbol{:} \AgdaDatatype{Bool}\AgdaSymbol{)} \AgdaSymbol{→} \AgdaSymbol{(}\AgdaBound{v₂} \AgdaSymbol{:} \AgdaDatatype{Bool}\AgdaSymbol{)} \AgdaSymbol{→} \AgdaDatatype{Dec} \AgdaSymbol{(}\AgdaBound{v₁} \AgdaDatatype{≡} \AgdaBound{v₂}\AgdaSymbol{)}\<%
\\
\>\AgdaInductiveConstructor{true} \<[6]%
\>[6]\AgdaFunction{≟} \AgdaInductiveConstructor{true} \<[14]%
\>[14]\AgdaSymbol{=} \AgdaInductiveConstructor{yes} \AgdaInductiveConstructor{refl}\<%
\\
\>\AgdaInductiveConstructor{false} \AgdaFunction{≟} \AgdaInductiveConstructor{false} \AgdaSymbol{=} \AgdaInductiveConstructor{yes} \AgdaInductiveConstructor{refl}\<%
\\
\>\AgdaInductiveConstructor{true} \<[6]%
\>[6]\AgdaFunction{≟} \AgdaInductiveConstructor{false} \AgdaSymbol{=} \AgdaInductiveConstructor{no} \AgdaSymbol{λ()}\<%
\\
\>\AgdaInductiveConstructor{false} \AgdaFunction{≟} \AgdaInductiveConstructor{true} \<[14]%
\>[14]\AgdaSymbol{=} \AgdaInductiveConstructor{no} \AgdaSymbol{λ()}\<%
\end{code}

\paragraph{}
While these types of lemmas are in principle trivial to generate for most
datatypes, there is a problem: required number of lines is \emph{quadratic} in
the number of constructors for the datatype. The reason for this quadratic
explosion is the need to the proofs for $v_1 \neq v_2$.

I found a solution to the problem in \cite{deceq}, though I generalized his
solution slightly. The idea is a follows:

\begin{itemize}
\item Construct an injective function $f$ from the original type $T$ to another
  type $T'$.
\item Prove decidable equality for $T'$.
\item Use this to decide $f(v_1) \equiv f(v_2)$.
\item We know that $f(v_1) \not\equiv f(v_2)$ implies $v_1 \not\equiv v_2$,
  since $f$ is a function.
\item We also know that $f(v_1) \equiv f(v_2)$ implies $v_1 \equiv v_2$, since
  $f$ is injective.
\item If we combine these, we get decidable equality for $T$.
\end{itemize}

By choosing $T'$ to be the same for every $T$, we can reuse a lot code. In
practice my code does the following:

\begin{itemize}
\item Define a type \texttt{Tree}, which is the set of rooted, ordered tree with
  numbers attached to each node. This datatype was chosen since this makes it
  easy to construct the injective functions.
\item Prove decidable equality for \texttt{Tree}.
\item Prove that given $T$ and an injective functions $f : T \to \mathtt{Tree}$,
  we have decidable equality for $T$.
\item Prove that given $T$ and a surjective function $g : \mathtt{Tree} \to T$,
  we can construct an injective function $f : T \to \mathtt{Tree}$, and thus
  this also implies decidable equality on $T$.
\end{itemize}

This last step is a slight further optimization, as it in practice takes fewer
lines of code in Agda to specify a function $g : \mathtt{Tree} \to T$ and prove
it surjective, than to specify a function $f : T \to \mathtt{Tree}$ and prove it
injective.

\later[inline]{Insert reference to appendix, where usage of this can be seen.}
