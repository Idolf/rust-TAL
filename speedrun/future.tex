\chapter{Related Work and Future Work}
\label{chap:future}
\section{Related Work}
\subsection{Relation to STAL}
\label{sec:rel-stal}

\section{Original Goal}

The original title of this project was ``A Rust-like Typed Assembly Language --
Using affine types to avoid garbage collection''.

The goal of this proposed thesis was to design a Typed Assembly Language based
on linear types. Linear types have been tried before in the context of TAL, in
particular in the paper ``A linearly typed assembly language''\cite{ltal} from
2003. In this paper the authors design a TAL based in linear types, but they
conclude the approach has serious drawbacks in particular with regards to
sharing.

I thought it interesting to revisit the idea, since the programming language
Rust\cite{rust} have proven that is \emph{can} be practical to use linear type
systems for memory management\footnote{Technically Rust uses an affine type
  system, not a linear one. In this particular case the differences are minor
  and mostly related to usability of the language.}. I found the paper ``Patina:
A Formalization of the Rust Programming Language''\cite{patina} by Eric Reed,
and after skimming it, it was my hope that I could use the same techniques in a
TAL.

I changed focus during the project for a number of reasons. One of my issues was
that the original paper fell short of actually proving soundness, as several of
the later proofs are missing. This was quite naturally detrimental to my effort
and I should have noticed this before settling on a thesis subject.

However the much bigger issue in my thesis turned out to be implementing
variables and substitutions correctly in Agda using de Bruijn
indices. Substitutions take up about 25\% of my final code-base (including the
utilities not specifically related to TAL. This is after rewriting that part
several times and removing unused lemmas.

I do not think that this is an issue with Agda, but rather I think that de
Bruijn indices could simply be hard to formalize correctly -- at least they were
to me. This seems consistent with the results from ``An Untrusted Verifier for
Typed Assembly Language''\cite{untrusted}[5.1], in which the authors had similar
problems when formalizing a TAL in Coq also using a similar approach.

This meant that I used most of my time simply re-implementing known results in a
different framework.

\section{Future Work}
