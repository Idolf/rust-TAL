\section{Formal Definition}

\begin{definition}
  A \textbf{simplfied assembly language} is a tuple

  $L = (\mathcal{G}, \mathcal{H}, \mathcal{R}, \mathcal{I}, \mathcal{T}, R_\to)$
  such that:

  \begin{itemize}
  \item $\mathcal{G}$ a set, called the \emph{set of possible global states}.
  \item $\mathcal{H}$ a set, called the \emph{set of possible heap states}.
  \item $\mathcal{R}$ a set, called the \emph{set of possible register files}.
  \item $\mathcal{I}$ a set, called the \emph{set of possible instruction
      sequences}.
  \item $\mathcal{T}$ a set, called the \emph{set of possible termination
      states}.
  \item Write as $\mathcal{S}$ short hand for
    $\mathcal{H} \times \mathcal{R} \times \mathcal{I}$ and $\mathcal{S}^T$ as
    shorthand for $\mathcal{S} \cup \mathcal{T}$.
  \item $\mathcal{S} \cap \mathcal{T} = \emptyset$.
  \item $R_\to$ is a partial function
    $\mathcal{G} \times \mathcal{S} \to \mathcal{S}^T$,
    called the \emph{transition relationtion}.
  \item Write $R_\to^\ast$ for the transitive closure of $R_\to$.
  \end{itemize}

  Assume $S_1 \in \mathcal{S}$ and $S_2 \in \mathcal{S}^T$. We will write
  $G |- S_1 \to_L S_2$ as shorthand for $(G, S_1, S_2) \in R_\to$.  When $L$ can
  be inferred without ambiguity, we will leave it out. We will use similar
  notation with $\to^\ast$ and $R_\to^\ast$.
\end{definition}

\begin{definition}
  Let $L$ be a simplified assembly language. We say that
  $(\mathbf{\Psi}_g, \mathbf{\Psi}_h, \mathbf{\Gamma})$ is a \emph{typing
    scheme} for $L$ iff:

  \begin{itemize}
  \item $\mathbf{\Psi}_g$ is a set, called the \emph{set of global state
      types}. Let $\Psi_g$ be the typical member of this set.
  \item $\mathbf{\Psi}_h$ is a set, called the \emph{set of heap state
      types}. Let $\Psi_h$ be the typical member of this set.
  \item $\mathbf{\Gamma}$ is a set, called the \emph{set of register types}. Let
    $\Gamma$ be the typical member of this set.
  \item There is a judgment of the form $|- G : \Psi_g$.
  \item There is a judgment of the form $\Psi_g |- H : \Psi_h$.
  \item There is a judgment of the form $\Psi_g , \Psi_h |- R : \Gamma$.
  \item There is a judgment of the form
    $\Psi_g , \Psi_h , \Gamma |- I\ \mathbf{Valid}$.
  \end{itemize}

  We define $G |- S\ \mathbf{Valid}$ as a judgment only one rule:
  \begin{mathpar}
    \infer{
      |- G : \Psi_g \and
      \Psi_g |- H : \Psi_h \and
      \Psi_g , \Psi_h |- R : \Gamma \and
      \Psi_g , \Psi_h , \Gamma |- I\ \mathbf{Valid}
    }{
      G |- (H, R, I)\ \mathbf{Valid}
    }
  \end{mathpar}

  We call $L$ for a \emph{typed assembly language}, if it has a typing scheme.
\end{definition}

\begin{definition}
  Let $L$ be a typed assembly language. We say that it is \emph{progressing} iff
  for any given $G \in \mathcal{G}, S \in \mathcal{S}$ with

  \begin{itemize}
  \item $G |- S\ \mathbf{Valid}$
  \end{itemize}

  we can conclude that there is some $S' \in \mathcal{S}^T$ such that

  \begin{itemize}
  \item $G |- S \to S'$.
  \end{itemize}
\end{definition}

\begin{definition}
  Let $L$ be a typed assembly language. We say that it is \emph{reducing} iff
  for any given $G \in \mathcal{G}, S, S' \in \mathcal{S}$ with

  \begin{itemize}
  \item $G |- S\ \mathbf{Valid}$
  \item $G |- S \to S'$.
  \end{itemize}

  we can conclude that:

  \begin{itemize}
  \item $G |- S'\ \mathbf{Valid}$
  \end{itemize}
\end{definition}

\begin{definition}
  Let $L$ be a typed assembly language. We say that it is \emph{sound} iff it is
  both progressing and reducing.
\end{definition}

\begin{theorem}[Soundness]
  Let $L$ be a sound typed assembly language. Let $G \in \mathcal{G}$ and
  $S, S' \in \mathcal{S}$ and assume $G |- S\ \mathbf{Valid}$ along with
  $G |- S \to\^ast S'$. We now conclude that there exists some
  $S'' \in \mathcal{S}^T$ with $G |- S' \to S''$.
\end{theorem}

\begin{proof}
  We first conclude that $G |- S'\ \mathbf{Valid}$. This can be shown by
  induction over the derivation of $G |- S \to^\ast S'$ along with the fact that
  $L$ is reducing.

  Since $L$ is also progressing, this implies the result.
\end{proof}

\subsection{Bisimulation}

\begin{definition}
  Assume the following:
  \begin{itemize}
  \item
    $L_1 = (\mathcal{G}_1, \mathcal{H}_1, \mathcal{R}_1, \mathcal{I}_1,
    \mathcal{T}, R_\to)$.
  \item
    $L_2 = (\mathcal{G}_2, \mathcal{H}_2, \mathcal{R}_2, \mathcal{I}_2,
    \mathcal{T}, R_\to)$.
  \item Let $E = (f_G, f_H, f_R, f_I, f_T)$ be tuple of total functions. The
    functions should be defined over $\mathcal{G}_1 \to \mathcal{G}_2$,
    $\mathcal{H}_1 \to \mathcal{H}_2$, $\mathcal{R}_1 \to \mathcal{R}_2$,
    $\mathcal{I}_1 \to \mathcal{I}_2$ and $\mathcal{T}_1 \to \mathcal{T}_2$
    respectively.
  \end{itemize}

  Define $f_S : \mathcal{S}_1 \to \mathcal{S}_2$ and
  $f_{S^T} : \mathcal{S}_1^T \to \mathcal{S}_1^T$ as:

  \begin{align*}
    f_S(H, R, I) &= (f_H(H), f_R(R), f_I(I)) \\
    f_{S^T}(v) &=
                 \begin{cases}
                   f_S(v) & \text{if $v \in \mathcal{S}_1$} \\
                   f_T(v) & \text{if $v \in \mathcal{T}_1$} \\
                 \end{cases}
  \end{align*}

  We say that $E$ is a bisimulation between $L_1$ and $L_2$ iff the following
  holds:

  \begin{itemize}
  \item Assume
    $G \in \mathcal{G}, S_1 \in \mathcal{S}_1, S_1' \in \mathcal{S}_1^T, S_2 \in
    \mathcal{S}_2, S_2' \in \mathcal{S}_2^T$.
  \item Assume we have a derivationIf $\alpha$ is a derivation $\alpha$ of $G |- S_1 \to S_1'$, then we can
    find a derivation $\beta$ of $f_G(G) |- S_2 \to S_2'$ such that the diagram
    below commutes.
  \item The converse also holds: If $\beta$ is a derivation of
    $f_G(G) |- S_2 \to S_2'$, then we can find a derivation $\alpha$ of
    $G |- S_1 \to S_1'$ such that the diagram below commutes.
  \end{itemize}
\end{definition}

% Define $f : \mathcal{S}^T \to \mathcal{S}
% \begin{theorem}[Soundness of embeddings]
%   Let $L_1$ by a sound typed assembly language and $L_2$ be a simplified
%   assembly language. Assume there
