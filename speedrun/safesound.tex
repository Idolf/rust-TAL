\chapter{Safety and Soundness}
\label{chap:safesound}

In this chapter we will discuss the construction of our sandbox. We will discuss
how it would work in practice including why we believe that it should be safe
given reasonable assumptions.

\section{Defining Safety}

Before going into the detail, it is beneficial to discuss exactly what we mean
by ``safety''. We will first discuss what we mean by a safe system. As a naive
definition one might the following:

\vspace{0.3cm}
\begin{addmargin}{1cm}
  {\it A ``safe'' system must never do anything that could jeopardize the
    safety of the user.}
\end{addmargin}
\vspace{0.3cm}

Besides being close to a circular definition, this definition also highlights
another issue: Any system fulfilling this definition must be highly customized
to the specific user. A slightly better definition would be:

\vspace{0.3cm}
\begin{addmargin}{1cm}
  {\it A ``safe'' system will never do anything potentially dangerous that a
    reasonably well-educated user would not expect it to.}
\end{addmargin}
\vspace{0.3cm}

While somewhat impractical, this definition has the nice feature, that one can
actually strive to implement it, even if one could never prove such a general
statement. An interesting detail is that it shows that, while safety can depend
on context and is often very hard to prove, some systems would always be
considered unsafe.

The definition is still too vague though -- to get closer to a \emph{provable}
definition, we will avoid mentioning the user:

\vspace{0.3cm}
\begin{addmargin}{1cm}
  {\it A ``safe'' system will never do any of the following things: [long list
    of potentially dangerous things]}
\end{addmargin}
\vspace{0.3cm}

The biggest remaining issue is that this blacklist approach is unlikely to be
practical. The list of bad actions for a system is likely near-infinite, so even
if we somehow manage to specify all the bad properties correctly, the list will
be unimaginably large. Instead of a blacklist we will use a whitelist:

\vspace{0.3cm}
\begin{addmargin}{1cm}
  {\it \textbf{A safe system} is a system will only do things from the following
    list: [long list of non-dangerous things]}
\end{addmargin}
\vspace{0.3cm}

In principle this list would likely be just as large as the blacklist -- however
in this case we can approximate it by using a subset of the list! Indeed, if we
used the empty list, then we would only have safe systems (the ones that do
nothing).

We will now use this definition to define a safe program. One might try
something similar to this:

\vspace{0.3cm}
\begin{addmargin}{1cm}
  {\it Assume we have a platform on which we can run programs. A ``safe
    program'' for that platform is any program where running it on the platform
    results in a safe system.}
\end{addmargin}
\vspace{0.3cm}

\section{Assumptions}
It turns that it is not possible to prove safety unconditionally. The system
might for instance be hit by cosmic radiation causing a bit-flip at just the
wrong time. We will instead prove safety given a collection of reasonable
assumptions, that would hold in most real-world situations. Which assumptions we
depend on our circumstances or exactly what we consider dangerous, so we could
consider many potentially assumptions:

\begin{enumerate}
\item We might assume that the CPU follows the CPU specification.
\item We might assume that the kernel is bug-free.
\item We might assume that any potential attacker does not have access to more
  than 1 zetaflop of computations per second or more than 1 zetabyte of storage.
\item We might assume that a specific trusted third party is indeed trustworthy.
\item We might assume that the attacker is merely curious but not malicious.
\item We might assume that the user will never make a mistake.
\end{enumerate}

We might then prove that our system is safe under these assumptions. For
instance we could imagine a system that was proven safe assuming \textbf{(1),
  (2), (6)} and either \textbf{(4)} or \textbf{(5)} holds.

Clearly we want to depend on as few assumptions as possible. Traditionally most
of these assumptions have been grouped into one of these categories:

\begin{itemize}
\item The \emph{Trusted Computing Base} is the collection of all software that
  we have assumed correct.
\item The \emph{Threat Model} is the description of what dangerous behavior we
  wish to avoid along with assumptions about any potential attackers'
  motivations and capabilities.
\end{itemize}

While these two concepts are not enough to capture all combinations of
assumptions we might encounter, they are a very good approximation in most
cases.

\section{The Case for Building a TAL}
Lets us consider the following scenario: We would like to protect a user. This
user has a computer on which she stores some secret information. The user is
able to download untrusted files from the internet in a secure manner (i.e., she
can download them without compromising her safety, but she cannot currently
interact with them in any way).

We would like to build her platform, such that she can run programs downloaded
from the internet without compromising the security of her information. These
programs should not interact with any other part of the system unless permission
for that interaction have been given by the user. Our threat model includes a
very capable adversary: He can change any binary the user downloads and forge
any cryptographic signature she might wish to verify -- however he cannot
interact with the system in any ways besides affecting the files downloaded.

How could we build such a platform for her? One solution would be as follows:

\begin{itemize}
\item Build a sandbox.
\item Let the sandbox, OS and CPU be part of the Trusted Computing Base.
\item Run the downloaded programs inside the sandbox.
\end{itemize}

While this solution will work it is certainly not optimal: The OS and sandbox
are fairly large components, so assuming them correct seems dishonest. Certainly
we can do better:

\begin{itemize}
\item Design a type system for our programs. This should be a type system for
  real-world programs, which can be run natively (though the system might
  contain programs in simplified form as part of its design).
\item Design a type-checker for the type system.
\item Let the type-checker, OS and CPU be part of the Trusted Computing Base.
\item Prove by hand that the TAL is sound.
\item Additionally assume that programs that are well-typed will also be safe
  programs.
\item Run the type-checker on the downloaded programs, but only run the program
  itself if it is well-typed.
\end{itemize}

This might be a better solution than the sandbox solution, however it has
several weak points:

\begin{itemize}
\item The type-checker have been assumed correct, but we have no real reason to
  believe it to be so. Not only might it accept programs that are not
  well-typed, but it might even contain critical bugs that will allow the
  attacker to get the secret information before even running the program.
\item We simply assumed without proof that well-typed program are safe.
\end{itemize}

It is possible to solve both of these problems, though not without introducing
additional (smaller) problems:

\begin{itemize}
\item Design a type system for our programs as we did before.
\item Design a proof system with a proof checker, which supports code
  extraction.\footnote{Technically we do not need code extraction, just a way to
    execute programs inside the system in real-world data.}
\item Implement a type-checker inside the proof system.
\item Implement a model for how programs are evaluated inside the proof system
  (i.e., a small-step semantics). This should include a definition of that
  constitutes a safe program inside the model.
\item Use the proof system to prove that the type-checker is correct.
\item Use the proof system to prove that well-typed programs are safe
  \emph{inside the model} (i.e., a proof of soundness).
\item Assume that our model is at least somewhat correct. Specifically assume
  that programs that are safe in the model are also safe in the real-world.
\item Let the proof system, OS and CPU be part of the Trusted Computing Base.
\item Run the type-checker on the downloaded programs, but only run the program
  itself if it is well-typed.
\end{itemize}

This is significantly better than before. Besides our Trusted Computing Base, we
now only have a single assumption, and this assumption is at least somewhat
reasonable. The only real issues is whether we can trust the proof
system.

There are even some additional advantages to this model: Most of our components
do not need to be trusted beyond limitations. The proof checker needs only check
\emph{our proofs} correctly and have a correct code extraction
procedure. Furthermore OS and CPU do not need to be correct in the face of
\emph{any possible} program, only those permitted by the type system.

If this procedure ever becomes efficient and usable enough to take of, it should
improve real-world security significantly. However there are still some possible
points of improvement.

\section{Further improvements}

There are still a few avenues available to us, for reducing assumptions even
further:

\paragraph{Simplify the model.} We can try to simplify the model and bring it
closer to the real world. The closer the two models are, the fewer things could
possibly be wrong. We can achieve some of these benefits by implementing erasure
theorems inside the proof system, however it can never bridge the gap
completely, as we still need to assume that our models definition of safety
corresponds to the real-world definition, which is unfortunately somewhat
underspecified.

\paragraph{Improve the proof system.} We might try to improve the proof system
in several ways. We might try to simplify the implementation or get it peer
reviewed to alleviate some of our uncertainties. This might be an easier task
than reviewing e.g. a sandbox, and might be applicable in a broader set of
circumstances. We might also try to implement the proofs in other proof systems,
though we will need to take care that the proofs are equivalent. Alternatively
we could try to verify the proof system inside another proof system.

\paragraph{Simplify the OS.} One can to a certain extend simplify the OS,
especially if one is only optimizing for this specific scenario. Many
OS-specific features (and in particular protections) might be better implemented
using our proof system instead of the OS. This move of responsibility from the
OS to the proof system will strictly decrease our TCB, since the proof system
does not have to increase in size to accommodate the change.

\section{Erasure and Safety}
