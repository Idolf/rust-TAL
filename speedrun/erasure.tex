\chapter{Erasure Semantics}
\label{chap:erasure}

In Types and Programming Languages\cite{typesandprog} erasure is defined as the following
property between a typed language and an untyped version of the same language:

\begin{itemize}
\item There must be a function $\erase$ which maps values from the typed language
  to the untyped language.
\item If we have $t -> t'$ in the typed language, then we must have
  $\erase(t) -> \erase(t')$ in the untyped language.
\item If we have $\erase(t) -> m'$ in the untyped semantics, then there is a $t'$
  such that $t -> t'$ and $\erase(t') = m'$.
\end{itemize}

Unfortunately this property does not quite hold in our case. Instead we will
define an $\erase$ function such that the following holds:

\begin{itemize}
\item If we have $\steps{\high P}{\high P'}$ in \ATAL, then we have
  $\steps{\erase(\high P)}{\erase(\high P')}$ in \ATALe.
\item If we have $\steps{\erase(\high P)}{\simple{P'}}$ in \ATALe, then one of
  these two holds:
  \begin{itemize}
  \item There is a $\high P'$ such that $\steps{\high P}{\high P'}$ and
    $\erase(\high P') = \simple{P'}$ or
  \item $\high P$ is in a bad state, i.e. it is neither halting nor
    progressing. Instead it is stuck, and specifically this occurs because some
    substitution was impossible. We will later prove that cannot occur for
    well-typed terms.
  \end{itemize}
\end{itemize}

At this point it is useful to take a step back and ask, \emph{why} erasure is a
useful property when working with TALs. There are a few good reasons for
including erasure as part of the work on \ATAL:

\begin{itemize}
\item \ATAL depends on types at run-time for its execution. while \ATALe does
  not. Since the two languages are essentially equivalent, one can simply
  translate from \ATAL to \ATALe before running the program.
\item A real-world CPU is not able to run \ATAL. While it is not able to \ATALe
  either, it is closer to the real-world model. If one wanted to justify
  correctness of our semantics it is much easier to argue why the small-step
  semantics of \ATALe correctly models the world than that of \ATAL.
\item We can once and for all define \ATALe and convince ourselves of its
  adequacy. We are then free to experiment with different type-systems as long
  as they all have the erasure property.
\item In fact it is possible to have several different type-checkers as part of
  the same sandbox system. This might be useful, as it would allow different
  TALs with different features to use the same platform.
\end{itemize}

\section{The erasure function for \ATAL}

The erasure function is mostly the obvious one: Simply drop all occurrences of
types or instantiations. In the Agda implementation
(\texttt{Judgments/Erase.agda}) this has been implemented as several different
functions having the same interface -- one for each syntactic category. We will
here write a few of the functions; for a complete listing see the
appendix.\todo{Where in the appendix?}

{\footnotesize
\[\begin{array}{l}
\begin{array}{ll}
\rlap{$\erase_w : \high w -> \simple{w}$} \\
\erase_w(\mathtt{globval}\ \ell_g) &= \mathtt{globval}\ \ell_g \\
\erase_w(\mathtt{heapval}\ \ell_h) &= \mathtt{heapval}\ \ell_h \\
\erase_w(\mathtt{int}\ i) &= \mathtt{int}\ i \\
\erase_w(\mathtt{uninit}) &= \mathtt{uninit} \\
\erase_w(\Lambda\ \Delta \cdot \high w[\Theta]) &= \erase_w(\high w) \\
\end{array} \\\\

\begin{array}{l}
\rlap{$\erase_g : \high g -> \simple{g}$} \\
\erase_g(\mathtt{code}[\Delta]\Gamma \cdot I) = \mathtt{code}\ (\erase_I(I)) \\\\

\rlap{$\erase_h : \high h -> \simple{h}$} \\
\erase_h(\mathtt{tuple}\ \langle \tau_1, \dots, \tau_m \rangle\ \langle \high w_1, \dots, \high w_n \rangle) = \mathtt{tuple}\ \langle \erase_w(\high w_1), \dots, \erase_w(\high w_n) \rangle \\
\end{array} \\\\

\begin{array}{ll}
\rlap{$\erase_{\iota} : \high\iota -> \simple{\iota}$} \\
\erase_{\iota}(\mathtt{add}\ r_a, r_b, \high v) &= \mathtt{add}\ r_a, r_b, \erase_v(\high v) \\
\erase_{\iota}(\mathtt{sub}\ r_a, r_b, \high v) &= \mathtt{sub}\ r_a, r_b, \erase_v(\high v) \\
\erase_{\iota}(\mathtt{salloc}\ n) &= \mathtt{salloc}\ n \\
\erase_{\iota}(\mathtt{sfree}\ n) &= \mathtt{sfree}\ n \\
\erase_{\iota}(\mathtt{ld}\ r, \mathtt{sp}(n)) &= \mathtt{ld}\ r, \mathtt{sp}(n) \\
\erase_{\iota}(\mathtt{st}\ \mathtt{sp}(n), r) &= \mathtt{st}\ \mathtt{sp}(n), r \\
\erase_{\iota}(\mathtt{ld}\ r_a, r_b(n)) &= \mathtt{ld}\ r_a, r_b(n) \\
\erase_{\iota}(\mathtt{st}\ r_a(n), r_b) &= \mathtt{st}\ r_a(n), r_b \\
\erase_{\iota}(\mathtt{malloc}\ r, \langle \tau_1, \dots, \tau_n \rangle) &= \mathtt{malloc}\ r, n \\
\erase_{\iota}(\mathtt{mov}\ r, \high v) &= \mathtt{mov}\ r, \erase_v(\high v) \\
\erase_{\iota}(\mathtt{beq}\ r, \high v) &= \mathtt{beq}\ r, \erase_v(\high v) \\
\end{array} \\\\

\begin{array}{l}
\erase_P : \high P -> \simple{P} \\
\erase_P(\high G, \high H, \high R, \high I) = (\erase_G(\high G), \erase_H(\high H), \erase_R(\high R), \erase_I(\high I)) \\\\
\end{array} \\\\

\end{array}\]
}

As the subscript can be inferred unambigiously in most cases, we will leave them
out when possible.

\section{Sketch proof of erasure}

The proof itself is contained in the file \texttt{Lemmas/Erase.agda}. It
contains a lot of boilerplate lemmas, so we will only include a sketch of the
proof here.

\begin{lemma}
  Erasure commutes with list-lookup. Specifically if
  $L = \langle \high w_1, \dots \high w_n \rangle$ then
  $\erase(L[i]) = \erase(L)[i]$.
\end{lemma}
\begin{proof}
  By induction over $i$.
\end{proof}

\begin{lemma}
  Erasure commutes with list-update. Specifically if
  $L = \langle \high w_1, \dots \high w_n \rangle$ then
  $\erase(L[i |-> \high w]) = \erase(L)[i |-> \erase(\high w)]$.
\end{lemma}
\begin{proof}
  By induction over $i$.
\end{proof}

At this point we have many similar, small lemmas. After those we can prove that
a substitution will only affect parts that are not preserved by erasure:

\begin{lemma}
  \label{lemma:suberase}
  Let $\high v, \high v'$ be small values such that
  $\high v[\theta] = \high v'$. In that case we have
  $\erase(\high v) = \erase(\high v')$. Similar lemmas hold for instructions and
  instruction sequences.
\end{lemma}
\begin{proof}
  By structural induction over $\high v$ (and similarly for the other syntactic
  categories).
\end{proof}

After this we can start on the ``real'' erasure lemmas:

\begin{lemma}
  Assume we have $\evalregs{\high R}{\high v}{\high w}$, then we also have
  $\evalregs{\erase(\high R)}{\erase(\high v)}{\erase(\high w)}$.
\end{lemma}
\begin{proof}
  By structural induction over $\high v$. The only non-trivial case is
  $v = \mathtt{reg}\ r$, where we use one of the trivial lemmas.
\end{proof}

\begin{lemma}
  Assume we have $\evalcode{\high G}{\high w}{\high I}$, then we also have
  $\evalcode{\erase(\high G)}{\erase(\high w)}{\erase(\high I)}$.
\end{lemma}
\begin{proof}
  By structural induction over $\high w$ while using \cref{lemma:suberase}.
\end{proof}

Finally, the core lemma of our proof:

\begin{lemma}
  \label{lemma:erasestep}
  Assume we have
  $\execis{\high G}{\high H, \high R, \high I}{\high H', \high R', \high
    I'}$. In that case we also have
  $\execis{\erase(\high G)}{\erase(\high H, \high R, \high I)}{\erase(\high H',
    \high R', \high I')}$.
\end{lemma}
\begin{proof}
  We will only do a sketch a few of the cases, as the actual proof depends on
  many lemmas and has many cases. In the body of this particular proof we will
  write $x^{\mathrm{e}}$ as shorthand for $\erase(x)$, as some lines would be

  Let $\mathcal{S}$ be our derivation of
  $\execis{\high G}{\high H, \high R, \high I}{\high I', \high R', \high
    I'}$. Consider the cases of $\high I$:

  \paragraph{Assume $\high I = \mathtt{add}\ r_a, r_b, \high v ; \high I''$.}
  Thus $\mathcal{S}$ must be of the form:

  \begin{mathpar}
    \infer{
      \stackrel{\mathcal{E}_0}{\high R[r_b] = \mathtt{int}\ i_1} \and
      \stackrel{\mathcal{E}_1}{\evalregs{\high R}{\high v}{\mathtt{int}\ i_2}}
    }{
      \execis{\high G}
      {\high H, \high R, (\mathtt{add}\ r_a, r_b, \high v) ; \high I''}
      {\high H, \high R[r_a \mapsto \mathtt{int}\ (i_1 + i_2)], \high I''}
    }
  \end{mathpar}

  By applying lemmas on $\mathcal{E}_0$ and $\mathcal{E}_1$ we get derivations
  $\mathcal{E}_0', \mathcal{E}_1'$ of $R^{\mathrm{e}}[r_b] = \mathtt{int}\ i_1$ and
  $\evalregs{\high R^{\mathrm{e}}}{\high v^{\mathrm{e}}}{\mathtt{int}\ i_2}$ respectively. From this
  we can construct a derivation:

  \begin{mathpar}
    \infer{
      \stackrel{\mathcal{E}_0'}{\high R^{\mathrm{e}}[r_b] = \mathtt{int}\ i_1} \and
      \stackrel{\mathcal{E}_1'}{\evalregs{\high R^{\mathrm{e}}}{\high v^{\mathrm{e}}}{\mathtt{int}\ i_2}}
    }{
      \execis{\high G^{\mathrm{e}}}
      {\high H^{\mathrm{e}}, \high R^{\mathrm{e}}, (\mathtt{add}\ r_a, r_b, \high v^{\mathrm{e}}) ; \high I''{}^{\mathrm{e}}}
      {\high H^{\mathrm{e}}, \high R^{\mathrm{e}}[r_a \mapsto \mathtt{int}\ (i_1 + i_2)], \high I''{}^{\mathrm{e}}}
    }
  \end{mathpar}

  We also have a lemma stating that
  $\erase(\high R[r_a \mapsto \mathtt{int}\ (i_1 + i_2)]) = \high R^{\mathrm{e}}[r_a
  \mapsto \mathtt{int}\ (i_1 + i_2)]$, so this derivation is what we wanted.

  \paragraph{Assume $\high I = \mathtt{malloc}\ r, \langle \tau_1, \dots, \tau_n \rangle ; \high I''$.}

  Thus $\mathcal{S}$ must be of the form:

  \begin{mathpar}
    \infer{
      \stackrel{\mathcal{E}_0}{\ell_{\mathrm{h}} \notin \mathbf{dom}(\high H)} \and
      \stackrel{\mathcal{E}_1}{\high h = \mathtt{tuple}\ \langle \tau_1, \dots, \tau_n \rangle\ \overbrace{\langle \mathtt{uninit}, \dots, \mathtt{uninit} \rangle}^n}
    }{
      \execis{\high G}
      {\high H, \high R, (\mathtt{malloc}\ r, \langle \tau_1, \dots, \tau_n \rangle) ; \high I''}
      {\high H \cup \{\ell_{\mathrm{h}} |-> \high h\}, R[r |-> \mathtt{heapval}\ \ell_{\mathrm{h}}], \high I''}
    }
  \end{mathpar}

  By applying a lemmas on $\mathcal{E}_0, \mathcal{E}_1$ we get a derivation
  $\mathcal{E}_0', \mathcal{E}_1'$ of
  $\ell_{\mathrm{h}} \notin \mathbf{dom}(\high H^{\mathrm{e}})$ and
  $\high h^{\mathrm{e}} = \mathtt{tuple}\ \overbrace{\langle \mathtt{uninit}, \dots,
    \mathtt{uninit} \rangle}^n$ respectively. From this we can construct a derivation:

  \begin{mathpar}
    \infer{
      \stackrel{\mathcal{E}_0}{\ell_{\mathrm{h}} \notin \mathbf{dom}(\high H^{\mathrm{e}})} \and
      \stackrel{\mathcal{E}_1'}{\high h^{\mathrm{e}} = \mathtt{tuple}\ \overbrace{\langle \mathtt{uninit}, \dots, \mathtt{uninit} \rangle}^n}
    }{
      \execis{\high G^{\mathrm{e}}}
      {\high H^{\mathrm{e}}, \high R^{\mathrm{e}}, (\mathtt{malloc}\ r, n) ; \high I''{}^{\mathrm{e}}}
      {\high H^{\mathrm{e}} \cup \{\ell_{\mathrm{h}} |-> \high h^{\mathrm{e}}\}, \high R^{\mathrm{e}}[r |-> \mathtt{heapval}\ \ell_{\mathrm{h}}], \high I''{}^{\mathrm{e}}}
    }
  \end{mathpar}

  Since we have a lemma showing that
  $\erase(\high H \cup \{\ell_{\mathrm{h}} |-> \high h\}) = \high H^{\mathrm{e}}
  \cup \{\ell_{\mathrm{h}} |-> \high h^{\mathrm{e}}\}$ and another showing that
  $\erase(\high R[r |-> \mathtt{heapval}\ \ell_{\mathrm{h}}]) = \high
  R^{\mathrm{e}}[r |-> \mathtt{heapval}\ \ell_{\mathrm{h}}]$, this derivation is
  what we wanted.
\end{proof}

The first part of the erasure property follows directly from this.

\begin{corollary}[Erasure part 1]
  \label{lemma:erasesteps}
  If we have $\steps{\high P}{\high P'}$ in \ATAL, then we have
  $\steps{\erase(\high P)}{\erase(\high P')}$ in \ATALe.
\end{corollary}
\begin{proof}
  There is only a single case for the value of $\steps{\high P}{\high P'}$, and
  in this case we can simply use \cref{lemma:erasestep}.
\end{proof}

The second part of the erasure property follows almost directly.
\begin{lemma}[Erasure part 2]
  \item If we have $\steps{\erase(\high P)}{\simple{P'}}$ in \ATALe, then one of
  these two holds:
  \begin{itemize}
  \item There is a $\high P'$ such that $\steps{\high P}{\high P'}$ and
    $\erase(\high P') = \simple{P'}$ or
  \item $\high P$ is in a bad state, i.e. it is neither halting nor
    progressing. Instead it is stuck, and specifically this occurs because some
    substitution was impossible. We will later prove that cannot occur for
    well-typed terms.
  \end{itemize}
\end{lemma}
\begin{proof}
  Assume we have $\steps{\erase(\high P)}{\simple{P'}}$. We have two
  cases\footnote{Note that for this lemma to be provable in Agda, we need a
    \emph{constructive} proof -- that is we cannot simply assume $Q \lor \neg Q$
    for any $Q$. Luckily we have already proved that our small-step relation is
    decidable, which is what we need here.}

  \paragraph{Assume $\steps{\high P}{\high P'}$ for some $\high P'$.} By
  \cref{lemma:erasesteps} this implies that
  $\steps{\erase(\high P)}{\erase(\high P')}$. We have previously proven that
  the small-step relation is a partial function, which implies
  $\erase(\high P') = \simple{P'}$. This concludes the case, as we fulfill the
  requirements for the first clause.

  \paragraph{Assume no such $\high P'$ exists.} To fulfill the second clause, it
  remains to be shown that $\high P$ is not halting. If $\high P$ was halting
  then we would have
  $\erase(\high P) = (\erase(\high G), \erase(\high H), \erase(\high R),
  \mathtt{halt})$. However this is a contradiction, since it the derivation
  $\steps{(\erase(\high G), \erase(\high H), \erase(\high R),
    \mathtt{halt})}{\simple{P'}}$ can clearly not exist.
\end{proof}
