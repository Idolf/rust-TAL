\section{Motivation}

Let us consider a software developer, with the job of developing a piece of code
for a client. She is considering in what format to distribute her final product.
As the choices seem overwhelming, she decides to list a few properties she would
like her final product to have, before she begins developing it.

\paragraph{Low overhead:}
Her product should run as fast as possible while using as few resources as
possible.

\paragraph{Low latency:}
Her product should be executable as soon as possible after her client receives
it. So while she considers herself an Open Source advocate, she concludes that
distributing the source code is not a solution to the \emph{current} problem.

\paragraph{Reusability:}
Whatever method she uses to create her product should be reusable for other
projects. She concludes that her method should support general-purpose code.

\paragraph{Security:}
Ideally she would like to convince her client beyond any doubt, that her product
serves the intended function. However her client have had problems even settling
on a consistent design specification, much less a formalized one.

She decides to settle for a slightly less ambiguous goal instead: Absolute
certainty that it is not \emph{malicious}.

\paragraph{}
After combining the first three requirements, she is left with very few options:

\begin{enumerate}
\item She could distribute raw machine code (inside a suitable container format
  such as \texttt{PE} or \texttt{ELF}).
\item She could distribute a bytecode format intended to be run through a fast
  interpreter or JIT-compiler.
\end{enumerate}

She are now left with the problem of convincing her client that her code is
indeed not malicious. She decides to do a small real-world survey to see how
other people have tried to solve this problem:

\begin{itemize}
\item Do nothing! Given that most people are very trusting, many will simply run
  whatever code they receive.
\item Expect the client to run anti-virus software to detect malicious code.
\item Buy a code-signing certificate and sign the code. The security of this
  system is based on the fact that a code-signing certificates are
  semi-expensive and only issued to entities verified to be non-malicious by a
  trusted Certificate Authority (CA).
\item Have the client run the code inside a sandboxed environment, by using
  e.g. the Java VM, Docker or Native Client from the Chromium project.
\item Publish the code as open source along with a method for compiling that
  source into a bitwise identical result.
\item Some combination of the above.
\end{itemize}

One notices that none of the real-world solutions actually \emph{solves} the
security requirement as stated!

\begin{itemize}
\item Rice's Theorem states that the anti-virus model is inherently
  imperfect. In fact, it is a quite common occurrence for real-world anti-virus
  software to produce both false positives and false negatives.
\item There are multiple examples of code-signing certificate being stolen or
  created under fake pseudonyms. There have also been examples of CAs being
  hacked or coerced to create certificates.
\item None of the sandboxes we might consider have perfect security without
  placing unreasonable demands on the other properties. In fact many of them
  have known public exploits for previous versions.
\item Even open source software with security audits sometimes have bugs -- and
  even backdoors are not completely unheard of.
\end{itemize}

Desperately she turns to research: Is there really no solution in sight? It
turns out there is: Proof-carrying code.

\later[inline]{A lot of citations.}
