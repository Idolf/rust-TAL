\chapter{Lessons Learned From Using Agda}
\label{chap:hindsight}

The purpose of this chapter is to give document some of my experiences with
learning Agda and using it to implement my TAL. This has been my first real
project in Agda, and as such I have made a lot of errors that could have been
avoided had I known the language better.

As such this chapter will contain some ricks I have discovered mixed with some
of my (hopefully) interesting observations about learning the language.

\section{Extending the Standard Library}

When initially learning Agda, the standard library likely to be ones biggest
guide for writing practical Agda-code. The good news is that the standard
library is generally of high quality and contain many useful functions for doing
very general-purpose things. The bad news is that 1) the libraries have often
been generalized to the point where they are completely unusable by new users
and 2) for some of the library it seems somewhat random which lemmas have been
deemed important enough to be included.

As such I was somewhat forced to implement my own standard library on top of the
Agda standard library. This was very good practice for learning Agda, however it
was also somewhat wasteful. As I continued learning the language I noticed that
considerable parts of my own library was actually present in the Agda standard
library. This unfortunately did not mean that I could simply stop using my own
implementation, as the version in the standard library was often slightly
different, harder to use or had issues related to scoping in the module system.

\section{Decidable Equality}
\label{sec:deceq}

One of the general-purpose techniques I am most proud of is my solution to
proving decidable equality. In Agda one often wants to prove decidable equality
for some type $T$. In particular one wishes to prove a lemma of the following
form:

\begin{lemma}[Decidable Equality]
  Let $v_1, v_2 \in T$. We can now derive a proof of either $v_1 \equiv v_2$ or
  $v_1 \not\equiv v_2$.
\end{lemma}

This lemma is in Agda often written as a dependent function with the signature:

$$\cdot ≟ \cdot\;\;:\;\;(v_1 : T) \to (v_2 : T) \to \mathtt{Dec}(v_1 \equiv v_2)$$

An example from Agda standard library (with slight modifications):

\begin{code}%
\>\<%
\\
\>\AgdaFunction{\_≟\_} \AgdaSymbol{:} \AgdaSymbol{(}\AgdaBound{v₁} \AgdaSymbol{:} \AgdaDatatype{Bool}\AgdaSymbol{)} \AgdaSymbol{→} \AgdaSymbol{(}\AgdaBound{v₂} \AgdaSymbol{:} \AgdaDatatype{Bool}\AgdaSymbol{)} \AgdaSymbol{→} \AgdaDatatype{Dec} \AgdaSymbol{(}\AgdaBound{v₁} \AgdaDatatype{≡} \AgdaBound{v₂}\AgdaSymbol{)}\<%
\\
\>\AgdaInductiveConstructor{true} \<[6]%
\>[6]\AgdaFunction{≟} \AgdaInductiveConstructor{true} \<[14]%
\>[14]\AgdaSymbol{=} \AgdaInductiveConstructor{yes} \AgdaInductiveConstructor{refl}\<%
\\
\>\AgdaInductiveConstructor{false} \AgdaFunction{≟} \AgdaInductiveConstructor{false} \AgdaSymbol{=} \AgdaInductiveConstructor{yes} \AgdaInductiveConstructor{refl}\<%
\\
\>\AgdaInductiveConstructor{true} \<[6]%
\>[6]\AgdaFunction{≟} \AgdaInductiveConstructor{false} \AgdaSymbol{=} \AgdaInductiveConstructor{no} \AgdaSymbol{λ()}\<%
\\
\>\AgdaInductiveConstructor{false} \AgdaFunction{≟} \AgdaInductiveConstructor{true} \<[14]%
\>[14]\AgdaSymbol{=} \AgdaInductiveConstructor{no} \AgdaSymbol{λ()}\<%
\end{code}

\paragraph{}
While these types of lemmas are in principle trivial to generate for most
datatypes, there is a problem: required number of lines is \emph{quadratic} in
the number of constructors for the datatype. The reason for this quadratic
explosion is the need to generate proofs for $v_1 \neq v_2$.

I found a solution to the problem in a blog-post\cite{deceq}, though I
generalized the solution slightly. The idea is a follows:

\begin{itemize}
\item Construct an injective function $f$ from the original type $T$ to another
  type $T'$.
\item Prove decidable equality for $T'$.
\item Use this to decide $f(v_1) \equiv f(v_2)$.
\item We know that $f(v_1) \not\equiv f(v_2)$ implies $v_1 \not\equiv v_2$,
  since $f$ is a function.
\item We also know that $f(v_1) \equiv f(v_2)$ implies $v_1 \equiv v_2$, since
  $f$ is injective.
\item If we combine these, we get decidable equality for $T$.
\end{itemize}

By choosing $T'$ to be the same for every $T$, we can reuse a lot code. In
practice my code does the following:

\begin{itemize}
\item Define a type \texttt{Tree}, which is the set of rooted, ordered tree
  having a number attached to each node. This datatype was chosen since this
  makes it easy to construct the injective functions.
\item Prove decidable equality for \texttt{Tree}.
\item Prove that given $T$ and an injective functions $f : T \to \mathtt{Tree}$,
  we have decidable equality for $T$.
\item Prove that given $T$ and a surjective function $g : \mathtt{Tree} \to T$,
  we can construct an injective function $f : T \to \mathtt{Tree}$, and thus
  this also implies decidable equality on $T$.
\end{itemize}

This last step is a slight further optimization, as it turns out to take fewer
lines of code in Agda to specify a $g$ and prove it surjective, than to specify
a function $f$ and prove it injective.

For each additional type $T$ for which one want to prove decidable equality, one
now only need to:

\begin{itemize}
\item Create a function $g : \mathtt{Tree} \to T$.
\item Prove that $g$ is surjective.
\end{itemize}

For most datatypes, will only require $2k+3$ lines (or slightly more if we wrap
long lines), where $k$ is the number of number of constructors in the definition
of $T$. To illustrate how this method works in practice, here is small example
from \texttt{Lemmas/Equality.tex}:

\begin{code}
  \>[0]\AgdaIndent{2}{}\<[2]%
\>[2]\AgdaFunction{InstructionSequenceₕ-Tree} \AgdaSymbol{:} \AgdaRecord{ToTree} \AgdaFunction{H.InstructionSequence}\<%
\\
\>[0]\AgdaIndent{2}{}\<[2]%
\>[2]\AgdaFunction{InstructionSequenceₕ-Tree} \AgdaSymbol{=} \AgdaFunction{tree⋆} \AgdaFunction{from} \AgdaFunction{sur}\<%
\\
\>[2]\AgdaIndent{4}{}\<[4]%
\>[4]\AgdaKeyword{where} \AgdaFunction{from} \AgdaSymbol{:} \AgdaDatatype{Tree} \AgdaSymbol{→} \AgdaDatatype{Maybe} \AgdaFunction{H.InstructionSequence}\<%
\\
\>[4]\AgdaIndent{10}{}\<[10]%
\>[10]\AgdaFunction{from} \AgdaSymbol{(}\AgdaInductiveConstructor{node} \AgdaNumber{0} \AgdaSymbol{(}\AgdaBound{ι} \AgdaInductiveConstructor{∷} \AgdaBound{I} \AgdaInductiveConstructor{∷} \AgdaSymbol{\_))} \AgdaSymbol{=} \AgdaInductiveConstructor{\_\textasciitilde>\_} \AgdaFunction{<\$>} \AgdaFunction{fromTree} \AgdaBound{ι} \AgdaFunction{<*>} \AgdaFunction{from} \AgdaBound{I}\<%
\\
\>[4]\AgdaIndent{10}{}\<[10]%
\>[10]\AgdaFunction{from} \AgdaSymbol{(}\AgdaInductiveConstructor{node} \AgdaNumber{1} \AgdaSymbol{(}\AgdaBound{v} \AgdaInductiveConstructor{∷} \AgdaSymbol{\_))} \AgdaSymbol{=} \AgdaInductiveConstructor{jmp} \AgdaFunction{<\$>} \AgdaFunction{fromTree} \AgdaBound{v}\<%
\\
\>[4]\AgdaIndent{10}{}\<[10]%
\>[10]\AgdaFunction{from} \AgdaSymbol{(}\AgdaInductiveConstructor{node} \AgdaNumber{2} \AgdaSymbol{\_)} \AgdaSymbol{=} \AgdaInductiveConstructor{just} \AgdaInductiveConstructor{halt}\<%
\\
\>[4]\AgdaIndent{10}{}\<[10]%
\>[10]\AgdaFunction{from} \AgdaSymbol{\_} \AgdaSymbol{=} \AgdaInductiveConstructor{nothing}\<%
\\
\>[4]\AgdaIndent{10}{}\<[10]%
\>[10]\AgdaFunction{sur} \AgdaSymbol{:} \AgdaFunction{IsSurjective} \AgdaFunction{from}\<%
\\
\>[4]\AgdaIndent{10}{}\<[10]%
\>[10]\AgdaFunction{sur} \AgdaSymbol{(}\AgdaBound{ι} \AgdaInductiveConstructor{\textasciitilde>} \AgdaBound{I}\AgdaSymbol{)} \AgdaSymbol{=} \AgdaFunction{T₂} \AgdaNumber{0} \AgdaBound{ι} \AgdaSymbol{(}\AgdaField{proj₁} \AgdaSymbol{(}\AgdaFunction{sur} \AgdaBound{I}\AgdaSymbol{))} \AgdaInductiveConstructor{,}\<%
\\
\>[10]\AgdaIndent{12}{}\<[12]%
\>[12]\AgdaInductiveConstructor{\_\textasciitilde>\_} \AgdaFunction{<\$=>} \AgdaFunction{invTree} \AgdaBound{ι} \AgdaFunction{<*=>} \AgdaField{proj₂} \AgdaSymbol{(}\AgdaFunction{sur} \AgdaBound{I}\AgdaSymbol{)}\<%
\\
\>[0]\AgdaIndent{10}{}\<[10]%
\>[10]\AgdaFunction{sur} \AgdaSymbol{(}\AgdaInductiveConstructor{jmp} \AgdaBound{v}\AgdaSymbol{)} \AgdaSymbol{=} \AgdaFunction{T₁} \AgdaNumber{1} \AgdaBound{v} \AgdaInductiveConstructor{,} \AgdaInductiveConstructor{jmp} \AgdaFunction{<\$=>} \AgdaFunction{invTree} \AgdaBound{v}\<%
\\
\>[0]\AgdaIndent{10}{}\<[10]%
\>[10]\AgdaFunction{sur} \AgdaInductiveConstructor{halt} \AgdaSymbol{=} \AgdaFunction{T₀} \AgdaNumber{2} \AgdaInductiveConstructor{,} \AgdaInductiveConstructor{refl}\<%
\end{code}

\section{Working with Lists}
When doing the project, I needed many operations on lists and vectors that where
either not provided by the standard library or were very inconsistently
implemented.

Specifically I needed:

\begin{itemize}
\item Relations for looking up values along with updating them.
\item Applying a predicate to an entire list or vector (e.g., ``this list of types are all well-formed'').
\item Applying a predicate to two lists in parallel (e.g., ``this list of word values and list of types match each other'')
\item Looking up values and updating those said predicate lists.
\item Other ways to transform said predicate lists.
\item Decidability lemmas for all of the above.
\item Etc.
\end{itemize}

A few of those needs can be filled directly by the standard library, however
those that can have for the most part only been implemented for either lists or
vectors. I have implemented a mostly consistent interface for all of those
features which work with both lists and vectors.

I am considering cleaning up the interface somewhat and trying to get it
accepted into the Agda standard library.

\section{Extensions to Equality Reasoning}
Ideally one would prefer to deal with proving stuff about equalities as little
as possible when doing Agda, as they quickly become very cumbersome to work
with. As long as a single or two invocations of the congruence, transitivity or
reflexivity will suffice, it is somewhat managable, but sometimes much more
convoluted proofs are (unfortunately) necessary.

To deal with this, the standard library has a tool called Equality
Reasoning. This is some special-purpose functions for dealing with
equalities.\footnote{It is likely best thought of as an embedded domain-specific
  language, though the line between EDSL and ``useful functions'' are somewhat
  blurred in Agda.}

They are in principle simply a way of chaining a lot of equalities using
transitivity, however in practice it does not feel that way. It allows one to
write long-yet-managable proofs such as:

\begin{code}
\>[2]\AgdaIndent{4}{}\<[4]%
\>[4] \AgdaFunction{begin}\<%
\\
\>[4]\AgdaIndent{6}{}\<[6]%
\>[6]\AgdaSymbol{(}\AgdaBound{pos₂} \AgdaPrimitive{+} \AgdaInductiveConstructor{suc} \AgdaSymbol{(}\AgdaFunction{length} \AgdaBound{θs₁}\AgdaSymbol{))} \AgdaPrimitive{+} \AgdaBound{pos₁}\<%
\\
\>[0]\AgdaIndent{4}{}\<[4]%
\>[4]\AgdaFunction{≡⟨} \AgdaFunction{+-comm} \AgdaBound{pos₂} \AgdaSymbol{(}\AgdaInductiveConstructor{suc} \AgdaSymbol{(}\AgdaFunction{length} \AgdaBound{θs₁}\AgdaSymbol{))} \AgdaFunction{∥} \AgdaSymbol{(λ} \AgdaBound{v} \AgdaSymbol{→} \AgdaBound{v} \AgdaPrimitive{+} \AgdaBound{pos₁}\AgdaSymbol{)} \AgdaFunction{⟩}\<%
\\
\>[4]\AgdaIndent{6}{}\<[6]%
\>[6]\AgdaSymbol{(}\AgdaInductiveConstructor{suc} \AgdaSymbol{(}\AgdaFunction{length} \AgdaBound{θs₁}\AgdaSymbol{)} \AgdaPrimitive{+} \AgdaBound{pos₂}\AgdaSymbol{)} \AgdaPrimitive{+} \AgdaBound{pos₁}\<%
\\
\>[0]\AgdaIndent{4}{}\<[4]%
\>[4]\AgdaFunction{≡⟨} \AgdaFunction{+-comm} \AgdaSymbol{(}\AgdaFunction{length} \AgdaBound{θs₁}\AgdaSymbol{)} \AgdaBound{pos₂} \AgdaFunction{∥} \AgdaSymbol{(λ} \AgdaBound{v} \AgdaSymbol{→} \AgdaInductiveConstructor{suc} \AgdaBound{v} \AgdaPrimitive{+} \AgdaBound{pos₁}\AgdaSymbol{)} \AgdaFunction{⟩}\<%
\\
\>[4]\AgdaIndent{6}{}\<[6]%
\>[6]\AgdaSymbol{(}\AgdaInductiveConstructor{suc} \AgdaBound{pos₂} \AgdaPrimitive{+} \AgdaFunction{length} \AgdaBound{θs₁}\AgdaSymbol{)} \AgdaPrimitive{+} \AgdaBound{pos₁}\<%
\\
\>[0]\AgdaIndent{4}{}\<[4]%
\>[4]\AgdaFunction{≡⟨} \AgdaFunction{+-assoc} \AgdaSymbol{(}\AgdaInductiveConstructor{suc} \AgdaBound{pos₂}\AgdaSymbol{)} \AgdaSymbol{(}\AgdaFunction{length} \AgdaBound{θs₁}\AgdaSymbol{)} \AgdaBound{pos₁} \AgdaFunction{⟩}\<%
\\
\>[4]\AgdaIndent{6}{}\<[6]%
\>[6]\AgdaInductiveConstructor{suc} \AgdaBound{pos₂} \AgdaPrimitive{+} \AgdaSymbol{(}\AgdaFunction{length} \AgdaBound{θs₁} \AgdaPrimitive{+} \AgdaBound{pos₁}\AgdaSymbol{)}\<%
\\
\>[0]\AgdaIndent{4}{}\<[4]%
\>[4]\AgdaFunction{∎} \AgdaKeyword{where} \AgdaKeyword{open} \AgdaModule{Eq-Reasoning}\<%
\end{code}

In my opinion there are however a few problems with equality reasoning defined
in the standard library, for which reason I chose to implement my own version
instead:

\begin{itemize}
\item The standard library is generalized to the point that it is \emph{very}
  hard to understand for new-comers. I begun my implementation before I figured
  out how to even use version in the standard library (and I am still not very
  comfortable with the version in the standard library).
\item The version in the standard library has a fairly week interface. It is
  \emph{in principle} just as strong as my version -- though then again, so use
  direct manipulation of congruence, transitivity and reflexivity.
\end{itemize}

For a long time, I had simply written an entire sentence in place of this entire
section. I chosen to leave it, as it provides a fairly good illustration of my
ambivalence towards this technique:

\begin{quote}
``One would rather try to avoid equality reasoning, but if that is not possible,
one might as well have as strong a hacksaw as possible.''
\end{quote}

\section{Managing Name using Records and Modules}
A recent feature in Agda, is that records can be used in more-or-less the same
way as Haskell typeclasses. It should not be underestimated how useful it is,
even beyond how it is used in Haskell. This was used a lot during the project,
in particular for two purposes:

\begin{itemize}
\item It allowed use of overloaded functions, such as \texttt{erase} or lemmas
  such as \texttt{subst-subst}. These functions could then be used for many
  datatypes, just like one would expect from Haskell.
\item More interestingly it also allowed for overloaded syntax. For instance of
  one types $|-\ x\ \mathbf{Valid}$ in the Agda-code, then this can refer to
  many different datatypes/judgments depending on the type of $x$.
\end{itemize}

In addition to using records to overload syntax, I also using module scoping to
provide a uniform interface to both \ATAL and \ATALe. This allowed me to reuse
code in a non-traditional way. I did this by having nearly-identical files which
would only differ in what module why imported. While is not very satisfying from
a technical perspective, it is much nicer than the naive alternative of simply
having two diverging implementations.

\section{Decidable Propositions}

The interfaces in Agda for producing decidable interfaces is very nice to work
with. If one has some proposition $P : A -> \mathtt{Set}$ and want to prove that
it is decidable, then one simply creates a value of type
$(x : A) -> \mathtt{Dec} (P\ x)$. To do this, one pattern-matches on $x$ and
return either $\mathtt{yes}\ proof$ or $\mathtt{no}\ proof$, where $proof$ has
the type $P\ x$ and $P\ x -> \bot$ respectively.

Even though it is very nice to work with proving decidability, it is not in my
opinion very nice working with lemmas that prove decidability. In particular
there are two things one might want to do with such a lemma:

\begin{itemize}
\item Use it to prove other decidability lemmas.
\item Use it to get a proof of $P\ x$ for some specific value of $x$.
\end{itemize}

The first problem can be solved by using the \texttt{with}-syntax. One example
of doing this can be seen in $\texttt{Lemmas/TermDec.agda}$:

\begin{code}
  \>[0]\AgdaIndent{2}{}\<[2]%
\>[2]\AgdaFunction{instruction-dec} \AgdaSymbol{:} \AgdaSymbol{∀} \AgdaSymbol{\{}\AgdaBound{$\psi$₁} \AgdaBound{$\Delta$}\AgdaSymbol{\}} \AgdaBound{$\Gamma$} \AgdaBound{ι} \AgdaSymbol{→}\<%
\\
\>[2]\AgdaIndent{22}{}\<[22]%
\>[22]\AgdaDatatype{Dec} \AgdaSymbol{(}\AgdaFunction{∃} \AgdaSymbol{λ} \AgdaBound{$\Gamma$'} \AgdaSymbol{→} \AgdaBound{$\psi$₁} \AgdaInductiveConstructor{,} \AgdaBound{$\Delta$} \AgdaDatatype{⊢} \AgdaBound{ι} \AgdaDatatype{of} \AgdaBound{$\Gamma$} \AgdaDatatype{⇒} \AgdaBound{$\Gamma$'} \AgdaDatatype{instruction}\AgdaSymbol{)}\<%
\\
\>[0]\AgdaIndent{2}{}\<[2]%
\>[2]\AgdaFunction{instruction-dec} \AgdaSymbol{\{}\AgdaBound{$\psi$₁}\AgdaSymbol{\}} \AgdaSymbol{\{}\AgdaBound{$\Delta$}\AgdaSymbol{\}} \AgdaBound{$\Gamma$} \AgdaSymbol{(}\AgdaInductiveConstructor{add} \AgdaBound{♯rd} \AgdaBound{♯rs} \AgdaBound{v}\AgdaSymbol{)}\<%
\\
\>[2]\AgdaIndent{4}{}\<[4]%
\>[4]\AgdaKeyword{with} \AgdaFunction{vval-dec} \AgdaBound{v}\<%
\\
\>[0]\AgdaIndent{2}{}\<[2]%
\>[2]\AgdaSymbol{...} \AgdaSymbol{|} \AgdaInductiveConstructor{no} \AgdaBound{¬v⋆} \AgdaSymbol{=} \AgdaInductiveConstructor{no} \AgdaSymbol{(λ} \AgdaSymbol{\{} \AgdaSymbol{(}\AgdaSymbol{.\_} \AgdaInductiveConstructor{,} \AgdaInductiveConstructor{of-add} \AgdaBound{lookup≡int} \AgdaBound{v⋆}\AgdaSymbol{)} \AgdaSymbol{→} \AgdaBound{¬v⋆} \AgdaSymbol{(\_} \AgdaInductiveConstructor{,} \AgdaBound{v⋆}\AgdaSymbol{)} \AgdaSymbol{\})}\<%
\\
\>[0]\AgdaIndent{2}{}\<[2]%
\>[2]\AgdaSymbol{...} \AgdaSymbol{|} \AgdaInductiveConstructor{yes} \AgdaSymbol{(}\AgdaBound{$\tau$} \AgdaInductiveConstructor{,} \AgdaBound{v⋆}\AgdaSymbol{)}\<%
\\
\>[2]\AgdaIndent{4}{}\<[4]%
\>[4]\AgdaKeyword{with} \AgdaFunction{lookup-regs} \AgdaBound{♯rs} \AgdaBound{$\Gamma$} \AgdaFunction{≟} \AgdaInductiveConstructor{int} \AgdaSymbol{|} \AgdaBound{$\tau$} \AgdaFunction{≟} \AgdaInductiveConstructor{int}\<%
\\
\>[0]\AgdaIndent{2}{}\<[2]%
\>[2]\AgdaSymbol{...} \AgdaSymbol{|} \AgdaInductiveConstructor{no} \AgdaBound{lookup≢int} \AgdaSymbol{|} \AgdaSymbol{\_} \AgdaSymbol{=} \AgdaInductiveConstructor{no} \AgdaSymbol{(λ} \AgdaSymbol{\{} \AgdaSymbol{(}\AgdaSymbol{.\_} \AgdaInductiveConstructor{,} \AgdaInductiveConstructor{of-add} \AgdaBound{lookup≡int} \AgdaBound{v⋆'}\AgdaSymbol{)} \AgdaSymbol{→} \AgdaBound{lookup≢int} \AgdaBound{lookup≡int} \AgdaSymbol{\})}\<%
\\
\>[0]\AgdaIndent{2}{}\<[2]%
\>[2]\AgdaSymbol{...} \AgdaSymbol{|} \AgdaSymbol{\_} \AgdaSymbol{|} \AgdaInductiveConstructor{no} \AgdaBound{$\tau$≢int} \AgdaSymbol{=} \AgdaInductiveConstructor{no} \AgdaSymbol{(λ} \AgdaSymbol{\{} \AgdaSymbol{(}\AgdaSymbol{.\_} \AgdaInductiveConstructor{,} \AgdaInductiveConstructor{of-add} \AgdaBound{lookup≡int} \AgdaBound{v⋆'}\AgdaSymbol{)} \AgdaSymbol{→} \AgdaBound{$\tau$≢int} \AgdaSymbol{(}\AgdaFunction{vval-unique} \AgdaBound{v⋆} \AgdaBound{v⋆'}\AgdaSymbol{)} \AgdaSymbol{\})}\<%
\end{code}

While this method is function, it is somewhat impractical to work with. As an
alternative, I have created a function $\texttt{dec-inj}$, which can sometimes
make this easier to work with. Unfortunately it does not always work, but when
it does, it allows one simplify to something like this:

\begin{code}
  \>[0]\AgdaIndent{2}{}\<[2]%
\>[2]\AgdaFunction{instantiation-dec} \AgdaSymbol{:} \AgdaSymbol{∀} \AgdaSymbol{\{}\AgdaBound{$\Delta$}\AgdaSymbol{\}} \AgdaBound{θ} \AgdaBound{a} \AgdaSymbol{→}\<%
\\
\>[2]\AgdaIndent{24}{}\<[24]%
\>[24]\AgdaDatatype{Dec} \AgdaSymbol{(}\AgdaBound{$\Delta$} \AgdaDatatype{⊢} \AgdaBound{θ} \AgdaDatatype{of} \AgdaBound{a} \AgdaDatatype{instantiation}\AgdaSymbol{)}\<%
\\
\>[0]\AgdaIndent{2}{}\<[2]%
\>[2]\AgdaFunction{instantiation-dec} \AgdaSymbol{\{}\AgdaBound{$\Delta$}\AgdaSymbol{\}} \AgdaSymbol{(}\AgdaInductiveConstructor{$\alpha$} \AgdaBound{$\tau$}\AgdaSymbol{)} \AgdaInductiveConstructor{$\alpha$} \AgdaSymbol{=} \AgdaFunction{dec-inj} \AgdaInductiveConstructor{of-$\alpha$} \AgdaSymbol{(λ} \AgdaSymbol{\{} \AgdaSymbol{(}\AgdaInductiveConstructor{of-$\alpha$} \AgdaBound{$\tau$⋆}\AgdaSymbol{)} \AgdaSymbol{→} \AgdaBound{$\tau$⋆} \AgdaSymbol{\})} \AgdaSymbol{(}\AgdaBound{$\Delta$} \AgdaFunction{⊢?} \AgdaBound{$\tau$} \AgdaFunction{Valid}\AgdaSymbol{)}\<%
\\
\>[0]\AgdaIndent{2}{}\<[2]%
\>[2]\AgdaFunction{instantiation-dec} \AgdaSymbol{(}\AgdaInductiveConstructor{$\alpha$} \AgdaBound{$\tau$}\AgdaSymbol{)} \AgdaInductiveConstructor{$\rho$} \AgdaSymbol{=} \AgdaInductiveConstructor{no} \AgdaSymbol{(λ} \AgdaSymbol{())}\<%
\\
\>[0]\AgdaIndent{2}{}\<[2]%
\>[2]\AgdaFunction{instantiation-dec} \AgdaSymbol{(}\AgdaInductiveConstructor{$\rho$} \AgdaBound{$\sigma$}\AgdaSymbol{)} \AgdaInductiveConstructor{$\alpha$} \AgdaSymbol{=} \AgdaInductiveConstructor{no} \AgdaSymbol{(λ} \AgdaSymbol{())}\<%
\\
\>[0]\AgdaIndent{2}{}\<[2]%
\>[2]\AgdaFunction{instantiation-dec} \AgdaSymbol{\{}\AgdaBound{$\Delta$}\AgdaSymbol{\}} \AgdaSymbol{(}\AgdaInductiveConstructor{$\rho$} \AgdaBound{$\sigma$}\AgdaSymbol{)} \AgdaInductiveConstructor{$\rho$} \AgdaSymbol{=} \AgdaFunction{dec-inj} \AgdaInductiveConstructor{of-$\rho$} \AgdaSymbol{(λ} \AgdaSymbol{\{} \AgdaSymbol{(}\AgdaInductiveConstructor{of-$\rho$} \AgdaBound{$\sigma$⋆}\AgdaSymbol{)} \AgdaSymbol{→} \AgdaBound{$\sigma$⋆} \AgdaSymbol{\})} \AgdaSymbol{(}\AgdaBound{$\Delta$} \AgdaFunction{⊢?} \AgdaBound{$\sigma$} \AgdaFunction{Valid}\AgdaSymbol{)}\<%
\end{code}

The second example is slightly more tricky. Essentially we want to use our
decidability lemma as a compile-time guarantee that the predicate is
derivable. Additionally we may want to use the predicate in later
code. Conceptually we want a function $\mathtt{Dec}(P\ x) -> P\ x$ that can fail at compile-time. Having this function directly is clearly inconsistent (and it cannot fail at compile-time), however we can modify it slightly to get the behavior we want:

\begin{code}
\>\AgdaFunction{dec-force} \AgdaSymbol{:} \AgdaSymbol{∀} \AgdaSymbol{\{}\AgdaBound{a}\AgdaSymbol{\}} \AgdaSymbol{\{}\AgdaBound{A} \AgdaSymbol{:} \AgdaPrimitiveType{Set} \AgdaBound{a}\AgdaSymbol{\}} \AgdaSymbol{→}\<%
\\
\>[13]\AgdaIndent{14}{}\<[14]%
\>[14]\AgdaSymbol{(}\AgdaBound{p} \AgdaSymbol{:} \AgdaDatatype{Dec} \AgdaBound{A}\AgdaSymbol{)} \AgdaSymbol{→} \AgdaSymbol{\{}\AgdaBound{\_} \AgdaSymbol{:} \AgdaFunction{True} \AgdaBound{p}\AgdaSymbol{\}} \AgdaSymbol{→}\<%
\\
\>[13]\AgdaIndent{14}{}\<[14]%
\>[14]\AgdaBound{A}\<%
\end{code}

This function will not type-check unless it can derive an instance of
$\AgdaFunction{True} \AgdaBound{p}$, which is exactly the behavior we
wanted. This allows us to write e.g. the following function, which would fail if
our program was \emph{not} valid:

\begin{code}
\>\AgdaFunction{myprogram-valid} \AgdaSymbol{:} \AgdaDatatype{⊢} \AgdaFunction{myprogram} \AgdaDatatype{programstate}\<%
\\
\>\AgdaFunction{myprogram-valid} \AgdaSymbol{=} \AgdaFunction{dec-force} \AgdaSymbol{(}\AgdaFunction{programstate-dec} \AgdaSymbol{\_)}\<%
\end{code}

\section{Substitutions}
I think I should be able to same something semi-smart about this.

\todo{Write about substitutions}
