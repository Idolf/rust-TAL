\documentclass[a4paper]{report}
\usepackage[utf8x]{inputenc}
\usepackage{ucs}
\usepackage[greek,english]{babel}
\usepackage{amsmath, amssymb, amsthm}
\usepackage{stmaryrd}
\usepackage{mathpartir}
\usepackage{semantic}
\usepackage{titling}
\usepackage[pdftex,dvipsnames]{xcolor}
\usepackage[lang=en,grid]{kufront}
\usepackage{url}
\usepackage{xspace}
\usepackage{scrextend}
\usepackage{chngcntr}
\usepackage{apptools}
\usepackage{xargs}
\AtAppendix{\counterwithin{theorem}{chapter}}

% Agda stuff
\usepackage{latex/agda}
\usepackage{verbatim}
\DeclareUnicodeCharacter{8799}{\ensuremath{\stackrel{?}{=}}}
\DeclareUnicodeCharacter{8347}{\ensuremath{{}_{\mathrm{s}}}}
\DeclareUnicodeCharacter{8344}{\ensuremath{{}_{\mathrm{m}}}}
\DeclareUnicodeCharacter{8345}{\ensuremath{{}_{\mathrm{n}}}}
\DeclareUnicodeCharacter{8336}{\ensuremath{{}_{\mathrm{a}}}}
\DeclareUnicodeCharacter{8341}{\ensuremath{{}_{\mathrm{h}}}}
\DeclareUnicodeCharacter{8759}{\ensuremath{::}}

\usepackage[colorinlistoftodos,prependcaption,textsize=tiny]{todonotes}
\newcommandx{\hightodo}[2][1=]{\todo[linecolor=red,backgroundcolor=red!25,bordercolor=red,#1]{#2}}

\author{\Large Mathias Svensson \quad {\ttfamily\large tpx783@alumni.ku.dk}}
\date{\today}
\title{A formalization of the STAL meta-theory in Agda}
\project{\mdseries\LARGE Master's Thesis}
\supervisor{Supervisor: {\large Ken Friis Larsen} \texttt{kflarsen@di.ku.dk} \\[0.5cm]
            Co-supervisor: {\large Andrzej Filinski} \texttt{andrzej@di.ku.dk}}

\usepackage[pdftitle={\thetitle}, pdfauthor={Mathias Svensson}, linkbordercolor={0.8 0.8 0.8}]{hyperref}
\usepackage{cleveref}

\newtheorem{theorem}{Theorem}[section]
\newtheorem{definition}[theorem]{Definition}
\newtheorem{lemma}[theorem]{Lemma}
\newtheorem{corollary}[theorem]{Corollary}

\mathlig{|-}{\vdash}
\mathlig{|->}{\mapsto}
\mathlig{!}{\downarrow}
\mathlig{>>}{\gg}

\newcommand \embed {{\ensuremath{\mathbf{embed}}}}
\newcommand \embedp {\ensuremath{\mathbf{embed}^\ast}}

\newcommand \native[1] {{\ensuremath{#1_N}}\xspace}
\newcommand \simple[1] {{\ensuremath{#1_S}}\xspace}
\newcommand \high[1] {{\ensuremath{#1_H}}\xspace}
\newcommand \typed[1] {{\ensuremath{#1_T}}\xspace}

\newcommand \nativeo[1] {{{}\ensuremath{\stackrel{N}{#1}}{}}}
\newcommand \simpleo[1] {{{}\ensuremath{\stackrel{S}{#1}}{}}}
\newcommand \higho[1] {{{}\ensuremath{\stackrel{H}{#1}}{}}}

\newcommand \lang {\ensuremath{\mathcal{L}}}
\newcommand \Stuck {\ensuremath{\mathbf{Stuck}}}

\newcommand \nativelang {\ensuremath{\lang_N}\xspace}
\newcommand \simplelang {\ensuremath{\lang_S}\xspace}
\newcommand \highlang {\ensuremath{\lang_H}\xspace}
\newcommand \typedlang {\ensuremath{\lang_T}\xspace}

\newcommand \highembed {\ensuremath{\embed_{\highlang}}\xspace}
\newcommand \simpleembed {\ensuremath{\embed_{\simplelang}}\xspace}
\newcommand \simpleembedp {\ensuremath{\embedp_{\simplelang}}\xspace}


\newcommand \greek[1] {{\text{\selectlanguage{greek}#1\selectlanguage{english}}}}
\newcommand \ttalpha{\greek{\ttfamily\textalpha}}
\newcommand \ttrho{\greek{\ttfamily\textrho}}
\newcommand \malicious {{\ensuremath{\mathbf{malicious}}}}
\newcommand \halted {\ensuremath{\mathbf{halted}}}
\newcommand \running {\ensuremath{\mathbf{running}}}
\newcommand \halt {\ensuremath{\mathbf{halt}}}

\newcommand \valid[2] {{#1} |- {#2}\ \mathbf{Valid}}
\newcommand \subtype[3] {{#1} |- {#2} \le {#3}}
\newcommand \substitution[3] {{#1} \llbracket {#2} \rrbracket \equiv {#3}}
\newcommand \run[3] {\mathtt{Run}\ {#1} \llbracket {#2} \rrbracket \equiv {#3}}

\newcommand \ofglobs[2] {|- {#1} : {#2}}
\newcommand \ofheap[3] {{#1} |- {#2} : {#3}}
\newcommand \ofgval[3] {{#1} |- {#2} : {#3}}
\newcommand \ofhval[3] {{#1} |- {#2} : {#3}}
\newcommand \ofwval[3] {{#1} |- {#2} : {#3}}
\newcommand \ofwvaln[3] {{#1} |- {#2} : {#3}}
\newcommand \ofvval[3] {{#1} |- {#2} : {#3}}
\newcommand \ofstack[3] {{#1} |- {#2} : {#3}}
\newcommand \ofword[3] {{#1} |- {#2} : {#3}}
\newcommand \ofregister[3] {{#1} |- {#2} : {#3}}
\newcommand \ofinstructions[2] {{#1} |- {#2}\ \mathbf{Valid}}
\newcommand \ofinstruction[3] {{#1} |- {#2} : {#3}}
\newcommand \ofprogram[2] {{#1} |- {#2}}
\newcommand \oflang[1] {|- {#1}}
\newcommand \ofinstantiation[3] {{#1} |- {#2} : {#3}}
\newcommand \ofinstantiations[3] {{#1} |- {#2} : {#3}}

% Simple judgments
\newcommand \evalregs[3] {{#1} |- {#2} ! {#3}}
\newcommand \execis[3] {{#1} |- {#2} -> {#3}}
\newcommand \steps[2] {{#1} -> {#2}}

% Annotated judgments
\newcommand \evalrega[3] {{#1} |- {#2} ! {#3}}
\newcommand \evalcode[3] {{#1} |- {#2} ! {#3}}
\newcommand \execia[3] {{#1} |- {#2} => {#3}}
\newcommand \stepa[2] {{#1} -> {#2}}
\newcommand \donea[1] {{#1}\ \mathbf{done}}


\begin{document}

\begin{titlepage}
\maketitle
\end{titlepage}
\tableofcontents

\begin{abstract}
  Typed Assembly Language (TAL) is an interesting technique, as it allows
  provably safe distribution and execution of untrusted code, without assuming
  the presence of a trusted third-party. In this thesis we give a thorough
  introduction to TAL including how it is built ``on top'' of a real-world
  language. We then formalize a version of TAL that is mostly compatible with
  the one presented in the paper ``Stack-Based Typed Assembly Language''.

  Our contributions in this thesis is two-fold:

  \begin{itemize}
  \item We formalize the meta-theory of our language. Specifically we use Agda to
    prove that the type system is soundness and decidable along with proving an
    erasure theorem about the language.
  \item While creating the Agda-code several challenges were encountered and
    overcome. This gave rise to a number of smaller, but nonetheless interesting
    techniques that could likely be relevant outside Typed Assembly
    Language. These ``lessons learned'' have been included in a chapter meant to
    be readable independently of the rest of the thesis.
  \end{itemize}
\end{abstract}


% \chapter{Introduction}
% \section{Motivation}

Let us consider a software developer, with the job of developing a piece of code
for a client. She is considering in what format to distribute her final product.
As the choices seem overwhelming, she decides to list a few properties she would
like her final product to have, before she begins developing it.

\paragraph{Low overhead:}
Her product should run as fast as possible while using as few resources as
possible.

\paragraph{Low latency:}
Her product should be executable as soon as possible after her client receives
it. So while she considers herself an Open Source advocate, she concludes that
distributing the source code is not a solution to the \emph{current} problem.

\paragraph{Reusability:}
Whatever method she uses to create her product should be reusable for other
projects. She concludes that her method should support general-purpose code.

\paragraph{Security:}
Ideally she would like to convince her client beyond any doubt, that her product
serves the intended function. However her client have had problems even settling
on a consistent design specification, much less a formalized one.

She decides to settle for a slightly less ambiguous goal instead: Absolute
certainty that it is not \emph{malicious}.

\paragraph{}
After combining the first three requirements, she is left with very few options:

\begin{enumerate}
\item She could distribute raw machine code (inside a suitable container format
  such as \texttt{PE} or \texttt{ELF}).
\item She could distribute a bytecode format intended to be run through a fast
  interpreter or JIT-compiler.
\end{enumerate}

She are now left with the problem of convincing her client that her code is
indeed not malicious. She decides to do a small real-world survey to see how
other people have tried to solve this problem:

\begin{itemize}
\item Do nothing! Given that most people are very trusting, many will simply run
  whatever code they receive.
\item Expect the client to run anti-virus software to detect malicious code.
\item Buy a code-signing certificate and sign the code. The security of this
  system is based on the fact that a code-signing certificates are
  semi-expensive and only issued to entities verified to be non-malicious by a
  trusted Certificate Authority (CA).
\item Have the client run the code inside a sandboxed environment, by using
  e.g. the Java VM, Docker or Native Client from the Chromium project.
\item Publish the code as open source along with a method for compiling that
  source into a bitwise identical result.
\item Some combination of the above.
\end{itemize}

One notices that none of the real-world solutions actually \emph{solves} the
security requirement as stated!

\begin{itemize}
\item Rice's Theorem states that the anti-virus model is inherently
  imperfect. In fact, it is a quite common occurrence for real-world anti-virus
  software to produce both false positives and false negatives.
\item There are multiple examples of code-signing certificate being stolen or
  created under fake pseudonyms. There have also been examples of CAs being
  hacked or coerced to create certificates.
\item None of the sandboxes we might consider have perfect security without
  placing unreasonable demands on the other properties. In fact many of them
  have known public exploits for previous versions.
\item Even open source software with security audits sometimes have bugs -- and
  even backdoors are not completely unheard of.
\end{itemize}

Desperately she turns to research: Is there really no solution in sight? It
turns out there is: Proof-carrying code.

\later[inline]{A lot of citations.}

% \clearpage
\section{Proof Carrying Code}

\later[inline]{Definition of proof-carrying code}

\later[inline]{Informal definition of typed assembly language}

% \clearpage
\section{Problem Statement}

\later[inline]{Problem statement}



\chapter{Typed Assembly Language}
\section{Overview}

The purpose of this chapter is to introduce our typed assembly language along
with our proof of soundness. To give some context, we will first give a quick
run-through of our strategy.

\paragraph{Step 1:} Define three languages:

\begin{itemize}
\item \nativelang is a real-world language, e.g. MIPS as defined by
  \cite{mipssys}.
\item \simplelang is a simplification of \nativelang. The purpose of
  this simplification is to restrict the evaluation model. Our restricted model
  will only to only permit benevolent programs.
\item \highlang is an extension of \simplelang, which will
  introduce types.
\end{itemize}

To avoid dealing with halting semantics, we introduce the following states:
\begin{itemize}
\item $\malicious, \native\halt \in \nativelang$
\item $\simple\halt \in \simplelang$
\item $\high\halt \in \highlang$
\end{itemize}

$\halt$ represents every program that has halted in a non-malicious way (such as
intentional exiting or properly handled out-of-memory exceptions). $\malicious$
represents every any malicious state.

\paragraph{Step 2:} Define two translation functions and a relation:

\begin{itemize}
\item $\high\embed : \highlang -> \simplelang$. Intuition: Simply drop the type
  annotations.
\item $\simple\embed : \simplelang -> \nativelang$. Intuition: An assembler
  along with a code generator to construct the heap.
\item $\simple\embedp \subset \simplelang \times \nativelang$. Intuition:
  Multiple equivalent run-time equivalents of a single problem might exist. We
  use this relate multiple native programs to the same simplified program.
\end{itemize}

We observe that:

\begin{itemize}
\item $\high\embed(\high\halt) = \simple\halt$
\item $\simple\embedp(\simple\halt) = \{\native\halt\}$
\item $\malicious \notin \simple\embedp(p)$ for all p
\item $\simple\embed(p) \in \simple\embedp(p)$ for all p
\end{itemize}

\paragraph{Step 3:} For each of these languages, define a small-step relation:

\begin{itemize}
\item $\nativeo{->} \subset \nativelang \times \nativelang$.
\item $\simpleo{->} \subset \simplelang \times \simplelang$.
\item $\higho{->} \subset \highlang \times \highlang$.
\end{itemize}

Use $p_1 \nativeo{->} p_2 \nativeo{->} \dots \nativeo{->} p_{n-1} \nativeo{->} p_n$
as shorthand for $(p_1, p_2), \dots, (p_{n-1}, p_n) \in \nativeo{->}$. Similarly
for \simpleo{->} and \higho{->}.

We observe that:

\begin{itemize}
\item If $p \in \nativelang$ is halting for non-malicious reasons, then
  $p \nativeo{->} \native\halt$.
\item $\native\halt \nativeo{->} p \iff p = \native\halt$.
\item (Similarly for \simplelang and \highlang).
\item If $p \in \nativelang$ can transition in a malicious or undefined way,
  then $p \nativeo{->} \malicious$.
\item $\malicious \nativeo{->} p \iff p = \malicious$
\end{itemize}

\paragraph{Note:} This implies \nativelang will always progress (i.e. if
$p \in \nativelang$, then $p \nativeo{->} p'$ for some $p'$). This is \emph{not}
true for \simplelang or \highlang.

\paragraph{Step 4:} Define a typing judgment for \highlang and a sub-language:

\begin{itemize}
\item If $p \in \highlang$, then $\valid{}{p}$ is a judgment (for which there
  might or might not exist a derivation).
\item
  $\typedlang = \{p \in \highlang \mid \text{there exists a derivation of\ }
    \valid{}{p}\} \subset \highlang$.
\end{itemize}

\paragraph{Step 5:} Prove these theorems about \highlang:

\begin{itemize}
\item \textbf{Progress}: Assume $p \in \typedlang$. In that case there exists
  some $p' \in \highlang$ such that $p \higho{->} p'$.
\item \textbf{Preservation}: Assume $p \higho{->} p'$ and $p \in \typedlang$. In
  that case $p' \in \typedlang$.
\item \textbf{Soundness for \highlang}: If $p_1 \in \typedlang$ and
  $p_1 \higho{->} \dots \higho{->} p_n$, then there exists some $p_{n+1}$ such that
  $p_n \higho{->} p_{n+1}$. This follows directly from the two previous
  theorems.
\end{itemize}

\paragraph{Step 6:} Prove the following theorems about the relationship between
\highlang and \simplelang:

\begin{itemize}
\item \textbf{Bisimulation}: Define the following predicate on
  programs in \highlang:

  $$\Stuck(p) = (\exists p': p \simpleo{->} p' \land \Stuck(p')) \lor (\not\exists p': p \higho{->} p')$$

  Define the relation $R$ as:

  $$R = \{(p, \high\embed(p)) \mid \neg\Stuck(p)\} \subset \highlang \times
  \simplelang$$

  We see that $R$ is bisimulation between \higho{->} and \simpleo{->}. More
  specifically:\footnote{A more general definition of bisimulation exists,
    however introducing the generalization serves no real purpose in this
    instance.}
  \begin{itemize}
  \item Assume $(p_1, p_1') \in R$ and $p_1 \higho{->} p_2$. Then there exists
    some $p_2'$ such that $(p_2, p_2') \in R$ and $p_1' \simpleo{->} p_2'$.
  \item Assume $(p_1, p_1') \in R$ and $p_1' \simpleo{->} p_2'$. Then there
    exists some $p_2$ such that $(p_2, p_2') \in R$ and
    $p_1 \higho{->} p_2$.
  \end{itemize}
\item \textbf{Soundness for \simplelang}: Assume $p_1 \in \typedlang$ and
  $\high\embed(p_1) \simpleo{->} \dots \simpleo{->} p_2$. In that case there
  exists some $p_3$ such that $p_2 \simpleo{->} p_3$. This follows
  directly from the previous two theorems.
\end{itemize}

\paragraph{Step 7:} Prove the following theorem about the relationship between
\simplelang and \nativelang:

\begin{itemize}
\item \textbf{Stuttering bisimulation}: Define the following predicate on
  programs in \simplelang:

  $$\Stuck(p) = (\exists p': p \simpleo{->} p' \land \Stuck(p')) \lor (\not\exists p': p \simpleo{->} p')$$

  Note specifically that $\neg\Stuck(\simple\halt)$. Now define a relation $R$
  by:

  $$R = \{(p_1, p_2) \mid \neg\Stuck(p_1), p_2 \in \simple\embedp(p_1) \cup \{\native\halt\}\}$$

  This relation is a stuttering bisimulation of $\simpleo{->}$ and
  $\nativeo{->}$. More specifically there is a function
  $f : \simplelang \times \nativelang \to \mathbb{N}$ such that:\footnote{This is
    again a somewhat specialized definition of a much more general concept.}

  \begin{itemize}
  \item Assume $(p, p_1) \in R$ and $p \simpleo{->} p'$. Then there exists
    $1 \leq n \leq f(p, p_1)$ and a sequence
    $p_1 \nativeo{->} p_2 \nativeo{->} \dots \nativeo{->} p_n$ such that
    $(p', p_n) \in R$.
  \item Assume $(p, p_1) \in R$, assume $n \geq f(p, p_1)$ and that there is
    some sequence $p_1 \nativeo{->} \dots \nativeo{->} p_n$. Then there is some
    $1 \leq k \leq f(p, p_1)$ and a $p'$ such that $(p', p_k) \in R$.
  \end{itemize}

  (Note that since no programs are stuck in \nativelang, we can show a slightly
  different statement regarding sequences of any length.)
\end{itemize}

\paragraph{Step 8:} We finally prove the desired theorems:
\begin{itemize}
\item \textbf{Safety}: Assume $p \in \typedlang$,
  $p' \in \simple\embedp(\high\embed(p))$. There exists no path such that
  $p \nativeo{->} \dots \native{\malicious}$. In particular it is safe to
  execute $\simple\embed(\high\embed(p))$.

  This theorem follows directly from the two previous theorems.

\item \textbf{Correctness}: Assume $p_1 \in \typedlang$,
  $p_1' \in \simple\embedp(\high\embed(p_1))$ and assume
  $p_1 \higho{->} \dots \higho{->} p_n$. There now exists a finite sequence
  $p_1' \nativeo{->} \dots \nativeo{->} p_m'$ such that
  $p_m' \in \simple\embedp(\high\embed(p_n)) \cup \{\native\halt\}$. In
  particular this holds if $p_1' = \simple\embed(\high\embed(p_1)$.

\item \textbf{Decidability}: If $p \in \highlang$, then it is decidable to check
  if $p$ is a member of $\typedlang$.
\end{itemize}

\paragraph{Step 9: Performance} In practice one would also like to show that
these statements hold in the practical implementation:

\begin{itemize}
\item The implementation of the decision procedure from the previous theorem
  should be reasonably efficient and not have any pathological cases that would
  permit denial-of-service attacks.
\item It is possible to compile from a high-level language into
  $\typedlang$. This compiler should not be significantly worse in the relevant
  metrics than compilers that translate directly into assembler language.
\item The function $\simple\embedp(\high\embed(\cdot))$ does not degrade
  performance seriously.
\item None of our theorems restrict in what situations the programs in
  \nativelang could halt. In principle we could have $\simple\embed(p) = \nhalt$
  for all $p$. Ideally one would prove some guarantees about the conditions in
  which the translated programs could halt.
\end{itemize}

\paragraph{Scoping limitations} To limit the scope of this thesis, none of the
parts related to \nativelang will be attempted formalized. At best, we will
discuss how future work might address these issues.

Besides this, the parts related to \simplelang have not been formalized in Agda,
and a few steps of the proofs related to \highlang are not yet formalized. We
will be sure to mention which parts as they come up.

\section{Datatypes}

\subsubsection{Dictionaries}
A \textbf{dictionary} is an unordered set of the form
$\{x_1 |-> y_1, \dots, x_n |-> y_n\}$ such that $x_i \neq x_j$ whenever
$i \neq j$. By unordered we mean that e.g.
$\{x_1 |-> y_1, x_2 |-> y_2\} = \{x_2 |-> y_2, x_1 |-> y_1\}$.

Assume $D$ is some dictionary $\{x_1 |-> y_1, \dots, x_n |-> y_n\}$. In that
case $\mathbf{Keys}(D) = \{x_1, \dots, x_n\}$. We define as $D[x_k] = y_k$ and
$D[x_k |-> y_{new}] = \{x_1 |-> y_1, \dots x_k |-> y_{new}, \dots, x_n |->
y_n\}$.

Now let $D'$ be another dictionary $\{x_1' |-> y_1', \dots, x_m' |-> y_m'\}$
such that $\mathbf{Keys}(D_1) \cap \mathbf{Keys}(D_2) = \emptyset$. In this case
$D_1 \cup D_2$ is the dictionary
$\{x_1 |-> y_1, \dots, x_n |-> y_n, x_1' |-> y_1', \dots, x_m' |-> y_m'\}$.

\paragraph{}
Note: In the Agda-implementation, these maps are mostly implemented as ordered
lists with implicit keys, however we can ignore this fact when using
paper-notation.

\subsubsection{List-like objects}
A set $S$ is \textbf{list-like} iff there is a meaningful way to interpret the
elements of $S$ as ordered lists of elements drawn from a base-set $S_B$. In
other words iff there exist an injective function
$f : S \to \mathbf{List}\ S_B$, where $f$ may only depend on the superficial
syntax of the elements.

For instance heap values and stacks are list-like, as they are written
$\langle w_1, \dots, w_n \rangle$ and $w_1 :: \dots :: w_n :: \mathtt{nil}$
respectively.

Let $S$ be a list-like set and assume that $L$ is a typical member of this set
such that $f(L) = [x_1, \dots, x_n]$. In that case $\mathbf{Length}(L) = n$,
i.e. the number of elements in $L$.

We further define $L[k] = x_k$ and $L[k |-> x_{new}]$ to be the list-like member
of $S$ such that
$f(L[k |-> x_{new}]) = [x_1, \dots, x_{k-1}, x_{new}, \dots, x_n]$.

\section{Step 1: Defining the languages}

\nativelang is MIPS. \simplelang and \highlang are defined inductively by
these grammars. In the definitions, we use $wordsize$ and $max$ as CPU-specific
constants. They are concrete, low constants such as 32, however their exact
values are irrelevant for the later discussions.

{\footnotesize
\begin{tabular}{lrcl}
Variables: \\
\textit{global pointers} & $\ell_g$ \\
\textit{heap pointers}   & $\ell_h$ \\
\textit{type variable}   & $\alpha$ \\
\textit{stack variable}  & $\rho$ \\\\

Common definitions: \\
\textit{integer}            & $n,k$ & ::= & $0, 1, \dots$ \\
\textit{machine integers}   & $i$   & ::= & $0, 1, \dots, 2^{wordsize}-1$ \\
\textit{registers}          & $r$   & ::= & $\mathtt{r}_1 \mid \dots \mid \mathtt{r}_{max}$ \\\\

The grammar for \simplelang: \\
\textit{word values}        & $\simple w$ & ::= & $\mathtt{globval}\ \ell_g \mid \mathtt{heapval}\ \ell_h \mid \mathtt{int}\ i \mid \mathtt{ns} \mid \mathtt{uninit}$ \\
\textit{small values}       & $\simple v$ & ::= & $\mathtt{globval}\ \ell_g \mid \mathtt{reg}\ r \mid \mathtt{int}\ i$ \\
\textit{global values}      & $\simple g$ & ::= & $\mathtt{code}\ \simple I$ \\
\textit{global collections} & $\simple G$ & ::= & $\{\ell_g |-> \simple{g}, \dots, \ell_g |-> \simple{g}\}$ \\
\textit{heap values}        & $\simple h$ & ::= & $\langle \simple w, \dots, \simple w \rangle$ \\
\textit{heap collections}   & $\simple H$ & ::= & $\{\ell_h |-> \simple{h}, \dots, \ell_h |-> \simple{h}\}$ \\
\textit{stacks}             & $\simple S$ & ::= & $\mathtt{nil} \mid \simple w :: \simple S$ \\
\textit{register files}     & $\simple R$ & ::= & $\{\mathtt{sp} |-> \simple{S}, \mathtt{r}_1 |-> \simple{w}, \dots, \mathtt{r}_{max} |-> \simple{w}\}$ \\\\

\textit{instructions} & $\simple \iota$ & ::= & $\mathtt{add}\ r, r, \simple v \mid \mathtt{sub}\ r, r, \simple v \mid$ \\
        &&& $\mathtt{salloc}\ n \mid \mathtt{sfree}\ n \mid$ \\
        &&& $\mathtt{ld}\ r, \mathtt{sp}(n) \mid \mathtt{ld}\ r, r(n) \mid$ \\
        &&& $\mathtt{st}\ \mathtt{sp}(n), r \mid \mathtt{st}\ r(n), r \mid$ \\
        &&& $\mathtt{malloc}\ r,\ n \mid $ \\
        &&& $\mathtt{mov}\ r, \simple v \mid \mathtt{beq}\ r, \simple v$ \\
\textit{instruction sequences} & $\simple I$ & ::= & $\simple\iota ; \simple I \mid \mathtt{jmp}\ \simple v \mid \mathtt{halt}$ \\
\textit{program states} & $\simple P$ & ::= & $(\simple H, \simple R, \simple I)$ \\
\textit{programs} & $\simplelang$ & ::= & $\simple\running (\simple G, \simple P) \mid \simple\halted$ \\\\

The type system for \highlang: \\
\textit{types}                    & $\tau$ & ::= & $\alpha \mid \mathtt{int} \mid \mathtt{ns} \mid \mathtt\forall[ \Delta ] \Gamma \mid \langle\tau^\phi,\dots,\tau^\phi\rangle$ \\
\textit{initialization flags}     & $\phi$ & ::= & $\mathtt{init} \mid \mathtt{uninit}$ \\
\textit{stack types}              & $\sigma$ & ::= & $\rho \mid \mathtt{nil} \mid \tau :: \sigma$ \\
\textit{type assignments}         & $\Delta$ & ::= & $\mathtt{nil} \mid a :: \Delta$ \\
\textit{type assignment value}    & $a$ & ::= & $\alpha \mid \rho$ \\
\textit{global label assignments} & $\Psi_g$ & ::= & $\{\ell_g |-> \tau, \dots, \ell_g |-> \tau\}$ \\
\textit{heap label assignmentss}  & $\Psi_h$ & ::= & $\{\ell_h |-> \tau, \dots, \ell_h |-> \tau\}$ \\
\textit{register assignments}     & $\Gamma$ & ::= & $\{\mathtt{sp} |-> \sigma, \mathtt{r}_1 |-> \tau, \dots, \mathtt{r}_{max} |-> \tau\}$ \\
\textit{instantiation}            & $\theta$ & ::= & $\tau/\alpha \mid \sigma/\rho$ \\\\

The grammar for \highlang: \\
\textit{word values}              & $\high w$ & ::= & $\mathtt{globval}\ \ell_g \mid \mathtt{heapval}\ \ell_h \mid \mathtt{int}\ i \mid \mathtt{ns} \mid \mathtt{uninit}\ \tau \mid \Lambda\ \Delta \cdot \high{w}[\theta, \dots, \theta]$ \\
\textit{small values}             & $\high v$ & ::= & $\mathtt{globval}\ \ell_g \mid \mathtt{reg}\ r \mid \mathtt{int}\ i \mid \Lambda\ \Delta \cdot \high{v}[\theta, \dots, \theta]$ \\
\textit{global values}            & $\high g$ & ::= & $\mathtt{code}[ \Delta ] \Gamma \cdot \high I$ \\
\textit{global collections}       & $\high G$ & ::= & $\{\ell_g |-> \high{g}, \dots, \ell_g |-> \high{g}\}$ \\
\textit{heap values}              & $\high h$ & ::= & $\langle \high w, \dots, \high w \rangle$ \\
\textit{heap collections}         & $\high H$ & ::= & $\{\ell_h |-> \high{h}, \dots, \ell_h |-> \high{h}\}$ \\
\textit{stacks}                   & $\high S$ & ::= & $\mathtt{nil} \mid \high w :: \high S$ \\
\textit{register files}           & $\high R$ & ::= & $\{\mathtt{sp} |-> \high{S}, \mathtt{r}_1 |-> \high{w}, \dots, \mathtt{r}_{max} |-> \high{w}\}$ \\\\

\textit{instructions} & $\high\iota$ & ::= & $\mathtt{add}\ r, r, \high v \mid \mathtt{sub}\ r, r, \high v \mid$ \\
        &&& $\mathtt{salloc}\ n \mid \mathtt{sfree}\ n \mid$ \\
        &&& $\mathtt{ld}\ r, \mathtt{sp}(n) \mid \mathtt{ld}\ r, r(n) \mid$ \\
        &&& $\mathtt{st}\ \mathtt{sp}(n), r \mid \mathtt{st}\ r(n), r \mid$ \\
        &&& $\mathtt{malloc}\ r,\ \langle \tau, \dots, \tau \rangle \mid $ \\
        &&& $\mathtt{mov}\ r, \high v \mid \mathtt{beq}\ r, \high v$ \\
\textit{instruction sequences} & $\high I$ & ::= & $\high\iota ; \high I \mid \mathtt{jmp}\ \high v \mid \mathtt{halt}$ \\
\textit{program states} & $\high P$ & ::= & $(\high H, \high R, \high I)$ \\
\textit{programs} & $\highlang$ & ::= & $\high\running (\high G, \high P) \mid \high\halted$ \\\\
\end{tabular}
}


We will occasionally refer to a specific heap collection as a \textbf{heap} or
perhaps \textbf{the heap} for a specific program state. Similarly a specific
global collection might be referred to as \textbf{the globals} of a program. We
will however not use the term ``global'' without qualification, to avoid
ambiguity between global collections and global values.

The fundamental assumption of this language is that the state of a program is
totally defined by a collection of immutable global values, the current state of
the heap, the current stack, the current registers and the current instruction
pointer. This is represented in our language as:

\begin{itemize}
\item The immutable global values are places in a global collection
  consisting of individual global items.
\item The mutable global values are places in a heap collection
  consisting of individual heap items.
\item The current stack is encoded as a stack consisting and placed inside a
  register file.
\item The value of the current registers are encoded in a register file together
  with the stack.
\item The instruction pointer is not encoded directly as a pointer, it is
  instead represented as the instructions to be run until the next block.
\end{itemize}

While this representation is enough for many programs, it also excludes many
programs by design. For instance it excludes mutable global variables not stored
on the heap and self-modifying code (such as just-in-time compilers).

It should be possible to expand the language to allow more programs, however
some sacrifice will always be needed, as we are not able to exclude malicious
programs from our semantics without also excluding some valid programs.

\subsection{Notes on the Agda implementation}

The Agda implementation is mostly a faithful representation of the grammar
presented here, though the notation used for it is slightly different in places.

The biggest difference is how we represent the variables (i.e.
$\ell_g, \ell_h, \alpha, \rho$) and the stores that use these variables
(i.e. $\Psi_g, \Psi_h, \Delta$). In this report we will mostly just assume that
we have an endless supply of unique variables (e.g.
$\alpha_1, \rho_1, \ell_{g,1}, \ell_{h,1}, \alpha_2, \dots$), however this is
not a practical solution in Agda.

\todo{Bad font for $\mathbb{\alpha}$}

For the $\alpha$ and $\rho$ we will use De Bruijn indices. The $\Delta$ values
will simply be lists composed over two elements $\mathbb{\alpha}$ and
$\mathtt{\rho}$. For an $\alpha$-variable to be valid in a given
$\Delta$-context we must have that that $\Delta[\alpha] = \mathtt{\alpha}$ (and
similarly for $\rho$). This means that while on paper we might simply allow
assume weakening or reordering of contexts as fundamental rules, this approach
will not work in Agda. In this particular case we will instead have to prove a
explicit weakening lemma, which depends on a renaming function.

For the heap and globals we will simply compose them into lists and have the
variables be indices into this list. This will not cause any significant
complications, as our semantics will never remove anything from either of the
lists.

\section{Step 2: Translations}

The a (very informal) definition of \simpleembed is as follows:

\begin{itemize}
\item $\simple{H}$: Create assembler code to malloc and initialize the heap
  values.
\item $\simple{R}$: Append assembler code to setup the registers.
\item $\simple{I}$: Convert $\simple{I}$ to assembler and append it.
\item $\simple{G}$: Convert each each global into assembler-code and append them.
\item Finally run an assembler over the generated text to create machine code.
\end{itemize}

Defining \simpleembedp is somewhat more complicated, and is outside the scope
of this thesis. Finally \highembed is this function:

{\footnotesize
\[\begin{array}{l}
\begin{array}{ll}
\rlap{$\embed_w : \high w -> \simple{w}$} \\
\embed_w(\mathtt{globval}\ \ell_g) &= \mathtt{globval}\ \ell_g \\
\embed_w(\mathtt{heapval}\ \ell_h) &= \mathtt{heapval}\ \ell_h \\
\embed_w(\mathtt{int}\ i) &= \mathtt{int}\ i \\
\embed_w(\mathtt{uninit}) &= \mathtt{uninit} \\
\embed_w(\Lambda\ \Delta \cdot w[\theta_1, \dots, \theta_n]) &= \embed_w(w) \\\\

\rlap{$\embed_v : \high v -> \simple{v}$} \\
\embed_v(\mathtt{globval}\ \ell_g) &= \mathtt{globval}\ \ell_g \\
\embed_v(\mathtt{reg}\ r) &= \mathtt{reg}\ r \\
\embed_v(\mathtt{int}\ i) &= \mathtt{int}\ i \\
\embed_v(\Lambda\ \Delta \cdot v[\theta_1, \dots, \theta_n]) &= \embed_v(v)
\end{array} \\\\

\begin{array}{l}
\embed_g : \high g -> \simple{g} \\
\embed_g(\mathtt{code}[\Delta]\Gamma \cdot I) = \mathtt{code}\ (\embed_I(I)) \\\\

\embed_G : \high G -> \simple{G} \\
\embed_G(\{\ell_{g,1} |-> g_1, \dots, \ell_{g,n} |-> g_n\}) = \{\ell_{g,1} |-> \embed_g(g_1), \dots, \ell_{g,n} |-> \embed_g(g_n)\} \\\\

\embed_h : \high h -> \simple{h} \\
\embed_h(\langle w_1, \dots, w_n \rangle) = \langle \embed_w(w_1), \dots \embed_w(w_n)\rangle \\\\

\embed_H : \high H -> \simple{H} \\
\embed_H(\{\ell_{h,1} |-> h_1, \dots, \ell_{h,n} |-> h_n\}) = \{\ell_{h,1} |-> \embed_h(h_1), \dots, \ell_{h,n} |-> \embed_h(h_n)\}
\end{array} \\\\

\begin{array}{ll}
\rlap{$\embed_S : \high S -> \simple{S}$} \\
\embed_S(\mathtt{nil}) &= \mathtt{nil} \\
\embed_S(w :: S) &= \embed_w(w) :: \embed_S(S) \\
\end{array} \\\\

\begin{array}{l}
\embed_R : \high R -> \simple{R} \\
\embed_R(\{\mathtt{sp} |-> S, \mathtt{r}_1 |-> w_1, \dots, \mathtt{r}_{r_{max}} |-> w_{r_{max}}\})\\
\quad\quad\quad\quad = \{\mathtt{sp} |-> \embed_S(S), \mathtt{r}_1 |-> \embed_w(w_1), \dots, \mathtt{r}_{r_{max}} |-> \embed_w(w_{r_{max}})\}
\end{array} \\\\

\begin{array}{ll}
\rlap{$\embed{\iota} : \high\iota -> \simple\iota$} \\
\embed{\iota}(\mathtt{add}\ r_a, r_b, v) &= \mathtt{add}\ r_a, r_b, \embed_v(v) \\
\embed{\iota}(\mathtt{sub}\ r_a, r_b, v) &= \mathtt{sub}\ r_a, r_b, \embed_v(v) \\
\embed{\iota}(\mathtt{salloc}\ n) &= \mathtt{salloc}\ n \\
\embed{\iota}(\mathtt{sfree}\ n) &= \mathtt{sfree}\ n \\
\embed{\iota}(\mathtt{ld}\ r, \mathtt{sp}(n)) &= \mathtt{ld}\ r, \mathtt{sp}(n) \\
\embed{\iota}(\mathtt{st}\ \mathtt{sp}(n), r) &= \mathtt{st}\ \mathtt{sp}(n), r \\
\embed{\iota}(\mathtt{ld}\ r_a, r_b(n)) &= \mathtt{ld}\ r_a, r_b(n) \\
\embed{\iota}(\mathtt{st}\ r_a(n), r_b) &= \mathtt{st}\ r_a(n), r_b \\
\embed{\iota}(\mathtt{malloc}\ r, \langle \tau_1, \dots, \tau_n \rangle) &= \mathtt{malloc}\ r, n \\
\embed{\iota}(\mathtt{mov}\ r, v) &= \mathtt{mov}\ r, \embed_v(v) \\
\embed{\iota}(\mathtt{beq}\ r, v) &= \mathtt{beq}\ r, \embed_v(v) \\
\end{array} \\\\

\begin{array}{ll}
\embed_I : \high I -> \simple{I} \\
\embed_I(\iota ; I) &= \embed{\iota}(\iota) ; \embed_I(I) \\
\embed_I(\mathtt{jmp}\ v) &= \mathtt{jmp}\ (\embed_v(v)) \\
\embed_I(\mathtt{halt}) &= \mathtt{halt}
\end{array} \\\\

\begin{array}{l}
\embed_P : \high P -> \simple{P} \\
\embed_P(H, R, I) = (\embed_H(H), \embed_R(R), \embed_I(I)) \\\\

\highembed : \highlang -> \simplelang \\
\highembed(\high\running(G, P)) = \simple\running(\embed_G(G), \embed_P(P)) \\
\highembed(\high\halted) = \simple\halted \\
\end{array} \\\\

\end{array}\]
}

\subsection{Notes on the Agda implementation}

The Agda-implementation\footnote{The relevant file is
  \texttt{Judgments/Embed.agda}} for this is more or less a direct transcription
of this section.


% Each of these languages has an associated small-step semantic. For
% \native{\lang}, this semantic is as defined in the user manual, while the exact semantics of the

% This chapter has three parts. First we will formally introduce the concepts of
% Typed Assembly Language in an informal setting. We will then give a formal
% definition of the same concepts. Finally we will introduce the specific instance
% of Typed Assembly Language used in \cite{STAL}.

% Note that I have chosen a specific way of presenting Typed Assembly Language,
% that lends itself well to formalization. While this approach is not completely
% similar to the approach used in \cite{STAL}, I expect that the concepts should
% be fairly recognizable.

% \section{Informal definition}
\paragraph{The native language} is the language on which the rest of the work is
based, typically a real-world assembler language. It comes with a notion of a
\textbf{native semantics}, defined by an external party. This means that we can
model the language as a non-deterministic state machine. Note that in this
semantics, a program can \emph{always} progress, however sometimes this progress
results termination for running invalid code.

In practice this native language can be a lot of different things. Some of these
are:

\begin{itemize}
\item The set of possible states a specific Pentium 4 can be in. The semantics
  are mostly those defined in the Intel Specification\cite{intelsys}, however
  the specific CPU might deviate slightly from this specification because of
  bugs or unspecified behavior.
\item The set of possible states of a process running on Linux version 4.2.0-18
  on a specific Mips-CPU. While the semantics will obviously be mostly
  equivalent to those defined in the Mips Specification\cite{mipssys} and in the
  Posix Standard\cite{posix}, there are a lot of undefined Linux-specific
  behavior that will depend heavily on the kernel version.
\end{itemize}


\paragraph{The simplified language} is a simplification of the raw language
along with \emph{simplified semantics} to match it. While the native language
typically deal with a complicated memory model, where there is no real
difference between code and data, pointers and numbers, this is not the case in
the simplified semantics. This simplification serves multiple purposes:

\begin{itemize}
\item The native semantics are typically not written as a formal
  specification. This means that they might contain both under-specified
  behavior in certain corner cases and very complicated behavior in others.

  By simplifying we avoid both of these problems.
\item Similarly we might be interested in talking about multiple different
  native languages, that only differ very slightly in their behavior. By
  simplifying the semantics, we are able to talk about multiple similarly
  machines are the same time.
\item The native semantics might contain semantics that are undesirable. For
  instance it might halt or crash when certain actions occur. In our
  simplification we purposefully avoid undesirable in our semantics, and instead
  leave these behaviors undefined.

  \textbf{Note:} This means that the semantic relation is no longer a total
  left-total.
\end{itemize}

\paragraph{The typed language} is an extension of this simplified language, with
the purpose of introducing typing judgments. Specifically it differs from the
simplified semantics in the following ways:

\begin{itemize}
\item The typed language contains annotations not present in this simplified
  language. If these annotations are thrown away, then we arrive at the
  simplified language.

\item The typed semantics might be more restrictive than the simplified
  semantics, however they are still in a certain sense ``compatible'' with the
  simplified semantics.

\item The typed language also comes with a number of typing judgments.
\end{itemize}

% \section{Formal Definition}

\begin{definition}
  A \textbf{simplfied assembly language} is a tuple

  $L = (\mathcal{G}, \mathcal{H}, \mathcal{R}, \mathcal{I}, \mathcal{T}, R_\to)$
  such that:

  \begin{itemize}
  \item $\mathcal{G}$ a set, called the \emph{set of possible global states}.
  \item $\mathcal{H}$ a set, called the \emph{set of possible heap states}.
  \item $\mathcal{R}$ a set, called the \emph{set of possible register files}.
  \item $\mathcal{I}$ a set, called the \emph{set of possible instruction
      sequences}.
  \item $\mathcal{T}$ a set, called the \emph{set of possible termination
      states}.
  \item Write as $\mathcal{S}$ short hand for
    $\mathcal{H} \times \mathcal{R} \times \mathcal{I}$ and $\mathcal{S}^T$ as
    shorthand for $\mathcal{S} \cup \mathcal{T}$.
  \item $\mathcal{S} \cap \mathcal{T} = \emptyset$.
  \item $R_\to$ is a partial function
    $\mathcal{G} \times \mathcal{S} \to \mathcal{S}^T$,
    called the \emph{transition relationtion}.
  \item Write $R_\to^\ast$ for the transitive closure of $R_\to$.
  \end{itemize}

  Assume $S_1 \in \mathcal{S}$ and $S_2 \in \mathcal{S}^T$. We will write
  $G |- S_1 \to_L S_2$ as shorthand for $(G, S_1, S_2) \in R_\to$.  When $L$ can
  be inferred without ambiguity, we will leave it out. We will use similar
  notation with $\to^\ast$ and $R_\to^\ast$.
\end{definition}

\begin{definition}
  Let $L$ be a simplified assembly language. We say that
  $(\mathbf{\Psi}_g, \mathbf{\Psi}_h, \mathbf{\Gamma})$ is a \emph{typing
    scheme} for $L$ iff:

  \begin{itemize}
  \item $\mathbf{\Psi}_g$ is a set, called the \emph{set of global state
      types}. Let $\Psi_g$ be the typical member of this set.
  \item $\mathbf{\Psi}_h$ is a set, called the \emph{set of heap state
      types}. Let $\Psi_h$ be the typical member of this set.
  \item $\mathbf{\Gamma}$ is a set, called the \emph{set of register types}. Let
    $\Gamma$ be the typical member of this set.
  \item There is a judgment of the form $|- G : \Psi_g$.
  \item There is a judgment of the form $\Psi_g |- H : \Psi_h$.
  \item There is a judgment of the form $\Psi_g , \Psi_h |- R : \Gamma$.
  \item There is a judgment of the form
    $\Psi_g , \Psi_h , \Gamma |- I\ \mathbf{Valid}$.
  \end{itemize}

  We define $G |- S\ \mathbf{Valid}$ as a judgment only one rule:
  \begin{mathpar}
    \infer{
      |- G : \Psi_g \and
      \Psi_g |- H : \Psi_h \and
      \Psi_g , \Psi_h |- R : \Gamma \and
      \Psi_g , \Psi_h , \Gamma |- I\ \mathbf{Valid}
    }{
      G |- (H, R, I)\ \mathbf{Valid}
    }
  \end{mathpar}

  We call $L$ for a \emph{typed assembly language}, if it has a typing scheme.
\end{definition}

\begin{definition}
  Let $L$ be a typed assembly language. We say that it is \emph{progressing} iff
  for any given $G \in \mathcal{G}, S \in \mathcal{S}$ with

  \begin{itemize}
  \item $G |- S\ \mathbf{Valid}$
  \end{itemize}

  we can conclude that there is some $S' \in \mathcal{S}^T$ such that

  \begin{itemize}
  \item $G |- S \to S'$.
  \end{itemize}
\end{definition}

\begin{definition}
  Let $L$ be a typed assembly language. We say that it is \emph{reducing} iff
  for any given $G \in \mathcal{G}, S, S' \in \mathcal{S}$ with

  \begin{itemize}
  \item $G |- S\ \mathbf{Valid}$
  \item $G |- S \to S'$.
  \end{itemize}

  we can conclude that:

  \begin{itemize}
  \item $G |- S'\ \mathbf{Valid}$
  \end{itemize}
\end{definition}

\begin{definition}
  Let $L$ be a typed assembly language. We say that it is \emph{sound} iff it is
  both progressing and reducing.
\end{definition}

\begin{theorem}[Soundness]
  Let $L$ be a sound typed assembly language. Let $G \in \mathcal{G}$ and
  $S, S' \in \mathcal{S}$ and assume $G |- S\ \mathbf{Valid}$ along with
  $G |- S \to\^ast S'$. We now conclude that there exists some
  $S'' \in \mathcal{S}^T$ with $G |- S' \to S''$.
\end{theorem}

\begin{proof}
  We first conclude that $G |- S'\ \mathbf{Valid}$. This can be shown by
  induction over the derivation of $G |- S \to^\ast S'$ along with the fact that
  $L$ is reducing.

  Since $L$ is also progressing, this implies the result.
\end{proof}

\subsection{Bisimulation}

\begin{definition}
  Assume the following:
  \begin{itemize}
  \item
    $L_1 = (\mathcal{G}_1, \mathcal{H}_1, \mathcal{R}_1, \mathcal{I}_1,
    \mathcal{T}, R_\to)$.
  \item
    $L_2 = (\mathcal{G}_2, \mathcal{H}_2, \mathcal{R}_2, \mathcal{I}_2,
    \mathcal{T}, R_\to)$.
  \item Let $E = (f_G, f_H, f_R, f_I, f_T)$ be tuple of total functions. The
    functions should be defined over $\mathcal{G}_1 \to \mathcal{G}_2$,
    $\mathcal{H}_1 \to \mathcal{H}_2$, $\mathcal{R}_1 \to \mathcal{R}_2$,
    $\mathcal{I}_1 \to \mathcal{I}_2$ and $\mathcal{T}_1 \to \mathcal{T}_2$
    respectively.
  \end{itemize}

  Define $f_S : \mathcal{S}_1 \to \mathcal{S}_2$ and
  $f_{S^T} : \mathcal{S}_1^T \to \mathcal{S}_1^T$ as:

  \begin{align*}
    f_S(H, R, I) &= (f_H(H), f_R(R), f_I(I)) \\
    f_{S^T}(v) &=
                 \begin{cases}
                   f_S(v) & \text{if $v \in \mathcal{S}_1$} \\
                   f_T(v) & \text{if $v \in \mathcal{T}_1$} \\
                 \end{cases}
  \end{align*}

  We say that $E$ is a bisimulation between $L_1$ and $L_2$ iff the following
  holds:

  \begin{itemize}
  \item Assume
    $G \in \mathcal{G}, S_1 \in \mathcal{S}_1, S_1' \in \mathcal{S}_1^T, S_2 \in
    \mathcal{S}_2, S_2' \in \mathcal{S}_2^T$.
  \item Assume we have a derivationIf $\alpha$ is a derivation $\alpha$ of $G |- S_1 \to S_1'$, then we can
    find a derivation $\beta$ of $f_G(G) |- S_2 \to S_2'$ such that the diagram
    below commutes.
  \item The converse also holds: If $\beta$ is a derivation of
    $f_G(G) |- S_2 \to S_2'$, then we can find a derivation $\alpha$ of
    $G |- S_1 \to S_1'$ such that the diagram below commutes.
  \end{itemize}
\end{definition}

% Define $f : \mathcal{S}^T \to \mathcal{S}
% \begin{theorem}[Soundness of embeddings]
%   Let $L_1$ by a sound typed assembly language and $L_2$ be a simplified
%   assembly language. Assume there

% \section{STAL}

% In this chapter we will introduce the concept of a Typed Assembly Language,
% specifically how it relates to a pure assembly language, how one designs it, and
% what it achieves.

% Throughout the section, we will assume that we have been given the task of
% designing a type-system for the real-world assembly language similar to
% MIPS. The details of the language does not matter, much and indeed most
% real-world assembler languages (or even some bytecode languages) could have been
% chosen instead. We choose MIPS primarily to have an easier comparison with existing

%   choice of language does not matter much, as the same techniques
% should be applicable to other hardware languages, and likely even to many
% bytecode languages. However using the choice of MIPS is advantagous, as it is
% very close to the language used in the next section.

% Before beginning, let us take a step back and look at what we have and we are trying to
% accomplish. We have been

% As stated previously\later[]{Put in reference}, we want to create a type system
% for a low-level language.

% In this section we will assume that we are trying to design a type system for a
% variant of MIPS, specified by an external party. This was chosen, because

%  as a base
% for our study. The actual chosen language does not matter much for the
% discussion at hand, as the same techniques should be applicable to other
% assembler language and likely even to some bytecode languages as well.

% Before we even begin to consider what it would mean to introduce types to an
% assembly languages, let us take a step back and look at the big-picture view.

% What we have is as design specification for the CPU in question. This design
% specification consist roughly of the following:

% \begin{itemize}
% \item Some description of how the state present in the CPU at any given
%   time. For most CPUs this will include at least the RAM, the register file and
%   the program counter.
% \item A partial function mapping the current state to an instruction $\iota$. For most CPUs
% \item A partial function mapping

% \begin{tabular}{lrcl}
% $instructions$ & $\iota$ & ::= & $\mathtt{add}\ \mathtt{r}_d, \mathtt{r}_s, v \mid \mathtt{sub}\ \mathtt{r}_d, \mathtt{r}_s, v \mid \dots$ \\
% \end{tabular}

% \later[inline]{Complete the above and add judgments too}

% It turns out that while the above semantics are very general in


% \chapter{Stack-Based Typed Assembly Language}
% This chapter will define the two languages $STAL$ and $STAL_S$. The design bla
% bla previous section.

% For the most part $STAL$ is equivalent to the one defined in \cite{STAL},
% however a few changes have been made to simplify the proofs. It is our opinion
% that the resulting language is morally equivalent to the original one.

% \subsection{Related judgments}

\todo[inline]{This is a few syntax definitions. While they might be introduced
  here, they should be defined more formally elsewhere.}

\subsubsection{Dictionaries}
A \textbf{dictionary} is an unordered set of the form
$\{x_1 |-> y_1, \dots, x_n |-> y_n\}$ with the restriction that $x_i \neq x_j$
whenever $i \neq j$.

Assume that $D_1$ and $D_2$ are arbitrary dictionaries of the form
$\{x_1 |-> y_1, \dots, x_n |-> y_n\}$ and
$\{x_1' |-> y_1', \dots, x_m' |-> y_m'\}$ respectively.

Define $\mathbf{Keys}(D_1)$ to be the set $\{x_1, \dots, x_n\}$. Whenever
$\mathbf{Keys}(D_1) \cap \mathbf{Keys}(D_2) = \emptyset$, then the expression
$D_1 \cup D_2$ denotes the the dictionary
$\{x_1 |-> y_1, \dots, x_n |-> y_n, x_1' |-> y_1', \dots, x_m' |-> y_m'\}$.

We define $D_1\{x |-> y\}$ whenever $x \in \mathbf{Keys}(D_1)$. This expression
is the dictionary $\{x_1 |-> y_1', \dots, x_n |-> y_n'\}$ subject to the
restriction that $y_k' = y$ whenever $x_k = x$ and $y_k' = y_k$ otherwise.

By $(x |-> y) \in D$ we mean a judgment asserting that $D$ is of the form
$\{x_1 |-> y_1, \dots, x_n |-> y_n\}$ and that there exists some $k$ such that
$x = x_k$ and $y = y_k$. This syntax is meaning for e.g. global collections and
register files.

\subsubsection{List-like objects}
A set $S$ is \textbf{list-like} iff there is a meaningful way to interpret the
elements of $S$ as ordered lists of elements drawn from a base-set $S_B$. In
other words, there should exist a injective function
$f : S \to \mathbf{List}\ S_B$, where $f$ may only depend on the superficial
syntax of the elements.

For instance heap values and stacks are list-like, as they are written
$\langle w_1, \dots, w_n \rangle$ and $w_1 :: \dots :: w_n :: \mathtt{nil}$
respectively.

For the remainder of this subsection, assume $L_1$ and $L_2$ are typical
elements of the same list-like set $S$. Further assume that
$f(L_1) = [x_1, \dots, x_n]$ and $f(L_2) = [y_1, \dots, y_m]$.

Define $\mathbf{Length}(L_1) = n$.

Define $L_1 >> L_2$ be the element $L_3 \in S$ such that
$f(L_3) = [x_1, \dots, x_n, y_1, \dots, y_m]$, if such an element exists.

By $L_1 ! k => v$ we mean a judgment asserting $x_k = v$.

By $L_1[k]<- v => L_2$ we mean a judgment asserting that $n=m$, $y_k=v$ and
$x_i = y_i$ if $i \neq k$.

% \section{Base language}

In the following $r_{max}$ and $w$ ares a global constants, representing
respectively the number of registers on the machine and the number of bits in a
machine-word. While it is a constant, its concrete value is never used and it
can thus vary between implementation.
\\

\subsection{Grammar}

\begin{tabular}{lrcl}
Variables: \\
\textit{global pointers}    & $\ell_g$ \\
\textit{heap pointers}      & $\ell_h$ \\
\\
\textit{integer}            & $n$ & ::= & $0, 1, \dots$ \\
\textit{machine integers}   & $i$ & ::= & $0, 1, \dots, 2^{w}-1$ \\
\textit{registers}          & $r$ & ::= & $\mathtt{r}_1 \mid \dots \mid \mathtt{r}_{r_{max}}$ \\
\textit{word values}        & $w$ & ::= & $\mathtt{globval}\ \ell_g \mid \mathtt{heapval}\ \ell_h \mid \mathtt{int}\ i \mid \mathtt{ns} \mid \mathtt{uninit}$ \\
\textit{small values}       & $v$ & ::= & $\mathtt{globval}\ \ell_g \mid \mathtt{reg}\ r \mid \mathtt{int}\ i \mid \mathtt{ns} \mid \mathtt{uninit}$ \\
\textit{global values}      & $g$ & ::= & $\mathtt{code}\ I$ \\
\textit{global collections} & $G$ & ::= & $\{\ell_g |-> g, \dots, \ell_g |-> g\}$ \\
\textit{heap values}        & $h$ & ::= & $\langle w, \dots, w \rangle$ \\
\textit{heap collections}   & $H$ & ::= & $\{\ell_h |-> h, \dots, \ell_h |-> h\}$ \\
\textit{stacks}             & $S$ & ::= & $\mathtt{nil} \mid w :: S$ \\
\textit{register files}     & $R$ & ::= & $\{\mathtt{sp} |-> S, \mathtt{r}_1 |-> w, \dots, \mathtt{r}_{r_{max}} |-> w\}$ \\
\\
\textit{instructions} & $\iota$ & ::= & $\mathtt{add}\ r_d, r_s, v \mid \mathtt{sub}\ r_d, r_s, v \mid$ \\
        &&& $\mathtt{salloc}\ n \mid \mathtt{sfree}\ n \mid$ \\
        &&& $\mathtt{ld}\ r_d, \mathtt{sp}(n) \mid \mathtt{st}\ \mathtt{sp}(n), r_s \mid$\\
        &&& $\mathtt{ld}\ r_d, r_s(n) \mid \mathtt{st}\ r_d(n), r_s \mid$\\
        &&& $\mathtt{malloc}\ r_d,\ n \mid $ \\
        &&& $\mathtt{mov}\ r_d, r_s \mid \mathtt{beq}\ r, v$ \\
\textit{instruction sequences} & $I$ & ::= & $\iota ; I \mid \mathtt{jmp}\ v$ \\
\textit{programs} & $P$ & ::= & $(G, H, R, I)$ \\
\end{tabular}

\paragraph{}
The mappings of the form $\{x |-> y, \dots\}$ are unordered and do not have
repeating keys. In the Agda-implementation, these maps are mostly implemented as
ordered lists with implicit keys, however we can ignore this fact when using
paper-notation.

We will occasionally refer to a specific heap collection as a \textbf{heap} or
perhaps \textbf{the heap} for a specific program state. Similarly a specific
global collection might be referred to as \textbf{the globals} of a program. We
will however not use the term ``global'' without qualification, to avoid
ambiguity between global collections and global values.

The fundamental assumption of this language is that the state of a program is
totally defined by a collection of immutable global values, the current state of
the heap, the current stack, the current registers and the current instruction
pointer. This is represented in our language as:

\begin{itemize}
\item The immutable global values are places in a global collection
  consisting of individual global items.
\item The mutable global values are places in a heap collection
  consisting of individual heap items.
\item The current stack is encoded as a stack consisting and placed inside a
  register file.
\item The value of the current registers are encoded in a register file together
  with the stack.
\item The instruction pointer is not encoded directly as a pointer, it is
  instead represented as the instructions to be run until the next block.
\end{itemize}

While this representation is enough for many programs, it also excludes many
programs by design. For instance it excludes mutable global variables not stored
on the heap and self-modifying code (such as just-in-time compilers).

It should be possible to expand the language to allow more programs, however
some sacrifice will always be needed, as we are not able to exclude malicious
programs from our semantics without also excluding some valid programs.

\subsection{Related judgments}

\todo[inline]{This is a few syntax definitions. While they might be introduced
  here, they should be defined more formally elsewhere.}

\subsubsection{Dictionaries}
A \textbf{dictionary} is an unordered set of the form
$\{x_1 |-> y_1, \dots, x_n |-> y_n\}$ with the restriction that $x_i \neq x_j$
whenever $i \neq j$.

Assume that $D_1$ and $D_2$ are arbitrary dictionaries of the form
$\{x_1 |-> y_1, \dots, x_n |-> y_n\}$ and
$\{x_1' |-> y_1', \dots, x_m' |-> y_m'\}$ respectively.

Define $\mathbf{Keys}(D_1)$ to be the set $\{x_1, \dots, x_n\}$. Whenever
$\mathbf{Keys}(D_1) \cap \mathbf{Keys}(D_2) = \emptyset$, then the expression
$D_1 \cup D_2$ denotes the the dictionary
$\{x_1 |-> y_1, \dots, x_n |-> y_n, x_1' |-> y_1', \dots, x_m' |-> y_m'\}$.

We define $D_1\{x |-> y\}$ whenever $x \in \mathbf{Keys}(D_1)$. This expression
is the dictionary $\{x_1 |-> y_1', \dots, x_n |-> y_n'\}$ subject to the
restriction that $y_k' = y$ whenever $x_k = x$ and $y_k' = y_k$ otherwise.

By $(x |-> y) \in D$ we mean a judgment asserting that $D$ is of the form
$\{x_1 |-> y_1, \dots, x_n |-> y_n\}$ and that there exists some $k$ such that
$x = x_k$ and $y = y_k$. This syntax is meaning for e.g. global collections and
register files.

\subsubsection{List-like objects}
A set $S$ is \textbf{list-like} iff there is a meaningful way to interpret the
elements of $S$ as ordered lists of elements drawn from a base-set $S_B$. In
other words, there should exist a injective function
$f : S \to \mathbf{List}\ S_B$, where $f$ may only depend on the superficial
syntax of the elements.

For instance heap values and stacks are list-like, as they are written
$\langle w_1, \dots, w_n \rangle$ and $w_1 :: \dots :: w_n :: \mathtt{nil}$
respectively.

For the remainder of this subsection, assume $L_1$ and $L_2$ are typical
elements of the same list-like set $S$. Further assume that
$f(L_1) = [x_1, \dots, x_n]$ and $f(L_2) = [y_1, \dots, y_m]$.

Define $\mathbf{Length}(L_1) = n$.

Define $L_1 >> L_2$ be the element $L_3 \in S$ such that
$f(L_3) = [x_1, \dots, x_n, y_1, \dots, y_m]$, if such an element exists.

By $L_1 ! k => v$ we mean a judgment asserting $x_k = v$.

By $L_1[k]<- v => L_2$ we mean a judgment asserting that $n=m$, $y_k=v$ and
$x_i = y_i$ if $i \neq k$.

\subsection{Semantics}

We are now able to implement the semantics for our language: \\

\fbox{$\evalbig{R}{v}{w}$}
\begin{mathpar}
\infer{ }{
  \evalbig{R}{\mathtt{globval}\ \ell_g}{\mathtt{globval}\ \ell_g}
} \and
\infer{
  (r \mapsto w) \in R
}{
  \evalbig{R}{\mathtt{reg}\ r}{w}
} \and
\infer{ }{
  \evalbig{R}{\mathtt{int}\ i}{\mathtt{int}\ i}
} \and
\infer{ }{
  \evalbig{R}{\mathtt{ns}}{\mathtt{ns}}
} \and
\infer{ }{
  \evalbig{R}{\mathtt{uninit}}{\mathtt{uninit}}
}
\end{mathpar}

\fbox{$\execinstruction{G}{H, R, I}{H', R', I'}$}
\begin{mathpar}
\infer{
  \evalbig{R}{v}{\mathtt{int}\ i_1} \and
  (\mathtt{r}_s \mapsto \mathtt{int}\ i_2) \in R
}{
  \execinstruction{G}
    {H, R, (\mathtt{add}\ r_d, r_s, v) ; I}
    {H, R\{r_d \mapsto \mathtt{int}\ (i_1 + i_2)\}, I}
} \and
\infer{
  \evalbig{R}{v}{\mathtt{int}\ i_1} \and
  (r_s \mapsto \mathtt{int}\ i_2) \in R
}{
  \execinstruction{G}
    {H, R, (\mathtt{sub}\ r_d, r_s, v) ; I}
    {H, R\{r_d |-> \mathtt{int}\ (i_1 - i_2)\}, I}
} \and
\infer{
  (\mathtt{sp} |-> S) \in R \and
  S' = \overbrace{\mathtt{ns} :: \dots :: \mathtt{ns}}^n :: S
}{
  \execinstruction{G}
    {H, R, \mathtt{salloc}\ n ; I}
    {H, R\{\mathtt{sp} |-> S'\}, I}
} \and
\infer{
  (\mathtt{sp} |-> S) \in R \and
  S = \overbrace{w_1 :: \dots :: w_n}^n :: S'
}{
  \execinstruction{G}
    {H, R, \mathtt{sfree}\ n ; I}
    {H, R\{\mathtt{sp} |-> S'\}, I}
} \and
\infer{
  (\mathtt{sp} |-> S) \in R \and
  S ! n => w
}{
  \execinstruction{G}
    {H, R, (\mathtt{ld}\ r_d, \mathtt{sp}(n)) ; I}
    {H, R\{r_d |-> w\}, I}
} \and
\infer{
  (r_s |-> \mathtt{heapval}\ \ell_h) \in R \and
  (\ell_h |-> h) \in H \and
  h ! n => w
}{
  \execinstruction{G}
    {H, R, (\mathtt{ld}\ r_d, r_s(n)) ; I}
    {H, R\{r_d |-> w\}, I}
} \and
\infer{
  (r_s |-> w) \in R \and
  (\mathtt{sp} |-> S) \in R \and
  S[n] <- w => S'
}{
  \execinstruction{G}
    {H, R, (\mathtt{st}\ \mathtt{sp}(n), r_s) ; I}
    {H, R\{\mathtt{sp} |-> S'\}, I}
} \and
\infer{
  (r_s |-> w) \in R \and
  (r_d |-> \mathtt{heapval}\ \ell_h) \in R \and
  (\ell_h |-> h) \in H \and
  h[n] <- w => h'
}{
  \execinstruction{G}
    {H, R, (\mathtt{st}\ r_d(n), r_s) ; I}
    {H\{\ell_h |-> h'\}, R, I}
} \and
\infer{
  \ell_h \not\in \mathbf{Keys}(H) \and
  h = \overbrace{\langle \mathtt{uninit}, \dots, \mathtt{uninit} \rangle}^n
}{
  \execinstruction{G}
    {H, R, (\mathtt{malloc}\ r_d, n) ; I}
    {H \cup \{\ell_h |-> h\}, R\{r_d |-> \mathtt{heapval}\ \ell_h\}, I}
} \and
\infer{
  \evalbig{R}{v}{w}
}{
  \execinstruction{G}
    {H, R, (\mathtt{mov}\ r_d, v) ; I}
    {H, R\{r_d |-> w\}, I}
} \and
\infer{
  (r |-> \mathtt{int}\ 0) \in R \and
  \evalbig{R}{v}{\mathtt{code}\ I'}
}{
  \execinstruction{G}
    {H, R, (\mathtt{beq}\ r, v) ; I}
    {H, R, I'}
} \and
\infer{
  (r |-> \mathtt{int}\ i) \in R \and
  i \neq 0
}{
  \execinstruction{G}
    {H, R, (\mathtt{beq}\ r, v) ; I}
    {H, R, I}
} \and
\infer{
  \evalbig{R}{v}{\mathtt{code}\ I'}
}{
  \execinstruction{G}
    {H, R, \mathtt{jmp}\ v}
    {H, R, I'}
}
\end{mathpar}

\fbox{$(G, H, R, I) -> (G', H', R', I')$}
\begin{mathpar}
\infer{
  \execinstruction{G}{H, R, I}{H', R', I'}
}{
  (G, H, R, I) -> (G, H', R', I')
}
\end{mathpar}

% \section{Annotated language}

\subsection{Grammar}

\begin{tabular}{lrcl}
Variables: \\
\textit{global pointers}          & $\ell_g$ \\
\textit{heap pointers}            & $\ell_h$ \\
\textit{type assumption}          & $\alpha$ \\
\textit{stack assumption}         & $\rho$ \\
\\
\textit{integer}                  & $n$ & ::= & $0, 1, \dots$ \\
\textit{machine integers}         & $i$ & ::= & $0, 1, \dots, 2^{w}-1$ \\
\textit{registers}                & $r$ & ::= & $\mathtt{r}_1 \mid \dots \mid \mathtt{r}_{r_{max}}$ \\
\\
\textit{types}                    & $\tau$ & ::= & $\alpha \mid \mathtt{int} \mid \mathtt{ns} \mid \mathtt\forall[ \Delta ] \Gamma \mid \langle\tau^\phi,\dots,\tau^\phi\rangle$ \\
\textit{stack types}              & $\sigma$ & ::= & $\rho \mid \mathtt{nil} \mid \tau :: \sigma$ \\
\textit{initialization flags}     & $\phi$ & ::= & $\mathtt{init} \mid \mathtt{uninit}$ \\
\textit{type assignments}         & $\Delta$ & ::= & $\mathtt{nil} \mid a :: \Delta$ \\
\textit{type assignment value}    & $a$ & ::= & $\alpha \mid \rho$ \\
\textit{global label assignments} & $\Psi_g$ & ::= & $\{\ell_g |-> \tau, \dots, \ell_g |-> \tau\}$ \\
\textit{heap label assignmentss}  & $\Psi_h$ & ::= & $\{\ell_h |-> \tau, \dots, \ell_h |-> \tau\}$ \\
\textit{label assignments}        & $\Psi$ & ::= & $(\Psi_g , \Psi_h)$ \\
\textit{register assignments}     & $\Gamma$ & ::= & $\{\mathtt{sp} |-> \sigma, \mathtt{r}_1 |-> \tau, \dots, \mathtt{r}_{r_{max}} |-> \tau\}$ \\
\\
\textit{instantiation}            & $i$ & ::= & $\tau/\alpha \mid \sigma/\rho$ \\
\textit{word values}              & $w$ & ::= & $\mathtt{globval}\ \ell_g \mid \mathtt{heapval}\ \ell_h \mid \mathtt{int}\ i \mid \mathtt{ns} \mid \mathtt{uninit}\ \tau \mid \Lambda\ \Delta.w[i, \dots, i]$ \\
\textit{small values}             & $v$ & ::= & $\mathtt{globval}\ \ell_g \mid \mathtt{reg}\ r \mid \mathtt{int}\ i \mid \mathtt{ns} \mid \mathtt{uninit}\ \tau \mid \Lambda\ \Delta.v[i, \dots, i]$ \\
\textit{global values}            & $g$ & ::= & $\mathtt{code}\ \forall[ \Delta ] \Gamma \cdot I$ \\
\textit{global collections}       & $G$ & ::= & $\{\ell_g |-> g, \dots, \ell_g |-> g\}$ \\
\textit{heap values}              & $h$ & ::= & $\langle w, \dots, w \rangle$ \\
\textit{heap collections}         & $H$ & ::= & $\{\ell_h |-> h, \dots, \ell_h |-> h\}$ \\
\textit{stacks}                   & $S$ & ::= & $\mathtt{nil} \mid w :: S$ \\
\textit{register files}           & $R$ & ::= & $\{\mathtt{sp} |-> S, \mathtt{r}_1 |-> w, \dots, \mathtt{r}_{r_{max}} |-> w\}$ \\
\\
\textit{instructions} & $\iota$ & ::= & $\mathtt{add}\ r_d, r_s, v \mid \mathtt{sub}\ r_d, r_s, v \mid$ \\
        &&& $\mathtt{salloc}\ n \mid \mathtt{sfree}\ n \mid$ \\
        &&& $\mathtt{ld}\ r_d, \mathtt{sp}(n) \mid \mathtt{st}\ \mathtt{sp}(n), r_s \mid$\\
        &&& $\mathtt{ld}\ r_d, r_s(n) \mid \mathtt{st}\ r_d(n), r_s \mid$\\
        &&& $\mathtt{malloc}\ r_d,\ \langle \tau_1, \dots, \tau_n \rangle \mid $ \\
        &&& $\mathtt{mov}\ r_d, r_s \mid \mathtt{beq}\ r, v$ \\
        &&& $\mathtt{halt}$ \\
\textit{instruction sequences} & $I$ & ::= & $\iota ; I \mid \mathtt{jmp}\ v$ \\
\textit{programs} & $P$ & ::= & $(G, H, R, I)$ \\

\end{tabular}

\subsection{Semantics}

\paragraph{}
\fbox{$\evalrega{R}{v}{w}$}
\begin{mathpar}
\infer{ }{
  \evalrega{R}{\mathtt{globval}\ \ell_g}{\mathtt{globval}\ \ell_g}
} \and
\infer{
  (r \mapsto w) \in R
}{
  \evalrega{R}{\mathtt{reg}\ r}{w}
} \and
\infer{ }{
  \evalrega{R}{\mathtt{int}\ i}{\mathtt{int}\ i}
} \and
\infer{ }{
  \evalrega{R}{\mathtt{ns}}{\mathtt{ns}}
} \and
\infer{ }{
  \evalrega{R}{\mathtt{uninit}\ \tau}{\mathtt{uninit}\ \tau}
} \and
\infer{
  \evalrega{R}{v}{w}
}{
  \evalrega{R}{\Lambda\ \Delta.v[i_1, \dots, i_n]}{\Lambda\ \Delta.w[i_1, \dots, i_n]}
}
\end{mathpar}

\fbox{$\evalcodea{G}{w}{I}$}
\begin{mathpar}
\infer{
  \lookup{G}{\ell_g}{\mathtt{code}\ \forall[\Delta]\Gamma \cdot I}
}{
  \evalcodea{G}{\mathtt{globval}\ \ell_g}{I}
} \and
\infer{
  \evalcodea{G}{w}{I} \and
  I_1 = I[v_1 / x_1] \and
  \dots \and
  I_n = I_{n-1}[v_n / x_n]
}{
  \evalcodea{G}{\Lambda\ \Delta . w[x_1 / y_1, \dots, x_n / y_n]}{I_n}
}
\end{mathpar}

\fbox{$\execia{G}{C}{C}$}
\begin{mathpar}
\infer{
  \evalrega{R}{v}{\mathtt{int}\ i_1} \and
  (\mathtt{r}_s \mapsto \mathtt{int}\ i_2) \in R
}{
  \execia{G}
    {H, R, (\mathtt{add}\ r_d, r_s, v) ; I}
    {H, R\{r_d \mapsto \mathtt{int}\ (i_1 + i_2)\}, I}
} \and
\infer{
  \evalrega{R}{v}{\mathtt{int}\ i_1} \and
  (r_s \mapsto \mathtt{int}\ i_2) \in R
}{
  \execia{G}
    {H, R, (\mathtt{sub}\ r_d, r_s, v) ; I}
    {H, R\{r_d |-> \mathtt{int}\ (i_1 - i_2)\}, I}
} \and
\infer{
  (\mathtt{sp} |-> S) \in R \and
  S' = \overbrace{\mathtt{ns} :: \dots :: \mathtt{ns}}^n :: S
}{
  \execia{G}
    {H, R, \mathtt{salloc}\ n ; I}
    {H, R\{\mathtt{sp} |-> S'\}, I}
} \and
\infer{
  (\mathtt{sp} |-> S) \in R \and
  S = \overbrace{w_1 :: \dots :: w_n}^n :: S'
}{
  \execia{G}
    {H, R, \mathtt{sfree}\ n ; I}
    {H, R\{\mathtt{sp} |-> S'\}, I}
} \and
\infer{
  (\mathtt{sp} |-> S) \in R \and
  \lookup{S}{n}{w}
}{
  \execia{G}
    {H, R, (\mathtt{ld}\ r_d, \mathtt{sp}(n)) ; I}
    {H, R\{r_d |-> w\}, I}
} \and
\infer{
  (r_s |-> \mathtt{heapval}\ \ell_h) \in R \and
  (\ell_h |-> h) \in H \and
  \lookup{h}{n}{w}
}{
  \execia{G}
    {H, R, (\mathtt{ld}\ r_d, r_s(n)) ; I}
    {H, R\{r_d |-> w\}, I}
} \and
\infer{
  (r_s |-> w) \in R \and
  (\mathtt{sp} |-> S) \in R \and
  \update{S}{n}{w}{S'}
}{
  \execia{G}
    {H, R, (\mathtt{st}\ \mathtt{sp}(n), r_s) ; I}
    {H, R\{\mathtt{sp} |-> S'\}, I}
} \and
\infer{
  (r_s |-> w) \in R \and
  (r_d |-> \mathtt{heapval}\ \ell_h) \in R \and
  (\ell_h |-> h) \in H \and
  \update{h}{n}{w}{h'}
}{
  \execia{G}
    {H, R, (\mathtt{st}\ r_d(n), r_s) ; I}
    {H\{\ell_h |-> h'\}, R, I}
} \and
\infer{
  \ell_h \not\in \mathbf{Keys}(H) \and
  h = \overbrace{\langle \mathtt{uninit}\ \tau_1, \dots, \mathtt{uninit}\ \tau_n \rangle}^n
}{
  \execia{G}
    {H, R, (\mathtt{malloc}\ r_d, \langle \tau_1, \dots, \tau_n \rangle) ; I}
    {H \cup \{\ell_h |-> h\}, R\{r_d |-> \mathtt{heapval}\ \ell_h\}, I}
} \and
\infer{
  \evalrega{R}{v}{w}
}{
  \execia{G}
    {H, R, (\mathtt{mov}\ r_d, v) ; I}
    {H, R\{r_d |-> w\}, I}
} \and
\infer{
  (r |-> \mathtt{int}\ 0) \in R \and
  \evalrega{R}{v}{w} \and
  \evalcodea{G}{w}{I'}
}{
  \execia{G}
    {H, R, (\mathtt{beq}\ r, v) ; I}
    {H, R, I'}
} \and
\infer{
  (r |-> \mathtt{int}\ i) \in R \and
  i \neq 0
}{
  \execia{G}
    {H, R, (\mathtt{beq}\ r, v) ; I}
    {H, R, I}
} \and
\infer{
  \evalrega{R}{v}{w} \and
  \evalcodea{G}{w}{I'}
}{
  \execia{G}
    {H, R, \mathtt{jmp}\ v}
    {H, R, I'}
}
\end{mathpar}

\fbox{$\stepa{P}{P'}$}
\begin{mathpar}
\infer{
  \execia{G}{H, R, I}{H', R', I'}
}{
  (G, H, R, I) -> (G, H', R', I')
}
\end{mathpar}

\fbox{$\donea{P}$}
\begin{mathpar}
\infer{ }{
  \donea{(G, H, R, \mathtt{halt})}
}
\end{mathpar}

\subsection{Drop relation}

Relation between simple and annotated language. The function maps constructs
from the annotated language to equivalent constructs of the simple language.

{\footnotesize
\[\begin{array}{l}
\begin{array}{ll}
\rlap{$\mathbf{Drop}_w : w -> w^\circ$} \\
\drop{w}{\mathtt{globval}\ \ell_g} &= \mathtt{globval}\ \ell_g \\
\drop{w}{\mathtt{heapval}\ \ell_h} &= \mathtt{heapval}\ \ell_h \\
\drop{w}{\mathtt{int}\ i} &= \mathtt{int}\ i \\
\drop{w}{\mathtt{ns}} &= \mathtt{ns} \\
\drop{w}{\mathtt{uninit}\ \tau} &= \mathtt{uninit} \\
\drop{w}{\Lambda\ \Delta.w[i_1, \dots, i_n]} &= \drop{w}{w} \\\\

\rlap{$\mathbf{Drop}_v : v -> v^\circ$} \\
\drop{v}{\mathtt{globval}\ \ell_g} &= \mathtt{globval}\ \ell_g \\
\drop{v}{\mathtt{reg}\ r} &= \mathtt{reg}\ r \\
\drop{v}{\mathtt{int}\ i} &= \mathtt{int}\ i \\
\drop{v}{\mathtt{ns}} &= \mathtt{ns} \\
\drop{v}{\mathtt{uninit}\ \tau} &= \mathtt{uninit} \\
\drop{v}{\Lambda\ \Delta.v[i_1, \dots, i_n]} &= \drop{v}{v}
\end{array} \\\\

\begin{array}{l}
\mathbf{Drop}_g : g -> g^\circ \\
\drop{g}{\mathtt{code}\ \forall[\Delta]\Gamma.I} = \mathtt{code}\ \drop{I}{I} \\\\

\mathbf{Drop}_G : G -> G^\circ \\
\drop{G}{\{\ell_{g,1} |-> g_1, \dots, \ell_{g,n} |-> g_n\}} = \{\ell_{g,1} |-> \drop{g}{g_1}, \dots, \ell_{g,n} |-> \drop{g}{g_n}\} \\\\

\mathbf{Drop}_h : h -> h^\circ \\
\drop{h}{\langle w_1, \dots, w_n \rangle} = \langle \drop{w}{w_1}, \dots \drop{w}{w_n}\rangle \\\\

\mathbf{Drop}_H : H -> H^\circ \\
\drop{H}{\{\ell_{h,1} |-> h_1, \dots, \ell_{h,n} |-> h_n\}} = \{\ell_{h,1} |-> \drop{h}{h_1}, \dots, \ell_{h,n} |-> \drop{h}{h_n}\}
\end{array} \\\\

\begin{array}{ll}
\rlap{$\mathbf{Drop}_S : S -> S^\circ$} \\
\drop{S}{\mathtt{nil}} &= \mathtt{nil} \\
\drop{S}{w :: S} &= \drop{w}{w} :: \drop{S}{S} \\
\end{array} \\\\

\begin{array}{l}
\mathbf{Drop}_R : R -> R^\circ \\
\drop{R}{\{\mathtt{sp} |-> S, \mathbf{r}_1 |-> w_1, \dots, \mathbf{r}_{r_{max}} |-> w_{r_{max}}\}}\\
\quad\quad\quad\quad = \{\mathtt{sp} |-> \drop{S}{S}, \mathbf{r}_1 |-> \drop{w}{w_1}, \dots, \mathbf{r}_{r_{max}} |-> \drop{w}{w_{r_{max}}}\}
\end{array} \\\\

\begin{array}{ll}
\rlap{$\mathbf{Drop}_\iota : \iota -> \iota^\circ$} \\
\drop{\iota}{\mathtt{add}\ r_d, r_s, v} &= \mathtt{add}\ r_d, r_s, \drop{v}{v} \\
\drop{\iota}{\mathtt{sub}\ r_d, r_s, v} &= \mathtt{sub}\ r_d, r_s, \drop{v}{v} \\
\drop{\iota}{\mathtt{salloc}\ n} &= \mathtt{salloc}\ n \\
\drop{\iota}{\mathtt{sfree}\ n} &= \mathtt{sfree}\ n \\
\drop{\iota}{\mathtt{ld}\ r_d, \mathbf{sp}(n)} &= \mathtt{ld}\ r_d, \mathbf{sp}(n) \\
\drop{\iota}{\mathtt{st}\ \mathbf{sp}(n), r_s} &= \mathtt{st}\ \mathbf{sp}(n), r_s \\
\drop{\iota}{\mathtt{ld}\ r_d, r_s(n)} &= \mathtt{ld}\ r_d, r_s(n) \\
\drop{\iota}{\mathtt{st}\ r_d(n), r_s} &= \mathtt{st}\ r_d(n), r_s \\
\drop{\iota}{\mathtt{malloc}\ r_d, \langle \tau_1, \dots, \tau_n \rangle} &= \mathtt{malloc}\ r_d, n \\
\drop{\iota}{\mathtt{mov}\ r_d, r_s} &= \mathtt{mov}\ r_d, r_s \\
\drop{\iota}{\mathtt{beq}\ r, v} &= \mathtt{beq}\ r, \drop{v}{v} \\
\end{array} \\\\

\begin{array}{l}
\mathbf{Drop}_I : I -> I^\circ \\
\drop{I}{\iota ; I} = \drop{\iota}{\iota} ; \drop{I}{I} \\
\drop{I}{\mathtt{jmp}\ v} = \mathtt{jmp}\ \drop{v}{v} \\\\

\mathbf{Drop}_P : P -> P^\circ \\
\drop{P}{(G, H, R, I)} = (\drop{G}{G}, \drop{H}{H}, \drop{R}{R}, \drop{I}{I})
\end{array} \\\\

\end{array}\]
}

\subsection{Bisimulation}

\begin{lemma}
  \label{thm:regeval-drop}
  Let $R$ be an annotated register file and assume that
  $\evalbig{R}{v}{w}$. Then $\evalbig{\drop{R}{R}}{\drop{v}{v}}{\drop{w}{w}}$.
\end{lemma}
\begin{proof}
  By structural induction over the derivation of $\evalbig{R}{v}{w}$.
\end{proof}

\begin{lemma}
  Let $f$ be one of the drop-functions (i.e.
  $\mathbf{Drop}_w, \mathbf{Drop}_v, \dots$). Assume
  $x_1, x_2 \in \mathrm{dom}(f)$ and that $x_1 = x_2[y / z]$. Then
  $f(x_1) = f(x_2)$.
\end{lemma}
\begin{proof}
  By structural induction over $x_1$.
\end{proof}

\begin{corollary}
  \label{thm:inst-drop}
  Let $I, I_1, \dots, I_n$ by annotated instruction sequences such that
  $I_1 = I[x_1 / y_1], \dots, I_n = I_{n-1}[x_n, y_n]$. Then
  $\drop{I}{I} = \drop{I}{I_n}$.
\end{corollary}

\begin{lemma}
  \label{thm:codeeval-drop}
  Assume $G$ is an annotated global collection and $\evalbig{G}{w}{I}$. Then
  there exists some $\ell_g$ such that $\drop{w}{w} = \mathtt{globval}\ \ell_g$
  and $\lookup{\drop{G}{G}}{\ell_g}{\mathtt{code}\ \drop{I}{I}}$.
\end{lemma}
\begin{proof}
  By structural induction over the $w$ using \autoref{thm:inst-drop}.
\end{proof}

\begin{lemma}
  \label{thm:exec-drop}
  Assume $G, H, \dots$ are annotated values such that
  $\execinstruction{G}{H, R, I}{H', R', I'}$. Then
  $\execinstruction{\drop{G}{G}}{\drop{H}{H}, \drop{R}{R},
    \drop{I}{I}}{\drop{H}{H'}, \drop{R}{R'}, \drop{I}{I'}}$
\end{lemma}
\begin{proof}
  By structural induction over the derivation of
  $\execinstruction{G}{H, R, I}{H', R', I'}$ using \autoref{thm:regeval-drop}
  and \autoref{thm:codeeval-drop}.
\end{proof}

\begin{corollary}[Simulation]
  \label{thm:simulation}
  Let $\mathcal{R} = \{(P, \drop{P}{P}) : P \in \text{annotated
    programs}\}$. $\mathcal{R}$ is a simulation.
\end{corollary}

\begin{theorem}[Bisimulation]
  \label{thm:bisimulation}
  Let
  $\mathcal{R} = \{(P, \drop{P}{P}) : P \in \text{annotated programs}, \exists
  P' : P -> P'\}$. $\mathcal{R}$ is a bisimulation.
\end{theorem}
\begin{proof}
  We need to show that $\mathcal{R}$ and $\mathcal{R}^{-1}$ are simulations. The
  first part follows from \autoref{thm:exec-drop}.

  For the second part, assume that $(P_A, P_S) \in \mathcal{R}$ and
  $P_S -> P_S'$ for some $P_S'$. We need to show that there is a $P_A'$ such
  that $P_A -> P_A'$ and $(P_A', P_S') \in \mathcal{R}$.

  Let $P_A'$ be the witness of $\exists P' : P -> P'$ from the definition of
  $\mathcal{R}$. It remains to be shown that $(P_A', P_S') \in \mathcal{R}$,
  i.e. $\drop{P}{P_A'} = P_S'$.

  By \autoref{thm:exec-drop} we can get a derivation of
  $\drop{P}{P_A} -> \drop{P}{P_A'}$. By the definition of $\mathcal{R}$ we have
  $P_S = \drop{P}{P_A}$, so by an earlier assumption, we also have a derivation
  of $\drop{P}{P_A} -> P_S'$. By \autoref{thm:determinism}, this implies
  that $\drop{P}{P_A'} = P_S'$.
\end{proof}


% \subsection{Drop relation}

Relation between simple and annotated language. The function maps constructs
from the annotated language to equivalent constructs of the simple language.

{\footnotesize
\[\begin{array}{l}
\begin{array}{ll}
\rlap{$\mathbf{Drop}_w : w -> w^\circ$} \\
\drop{w}{\mathtt{globval}\ \ell_g} &= \mathtt{globval}\ \ell_g \\
\drop{w}{\mathtt{heapval}\ \ell_h} &= \mathtt{heapval}\ \ell_h \\
\drop{w}{\mathtt{int}\ i} &= \mathtt{int}\ i \\
\drop{w}{\mathtt{ns}} &= \mathtt{ns} \\
\drop{w}{\mathtt{uninit}\ \tau} &= \mathtt{uninit} \\
\drop{w}{\Lambda\ \Delta.w[i_1, \dots, i_n]} &= \drop{w}{w} \\\\

\rlap{$\mathbf{Drop}_v : v -> v^\circ$} \\
\drop{v}{\mathtt{globval}\ \ell_g} &= \mathtt{globval}\ \ell_g \\
\drop{v}{\mathtt{reg}\ r} &= \mathtt{reg}\ r \\
\drop{v}{\mathtt{int}\ i} &= \mathtt{int}\ i \\
\drop{v}{\mathtt{ns}} &= \mathtt{ns} \\
\drop{v}{\mathtt{uninit}\ \tau} &= \mathtt{uninit} \\
\drop{v}{\Lambda\ \Delta.v[i_1, \dots, i_n]} &= \drop{v}{v}
\end{array} \\\\

\begin{array}{l}
\mathbf{Drop}_g : g -> g^\circ \\
\drop{g}{\mathtt{code}\ \forall[\Delta]\Gamma.I} = \mathtt{code}\ \drop{I}{I} \\\\

\mathbf{Drop}_G : G -> G^\circ \\
\drop{G}{\{\ell_{g,1} |-> g_1, \dots, \ell_{g,n} |-> g_n\}} = \{\ell_{g,1} |-> \drop{g}{g_1}, \dots, \ell_{g,n} |-> \drop{g}{g_n}\} \\\\

\mathbf{Drop}_h : h -> h^\circ \\
\drop{h}{\langle w_1, \dots, w_n \rangle} = \langle \drop{w}{w_1}, \dots \drop{w}{w_n}\rangle \\\\

\mathbf{Drop}_H : H -> H^\circ \\
\drop{H}{\{\ell_{h,1} |-> h_1, \dots, \ell_{h,n} |-> h_n\}} = \{\ell_{h,1} |-> \drop{h}{h_1}, \dots, \ell_{h,n} |-> \drop{h}{h_n}\}
\end{array} \\\\

\begin{array}{ll}
\rlap{$\mathbf{Drop}_S : S -> S^\circ$} \\
\drop{S}{\mathtt{nil}} &= \mathtt{nil} \\
\drop{S}{w :: S} &= \drop{w}{w} :: \drop{S}{S} \\
\end{array} \\\\

\begin{array}{l}
\mathbf{Drop}_R : R -> R^\circ \\
\drop{R}{\{\mathtt{sp} |-> S, \mathbf{r}_1 |-> w_1, \dots, \mathbf{r}_{r_{max}} |-> w_{r_{max}}\}}\\
\quad\quad\quad\quad = \{\mathtt{sp} |-> \drop{S}{S}, \mathbf{r}_1 |-> \drop{w}{w_1}, \dots, \mathbf{r}_{r_{max}} |-> \drop{w}{w_{r_{max}}}\}
\end{array} \\\\

\begin{array}{ll}
\rlap{$\mathbf{Drop}_\iota : \iota -> \iota^\circ$} \\
\drop{\iota}{\mathtt{add}\ r_d, r_s, v} &= \mathtt{add}\ r_d, r_s, \drop{v}{v} \\
\drop{\iota}{\mathtt{sub}\ r_d, r_s, v} &= \mathtt{sub}\ r_d, r_s, \drop{v}{v} \\
\drop{\iota}{\mathtt{salloc}\ n} &= \mathtt{salloc}\ n \\
\drop{\iota}{\mathtt{sfree}\ n} &= \mathtt{sfree}\ n \\
\drop{\iota}{\mathtt{ld}\ r_d, \mathbf{sp}(n)} &= \mathtt{ld}\ r_d, \mathbf{sp}(n) \\
\drop{\iota}{\mathtt{st}\ \mathbf{sp}(n), r_s} &= \mathtt{st}\ \mathbf{sp}(n), r_s \\
\drop{\iota}{\mathtt{ld}\ r_d, r_s(n)} &= \mathtt{ld}\ r_d, r_s(n) \\
\drop{\iota}{\mathtt{st}\ r_d(n), r_s} &= \mathtt{st}\ r_d(n), r_s \\
\drop{\iota}{\mathtt{malloc}\ r_d, \langle \tau_1, \dots, \tau_n \rangle} &= \mathtt{malloc}\ r_d, n \\
\drop{\iota}{\mathtt{mov}\ r_d, r_s} &= \mathtt{mov}\ r_d, r_s \\
\drop{\iota}{\mathtt{beq}\ r, v} &= \mathtt{beq}\ r, \drop{v}{v} \\
\end{array} \\\\

\begin{array}{l}
\mathbf{Drop}_I : I -> I^\circ \\
\drop{I}{\iota ; I} = \drop{\iota}{\iota} ; \drop{I}{I} \\
\drop{I}{\mathtt{jmp}\ v} = \mathtt{jmp}\ \drop{v}{v} \\\\

\mathbf{Drop}_P : P -> P^\circ \\
\drop{P}{(G, H, R, I)} = (\drop{G}{G}, \drop{H}{H}, \drop{R}{R}, \drop{I}{I})
\end{array} \\\\

\end{array}\]
}

% \subsection{Bisimulation}

\begin{lemma}
  \label{thm:regeval-drop}
  Let $R$ be an annotated register file and assume that
  $\evalbig{R}{v}{w}$. Then $\evalbig{\drop{R}{R}}{\drop{v}{v}}{\drop{w}{w}}$.
\end{lemma}
\begin{proof}
  By structural induction over the derivation of $\evalbig{R}{v}{w}$.
\end{proof}

\begin{lemma}
  Let $f$ be one of the drop-functions (i.e.
  $\mathbf{Drop}_w, \mathbf{Drop}_v, \dots$). Assume
  $x_1, x_2 \in \mathrm{dom}(f)$ and that $x_1 = x_2[y / z]$. Then
  $f(x_1) = f(x_2)$.
\end{lemma}
\begin{proof}
  By structural induction over $x_1$.
\end{proof}

\begin{corollary}
  \label{thm:inst-drop}
  Let $I, I_1, \dots, I_n$ by annotated instruction sequences such that
  $I_1 = I[x_1 / y_1], \dots, I_n = I_{n-1}[x_n, y_n]$. Then
  $\drop{I}{I} = \drop{I}{I_n}$.
\end{corollary}

\begin{lemma}
  \label{thm:codeeval-drop}
  Assume $G$ is an annotated global collection and $\evalbig{G}{w}{I}$. Then
  there exists some $\ell_g$ such that $\drop{w}{w} = \mathtt{globval}\ \ell_g$
  and $\lookup{\drop{G}{G}}{\ell_g}{\mathtt{code}\ \drop{I}{I}}$.
\end{lemma}
\begin{proof}
  By structural induction over the $w$ using \autoref{thm:inst-drop}.
\end{proof}

\begin{lemma}
  \label{thm:exec-drop}
  Assume $G, H, \dots$ are annotated values such that
  $\execinstruction{G}{H, R, I}{H', R', I'}$. Then
  $\execinstruction{\drop{G}{G}}{\drop{H}{H}, \drop{R}{R},
    \drop{I}{I}}{\drop{H}{H'}, \drop{R}{R'}, \drop{I}{I'}}$
\end{lemma}
\begin{proof}
  By structural induction over the derivation of
  $\execinstruction{G}{H, R, I}{H', R', I'}$ using \autoref{thm:regeval-drop}
  and \autoref{thm:codeeval-drop}.
\end{proof}

\begin{corollary}[Simulation]
  \label{thm:simulation}
  Let $\mathcal{R} = \{(P, \drop{P}{P}) : P \in \text{annotated
    programs}\}$. $\mathcal{R}$ is a simulation.
\end{corollary}

\begin{theorem}[Bisimulation]
  \label{thm:bisimulation}
  Let
  $\mathcal{R} = \{(P, \drop{P}{P}) : P \in \text{annotated programs}, \exists
  P' : P -> P'\}$. $\mathcal{R}$ is a bisimulation.
\end{theorem}
\begin{proof}
  We need to show that $\mathcal{R}$ and $\mathcal{R}^{-1}$ are simulations. The
  first part follows from \autoref{thm:exec-drop}.

  For the second part, assume that $(P_A, P_S) \in \mathcal{R}$ and
  $P_S -> P_S'$ for some $P_S'$. We need to show that there is a $P_A'$ such
  that $P_A -> P_A'$ and $(P_A', P_S') \in \mathcal{R}$.

  Let $P_A'$ be the witness of $\exists P' : P -> P'$ from the definition of
  $\mathcal{R}$. It remains to be shown that $(P_A', P_S') \in \mathcal{R}$,
  i.e. $\drop{P}{P_A'} = P_S'$.

  By \autoref{thm:exec-drop} we can get a derivation of
  $\drop{P}{P_A} -> \drop{P}{P_A'}$. By the definition of $\mathcal{R}$ we have
  $P_S = \drop{P}{P_A}$, so by an earlier assumption, we also have a derivation
  of $\drop{P}{P_A} -> P_S'$. By \autoref{thm:determinism}, this implies
  that $\drop{P}{P_A'} = P_S'$.
\end{proof}

% \begin{lemma}[Progress]
  \label{thm:progress}
  Assume $P_1$ is an annotated program such that $|- P_1$.
  Then we can either get a derivation of $\done{P_1}$ or there exists
  a $P_2$ such that $P_1 -> P_2$.
\end{lemma}

\begin{lemma}[Preservation]
  \label{thm:reduction}
  Assume $P_1, P_2$ are annotated programs such that $|- P_1$ and $P_1 ->
  P_2$. Then $|- P_2$.
\end{lemma}

\begin{lemma}[Soundness]
  \label{thm:soundness}
  Assume $P_1, \dots, P_n$ are annotated programs such that $|- P_1$ and
  $P_1 -> P_2, \dots, P_{n-1} -> P_n$. Then we can either get a derivation of
  $\done{P_n}$ or there exists a $P_{n+1}$ such that $P_n -> P_{n+1}$.
\end{lemma}
\begin{proof}
  Let $V(k)$ be the statement ``$k > n$ or there exists a derivation of
  $|- P_k$''. We use natural induction to show $V(k)$:

  \begin{itemize}
  \item $V(1)$: We have an assumption stating $|- P_1$, so we are done.
  \item $V(k) => V(k+1)$: If $k+1 > n$ we are done, so assume $k+1 \le n$. In
    that case $V(k)$ implies that we have a derivation of $|- P_k$. We also have
    a derivation of $P_k => P_{k+1}$ by assumption. We can now use the
    preservation lemma to get a derivation of $|- P_k$.
  \end{itemize}

  We can thus conclude that $|- P_n$. We use this on the progress lemma and are
  done.
\end{proof}

\begin{lemma}[Progress (simplified)]
  \label{thm:progress-simple}
  Let $P_1$ be an annotated program with $|- P_1$ and
  $P^\circ_1 = \drop{P}{P_1}$. Then we can either get a derivation of
  $\done{P^\circ_1}$ or there exists a $P^\circ_2$ such that
  $P^\circ_1 -> P^\circ_2$.
\end{lemma}
\begin{proof}
  We use the progress lemma on $|- P_1$, which results in two cases:

  \begin{itemize}
  \item Either we have a derivation of $\done{P_1}$. In that case
    $P_1 = (G, H, R, \mathtt{halt})$, so
    $P^\circ_1 = (\drop{G}{G}, \drop{H}{H}, \drop{R}{R}, \mathtt{halt})$. Thus
    we can construct a derivation of $\done{P^\circ_1}$.
  \item Alternatively we have a derivation of $P_1 -> P_2$ for some $P_2$. We
    see that $(P_1, P^\circ_1) = (P_1, \drop{P}{P_1}) \in \mathcal{R}$ from
    \autoref{thm:simulation}. Since $\mathcal{R}$ is a simulation, we have the
    desired result.
  \end{itemize}
\end{proof}

\begin{lemma}[Preservation (simplified)]
  \label{thm:reduction-simple}
  Let $P_1$ be an annotated program with $|- P_1$ and $P^\circ_1 = \drop{P}{P}$.
  Assume that $P^\circ_1 -> P^\circ_2$. Then there is some annotated program
  $P_2$ with $|- P_2$ and $P^\circ_2 = \drop{P}{P_2}$.
\end{lemma}
\begin{proof}
  We use the progress lemma on $|- P_1$, which results in two cases.

  We cannot have a derivation of $\done{P_1}$, as this would imply a
  contraction. The derivation would imply that $P_1 = (G, H, R, \mathtt{halt})$,
  which means that
  $P^\circ_1 = (\drop{G}{G}, \drop{H}{H}, \drop{R}{R}, \mathtt{halt})$. However
  this is not possible since $P^\circ_1 -> P^\circ_2$.

  So we conclude that there is some $P_t$ such that $P_1 -> P__t$. This means
  that $(P_1, P^\circ_1) = (P_1, \drop{P}{P_1}) \in \mathcal{R}$ from
  \autoref{thm:simulation}, by using $P_t$ as the witness. Since $\mathcal{R}$
  is a bisimulation, we there is a $P_2$ such that $P_1 -> P_2$ and
  $(P_2, P^\circ_2) \in \mathcal{R}$, which implies that
  $P^\circ_2 = \drop{P}{P_2}$.\footnote{It happens to be the case that
    $P_2 = P_t$ because both languages are deterministic, but our proof does not
    depend on this fact directly.}

  It just remains to be shown that $|- P_2$, however this follows directly from
  the preservation lemma.
  \end{itemize}
\end{proof}

\begin{theorem}[Soundness (simplified)]
  \label{thm:soundness-simple}
  Let $P_1$ be an annotated program with $|- P_1$ and let
  $P^\circ_1, \dots, P^\circ_n$ be simplified program such that
  $P^\circ_1 = \drop{P}{P_1}, P^\circ_1 -> P^\circ_2, \dots, P^\circ_{n-1} ->
  P^\circ_n$.
  Then we can either get a derivation of $\done{P^\circ_n}$ or there is some
  $P^\circ_{n+1}$ such that $P^\circ_n -> P^\circ_{n+1}$.
\end{theorem}
\begin{proof}
  Let $V(k)$ be the statement ``$k > n$ or there exists a $P_k$ such with
  $P^\circ_k = \drop{P}{P_k}$ and $|- P_k$''. We use natural induction to show
  $V(k)$:

  \begin{itemize}
  \item $V(1)$: Let $P = P_1$ and the result follows from the assumptions.
  \item $V(k) => V(k+1)$: If $k+1>n$ we are done, so assume $k \le n$. $V(k)$
    now implies that there is some $P_k$ such that $|- P_k$ and
    $P^\circ_k = \drop{P}{P_k}$. We also have a derivation of
    $P^\circ_k -> P^\circ_{k+1}$ by assumption. By applying the simplified
    preservation lemma, we get the result.
  \end{itemize}

  By $V(n)$ we can thus conclude that we have $P^\circ_n = \drop{P}{P_n}$ and
  $|- P_n$ for some $P_n$. We use this on the simplified progress lemma and get
  the desired result.
\end{proof}


% \chapter{Lessons Learned: Tips for doing formalizations in Agda}
% \section{Decidable Equality}

It is a common occurrence in Agda to want to prove decidable equality for a
specific type $T$. Decidable equality is a lemma of the following form:

\begin{lemma}[Decidable Equality]
  Let $v_1, v_2 \in T$. We can now derive a proof of either $v_1 \equiv v_2$ or
  $v_1 \not\equiv v_2$.
\end{lemma}

This lemma is in Agda often written as a dependent function with the signature:

$$\cdot ≟ \cdot\;\;:\;\;(v_1 : T) \to (v_2 : T) \to \mathtt{Dec}(v_1 \equiv v_2)$$

An example from Agda standard library (with slight modifications):

\begin{code}%
\>\<%
\\
\>\AgdaFunction{\_≟\_} \AgdaSymbol{:} \AgdaSymbol{(}\AgdaBound{v₁} \AgdaSymbol{:} \AgdaDatatype{Bool}\AgdaSymbol{)} \AgdaSymbol{→} \AgdaSymbol{(}\AgdaBound{v₂} \AgdaSymbol{:} \AgdaDatatype{Bool}\AgdaSymbol{)} \AgdaSymbol{→} \AgdaDatatype{Dec} \AgdaSymbol{(}\AgdaBound{v₁} \AgdaDatatype{≡} \AgdaBound{v₂}\AgdaSymbol{)}\<%
\\
\>\AgdaInductiveConstructor{true} \<[6]%
\>[6]\AgdaFunction{≟} \AgdaInductiveConstructor{true} \<[14]%
\>[14]\AgdaSymbol{=} \AgdaInductiveConstructor{yes} \AgdaInductiveConstructor{refl}\<%
\\
\>\AgdaInductiveConstructor{false} \AgdaFunction{≟} \AgdaInductiveConstructor{false} \AgdaSymbol{=} \AgdaInductiveConstructor{yes} \AgdaInductiveConstructor{refl}\<%
\\
\>\AgdaInductiveConstructor{true} \<[6]%
\>[6]\AgdaFunction{≟} \AgdaInductiveConstructor{false} \AgdaSymbol{=} \AgdaInductiveConstructor{no} \AgdaSymbol{λ()}\<%
\\
\>\AgdaInductiveConstructor{false} \AgdaFunction{≟} \AgdaInductiveConstructor{true} \<[14]%
\>[14]\AgdaSymbol{=} \AgdaInductiveConstructor{no} \AgdaSymbol{λ()}\<%
\end{code}

\paragraph{}
While these types of lemmas are in principle trivial to generate for most
datatypes, there is a problem: required number of lines is \emph{quadratic} in
the number of constructors for the datatype. The reason for this quadratic
explosion is the need to the proofs for $v_1 \neq v_2$.

I found a solution to the problem in \cite{deceq}, though I generalized his
solution slightly. The idea is a follows:

\begin{itemize}
\item Construct an injective function $f$ from the original type $T$ to another
  type $T'$.
\item Prove decidable equality for $T'$.
\item Use this to decide $f(v_1) \equiv f(v_2)$.
\item We know that $f(v_1) \not\equiv f(v_2)$ implies $v_1 \not\equiv v_2$,
  since $f$ is a function.
\item We also know that $f(v_1) \equiv f(v_2)$ implies $v_1 \equiv v_2$, since
  $f$ is injective.
\item If we combine these, we get decidable equality for $T$.
\end{itemize}

By choosing $T'$ to be the same for every $T$, we can reuse a lot code. In
practice my code does the following:

\begin{itemize}
\item Define a type \texttt{Tree}, which is the set of rooted, ordered tree with
  numbers attached to each node. This datatype was chosen since this makes it
  easy to construct the injective functions.
\item Prove decidable equality for \texttt{Tree}.
\item Prove that given $T$ and an injective functions $f : T \to \mathtt{Tree}$,
  we have decidable equality for $T$.
\item Prove that given $T$ and a surjective function $g : \mathtt{Tree} \to T$,
  we can construct an injective function $f : T \to \mathtt{Tree}$, and thus
  this also implies decidable equality on $T$.
\end{itemize}

This last step is a slight further optimization, as it in practice takes fewer
lines of code in Agda to specify a function $g : \mathtt{Tree} \to T$ and prove
it surjective, than to specify a function $f : T \to \mathtt{Tree}$ and prove it
injective.

For each additional type $T$ for which you want to prove decidable equality, you
now only need to:

\begin{itemize}
\item Create a function $g : \mathtt{Tree} \to T$.
\item Prove that $g$ is surjective.
\end{itemize}

Unless we do wrapping of long lines, this will only require $2k+3$ lines, where
$k$ is the number of number of constructors in the definition of $T$.

\later[inline]{Insert reference to appendix, where usage of this can be seen.}

% \section{Extensions to Equality Reasoning}
Not that smart or interesting. One would rather try to avoid equality reasoning,
but if that is not possible, one might as well have as strong a hacksaw as
possible.

\later[inline]{Write about equality reasoning}

% \section{Substitutions}
I think I should be able to same something semi-smart about this.

\later[inline]{Write about substitutions}


% \chapter{Judgments}
% \section{Grammar}
\label{sec:grammar}

\begin{tabular}{lrcl}
$types$ & $\tau$ & ::= & $\alpha \mid \mathtt{int} \mid \mathtt{ns} \mid \mathtt\forall[ \Delta ] \Gamma \mid \langle\tau_1^{\phi_1},\dots,\tau_n^{\phi_n}\rangle$ \\
$stack\ types$ & $\sigma$ & ::= & $\rho \mid \mathtt{nil} \mid \tau :: \sigma$ \\
$initialization\ flags$ & $\phi$ & ::= & $\mathtt{init} \mid \mathtt{uninit}$ \\
$type\ assignments$ & $\Delta$ & ::= & $\mathtt{nil} \mid a :: \Delta$ \\
$type\ assignment\ value$ & $a$ & ::= & $\alpha \mid \rho$ \\
$global\ label\ assignments$ & $\Psi_g$ & ::= & $\{\ell_{g,1}: \tau_1,\dots,\ell_{g,n}:\tau_n\}$ \\
$heap\ label\ assignmentss$ & $\Psi_h$ & ::= & $\{\ell_{h,1}: \tau_1,\dots,\ell_{h,n}:\tau_n\}$ \\
$label\ assignments$ & $\Psi$ & ::= & $(\Psi_g , \Psi_h)$ \\
$register\ assignments$ & $\Gamma$ & ::= & $\{\mathtt{sp} : \sigma, \mathtt{r}_1: \tau_1, \dots, \mathtt{r}_{regs}: \tau_{regs}\}$ \\
\\
$registers$ & $r$ & ::= & $\mathtt{r}_1 \mid \dots \mid \mathtt{r}_{regs}$ \\
$casts$ & $c$ & ::= & $\mathtt{\alpha}\Rightarrow\tau \mid \rho\Rightarrow\sigma \mid \mathtt{weaken}\ \Delta$ \\
$word\ values$ & $w$ & ::= & $\mathtt{globval}\ \ell_g \mid \mathtt{heapval}\ \ell_h \mid \mathtt{int}\ i \mid \mathtt{ns} \mid \mathtt{uninit}\ \tau \mid w[c]$ \\
$small\ values$ & $v$ & ::= & $\mathtt{reg}\ \mathtt{r}_k \mid \mathtt{word}\ w \mid v[c]$ \\
$global\ values$ & $g$ & ::= & $\mathtt{code}[\Delta]\Gamma.I$ \\
$globals$ & $G$ & ::= & $\{\ell_{g,1}\mapsto g_1, \dots, \ell_{g,n} \mapsto g_n\}$ \\
$heap\ values$ & $h$ & ::= & $\langle w_1, \dots, w_n \rangle$ \\
$heaps$ & $H$ & ::= & $\{\ell_{h,1}\mapsto h_1, \dots, \ell_{h,n} \mapsto h_n\}$ \\
$stacks$ & $S$ & ::= & $\mathtt{nil} \mid w :: S$ \\
$register\ files$ & $R$ & ::= & $\{sp \mapsto S, r_1 \mapsto w_1, \dots, r_{regs} \mapsto w_{regs}\}$ \\
\\
$instructions$ & $\iota$ & ::= & $\mathtt{add}\ \mathtt{r}_d, \mathtt{r}_s, v \mid \mathtt{sub}\ \mathtt{r}_d, \mathtt{r}_s, v \mid$ \\
        &&& $\mathtt{push}\ v \mid \mathtt{pop} \mid$ \\
        &&& $\mathtt{ld}\ \mathtt{r}_d, \mathtt{sp}, i \mid \mathtt{st}\ \mathtt{sp}, i, \mathtt{r}_s \mid$\\
        &&& $\mathtt{ld}\ \mathtt{r}_d, \mathtt{r}_s, i \mid \mathtt{st}\ \mathtt{r}_d, i, \mathtt{r}_s \mid$\\
        &&& $\mathtt{malloc}\ \mathtt{r}_d,\ \langle \tau_1, \dots, \tau_n \rangle \mid $ \\
        &&& $\mathtt{mov}\ \mathtt{r}_d, \mathtt{r}_s \mid \mathtt{beq}\ \mathtt{r}_k, v$ \\
$instruction\ sequences$ & $I$ & ::= & $\iota ; I \mid \mathtt{jmp}\ v$ \\
$program\ states$ & $P$ & ::= & $(H, R, I)$ \\
\end{tabular}

\begin{definition}
  The above set of datatypes are called the \textbf{core datatypes} and referred
  to by $\mathbb{C}$.
\end{definition}

\subsection{Comments}
In the previous, $regs$ is a constant. Its value of the number of registers on
the machine in question. The concrete value of $regs$ does not matter for any of
the proofs, but in my code, I set the value to $4$.

The references $\alpha$ and $\rho$ are implemented using De Bruijn indexes. This
is relevant both in $types$, $stack\ types$ and in $casts$, where they are
used. It is also relevant in the $\mathtt{weaken}$ case of $casts$, as we need
to know \emph{where} to weaken.

Note also that for technical reasons, a lot of the constructors in Agda does not
actually look like this. For instance we use $\iota \sim> I$ instead of
$\iota ; I$ and $v \llbracket c \rrbracket$ instead of $v [ c ]$. We also use
$\mathtt{sld}\ r_d, i$ and $\mathtt{sst}\ i, r_s$ instead of
$\mathtt{ld}\ r_d, \mathtt{sp}, i$ and $\mathtt{st}\ \mathtt{sp}, i, r_s$.

Note also that in the Agda-code, the $\mathtt{globval}\ l_{g,k}$ of $w$ also
includes the number of assumptions used in the corresponding code global. This
would ideally be removed later.

\subsection{Differences from STAL}

\begin{itemize}
\item No existential types
\item No stack-pointer types
\item No compound stacks
\item The original paper has an unlimited number of registers, with only a
  finite amount used at any time. We have a fixed number.
\item The heap has been split up into the \texttt{Globals}, containing immutable
  code and, and \texttt{Heap} containing only mutable tuples.
\item The instructions are slightly different -- for instance there is no \texttt{salloc}/\texttt{sfree}, only \texttt{push}/\texttt{pop}.
\item The small step semanics will not update the globals.
\end{itemize}

% \documentclass{article}
\usepackage[a4paper, hmargin={2.8cm, 2.8cm}, vmargin={2.8cm, 2.8cm}]{geometry}
\usepackage[utf8]{inputenc}
\usepackage[english]{babel}
\usepackage{listingsutf8}
\usepackage{amsmath, amssymb}
\usepackage{mathtools}
\usepackage{mathpartir}

\newcommand \isctx[1] {{#1}\ \ \mathbf{ctx}}
\newcommand \isstack[2] {{#1} \vdash {#2}\ \ \mathbf{stack}}
\newcommand \islifetime[3] {{#1}, {#2} \vdash {#3}\ \ \mathbf{lifetime}}
\newcommand \issub[4]{{#1}, {#2} \vdash {#3} \leq {#4}}
\newcommand \istype[3] {{#1}, {#2} \vdash {#3}\ \ \mathbf{type}}
\newcommand \istypen[4] {{#1}, {#2} \vdash {#3}\ \ \mathbf{type}_{#4}}
\newcommand \isregister[2] {{#1} \vdash {#2}\ \ \mathbf{register}}
\newcommand \isqualifier[1] {{#1}\ \ \mathbf{qualifer}}
\newcommand \ownedptr[1] {\sim {#1}}
\newcommand \refptr[3] {\&\,{#1}\ {#2}\ {#3}}
\newcommand \nil {\mathtt{nil}}
\newcommand \cons {\dblcolon}
\newcommand \static {\mathtt{static}}
\newcommand \intt {\mathtt{int}}
\newcommand \void[1] {\mathtt{void}_{#1}}
\newcommand \mut {\mathtt{mut}}
\newcommand \imm {\mathtt{imm}}
\newcommand{\xmid}{\;\mid\;}
\newcommand{\defeq}{\;\Coloneqq\;\;}
\begin{document}

(This is still leaving out initialization-state for simplicity)

\fbox{$\isctx{\Delta}$}
\begin{mathpar}
\infer{ }{\isctx{\cdot}} \and
\infer{\isctx{\Delta}}{\isctx{\Delta, \rho}} \and
\infer{\isctx{\Delta} \and \isstack{\Delta}{\sigma}}{\isctx{\Delta, \alpha_{n,\sigma}}} \and
\infer{\isctx{\Delta} \and \isstack{\Delta}{\sigma}}{\isctx{\Delta, \ell_\sigma}} \and
\infer{\isctx{\Delta} \and \islifetime{\Delta}{\sigma}{\ell_1} \and \islifetime{\Delta}{\sigma}{\ell_2}}{\isctx{\Delta, \ell_1 \leq_\sigma \ell_2}}
\end{mathpar}

\fbox{$\isstack{\Delta}{\sigma}$}
\begin{mathpar}
\infer{ }{\isstack{\Delta}{\nil}} \and
\infer{\rho \in \Delta}{\isstack{\Delta}{\rho}} \and
\infer{\isstack{\Delta}{\sigma} \and \istype{\Delta}{\sigma}{\tau}}{\isstack{\Delta}{\tau \cons \sigma}} \and
\infer{\isstack{\Delta}{\sigma}}{\isstack{\Delta}{\ell \cons \sigma}}
\end{mathpar}

\fbox{$\islifetime{\Delta}{\sigma}{\ell}$}
\begin{mathpar}
\infer{\islifetime{\Delta}{\sigma}{\ell}}{\islifetime{\Delta}{\_ \cons \sigma}{\ell}} \and
\infer{\ell_\sigma \in \Delta}{\islifetime{\Delta}{\sigma}{\ell_\sigma}} \and
\infer{ }{\islifetime{\Delta}{\ell \cons \sigma}{\ell}} \and
\infer{ }{\islifetime{\Delta}{\sigma}{\static}}
\end{mathpar}

\fbox{$\istype{\Delta}{\sigma}{\tau}$}
\begin{mathpar}
\infer{\istypen{\Delta}{\sigma}{\tau}{n}}{\istype{\Delta}{\sigma}{\tau}}
\end{mathpar}

\fbox{$\istypen{\Delta}{\sigma}{\tau}{n}$}
\begin{mathpar}
\infer{\istypen{\Delta}{\sigma}{\tau}{n}}{\istypen{\Delta}{\_ \cons \sigma}{\tau}{n}} \and
\infer{\alpha_{n,\sigma} \in \Delta}
      {\istypen{\Delta}{\sigma}{\alpha_{n,\sigma}}{n}} \and
\infer{ }
      {\istypen{\Delta}{\sigma}{\intt}{4}} \and
\infer{ }
      {\istypen{\Delta}{\sigma}{\void{n}}{n}} \and
\infer{\istype{\Delta}{\sigma}{\tau}}
      {\istypen{\Delta}{\sigma}{\ownedptr{\tau}}{4}} \and
\infer{\islifetime{\Delta}{\sigma}{\ell}
       \and \isqualifier{q}
       \and \istype{\Delta}{\sigma}{\tau}}
      {\istypen{\Delta}{\sigma}{\refptr{\ell}{q}{\tau}}{4}} \and
\infer{\istypen{\Delta}{\sigma}{\tau_1}{n_1}
       \and \dots
       \and \istypen{\Delta}{\sigma}{\tau_k}{n_k}}
      {\istypen{\Delta}{\sigma}{[\tau_i]_{i \in \{1 \dots k\}}}{n_1 + \dots + n_k}} \and
\infer{\istypen{\Delta}{\sigma}{\tau_1}{n}
       \and \dots
       \and \istypen{\Delta}{\sigma}{\tau_k}{n}}
      {\istypen{\Delta}{\sigma}{\langle\tau_i\rangle_{i \in \{1 \dots k\}}}{4 + n}} \and
\infer{\isregister{\Delta + \Delta'}{\Gamma}}
      {\istypen{\Delta}{\sigma}{\forall[\Delta']\Gamma}{4}}
\end{mathpar}

\fbox{$\isregister{\Delta}{\Gamma}$}
\begin{mathpar}
\infer{\isstack{\Delta}{\sigma}
       \and \istypen{\Delta}{\sigma}{\tau_1}{4}
       \and \dots
       \and \istypen{\Delta}{\sigma}{\tau_k}{4}}
      {\isregister{\Delta}{\{r_1:\tau_1, \dots, r_k:\tau_k, \mathtt{sp}:\sigma\}}}
\end{mathpar}

\fbox{$\issub{\Delta}{\sigma}{\ell_1}{\ell_2}$}
\begin{mathpar}
\infer{\issub{\Delta}{\sigma}{\ell_1}{\ell_2}}
      {\issub{\Delta}{\_ \cons \sigma}{\ell_1}{\ell_2}} \and
\infer{\ell_1 \leq_\sigma \ell_2 \in \Delta}
      {\issub{\Delta}{\sigma}{\ell_1}{\ell_2}} \and
\infer{ \islifetime{\Delta}{\ell_1 \cons \sigma}{\ell_2}}
      {\issub{\Delta}{\ell_1 \cons \sigma}{\ell_1}{\ell_2}} \and
\infer{ \islifetime{\Delta}{\sigma}{\ell}}
      {\issub{\Delta}{\sigma}{\ell}{\static}} \and
\end{mathpar}

\fbox{$\isqualifier{q}$}
\begin{mathpar}
\infer{ }{\isqualifier{\mut}} \and
\infer{ }{\isqualifier{\imm}}
\end{mathpar}
\end{document}

% \section{Subtyping judgments}

\newcommand \subtype[3] {{#1} |- {#2} \le {#3}}

\fbox{$\subtype{\Delta}{\tau_1}{\tau_2}$}
\begin{mathpar}
\infer{\alpha \in \Delta}{\subtype{\Delta}{\alpha}{\alpha}} \and
\infer{ }{\subtype{\Delta}{\mathtt{int}}{\mathtt{int}}} \and
\infer{ }{\subtype{\Delta}{\mathtt{ns}}{\mathtt{ns}}} \and
\infer
  {\valid{\Delta}{\Delta'} \and
    \subtype{\Delta', \Delta}{\Gamma_1}{\Gamma_2}}
  {\subtype{\Delta}{\forall[ \Delta' ]\Gamma_1}{\forall[ \Delta' ]\Gamma_2}} \and
\infer
  {\subtype{\Delta}{\tau_1^{\phi_1}}{\tau'_1{}^{\phi'_1}} \and
    \dots \and
    \subtype{\Delta}{\tau_n^{\phi_n}}{\tau'_n{}^{\phi'_n}}}
  {\subtype{\Delta}{\langle \tau'_1{}^{\phi_1}, \dots, \tau_n{}^{\phi_n} \rangle}
                   {\langle \tau'_1{}^{\phi'_1}, \dots, \tau'_n{}^{\phi'_n} \rangle}}
\end{mathpar}

\fbox{$\subtype{}{\phi_1}{\phi_2}$}
\begin{mathpar}
\infer{ }{\subtype{}{\mathtt{init}}{\mathtt{init}}} \and
\infer{ }{\subtype{}{\mathtt{\phi}}{\mathtt{uninit}}}
\end{mathpar}

\fbox{$\subtype{\Delta}{\tau_1^{\phi_1}}{\tau_2^{\phi_2}}$}
\begin{mathpar}
\infer{\valid{\Delta}{\tau} \and \subtype{}{\phi_1}{\phi_2}}{\subtype{\Delta}{\tau^{\phi_1}}{\tau^{\phi_2}}}
\end{mathpar}

% \section{Substitution judgments}

\fbox{$\substitution{\tau_1}{c}{\tau_2}$}
\begin{mathpar}
  \infer{ }{\substitution{\alpha}{\alpha => \tau}{\tau}} \and
  \infer{\alpha_1 \neq \alpha_2}{\substitution{\alpha_1}{\alpha_2 => \tau}{\alpha_1}} \and
  \infer{ }{\substitution{\alpha}{\rho => \sigma}{\alpha}} \and
  \infer{ }{\substitution{\alpha}{\mathtt{weaken}\ \Delta^{+}}{\alpha}}
\end{mathpar}

(This also includes obvious, mutually recursive definition.)\\\\
\fbox{$\substitution{\tau_1^{\phi_1}}{c}{\tau_2^{\phi_2}}$}\\

(The obvious, mutually recursive definitions.)\\\\
\fbox{$\substitution{\sigma_1}{c}{\sigma_2}$}
\begin{mathpar}
  \infer{ }{\substitution{\rho}{\rho => \sigma}{\sigma}} \and
  \infer{\rho_1 \neq \rho_2}{\substitution{\rho_1}{\rho_2 => \sigma}{\rho_1}} \and
  \infer{ }{\substitution{\rho}{\alpha => \tau}{\rho}} \and
  \infer{ }{\substitution{\rho}{\mathtt{weaken}\ \Delta^{+}}{\rho}}
\end{mathpar}

(This also includes obvious, mutually recursive definition.)\\\\
\fbox{$\substitution{\Delta_1}{c}{\Delta_2}$}\\

(The obvious, mutually recursive definitions.)\\\\
\fbox{$\substitution{a_1}{c}{a_2}$}\\

(The obvious, mutually recursive definitions.)\\\\
\fbox{$\substitution{\Gamma_1}{c}{\Gamma_2}$}\\
\begin{mathpar}
  \infer{ }{\substitution{\mathtt{nil}}{c}{\mathtt{nil}}} \and
  \infer{
    \substitution{a_1}{c}{a_2} \and
    \substitution{\Delta_1}{c}{\Delta_2}
  }{
    \substitution{(a_1 :: \Delta_1)}{c}{a_2 :: \Delta_2}
  }
\end{mathpar}\\
\fbox{$\substitution{c_1}{c}{c_2}$}\\

(The obvious, mutually recursive definitions.)\\\\
\fbox{$\substitution{w_1}{c}{w_2}$}\\

(The obvious, mutually recursive definitions.)\\\\
\fbox{$\substitution{v_1}{c}{v_2}$}\\

(The obvious, mutually recursive definitions.)\\\\
\fbox{$\substitution{\iota_1}{c}{\iota_2}$}\\

(The obvious, mutually recursive definitions.)\\\\
\fbox{$\substitution{I_1}{c}{I_2}$}\\

(The obvious, mutually recursive definitions.)


\begin{definition}
  Let $T \in \mathbb{C}$ and $v_1, v_2 \in T$ and assume we have some $c$. We define
  $\mathbb{SUB}$ to be the subset of $\mathbb{C}$ with the properties:
  \begin{itemize}
  \item $\substitution{v_1}{c}{v_2}$ is a meaningful expression iff $T \in \mathbb{SUB}$.
  \end{itemize}
\end{definition}

\subsection{Comments}

I write that the rules are obvious -- and they are, as long as you keep yourself
to using paper-notation. If you want to use De Bruijn indices they are not
nearly as obvious.

Note that the substitution for $\Delta$ assumes that the actual substitution is
to be done elsewhere -- it merely updates all references inside the $\Delta$. In
the current grammar there \emph{are} no references inside the assumptions, so we
get that $\substitution{\Delta}{c}{\Delta}$ for all $\Delta$. This would ideally
not hold in the final version.

For the substitution that actually updates the assumption list, see the next
section.

\subsection{Differences from STAL}

STAL handwaves even more that I do here. Rest assured that my Agda-code is
precise enough for a machine-verified proof.

% \section{Running judments}

\fbox{$\run{\Delta_1}{c}{\Delta_2}$}\\

\begin{mathpar}
\infer{ }{\run{(\alpha :: \Delta)}{\alpha => \tau}{\Delta}} \and
\infer{ }{\run{(\rho :: \Delta)}{\rho => \sigma}{\Delta}} \and
\infer{
  a_1 \neq a \and
  \substitution{a_1}{a => v}{a_2} \and
  \substitution{\Delta_1}{a => v}{\Delta_2}
}{
  \run{(a_1 :: \Delta_1)}{a => v}{(a_2 :: \Delta_2)}
} \and
\infer{
  \text{$\Delta^{+}$ should be inserted elsewhere} \and
  \substitution{a_1}{\mathtt{weaken}\ \Delta^{+}}{a_2} \and
  \substitution{\Delta_1}{\mathtt{weaken}\ \Delta^{+}}{\Delta_2}
}{
  \run{(a_1 :: \Delta_1)}{\mathtt{weaken}\ \Delta^{+}}{(a_2 :: \Delta_2)}
} \and
\infer{
  \text{$\Delta^{+}$ should be inserted here}
}{
  \run{\Delta}{\mathtt{weaken}\ \Delta^{+}}{(\Delta^{+}, \Delta)}
}
\end{mathpar}

\subsection{Comments}

We again have a bit of imprecision/handwaving because the actual Agda-code deals
with De Bruijn indices, and the paper notation deals with variable-names.

% \section{Term judgments}

Judgments that values are valid or of a specific type.

\fbox{$\ofglobs{G}{\Psi_g}$}
\begin{mathpar}
\infer{
  \ofgval{\Psi_g}{g_1}{\tau_1} \and
  \dots \and
  \ofgval{\Psi_g}{g_n}{\tau_n} \and
  \Psi_g = \{\ell_{g,1}:\tau_1, \dots, \ell_{g,n}:\tau_n\}
}{
  \ofglobs{\{\ell_{g,1}\mapsto g_1, \dots, \ell_{g,n} \mapsto g_n\}}{\Psi_g}
}
\end{mathpar}

\fbox{$\ofheap{\Psi_g}{H}{\Psi_h}$}
\begin{mathpar}
\infer{
  \ofhval{\Psi_g,\Psi_h}{h_1}{\tau_1} \and
  \dots \and
  \ofhval{\Psi_g,\Psi_h}{h_n}{\tau_n} \and
  \Psi_h = \{\ell_{h,1}:\tau_1, \dots, \ell_{h,n}:\tau_n\}
}{
  \ofheap{\Psi_g}{\{\ell_{h,1}\mapsto h_1, \dots, \ell_{h,n} \mapsto h_n\}}{\Psi_h}
}
\end{mathpar}

\fbox{$\ofstack{\Psi_g,\Psi_h}{S}{\sigma}$}
\begin{mathpar}
\infer{ }{\ofstack{\Psi_g,\Psi_h}{\mathtt{nil}}{\mathtt{nil}}} \and
\infer{
  \ofword{\Psi_g,\Psi_h}{w}{\tau} \and
  \ofstack{\Psi_g,\Psi_h}{S}{\sigma}
}{
  \ofstack{\Psi_g,\Psi_h}{w :: S}{\tau :: \sigma}
}
\end{mathpar}

\fbox{$\ofregister{\Psi}{R}{\Gamma}$}
\begin{mathpar}
\infer{
  \ofstack{\Psi_g,\Psi_h}{S}{\sigma} \and
  \ofwval{\Psi_g,\Psi_h,\mathtt{nil}}{w_1}{\tau_1} \and
  \dots
  \ofwval{\Psi_g,\Psi_h,\mathtt{nil}}{w_{regs}}{\tau_{regs}} \and
}{
  \ofregister{\Psi_g,\Psi_h}{\{\mathtt{sp}\mapsto S, \mathtt{r}_1\mapsto w_1, \dots, \mathtt{r}_{regs}\mapsto w_{regs}\}}{\{\mathtt{sp}:\sigma, \mathtt{r}_1:\tau_1, \dots, \mathtt{r}_{regs}:\tau_{regs}\}}
}
\end{mathpar}

\fbox{$\ofgval{\Psi_g}{g}{\tau}$}
\begin{mathpar}
\infer{
  \valid{\mathtt{nil}}{\Delta} \and
  \valid{\Delta}{\Gamma} \and
  \ofinstructions{\Psi_g, \Delta, \Gamma}{I}
}{
  \ofgval{\Psi_g}{\mathtt{code}[\Delta]\Gamma.I}{\forall[\Delta]\Gamma}
}
\end{mathpar}

\fbox{$\ofhval{\Psi_g,\Psi_h}{h}{\tau}$}
\begin{mathpar}
\infer{
  \ofwvaln{\Psi_g,\Psi_h}{w_1}{\tau_1^{\phi_1}} \and
  \dots \and
  \ofwvaln{\Psi_g,\Psi_h}{w_n}{\tau_n^{\phi_n}}
}{
  \ofgval{\Psi_g,\Psi_h}{\langle w_1, \dots, w_n \rangle}{\langle \tau_1^{\phi_1}, \dots, \tau_n^{\phi_n}\rangle}
}
\end{mathpar}

\fbox{$\ofwval{\Psi_g,\Psi_h,\Delta}{w}{\tau}$}
\begin{mathpar}
\infer{
  (\ell_g: \tau_1) \in \Psi_g \and
  \subtype{\mathtt{nil}}{\tau_1}{\tau_2}
}{
  \ofwval{\Psi_g,\Psi_h,\Delta}{\mathtt{globval}\ \ell_g}{\tau_2}
}\and
\infer{
  (\ell_h: \tau_1) \in \Psi_h \and
  \subtype{\mathtt{nil}}{\tau_1}{\tau_2}
}{
  \ofwval{\Psi_g,\Psi_h,\Delta}{\mathtt{heapval}\ \ell_h}{\tau_2}
}\and
\infer{ }{\ofwval{\Psi_g,\Psi_h,\Delta}{\mathtt{int}\ i}{\mathtt{int}}} \and
\infer{ }{\ofwval{\Psi_g,\Psi_h,\Delta}{\mathtt{ns}}{\mathtt{ns}}} \and
\infer{
  \ofwval{\Psi_g,\Psi_h,\Delta}{w}{\forall[\Delta_1]\Gamma_1} \and
  \valid{\Delta_1,\Delta}{c} \and
  \run{\Delta_1}{c}{\Delta_2} \\
  \substitution{\Gamma_1}{c}{\Gamma_2} \and
  \subtype{\Delta_2,\Delta}{\Gamma_2}{\Gamma_3}
}{
  \ofwval{\Psi_g,\Psi_h,\Delta}{w \llbracket c \rrbracket}{\forall[\Delta_2]\Gamma_3}
}
\end{mathpar}

\fbox{$\ofwvaln{\Psi_g,\Psi_h}{w}{\tau^\phi}$}
\begin{mathpar}
\infer{
  \valid{\mathtt{nil}}{\tau}
}{
  \ofwvaln{\Psi_g,\Psi_h}{\mathtt{uninit}\ \tau}{\tau^{\mathtt{uninit}}}
} \and
\infer{
  \ofwval{\Psi_g,\Psi_h,\mathtt{nil}}{w}{\tau}
}{
  \ofwvaln{\Psi_g,\Psi_h}{w}{\tau^{\mathtt{init}}}
}
\end{mathpar}

\fbox{$\ofvval{\Psi_g,\Delta,\Gamma}{v}{\tau}$}
\begin{mathpar}
\infer{
  (\mathtt{r}_k:\tau_k) \in \Gamma \and
  \subtype{\Delta}{\tau_k}{\tau'_k}
}{
  \ofvval{\Psi_g,\Delta,\Gamma}{\mathtt{reg}\ \mathtt{r}_k}{\tau'_k}
} \and
\infer{
  \ofwval{\Psi_g,\{\},\Delta}{w}{\tau}
}{
  \ofvval{\Psi_g,\Delta,\Gamma}{\mathtt{word}\ w}{\tau}
} \and
\infer{
  \ofvval{\Psi_g,\Delta,\Gamma}{w}{\forall[\Delta_1]\Gamma_1} \and
  \valid{\Delta_1,\Delta}{c} \and
  \run{\Delta_1}{c}{\Delta_2} \\
  \substitution{\Gamma_1}{c}{\Gamma_2} \and
  \subtype{\Delta_2,\Delta}{\Gamma_2}{\Gamma_3}
}{
  \ofvval{\Psi_g,\Delta,\Gamma}{w \llbracket c \rrbracket}{\forall[\Delta_2]\Gamma_3}
}
\end{mathpar}

\fbox{$\ofinstructions{\Psi_g, \Delta, \Gamma}{I}$}
\begin{mathpar}
\infer{
  \ofinstruction{\Psi_g, \Delta, \Gamma}{\iota}{\Gamma'} \and
  \ofinstructions{\Psi_g, \Delta, \Gamma'}{I}
}{
  \ofinstructions{\Psi_g, \Delta, \Gamma}{\iota ; I}
} \and
\infer{
  \ofvval{\Psi_1, \Delta, \Gamma}{v}{\forall[ \mathtt{nil} ]\Gamma'} \and
  \subtype{\Delta}{\Gamma}{\Gamma'}
}{
  \ofinstructions{\Psi_g, \Delta, \Gamma}{\mathtt{jmp}\ v}
}
\end{mathpar}

\fbox{$\ofinstruction{\Psi_g, \Delta, \Gamma}{\iota}{\Gamma'}$}
\begin{mathpar}
\infer{
  (\mathtt{r}_s : \mathtt{int}) \in \Gamma \and
  \ofvval{\Psi_g, \Delta, \Gamma}{v}{\mathtt{int}}
}{
  \ofinstruction{\Psi_g, \Delta, \Gamma}
    {\mathtt{add}\ \mathtt{r}_d, \mathtt{r}_s, v}
    {\Gamma\{\mathtt{r}_d:\mathtt{int}\}}
} \and
\infer{
  (\mathtt{r}_s : \mathtt{int}) \in \Gamma \and
  \ofvval{\Psi_g, \Delta, \Gamma}{v}{\mathtt{int}}
}{
  \ofinstruction{\Psi_g, \Delta, \Gamma}
    {\mathtt{sub}\ \mathtt{r}_d, \mathtt{r}_s, v}
    {\Gamma\{\mathtt{r}_d:\mathtt{int}\}}
} \and
\infer{
  \ofvval{\Psi_g, \Delta, \Gamma}{v}{\tau} \and
  (\mathtt{sp}:\sigma) \in \Gamma
}{
  \ofinstruction{\Psi_g, \Delta, \Gamma}
    {\mathtt{push}\ v}
    {\Gamma\{\mathtt{sp}: \tau :: \sigma\}}
} \and
\infer{
  (\mathtt{sp}: \tau::\sigma) \in \Gamma
}{
  \ofinstruction{\Psi_g, \Delta, \Gamma}
    {\mathtt{pop}}
    {\Gamma\{\mathtt{sp}: \sigma\}}
} \and
\infer{
  (\mathtt{sp}: \tau_1 :: \tau_2 :: \dots :: \tau_i :: \sigma) \in \Gamma
}{
  \ofinstruction{\Psi_g, \Delta, \Gamma}
    {\mathtt{ld}\ \mathtt{r}_d, \mathtt{sp}, i}
    {\Gamma\{\mathtt{r}_d: \tau_i\}}
} \and
\infer{
  (\mathtt{sp}: \tau_1 :: \tau_2 :: \dots :: \tau_i :: \sigma) \in \Gamma \and
  (\mathtt{r}_s: \tau_i') \in \Gamma
}{
  \ofinstruction{\Psi_g, \Delta, \Gamma}
    {\mathtt{st}\ \mathtt{sp}, i, \mathtt{r}_s}
    {\Gamma\{\mathtt{sp}: \tau_1 :: \tau_2 :: \dots :: \tau_i' :: \sigma\}}
} \and
\infer{
  (\mathtt{r}_s: \langle \tau_1^{\phi_1}, \dots, \tau_i^{\mathtt{init}}, \dots, \tau_n^{\phi_n}\rangle) \in \Gamma
}{
  \ofinstruction{\Psi_g, \Delta, \Gamma}
    {\mathtt{ld}\ \mathtt{r}_d, \mathtt{r}_s, i}
    {\Gamma\{\mathtt{r}_d: \tau_i\}}
} \and
\infer{
  (\mathtt{r}_d: \langle \tau_1^{\phi_1}, \dots, \tau_i^{\phi_i}, \dots, \tau_n^{\phi_n}\rangle) \in \Gamma \and
  (\mathtt{r}_s: \tau_i') \in \Gamma \and
  \subtype{\Delta}{\tau_i'}{\tau_i}
}{
  \ofinstruction{\Psi_g, \Delta, \Gamma}
    {\mathtt{st}\ \mathtt{r}_d, i, \mathtt{r}_s}
    {\Gamma\{\mathtt{r}_d: \langle \tau_1^{\phi_1}, \dots, \tau_i^{\mathtt{init}}, \dots, \tau_n^{\phi_n}\}}
} \and
\infer{
  \valid{\Delta}{\tau_1} \and
  \dots \and
  \valid{\Delta}{\tau_n}
}{
  \ofinstruction{\Psi_g, \Delta, \Gamma}
    {\mathtt{malloc}\ \mathtt{r}_d, \langle \tau_1, \dots, \tau_n \rangle}
    {\Gamma\{\mathtt{r}_d: \langle \tau_1^{\mathtt{uninit}}, \dots, \tau_n^{\mathtt{uninit}}\rangle\}}
} \and
\infer{
  \ofvval{\Psi_g, \Delta, \Gamma}{v}{\tau}
}{
  \ofinstruction{\Psi_g, \Delta, \Gamma}
    {\mathtt{mov}\ \mathtt{r}_d, v}
    {\Gamma\{\mathtt{r}_d: \tau\}}
} \and
\infer{
  (\mathtt{r}_k: \mathtt{int}) \in \Gamma \and
  \ofvval{\Psi_g, \Delta, \Gamma}{v}{\forall[ \mathtt{nil} ] \Gamma'} \and
  \subtype{\Delta}{\Gamma}{\Gamma'}
}{
  \ofinstruction{\Psi_g, \Delta, \Gamma}
    {\mathtt{beq}\ \mathtt{r}_k, v}
    {\Gamma}
}
\end{mathpar}

\fbox{$\ofinstruction{\Psi_g, \Delta, \Gamma}{I}{\Gamma'}$}
\begin{mathpar}
\infer{
  \ofinstruction{\Psi_g, \Delta, \Gamma}{\iota}{\Gamma'} \and
  \ofinstructions{\Psi_g, \Delta, \Gamma'}{I}
}{
  \ofinstructions{\Psi_g, \Delta, \Gamma}{\iota ; I}
}
\end{mathpar}

\fbox{$\ofprogram{G}{P}$}
\begin{mathpar}
\infer{
  \ofglobs{G}{\Psi_g} \and
  \ofheap{\Psi_g}{H}{\Psi_h} \\
  \ofregister{\Psi_g,\Psi_h}{R}{\Gamma} \and
  \ofinstructions{\Psi_g,\mathtt{nil},\Gamma}{I}
}{
  \ofprogram{G}{(H,R,I)}
}
\end{mathpar}

\subsection{Comments}

TODO

\subsection{Differences from STAL}

TODO

% \section{Semantics}

Finally: The semantics!\\

\fbox{$\evalsmall{R}{v}{w}$}
\begin{mathpar}
\infer{
  (\mathbf{r}_k \mapsto w) \in R
}{
  \evalsmall{R}{\mathtt{reg}_k}{w}
} \and
\infer{ }{\evalsmall{R}{\mathtt{word}\ w}{w}} \and
\infer{
  \evalsmall{R}{v}{w}
}{
  \evalsmall{R}{v \llbracket c \rrbracket}{w \llbracket c \rrbracket}
}
\end{mathpar}

\fbox{$\evalword{G}{w}{I}$}
\begin{mathpar}
\infer{
  (\ell_g \mapsto \mathtt{code}[\Delta]\Gamma.I) \in G
}{
  \evalword{G}{\mathtt{globval}\ \ell_g}{I}
} \and
\infer{
  \evalword{G}{w}{I} \and
  \substitution{I}{c}{I'}
}{
  \evalword{G}{w \llbracket c \rrbracket}{I'}
}
\end{mathpar}

\fbox{$\execinstruction{G}{P}{P'}$}
\begin{mathpar}
\infer{
  \evalsmall{R}{v}{\mathtt{int}\ n_1} \and
  (\mathtt{r}_s \mapsto \mathtt{int}\ n_2) \in R
}{
  \execinstruction{G}
    {H, R, (\mathtt{add}\ \mathtt{r}_d, \mathtt{r}_s, v) ; I}
    {(H, R\{\mathtt{r}_d \mapsto \mathtt{int}\ (n_1 + n_2)\}, I)}
} \and
\infer{
  \evalsmall{R}{v}{\mathtt{int}\ n_1} \and
  (\mathtt{r}_s \mapsto \mathtt{int}\ n_2) \in R
}{
  \execinstruction{G}
    {H, R, (\mathtt{sub}\ \mathtt{r}_d, \mathtt{r}_s, v) ; I}
    {(H, R\{\mathtt{r}_d \mapsto \mathtt{int}\ (n_1 - n_2)\}, I)}
} \and
\infer{
  \evalsmall{R}{v}{w} \and
  (\mathtt{sp} \mapsto S) \in R
}{
  \execinstruction{G}
    {H, R, \mathtt{push}\ v ; I}
    {(H, R\{\mathtt{sp} \mapsto w :: S\}, I)}
} \and
\infer{
  (\mathtt{sp} \mapsto w :: S) \in R
}{
  \execinstruction{G}
    {H, R, \mathtt{pop} ; I}
    {(H, R\{\mathtt{sp} \mapsto S\}, I)}
} \and
\infer{
  (\mathtt{sp} \mapsto w_1 :: w_2 :: \dots :: w_i :: S) \in R
}{
  \execinstruction{G}
    {H, R, (\mathtt{ld}\ \mathtt{r}_d, \mathtt{sp}, i) ; I}
    {(H, R\{\mathtt{r}_d \mapsto w_i\}, I)}
} \and
\infer{
  (\mathtt{sp} \mapsto w_1 :: w_2 :: \dots :: w_i :: S) \in R \and
  (\mathtt{r}_s \mapsto w_i') \in R
}{
  \execinstruction{G}
    {H, R, (\mathtt{st}\ \mathtt{sp}, i, \mathtt{r}_s) ; I}
    {(H, R\{\mathtt{sp} \mapsto w_1 :: w_2 \dots :: w_i' :: S\}, I)}
} \and
\infer{
  (\mathtt{r}_s \mapsto \mathtt{heapval}\ \ell_h) \in R \and
  (\ell_h \mapsto \langle w_1, \dots, w_i, \dots, w_n \rangle) \in H
}{
  \execinstruction{G}
    {H, R, (\mathtt{ld}\ \mathtt{r}_d, \mathtt{r}_s, i) ; I}
    {(H, R\{\mathtt{r}_d \mapsto w_i\}, I)}
} \and
\infer{
  (\mathtt{r}_d \mapsto \ell_h) \in R \and
  (\ell_h \mapsto \langle w_1, \dots, w_i, w_n \rangle) \in H \and
  (\mathtt{r}_s \mapsto w_i') \in R
}{
  \execinstruction{G}
    {H, R, (\mathtt{st}\ \mathtt{r}_d, i, \mathtt{r}_s) ; I}
    {(H\{\ell_h \mapsto \langle w_1, \dots, w_i', \dots, w_n \rangle\}, R, I)}
} \and
\infer{
  \text{$\ell_h$ is not present in $H$}
}{
  G |- \mathtt{eval} (H, R, (\mathtt{malloc}\ \mathtt{r}_d, \langle \tau_1, \dots, \tau_n \rangle) ; I) = \\
  (H\{\ell_h \mapsto \langle \mathtt{uninit}\ \tau_1, \dots, \mathtt{uninit}\ \tau_n \rangle\}, R\{\mathtt{r}_d \mapsto \mathtt{heapval}\ \ell_h\}, I)
} \and
\infer{
  \evalsmall{R}{v}{w}
}{
  \execinstruction{G}
    {H, R, (\mathtt{mov}\ \mathtt{r}_d, v) ; I}
    {(H, R\{\mathtt{r}_d \mapsto w\}, I)}
} \and
\infer{
  (\mathtt{r}_k \mapsto \mathtt{int}\ 0) \in R \and
  \evalsmall{R}{v}{w} \and
  \evalword{G}{w}{I'}
}{
  \execinstruction{G}
    {H, R, (\mathtt{beq}\ \mathtt{r}_k, v) ; I}
    {(H, R, I')}
} \and
\infer{
  (\mathtt{r}_k \mapsto \mathtt{int}\ i) \in R \and
  i \neq 0
}{
  \execinstruction{G}
    {H, R, (\mathtt{beq}\ \mathtt{r}_k, v) ; I}
    {(H, R, I)}
} \and
\infer{
  \evalsmall{R}{v}{w} \and
  \evalword{G}{w}{I'}
}{
  \execinstruction{G}
    {H, R, \mathtt{jmp}\ v}
    {(H, R, I')}
}
\end{mathpar}


% \chapter{Lemmas}
% \section{Miscellaneous small lemmas}

There are more small lemmas in the Agda file \texttt{Misc.agda}. I have only
included those that are not deemed trivial.

\begin{lemma}
  If we have $\substitution{\Delta_1}{c}{\Delta_1'}$ and
  $\substitution{\Delta_2}{c}{\Delta_2'}$, then we also have
  $\substitution{(\Delta_1,\Delta_2)}{c}{(\Delta_1',\Delta_2')}$.
\end{lemma}
\begin{proof}
  By induction over the derivation of $\substitution{\Delta_1}{c}{\Delta_1'}$.
\end{proof}

\begin{lemma}
  If we have $\substitution{\Delta_1}{c}{\Delta_1'}$ and
  $\run{\Delta_2}{c}{\Delta_2'}$, then we also have
  $\run{(\Delta_1,\Delta_2)}{c}{(\Delta_1',\Delta_2')}$.
\end{lemma}
\begin{proof}
  By induction over the derivation of $\substitution{\Delta_1}{c}{\Delta_1'}$.
\end{proof}

\begin{lemma}
  If we have $\run{\Delta_1}{c}{\Delta_1'}$, then we also have
  $\run{(\Delta_1,\Delta_2)}{c}{(\Delta_1',\Delta_2)}$.
\end{lemma}
\begin{proof}
  By induction over the derivation of $\run{\Delta_1}{c}{\Delta_1'}$.
\end{proof}

% \section{Decidable Equality}

\begin{definition}
  Let $\mathbb{T}$ be the set of all rooted trees where each node has been
  assigned a single value from $\mathbb{N}$.
\end{definition}

\begin{lemma}
  Let $t_1, t_2 \in \mathbb{T}$. Then it is decidable if $t_1 = t_2$.
\end{lemma}
\label{eq:tree}
\begin{proof}
  By induction over the structure of $t_1$ and $t_2$.
\end{proof}

\begin{lemma}
  Let $T$ be one of the core datatypes from \autoref{sec:grammar}. Then there
  exists a injective mapping from $T \to \mathbb{T}$.
\end{lemma}
\label{eq:inj}
\begin{proof}
  The mapping is obvious, since all of the definitions are finite, inductive
  datatypes.
\end{proof}

\begin{corollary}
  Let $T$ be one of the datatypes from \autoref{sec:grammar} and let
  $a, b \in T$. Then it is decidable if whether $a=b$ or $a \neq b$.
\end{corollary}
\label{dec:eq}
\begin{proof}
  Let $f$ be the function from \autoref{eq:inj}. Since
  $f(a), f(b) \in \mathbb{T}$ then by \autoref{eq:tree} we can decide if
  $f(a) = f(b)$. Because $f$ is injective, we have $a = b \iff f(a) = f(b)$, so
  we are done.
\end{proof}

% \documentclass{article}
\usepackage[a4paper, hmargin={2.8cm, 2.8cm}, vmargin={2.8cm, 2.8cm}]{geometry}
\usepackage[utf8]{inputenc}
\usepackage[english]{babel}
\usepackage{listingsutf8}
\usepackage{amsmath, amssymb}
\usepackage{mathtools}
\usepackage{mathpartir}

\newcommand \isctx[1] {{#1}\ \ \mathbf{ctx}}
\newcommand \isstack[2] {{#1} \vdash {#2}\ \ \mathbf{stack}}
\newcommand \islifetime[3] {{#1}, {#2} \vdash {#3}\ \ \mathbf{lifetime}}
\newcommand \issub[4]{{#1}, {#2} \vdash {#3} \leq {#4}}
\newcommand \istype[3] {{#1}, {#2} \vdash {#3}\ \ \mathbf{type}}
\newcommand \istypen[4] {{#1}, {#2} \vdash {#3}\ \ \mathbf{type}_{#4}}
\newcommand \isregister[2] {{#1} \vdash {#2}\ \ \mathbf{register}}
\newcommand \isqualifier[1] {{#1}\ \ \mathbf{qualifer}}
\newcommand \ownedptr[1] {\sim {#1}}
\newcommand \refptr[3] {\&\,{#1}\ {#2}\ {#3}}
\newcommand \nil {\mathtt{nil}}
\newcommand \cons {\dblcolon}
\newcommand \static {\mathtt{static}}
\newcommand \intt {\mathtt{int}}
\newcommand \void[1] {\mathtt{void}_{#1}}
\newcommand \mut {\mathtt{mut}}
\newcommand \imm {\mathtt{imm}}
\newcommand{\xmid}{\;\mid\;}
\newcommand{\defeq}{\;\Coloneqq\;\;}
\begin{document}

(This is still leaving out initialization-state for simplicity)

\fbox{$\isctx{\Delta}$}
\begin{mathpar}
\infer{ }{\isctx{\cdot}} \and
\infer{\isctx{\Delta}}{\isctx{\Delta, \rho}} \and
\infer{\isctx{\Delta} \and \isstack{\Delta}{\sigma}}{\isctx{\Delta, \alpha_{n,\sigma}}} \and
\infer{\isctx{\Delta} \and \isstack{\Delta}{\sigma}}{\isctx{\Delta, \ell_\sigma}} \and
\infer{\isctx{\Delta} \and \islifetime{\Delta}{\sigma}{\ell_1} \and \islifetime{\Delta}{\sigma}{\ell_2}}{\isctx{\Delta, \ell_1 \leq_\sigma \ell_2}}
\end{mathpar}

\fbox{$\isstack{\Delta}{\sigma}$}
\begin{mathpar}
\infer{ }{\isstack{\Delta}{\nil}} \and
\infer{\rho \in \Delta}{\isstack{\Delta}{\rho}} \and
\infer{\isstack{\Delta}{\sigma} \and \istype{\Delta}{\sigma}{\tau}}{\isstack{\Delta}{\tau \cons \sigma}} \and
\infer{\isstack{\Delta}{\sigma}}{\isstack{\Delta}{\ell \cons \sigma}}
\end{mathpar}

\fbox{$\islifetime{\Delta}{\sigma}{\ell}$}
\begin{mathpar}
\infer{\islifetime{\Delta}{\sigma}{\ell}}{\islifetime{\Delta}{\_ \cons \sigma}{\ell}} \and
\infer{\ell_\sigma \in \Delta}{\islifetime{\Delta}{\sigma}{\ell_\sigma}} \and
\infer{ }{\islifetime{\Delta}{\ell \cons \sigma}{\ell}} \and
\infer{ }{\islifetime{\Delta}{\sigma}{\static}}
\end{mathpar}

\fbox{$\istype{\Delta}{\sigma}{\tau}$}
\begin{mathpar}
\infer{\istypen{\Delta}{\sigma}{\tau}{n}}{\istype{\Delta}{\sigma}{\tau}}
\end{mathpar}

\fbox{$\istypen{\Delta}{\sigma}{\tau}{n}$}
\begin{mathpar}
\infer{\istypen{\Delta}{\sigma}{\tau}{n}}{\istypen{\Delta}{\_ \cons \sigma}{\tau}{n}} \and
\infer{\alpha_{n,\sigma} \in \Delta}
      {\istypen{\Delta}{\sigma}{\alpha_{n,\sigma}}{n}} \and
\infer{ }
      {\istypen{\Delta}{\sigma}{\intt}{4}} \and
\infer{ }
      {\istypen{\Delta}{\sigma}{\void{n}}{n}} \and
\infer{\istype{\Delta}{\sigma}{\tau}}
      {\istypen{\Delta}{\sigma}{\ownedptr{\tau}}{4}} \and
\infer{\islifetime{\Delta}{\sigma}{\ell}
       \and \isqualifier{q}
       \and \istype{\Delta}{\sigma}{\tau}}
      {\istypen{\Delta}{\sigma}{\refptr{\ell}{q}{\tau}}{4}} \and
\infer{\istypen{\Delta}{\sigma}{\tau_1}{n_1}
       \and \dots
       \and \istypen{\Delta}{\sigma}{\tau_k}{n_k}}
      {\istypen{\Delta}{\sigma}{[\tau_i]_{i \in \{1 \dots k\}}}{n_1 + \dots + n_k}} \and
\infer{\istypen{\Delta}{\sigma}{\tau_1}{n}
       \and \dots
       \and \istypen{\Delta}{\sigma}{\tau_k}{n}}
      {\istypen{\Delta}{\sigma}{\langle\tau_i\rangle_{i \in \{1 \dots k\}}}{4 + n}} \and
\infer{\isregister{\Delta + \Delta'}{\Gamma}}
      {\istypen{\Delta}{\sigma}{\forall[\Delta']\Gamma}{4}}
\end{mathpar}

\fbox{$\isregister{\Delta}{\Gamma}$}
\begin{mathpar}
\infer{\isstack{\Delta}{\sigma}
       \and \istypen{\Delta}{\sigma}{\tau_1}{4}
       \and \dots
       \and \istypen{\Delta}{\sigma}{\tau_k}{4}}
      {\isregister{\Delta}{\{r_1:\tau_1, \dots, r_k:\tau_k, \mathtt{sp}:\sigma\}}}
\end{mathpar}

\fbox{$\issub{\Delta}{\sigma}{\ell_1}{\ell_2}$}
\begin{mathpar}
\infer{\issub{\Delta}{\sigma}{\ell_1}{\ell_2}}
      {\issub{\Delta}{\_ \cons \sigma}{\ell_1}{\ell_2}} \and
\infer{\ell_1 \leq_\sigma \ell_2 \in \Delta}
      {\issub{\Delta}{\sigma}{\ell_1}{\ell_2}} \and
\infer{ \islifetime{\Delta}{\ell_1 \cons \sigma}{\ell_2}}
      {\issub{\Delta}{\ell_1 \cons \sigma}{\ell_1}{\ell_2}} \and
\infer{ \islifetime{\Delta}{\sigma}{\ell}}
      {\issub{\Delta}{\sigma}{\ell}{\static}} \and
\end{mathpar}

\fbox{$\isqualifier{q}$}
\begin{mathpar}
\infer{ }{\isqualifier{\mut}} \and
\infer{ }{\isqualifier{\imm}}
\end{mathpar}
\end{document}

% \section{Subtyping judgments}

\newcommand \subtype[3] {{#1} |- {#2} \le {#3}}

\fbox{$\subtype{\Delta}{\tau_1}{\tau_2}$}
\begin{mathpar}
\infer{\alpha \in \Delta}{\subtype{\Delta}{\alpha}{\alpha}} \and
\infer{ }{\subtype{\Delta}{\mathtt{int}}{\mathtt{int}}} \and
\infer{ }{\subtype{\Delta}{\mathtt{ns}}{\mathtt{ns}}} \and
\infer
  {\valid{\Delta}{\Delta'} \and
    \subtype{\Delta', \Delta}{\Gamma_1}{\Gamma_2}}
  {\subtype{\Delta}{\forall[ \Delta' ]\Gamma_1}{\forall[ \Delta' ]\Gamma_2}} \and
\infer
  {\subtype{\Delta}{\tau_1^{\phi_1}}{\tau'_1{}^{\phi'_1}} \and
    \dots \and
    \subtype{\Delta}{\tau_n^{\phi_n}}{\tau'_n{}^{\phi'_n}}}
  {\subtype{\Delta}{\langle \tau'_1{}^{\phi_1}, \dots, \tau_n{}^{\phi_n} \rangle}
                   {\langle \tau'_1{}^{\phi'_1}, \dots, \tau'_n{}^{\phi'_n} \rangle}}
\end{mathpar}

\fbox{$\subtype{}{\phi_1}{\phi_2}$}
\begin{mathpar}
\infer{ }{\subtype{}{\mathtt{init}}{\mathtt{init}}} \and
\infer{ }{\subtype{}{\mathtt{\phi}}{\mathtt{uninit}}}
\end{mathpar}

\fbox{$\subtype{\Delta}{\tau_1^{\phi_1}}{\tau_2^{\phi_2}}$}
\begin{mathpar}
\infer{\valid{\Delta}{\tau} \and \subtype{}{\phi_1}{\phi_2}}{\subtype{\Delta}{\tau^{\phi_1}}{\tau^{\phi_2}}}
\end{mathpar}

% \section{Substitution judgments}

\fbox{$\substitution{\tau_1}{c}{\tau_2}$}
\begin{mathpar}
  \infer{ }{\substitution{\alpha}{\alpha => \tau}{\tau}} \and
  \infer{\alpha_1 \neq \alpha_2}{\substitution{\alpha_1}{\alpha_2 => \tau}{\alpha_1}} \and
  \infer{ }{\substitution{\alpha}{\rho => \sigma}{\alpha}} \and
  \infer{ }{\substitution{\alpha}{\mathtt{weaken}\ \Delta^{+}}{\alpha}}
\end{mathpar}

(This also includes obvious, mutually recursive definition.)\\\\
\fbox{$\substitution{\tau_1^{\phi_1}}{c}{\tau_2^{\phi_2}}$}\\

(The obvious, mutually recursive definitions.)\\\\
\fbox{$\substitution{\sigma_1}{c}{\sigma_2}$}
\begin{mathpar}
  \infer{ }{\substitution{\rho}{\rho => \sigma}{\sigma}} \and
  \infer{\rho_1 \neq \rho_2}{\substitution{\rho_1}{\rho_2 => \sigma}{\rho_1}} \and
  \infer{ }{\substitution{\rho}{\alpha => \tau}{\rho}} \and
  \infer{ }{\substitution{\rho}{\mathtt{weaken}\ \Delta^{+}}{\rho}}
\end{mathpar}

(This also includes obvious, mutually recursive definition.)\\\\
\fbox{$\substitution{\Delta_1}{c}{\Delta_2}$}\\

(The obvious, mutually recursive definitions.)\\\\
\fbox{$\substitution{a_1}{c}{a_2}$}\\

(The obvious, mutually recursive definitions.)\\\\
\fbox{$\substitution{\Gamma_1}{c}{\Gamma_2}$}\\
\begin{mathpar}
  \infer{ }{\substitution{\mathtt{nil}}{c}{\mathtt{nil}}} \and
  \infer{
    \substitution{a_1}{c}{a_2} \and
    \substitution{\Delta_1}{c}{\Delta_2}
  }{
    \substitution{(a_1 :: \Delta_1)}{c}{a_2 :: \Delta_2}
  }
\end{mathpar}\\
\fbox{$\substitution{c_1}{c}{c_2}$}\\

(The obvious, mutually recursive definitions.)\\\\
\fbox{$\substitution{w_1}{c}{w_2}$}\\

(The obvious, mutually recursive definitions.)\\\\
\fbox{$\substitution{v_1}{c}{v_2}$}\\

(The obvious, mutually recursive definitions.)\\\\
\fbox{$\substitution{\iota_1}{c}{\iota_2}$}\\

(The obvious, mutually recursive definitions.)\\\\
\fbox{$\substitution{I_1}{c}{I_2}$}\\

(The obvious, mutually recursive definitions.)


\begin{definition}
  Let $T \in \mathbb{C}$ and $v_1, v_2 \in T$ and assume we have some $c$. We define
  $\mathbb{SUB}$ to be the subset of $\mathbb{C}$ with the properties:
  \begin{itemize}
  \item $\substitution{v_1}{c}{v_2}$ is a meaningful expression iff $T \in \mathbb{SUB}$.
  \end{itemize}
\end{definition}

\subsection{Comments}

I write that the rules are obvious -- and they are, as long as you keep yourself
to using paper-notation. If you want to use De Bruijn indices they are not
nearly as obvious.

Note that the substitution for $\Delta$ assumes that the actual substitution is
to be done elsewhere -- it merely updates all references inside the $\Delta$. In
the current grammar there \emph{are} no references inside the assumptions, so we
get that $\substitution{\Delta}{c}{\Delta}$ for all $\Delta$. This would ideally
not hold in the final version.

For the substitution that actually updates the assumption list, see the next
section.

\subsection{Differences from STAL}

STAL handwaves even more that I do here. Rest assured that my Agda-code is
precise enough for a machine-verified proof.

% \section{Running judments}

\fbox{$\run{\Delta_1}{c}{\Delta_2}$}\\

\begin{mathpar}
\infer{ }{\run{(\alpha :: \Delta)}{\alpha => \tau}{\Delta}} \and
\infer{ }{\run{(\rho :: \Delta)}{\rho => \sigma}{\Delta}} \and
\infer{
  a_1 \neq a \and
  \substitution{a_1}{a => v}{a_2} \and
  \substitution{\Delta_1}{a => v}{\Delta_2}
}{
  \run{(a_1 :: \Delta_1)}{a => v}{(a_2 :: \Delta_2)}
} \and
\infer{
  \text{$\Delta^{+}$ should be inserted elsewhere} \and
  \substitution{a_1}{\mathtt{weaken}\ \Delta^{+}}{a_2} \and
  \substitution{\Delta_1}{\mathtt{weaken}\ \Delta^{+}}{\Delta_2}
}{
  \run{(a_1 :: \Delta_1)}{\mathtt{weaken}\ \Delta^{+}}{(a_2 :: \Delta_2)}
} \and
\infer{
  \text{$\Delta^{+}$ should be inserted here}
}{
  \run{\Delta}{\mathtt{weaken}\ \Delta^{+}}{(\Delta^{+}, \Delta)}
}
\end{mathpar}

\subsection{Comments}

We again have a bit of imprecision/handwaving because the actual Agda-code deals
with De Bruijn indices, and the paper notation deals with variable-names.


\listoftodos[Notes]

\bibliographystyle{plain}
\bibliography{report}
\end{document}
