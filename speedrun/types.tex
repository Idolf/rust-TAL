\chapter{Type System for \ATAL}
\label{chap:types}

This chapter presents our type system. It is divided into two parts. The first
part is deals with validity of types and subtyping, while the second deals with
the actual typing relations.

\section{Judgments About Types}

This first part is relatively boring and can be summarized as:

\begin{itemize}
\item Types are valid in a specific context, long as the De Bruijn indices are
  valid.
\item Subtyping is a transitive relation. Subtyping implies validity and validity implies reflexivity.
\item
  $\langle \tau_1^{\phi_1}, \dots, \tau_k^{\mathtt{init}}, \dots,
  \tau_n^{\phi_n} \rangle$ is a subtype of
  $\langle \tau_1^{\phi_1}, \dots, \tau_k^{\mathtt{uninit}}, \dots,
  \tau_n^{\phi_n} \rangle$.
\item Other constructs form the obvious subtyping relation.
\item The Agda-version of the judgments can be seen in file
  \texttt{Judgments/Types.agda}.
\end{itemize}

\begin{tabular}{|c|p{7.5 cm}|}
  \hline
  Judgment & \multicolumn{1}{|c|}{Meaning} \\
  \hline

  $\valid{\Delta}{\tau}$ & $\tau$ is a well-formed type (i.e., does not contain invalid variables) \\
  $\valid{\Delta}{\sigma}$ & $\sigma$ is a well-formed stack type \\
  $\valid{\Delta}{\Gamma}$ & $\Gamma$ is a well-formed register assignment \\
  $\valid{}{\Psi_{\mathrm{h}}}$ & $\Psi_{\mathrm{h}}$ is a well-formed heap label assignments \\
  \hline

  $\subtype{\Delta}{\tau_1}{\tau_2}$ & $\tau_1$ is a subtype of $\tau_2$ \\
  $\subtype{\Delta}{\phi_1}{\phi_2}$ & $\phi_1$ is a subtype of $\phi_2$ \\
  $\subtype{\Delta}{\sigma_1}{\sigma_2}$ & $\sigma_1$ is a subtype of $\sigma_2$ \\
  $\subtype{\Delta}{\Gamma_1}{\Gamma_2}$ & $\Gamma_1$ is a subtype of $\Gamma_2$ \\
  $\subtype{}{\Psi_{\mathrm{h},1}}{\Psi_{\mathrm{h}_2}}$ & $\Psi_{\mathrm{h},1}$ is a subtype of $\Psi_{\mathrm{h},2}$ \\
  \hline
\end{tabular}

\subsection{Valid Types}
\fbox{$\valid{\Delta}{\tau}$}
\begin{mathpar}
\infer{\alpha \in \Delta}{\valid{\Delta}{\alpha}} \and
\infer{ }{\valid{\Delta}{\mathtt{int}}} \and
\infer{ }{\valid{\Delta}{\mathtt{uninit}}} \and
\infer{\valid{\Delta' ++ \Delta}{\Gamma}}{\valid{\Delta}{\forall[ \Delta' ] \Gamma}} \and
\infer{\valid{\Delta}{\tau_1} \and \dots \and \valid{\Delta}{\tau_n}}
      {\valid{\Delta}{\langle \tau_1^{\phi_1}, \dots, \tau_n^{\phi_n} \rangle}}
\end{mathpar}

\fbox{$\valid{\Delta}{\sigma}$}
\begin{mathpar}
\infer{\rho \in \Delta}{\valid{\Delta}{\rho}} \and
\infer{ }{\valid{\Delta}{\nil}} \and
\infer{
  \valid{\Delta}{\tau} \and \valid{\Delta}{\sigma}
}{
  \valid{\Delta}{\tau :: \sigma}
}
\end{mathpar}

\fbox{$\valid{\Delta}{\Gamma}$}
\begin{mathpar}
\infer{
  \valid{\Delta}{\sigma} \and
  \valid{\Delta}{\tau_1} \and
  \dots \and
  \valid{\Delta}{\tau_{\mathrm{\mathrm{max}}}}
}{
  \valid{\Delta}{\{\mathtt{sp} |-> \sigma, \mathtt{r}_1 |-> \tau_1, \dots, \mathtt{r}_{\mathrm{\mathrm{max}}} |-> \tau_{\mathrm{\mathrm{max}}}\}}
}
\end{mathpar}

\fbox{$\valid{}{\Psi_{\mathrm{h}}}$}
\begin{mathpar}
\infer{
  \valid{\nil}{\tau_1} \and
  \dots \and
  \valid{\nil}{\tau_{n}}
}{
  \valid{}{\{\ell_{\mathrm{h},1} |-> \tau_1, \dots, \ell_{\mathrm{h},n} |-> \tau_{n}\}}
}
\end{mathpar}

\subsection{Subtypes}
\fbox{$\subtype{\Delta}{\tau_1}{\tau_2}$}
\begin{mathpar}
\infer{\alpha \in \Delta}{\subtype{\Delta}{\alpha}{\alpha}} \and
\infer{ }{\subtype{\Delta}{\mathtt{int}}{\mathtt{int}}} \and
\infer{ }{\subtype{\Delta}{\mathtt{uninit}}{\mathtt{uninit}}} \and
\infer{
  \subtype{\Delta' ++ \Delta}{\Gamma_2}{\Gamma_1}
}{
  \subtype{\Delta}{\forall[\Delta'] \Gamma_1}{\forall[\Delta'] \Gamma_2}
} \and
\infer{
  \subtype{}{\phi_1}{\phi_1'} \and
  \dots \and
  \subtype{}{\phi_n}{\phi_n'} \\
  \valid{\Delta}{\tau_1} \and
  \dots \and
  \valid{\Delta}{\tau_n}
}{
  \subtype{\Delta}
          {\langle \tau_1^{\phi_1}, \dots, \tau_n^{\phi_n} \rangle}
          {\langle \tau_1^{\phi'_1}, \dots, \tau_n^{\phi'_n} \rangle}
}
\end{mathpar}

\fbox{$\subtype{}{\phi_1}{\phi_2}$}
\begin{mathpar}
\infer{ }{\subtype{}{\mathtt{init}}{\phi}} \and
\infer{ }{\subtype{}{\mathtt{uninit}}{\mathtt{uninit}}}
\end{mathpar}

\fbox{$\subtype{\Delta}{\sigma}{\sigma}$}
\begin{mathpar}
\infer{\rho \in \Delta}{\subtype{\Delta}{\rho}{\rho}} \and
\infer{ }{\subtype{\Delta}{\nil}{\nil}} \and
\infer{
  \subtype{\Delta}{\tau}{\tau'} \and
  \subtype{\Delta}{\sigma}{\sigma'} \and
}{
  \subtype{\Delta}{\tau :: \sigma}{\tau' :: \sigma'}
}
\end{mathpar}

\fbox{$\subtype{\Delta}{\Gamma}{\Gamma}$}
\begin{mathpar}
\infer{
  \subtype{\Delta}{\sigma}{\sigma'} \and
  \subtype{\Delta}{\tau_1}{\tau_1'} \and
  \dots \and
  \subtype{\Delta}{\tau_{\mathrm{\mathrm{max}}}}{\tau_{\mathrm{\mathrm{max}}}'}
}{
  \subtype{\Delta}
          {\{\mathtt{sp} |-> \sigma, \mathtt{r}_1 |-> \tau_1, \dots, \mathtt{r}_{\mathrm{\mathrm{max}}} |-> \tau_{\mathrm{\mathrm{max}}}\}}
          {\{\mathtt{sp} |-> \sigma', \mathtt{r}_1 |-> \tau_1', \dots, \mathtt{r}_{\mathrm{\mathrm{max}}} |-> \tau_{\mathrm{\mathrm{max}}}'\}}
}
\end{mathpar}

\fbox{$\subtype{}{\Psi_{\mathrm{h}}}{\Psi_{\mathrm{h}}}$}
\begin{mathpar}
\infer{
  \subtype{\nil}{\tau_1}{\tau_1'} \and
  \dots \and
  \subtype{\nil}{\tau_{n}}{\tau_{n}'}
}{
  \subtype{}
          {\{\ell_{\mathrm{h},1} |-> \tau_1, \dots, \ell_{\mathrm{h},n} |-> \tau_{n}\}}
          {\{\ell_{\mathrm{h},1} |-> \tau_1', \dots, \ell_{\mathrm{h},n} |-> \tau_{n}'\}}
}
\end{mathpar}

\subsection{Instantiations}

\fbox{$\ofinstantiation{\Delta}{\theta}{a}$}
\begin{mathpar}
\infer{
  \alpha \notin \Delta \and
  \valid{\Delta}{\tau}
}{
  \ofinstantiation{\Delta}{\tau / \alpha}{\alpha}
} \and
\infer{
  \rho \notin \Delta \and
  \valid{\Delta}{\sigma}
}{
  \ofinstantiation{\Delta}{\sigma / \rho}{\rho}
}
\end{mathpar}

\subsection{Lemmas}

The files \texttt{Lemmas/Types.agda} and \texttt{Lemmas/TypeSubstitution.agda}
contains some lemmas about types, the most important of which are:

\begin{lemma}
  \label{lemma:typdec}
  Validity and subtyping is decidable. So are the judgments
  $\ofinstantiation{\Delta}{\theta}{a}$ and
  $\ofinstantiations{\Delta}{\Theta}{\Delta'}$.
\end{lemma}

\begin{lemma}
  \label{lemma:typeq}
  Validity implies subtyping and vice versa.
\end{lemma}

\begin{lemma}
  \label{lemma:transitive}
  Subtyping is transitive.
\end{lemma}

\begin{lemma}
  \label{lemma:typ-context}
  Validity and subtyping is preserved by weakening or instantiation variables
  from the context.\footnote{Remember that substitution is a partial function in
    the Agda-code. In the Agda-code this lemma also implies totality of
    substitution given certain additional assumptions.}
\end{lemma}

\section{Judgments About Values}

This section is about typing judgments. There it has been split up into multiple
parts:

\begin{itemize}
\item The first part is about validity of instantiations.
\item The second part is about validity of code. There are three basic code
  primitives. Small values are a way to generate a word value from
  registers. Instructions can be though of as ``functions'' from register types
  to register types, and are typed as such. Finally instruction sequences can be
  thought of as ``functions'' that might not return. None of these depend on the
  heap, though they might depend on type parameters.
\item The third part is typing judgments about memory constructs. Most of them
  may depend on the heap, but none of them can be parameterized over types.
\item The final part is where we make everything fit together. It contains no
  real surprises, but be useful for understanding the other parts.
\end{itemize}

\begin{tabular}{|c|p{7.5 cm}|}
  \hline
  Judgment & \multicolumn{1}{|c|}{Meaning} \\
  \hline

  $\Delta |- \theta : a$ & $\theta$ is a valid instantiation of the variable $a$ \\
  $\Delta |- \Theta : \Delta'$ & $\Theta$ are valid instantiations of the variables $\Delta'$ \\
  \hline

  $\Psi_{\mathrm{g}}, \Delta |- \high v : \Gamma => \tau$ & $\high v$ is a well-formed small value evaluating register files of type $\Gamma$ to word values of type $\tau$ \\
  $\Psi_{\mathrm{g}}, \Delta |- \high \iota : \Gamma_1 => \Gamma_2$ & $\high \iota$ is a well-formed instruction which evaluating registers of type $\Gamma_1$ to registers of type $\Gamma_2$ \\
  $\Psi_{\mathrm{g}}, \Delta |- \high I : \Gamma$ & $\high I$ is a well-formed instruction sequence which will correctly evaluate registers of type $\Gamma$. \\
  \hline

  $\Psi_{\mathrm{g}}, \Psi_{\mathrm{h}} |- \high w : \tau$ & $\high w$ is a well-formed word value of type $\tau$ \\
  $\Psi_{\mathrm{g}}, \Psi_{\mathrm{h}} |- \high w : \tau^{\phi}$ & $\high w$ is a well-formed (and possibly uninitialized) word value of type $\tau^\phi$ (i.e., either $\high w : \tau$ or $\high w = \mathtt{uninit}$ and $\phi = \mathtt{uninit}$) \\
  $\Psi_{\mathrm{g}}, \Psi_{\mathrm{h}} |- \high S : \sigma$ & $\high S$ is a well-formed stack of type $\sigma$ \\
  $\Psi_{\mathrm{g}}, \Psi_{\mathrm{h}} |- \high R : \Gamma$ & $\high R$ is a well-formed register file of type $\Gamma$ \\
  $\Psi_{\mathrm{g}}, \Psi_{\mathrm{h}} |- \high h : \tau$ & $\high h$ is a well-formed heap value of type $\tau$ \\
  $\Psi_{\mathrm{g}} |- \high g : \tau$ & $\high g$ is a well-formed global value of type $\tau$. \\
  $\Psi_{\mathrm{g}} |- \high H : \Psi_{\mathrm{h}}$ & $\high H$ is a well-formed heap collection of type $\Psi_{\mathrm{h}}$ \\
  $|- \high G : \Psi_{\mathrm{g}}$ & $\high G$ is a well-formed global collection of type $\Psi_{\mathrm{g}}$. \\
  \hline

  $\Psi_{\mathrm{g}} |- (\high H, \high R, \high I) : (\Psi_{\mathrm{h}}, \Gamma)$ & $(\high H, \high R, \high I)$ are all valid while using types $\Psi_{\mathrm{h}}$ and $\Gamma$ for the heap and registers. \\
  $\valid{}{\high P}$ & $\high P$ is a well-formed and well-typed program. \\
  \hline
\end{tabular}

\subsection{Evaluation Types}

\fbox{$\ofinstantiations{\Delta}{\Theta}{\Delta}$}
\begin{mathpar}
\infer{ }{
  \ofinstantiations{\Delta}{\nil}{\nil}
} \and
\infer{
  \ofinstantiation{\Delta' ++ \Delta}{\theta}{a} \and
  \ofinstantiations{\Delta}{\Theta}{\Delta'}
}{
  \ofinstantiations{\Delta}{(\theta :: \Theta)}{(a :: \Delta')}
}
\end{mathpar}

\fbox{$\ofvval{\Psi_{\mathrm{g}},\Delta}{\high v}{\Gamma => \tau}$}
\begin{mathpar}
\infer{ }{
  \ofvval{\Psi_{\mathrm{g}},\Delta}{\mathtt{reg}\ r}{\Gamma => \Gamma[r]}
}\and
\infer{
}{
  \ofvval{\Psi_{\mathrm{g}},\Delta}{\mathtt{globval}\ \ell_{\mathrm{g}}}{\Gamma => \Psi_{\mathrm{g}}[\ell_{\mathrm{g}}]}
}\and
\infer{ }{\ofvval{\Psi_{\mathrm{g}},\Delta}{\mathtt{int}\ i}{\Gamma => \mathtt{int}}} \and
\infer{
  \ofvval{\Psi_{\mathrm{g}},\Delta}{\high v}{\Gamma => \forall[\Delta_1]\Gamma_1} \and
  \ofinstantiations{\Delta_2 ++ \Delta}{\Theta}{\Delta_1} \and
  \Gamma_2 = \Gamma_1[\Theta] \\
}{
  \ofvval{\Psi_{\mathrm{g}},\Delta}{\Lambda\ \Delta \cdot \high v[\Theta]}{\Gamma => \forall[\Delta_2]\Gamma_2}
}
\end{mathpar}

\fbox{$\ofinstruction{\Psi_{\mathrm{g}}, \Delta}{\high \iota}{\Gamma_1 => \Gamma_2}$}
\begin{mathpar}
\infer{
  \Gamma[r_b] = \mathtt{int} \and
  \ofvval{\Psi_{\mathrm{g}}, \Delta}{\high v}{\Gamma => \mathtt{int}}
}{
  \ofinstruction{\Psi_{\mathrm{g}}, \Delta}
    {\mathtt{add}\ r_a, r_b, \high v}
    {\Gamma => \Gamma[r_a |-> \mathtt{int}]}
} \and
\infer{
  \Gamma[r_b] = \mathtt{int} \and
  \ofvval{\Psi_{\mathrm{g}}, \Delta}{\high v}{\Gamma => \mathtt{int}}
}{
  \ofinstruction{\Psi_{\mathrm{g}}, \Delta}
    {\mathtt{sub}\ r_a, r_b, \high v}
    {\Gamma => \Gamma[r_a |-> \mathtt{int}]}
} \\
\infer{
  \sigma = \overbrace{\mathtt{uninit} :: \dots :: \mathtt{uninit}}^n :: \Gamma[\mathtt{sp}]
}{
  \ofinstruction{\Psi_{\mathrm{g}}, \Delta}
    {\mathtt{salloc}\ n}
    {\Gamma => \Gamma[\mathtt{sp} |-> \sigma]}
} \and
\infer{
  \Gamma[\mathtt{sp}] = \overbrace{\tau_1 :: \dots :: \tau_n}^n :: \sigma
}{
  \ofinstruction{\Psi_{\mathrm{g}}, \Delta}
    {\mathtt{sfree}\ n}
    {\Gamma => \Gamma[\mathtt{sp} |-> \sigma]}
} \and
\infer{
  \Gamma[\mathtt{sp}] = \sigma
}{
  \ofinstruction{\Psi_{\mathrm{g}}, \Delta}
    {\mathtt{ld}\ r, \mathtt{sp}(n)}
    {\Gamma => \Gamma[r |-> \sigma[n]]}
} \and
\infer{
  \Gamma[r_b] = \langle \tau_1^{\phi_1}, \dots, \tau_n^{\mathtt{init}}, \dots\rangle \and
}{
  \ofinstruction{\Psi_{\mathrm{g}}, \Delta}
    {\mathtt{ld}\ r_a, r_b(n)}
    {\Gamma => \Gamma[r_a |-> \tau_n]}
} \and
\infer{
  \Gamma[r] = \tau \and
  \Gamma[\mathtt{sp}] = \sigma
}{
  \ofinstruction{\Psi_{\mathrm{g}}, \Delta}
    {\mathtt{st}\ \mathtt{sp}(n), r}
    {\Gamma => \Gamma[\mathtt{sp} |-> \sigma[n |-> \tau]]}
} \and
\infer{
  \Gamma[r_a] = \langle \tau_1^{\phi_1}, \dots, \tau_n^{\phi_i}, \dots \rangle \and
  \Gamma[r_b] = \tau_n' \and
  \subtype{\Delta}{\tau_n'}{\tau_n}
}{
  \ofinstruction{\Psi_{\mathrm{g}}, \Delta}
    {\mathtt{st}\ r_a(n), r_b}
    {\Gamma => \Gamma[r_a |-> \langle \tau_1^{\phi_1}, \dots, \tau_n^{\mathtt{init}}, \dots, \rangle]}
} \and
\infer{
  \valid{\Delta}{\tau_1} \and
  \dots \and
  \valid{\Delta}{\tau_n}
}{
  \ofinstruction{\Psi_{\mathrm{g}}, \Delta}
    {\mathtt{malloc}\ r, \langle \tau_1, \dots, \tau_n \rangle}
    {\Gamma => \Gamma[r |-> \langle \tau_1^{\mathtt{uninit}}, \dots, \tau_n^{\mathtt{uninit}}\rangle]}
} \and
\infer{
  \ofvval{\Psi_{\mathrm{g}}, \Delta}{\high v}{\Gamma => \tau}
}{
  \ofinstruction{\Psi_{\mathrm{g}}, \Delta}
    {\mathtt{mov}\ r, \high v}
    {\Gamma => \Gamma[r |-> \tau]}
} \and
\infer{
  \Gamma[r] = \mathtt{int} \and
  \ofvval{\Psi_{\mathrm{g}}, \Delta}{\high v}{\Gamma => \forall[ \nil ] \Gamma'} \and
  \subtype{\Delta}{\Gamma}{\Gamma'}
}{
  \ofinstruction{\Psi_{\mathrm{g}}, \Delta}
    {\mathtt{beq}\ r, \high v}
    {\Gamma => \Gamma}
}
\end{mathpar}


\fbox{$\ofinstructions{\Psi_{\mathrm{g}}, \Delta}{\high I}{\Gamma}$}
\begin{mathpar}
\infer{
  \ofinstruction{\Psi_{\mathrm{g}}, \Delta}{\high \iota}{\Gamma => \Gamma'} \and
  \ofinstructions{\Psi_{\mathrm{g}}, \Delta}{\high I}{\Gamma'}
}{
  \ofinstructions{\Psi_{\mathrm{g}}, \Delta}{\high \iota ; \high I}{\Gamma}
} \and
\infer{
  \ofvval{\Psi_1, \Delta}{\high v}{\Gamma => \forall[ \nil ]\Gamma'} \and
  \subtype{\Delta}{\Gamma}{\Gamma'}
}{
  \ofinstructions{\Psi_{\mathrm{g}}, \Delta}{\mathtt{jmp}\ \high v}{\Gamma}
} \and
\infer{
}{
  \ofinstructions{\Psi_{\mathrm{g}}, \Delta}{\mathtt{halt}}{\Gamma}
}
\end{mathpar}

\subsection{Memory Constructs}

\fbox{$\ofwval{\Psi_{\mathrm{g}},\Psi_{\mathrm{h}}}{\high w}{\tau}$}
\begin{mathpar}
\infer{
  \Psi_{\mathrm{g}}[\ell_{\mathrm{g}}] = \tau_1 \and
  \subtype{\nil}{\tau_1}{\tau_2}
}{
  \ofwval{\Psi_{\mathrm{g}},\Psi_{\mathrm{h}}}{\mathtt{globval}\ \ell_{\mathrm{g}}}{\tau_2}
}\and
\infer{
  \Psi_{\mathrm{h}}[\ell_{\mathrm{h}}] = \tau_1 \and
  \subtype{\nil}{\tau_1}{\tau_2}
}{
  \ofwval{\Psi_{\mathrm{g}},\Psi_{\mathrm{h}}}{\mathtt{heapval}\ \ell_{\mathrm{h}}}{\tau_2}
}\and
\infer{ }{\ofwval{\Psi_{\mathrm{g}},\Psi_{\mathrm{h}}}{\mathtt{int}\ i}{\mathtt{int}}} \and
\infer{ }{\ofwval{\Psi_{\mathrm{g}},\Psi_{\mathrm{h}}}{\mathtt{uninit}}{\mathtt{uninit}}} \and
\infer{
  \ofwval{\Psi_{\mathrm{g}},\Psi_{\mathrm{h}}}{\high w}{\forall[\Delta_1]\Gamma_1} \and
  \ofinstantiations{\Delta_2}{\Theta}{\Delta_1} \and
  \subtype{\Delta_2}{\Gamma_2}{\Gamma_1[\Theta]} \\
}{
  \ofwval{\Psi_{\mathrm{g}},\Psi_{\mathrm{h}}}{\Lambda\ \Delta_2 \cdot \high w[\Theta]}{\forall[\Delta_2]\Gamma_2}
}
\end{mathpar}

\fbox{$\ofwvaln{\Psi_{\mathrm{g}},\Psi_{\mathrm{h}}}{\high w}{\tau^\phi}$}
\begin{mathpar}
\infer{
  \valid{\nil}{\tau}
}{
  \ofwvaln{\Psi_{\mathrm{g}},\Psi_{\mathrm{h}}}{\mathtt{uninit}}{\tau^{\mathtt{uninit}}}
} \and
\infer{
  \ofwval{\Psi_{\mathrm{g}},\Psi_{\mathrm{h}}}{\high w}{\tau}
}{
  \ofwvaln{\Psi_{\mathrm{g}},\Psi_{\mathrm{h}}}{\high w}{\tau^{\mathtt{init}}}
}
\end{mathpar}

\fbox{$\ofstack{\Psi_{\mathrm{g}},\Psi_{\mathrm{h}}}{\high S}{\sigma}$}
\begin{mathpar}
\infer{ }{\ofstack{\Psi_{\mathrm{g}},\Psi_{\mathrm{h}}}{\nil}{\nil}} \and
\infer{
  \ofword{\Psi_{\mathrm{g}},\Psi_{\mathrm{h}}}{\high w}{\tau} \and
  \ofstack{\Psi_{\mathrm{g}},\Psi_{\mathrm{h}}}{\high S}{\sigma}
}{
  \ofstack{\Psi_{\mathrm{g}},\Psi_{\mathrm{h}}}{\high w :: \high S}{\tau :: \sigma}
}
\end{mathpar}

\fbox{$\ofregister{\Psi_{\mathrm{g}},\Psi_{\mathrm{h}}}{\high R}{\Gamma}$}
\begin{mathpar}
\infer{
  \ofstack{\Psi_{\mathrm{g}},\Psi_{\mathrm{h}}}{\high S}{\sigma} \and
  \ofwval{\Psi_{\mathrm{g}},\Psi_{\mathrm{h}}}{\high w_1}{\tau_1} \and
  \dots \and
  \ofwval{\Psi_{\mathrm{g}},\Psi_{\mathrm{h}}}{\high w_{\mathrm{max}}}{\tau_{\mathrm{max}}} \and
}{
  \ofregister{\Psi_{\mathrm{g}},\Psi_{\mathrm{h}}}{\{\mathtt{sp} |->  \high S, \mathtt{r}_1 |->  \high w_1, \dots, \mathtt{r}_{\mathrm{max}} |->  \high w_{\mathrm{max}}\}}{\{\mathtt{sp} |-> \sigma, \mathtt{r}_1 |-> \tau_1, \dots, \mathtt{r}_{\mathrm{max}} |-> \tau_{\mathrm{max}}\}}
}
\end{mathpar}

\fbox{$\ofhval{\Psi_{\mathrm{g}},\Psi_{\mathrm{h}}}{\high h}{\tau}$}
\begin{mathpar}
\infer{
  \ofwvaln{\Psi_{\mathrm{g}},\Psi_{\mathrm{h}}}{\high w_1}{\tau_1^{\phi_1}} \and
  \dots \and
  \ofwvaln{\Psi_{\mathrm{g}},\Psi_{\mathrm{h}}}{\high w_n}{\tau_n^{\phi_n}}
}{
  \ofhval{\Psi_{\mathrm{g}},\Psi_{\mathrm{h}}}{\mathtt{tuple}\ \langle \high \tau_1, \dots, \high \tau_n \rangle\ \langle \high w_1, \dots, \high w_n \rangle}{\langle \tau_1^{\phi_1}, \dots, \tau_n^{\phi_n}\rangle}
}
\end{mathpar}

\fbox{$\ofgval{\Psi_{\mathrm{g}}}{\high g}{\tau}$}
\begin{mathpar}
\infer{
  \valid{\Delta}{\Gamma} \and
  \ofinstructions{\Psi_{\mathrm{g}}, \Delta}{\high I}{\Gamma}
}{
  \ofgval{\Psi_{\mathrm{g}}}{\mathtt{code}[\Delta]\Gamma \cdot \high I}{\forall[\Delta]\Gamma}
}
\end{mathpar}

\fbox{$\ofheap{\Psi_{\mathrm{g}}}{\high H}{\Psi_{\mathrm{h}}}$}
\begin{mathpar}
\infer{
  \ofhval{\Psi_{\mathrm{g}},\Psi_{\mathrm{h}}}{\high h_1}{\tau_1} \and
  \dots \and
  \ofhval{\Psi_{\mathrm{g}},\Psi_{\mathrm{h}}}{\high h_n}{\tau_n} \and
  \Psi_{\mathrm{h}} = \{\ell_{h,1} |-> \tau_1, \dots, \ell_{h,n} |-> \tau_n\}
}{
  \ofheap{\Psi_{\mathrm{g}}}{\{\ell_{h,1} |->  \high h_1, \dots, \ell_{h,n}  |->  \high h_n\}}{\Psi_{\mathrm{h}}}
}
\end{mathpar}

\fbox{$\ofglobs{\high G}{\Psi_{\mathrm{g}}}$}
\begin{mathpar}
\infer{
  \ofgval{\Psi_{\mathrm{g}}}{\high g_1}{\tau_1} \and
  \dots \and
  \ofgval{\Psi_{\mathrm{g}}}{\high g_n}{\tau_n} \and
  \Psi_{\mathrm{g}} = \{\ell_{g,1} |-> \tau_1, \dots, \ell_{g,n} |-> \tau_n\}
}{
  \ofglobs{\{\ell_{g,1} |->  \high g_1, \dots, \ell_{g,n}  |->  \high g_n\}}{\Psi_{\mathrm{g}}}
}
\end{mathpar}

\subsection{Program States}

\fbox{$\ofprogramstate{\Psi_{\mathrm{g}}}{(\high H, \high R, \high I)}{(\Psi_{\mathrm{h}},\Gamma)}$}
\begin{mathpar}
\infer{
  \ofheap{\Psi_{\mathrm{g}}}{\high H}{\Psi_{\mathrm{h}}} \\
  \ofregister{\Psi_{\mathrm{g}},\Psi_{\mathrm{h}}}{\high R}{\Gamma} \and
  \ofinstructions{\Psi_{\mathrm{g}},\nil}{\high I}{\Gamma}
}{
  \ofprogramstate{\Psi_{\mathrm{g}}}{(\high H,\high R,\high I)}{(\Psi_{\mathrm{h}},\Gamma)}
}
\end{mathpar}

\fbox{$\valid{}{\high P}$}
\begin{mathpar}
\infer{
  \ofglobs{\high G}{\Psi_{\mathrm{g}}} \and
  \ofprogramstate{\Psi_{\mathrm{g}}}{(\high H, \high R, \high I)}{(\Psi_{\mathrm{h}}, \Gamma)}
}{
  \valid{}{(\high G,\high H, \high R, \high I)}
}
\end{mathpar}

\subsection{Notes on the Agda Implementation}
\texttt{Judgments/Terms.agda}

\subsection{Lemmas}

\texttt{Lemmas/TermWeaken.agda} and \texttt{Lemmas/TermSubstitution.agda}.

\begin{lemma}
  \label{lemma:instructions-context}
  The judgment $\ofinstructions{\Psi_{\mathrm{g}}, \Delta}{\high I}{\Gamma}$ is
  preserved by weakening or instantiation variables
  from the context.\footnote{Remember that substitution is a partial function in
    the Agda-code. In the Agda-code this lemma also implies totality of
    substitution given certain additional assumptions.}
\end{lemma}

\texttt{Lemmas/TermCast.agda}
\begin{lemma}
  \label{lemma:evaluatin-casting}
  \begin{itemize}
  \item Assume $\ofvval{\Psi_{\mathrm{g}},\Delta}{\high v}{\Gamma => \tau}$
    and $\subtype{\Delta}{\Gamma'}{\Gamma}$. In that case there is some
    $\tau'$ such that
    $\ofvval{\Psi_{\mathrm{g}},\Delta}{\high v}{\Gamma' => \tau'}$.

  \item Assume $\ofinstruction{\Psi_{\mathrm{g}},\Delta}{\high \iota}{\Gamma_1 => \Gamma_2}$
    and $\subtype{\Delta}{\Gamma_1'}{\Gamma_1}$. In that case there is some
    $\Gamma_2'$ such that
    $\ofinstruction{\Psi_{\mathrm{g}},\Delta}{\high \iota}{\Gamma_1' => \Gamma_2'}$.

  \item Assume $\ofinstructions{\Psi_{\mathrm{g}},\Delta}{\high I}{\Gamma}$ and
    $\subtype{\Delta}{\Gamma'}{\Gamma}$. In that case
    $\ofinstructions{\Psi_{\mathrm{g}},\Delta}{\high I}{\Gamma'}$.
  \end{itemize}
\end{lemma}

\begin{lemma}
  \label{lemma:value-casting}
  Assume $\ofwval{\Psi_{\mathrm{g}},\Psi_{\mathrm{h}}}{\high w}{\tau}$ and that
  $\subtype{\nil}{\tau}{\tau'}$. In that case
  $\ofwval{\Psi_{\mathrm{g}},\Psi_{\mathrm{h}}}{\high w}{\tau'}$. Similar
  statements hold for stacks, register files and heap values.
\end{lemma}

\texttt{Lemmas/TermHeapCast.agda}
\begin{lemma}
  \label{lemma:heap-casting}
  Assume $\ofwval{\Psi_{\mathrm{g}},\Psi_{\mathrm{h}}}{\high w}{\tau}$ and that
  $\subtype{\nil}{\Psi_{\mathrm{h}}'}{\Psi_{\mathrm{h}}}$. In that case
  $\ofwval{\Psi_{\mathrm{g}},\Psi_{\mathrm{h}}'}{\high w}{\tau}$. Similar
  statements hold for stacks, register files and heap values.
\end{lemma}

\section{Soundness Proof}

This section will give a very broad outline of the proof of soundness. The
actual proof is implemented in \texttt{Lemmas/Soundness.agda} and has the
type-signature:

\begin{code}
\>\AgdaFunction{steps-soundness} \AgdaSymbol{:} \AgdaSymbol{∀} \AgdaSymbol{\{}\AgdaBound{n} \AgdaBound{P₁} \AgdaBound{P₂}\AgdaSymbol{\}} \AgdaSymbol{→}\<%
\\
\>[2]\AgdaIndent{20}{}\<[20]%
\>[20]\AgdaDatatype{⊢} \AgdaBound{P₁} \AgdaDatatype{programstate} \AgdaSymbol{→}\<%
\\
\>[2]\AgdaIndent{20}{}\<[20]%
\>[20]\AgdaFunction{⊢} \AgdaBound{P₁} \AgdaFunction{⇒ₙ} \AgdaBound{n} \AgdaFunction{/} \AgdaBound{P₂} \AgdaSymbol{→}\<%
\\
\>[2]\AgdaIndent{20}{}\<[20]%
\>[20]\AgdaFunction{GoodState} \AgdaBound{P₂}\<%
\end{code}

For the actual proof of this statement, we will refer to the file, however we
have also included a proof sketch in \cref{sec:soundness-proof}. This proof
sketch for the most part follows the structure of the proof in the file. Here we
will instead discuss the very broad lines included how the language was designed
to permit this result.

Soundness is defined in ``Types and Programming Languages''\cite{typesandprog}
as type systems that prevent terms from becoming stuck\footnote{In the context
  in the book, terms are either \emph{values}, \emph{stuck} or they can
  transition to another term. This corresponds loosely to our definitions of
  \emph{halting}, \emph{stuck} or \emph{progressing}.} during the
run-time. Instead of formalizing soundness directly, they instead require the
following two lemmas to be provable:

\begin{theorem}[Progress]
  Suppose $\mathtt{t}$ is a well-typed term (that is, $\mathtt{t} : \mathtt{T}$
  for some $\mathtt{T}$). Then either $\mathtt{t}$ is a value\footnote{In TAPL,
    this refers to a value in canonical form.} or else there is some
  $\mathtt{t'}$ with $\mathtt{t} -> \mathtt{t'}$.
\end{theorem}

\begin{theorem}[Preservation]
  If $\mathtt{t} : \mathtt{T}$ and $\mathtt{t} -> \mathtt{t'}$, then
  $\mathtt{t'} : \mathtt{T}$.
\end{theorem}

It is clear that progress and preservation implies soundness. By modifying these
theorems slightly, they can be fit into the our type system for \ATAL:

\begin{theorem}[\ATAL Progress]
  \label{lemma:atalprogress}
  Suppose $\high P$ is a well-typed program state (that is,
  $\valid{}{\high P}$). Then $\high P$ is not struck.
\end{theorem}

\begin{theorem}[\ATAL Preservation]
  \label{lemma:atalpreservation}
  If $\valid{}{P}$ and $\high P -> \high P'$, then $\valid{}{P'}$.
\end{theorem}

For \ATAL will not try to prove progress and preservation \emph{directly}
(though it should be possible to do so). We will instead prove the following
theorem:

\begin{theorem}[\ATAL Typed progress]
  \label{lemma:atalprogress+}
  Suppose $\high P$ is a well-typed program state (that is,
  $\valid{}{\high P}$). Then $\high P$ is not stuck. Additionally, if it
  transitions to some $\high P'$, then $\valid{}{\high P'}$.
\end{theorem}

In our Agda-code this is represented by the function:

\begin{code}
\>\AgdaFunction{step-progress⁺} \AgdaSymbol{:} \AgdaSymbol{∀} \AgdaSymbol{\{}\AgdaBound{P}\AgdaSymbol{\}} \AgdaSymbol{→}\<%
\\
\>[2]\AgdaIndent{19}{}\<[19]%
\>[19]\AgdaDatatype{⊢} \AgdaBound{P} \AgdaDatatype{programstate} \AgdaSymbol{→}\<%
\\
\>[2]\AgdaIndent{19}{}\<[19]%
\>[19]\AgdaFunction{Halting} \AgdaBound{P} \AgdaDatatype{∨}\<%
\\
\>[2]\AgdaIndent{19}{}\<[19]%
\>[19]\AgdaSymbol{(}\AgdaFunction{∃} \AgdaSymbol{λ} \AgdaBound{P'} \AgdaSymbol{→}\<%
\\
\>[19]\AgdaIndent{21}{}\<[21]%
\>[21]\AgdaDatatype{⊢} \AgdaBound{P'} \AgdaDatatype{programstate} \AgdaFunction{×}\<%
\\
\>[19]\AgdaIndent{21}{}\<[21]%
\>[21]\AgdaFunction{⊢} \AgdaBound{P} \AgdaFunction{⇒} \AgdaBound{P'}\AgdaSymbol{)}\<%
\end{code}

It is obvious that typed progress implies progress, since it is a strictly
stronger result. The proof that typed progress implies preservation depends on
our previous result that our small-step relation is deterministic:

\begin{proof}
  Assume $\valid{}{\high P}$ and $\high P -> \high P'$. By applying typed
  progress there are two possibilities:

  \begin{itemize}
  \item Either $\high P$ is halting. However this is a contradiction, since we
    have a derivation of $\high P -> \high P'$.

  \item Otherwise there is some $\high P''$ with $\high P -> \high P''$ and
    $\valid{}{\high P''}$. In this case use \cref{lemma:computhigh}, which
    implies that $\high P' = \high P''$. This means that our derivation of
    $\valid{}{\high P''}$ is also a derivation of $\valid{}{\high P'}$.
  \end{itemize}
\end{proof}

The proof of typed progress is not particularly interesting in itself -- we have
already seen most of the lemmas needed to prove it. For a sketch of how the
proof was structured, see \cref{sec:soundness-proof}.

\subsection{Soundness and Erasure}

Now that we have proven soundness for \ATAL along with our erasure property, we
are able to combine the two into a soundness proof for \ATALe:

\begin{theorem}[Soundness for \ATALe]
  Assume we have $\high P$ such that $\valid{}{\high P}$ and
  $\erase(\high P) -> \dots -> \simple{P}$ in the \ATALe small-step semantics
  using $n$ steps. In that case $\simple{P}$ is not stuck.
\end{theorem}
\begin{proof}
  By induction over the derivation of $\erase(\high P) -> \dots -> \simple{P}$.

  First assume that $n=0$. In that case we have $\erase(\high P) =
  \simple{P}$. By \cref{lemma:atalpreservation} we know that $\high P$ is not
  stuck, i.e. it is either halting or progressing. If $\high P$ halting, then it
  follows directly from the definition of $\erase$ that $\erase(\high P)$ is
  halting too. If $\high P$ is progressing, then by \cref{lemma:erasesteps} we
  know that $\erase(\high P)$ is too.

  Assume alternatively that $n>0$, in which case we must have
  $\erase(\high P) -> \simple{P'} -> \dots -> \simple{P}$. By using
  \cref{lemma:erasestepsbck} on the derivation for
  $\erase(\high P) -> \simple{P'}$ we know that either $\high P$ is stuck or it
  transitions to some $\high P'$. However by \cref{lemma:atalprogress} we know
  that $\high P$ is not stuck. We now have a derivation of
  $\erase(\high P') -> \dots -> \simple{P}$ of size $n-1$. To use the induction
  hypothesis on this derivation, we simply need a proof of $\valid{}{\high P'}$,
  however this follows from \cref{lemma:atalpreservation}.
\end{proof}

\section{Decidability}
\label{sec:decidability}

In this section we introduce our proof that the type system is decidable. In
particular the file \texttt{Lemmas/TermDec.agda} contains the function:

\begin{code}
\>\AgdaFunction{programstate-dec} \AgdaSymbol{:} \AgdaSymbol{∀} \AgdaBound{P} \AgdaSymbol{→} \AgdaDatatype{Dec} \AgdaSymbol{(}\AgdaDatatype{⊢} \AgdaBound{P} \AgdaDatatype{programstate}\AgdaSymbol{)}\<%
\end{code}

The actual proof is not particularly interesting, though we have given a sketch
of it in \cref{sec:decproof}. What is more interesting to discuss is how our
language was designed to allow this result to be achieved. To see this, it is
beneficial to interpret \texttt{programstate-dec} not as a proof, but as the
algorithm for our type-checker. Consider how it would type-check a program-state
$(\high G, \high H, \high R, \high I)$.

\paragraph{Step 1, type-check the globals.} We want to know if
$\ofglobs{\high G}{\Psi_{\mathrm{g}}}$ for some $\Psi_{\mathrm{g}}$. However we
cannot check each individual global value \emph{before} we know the value of
$\Psi_{\mathrm{g}}$; an apparent contradiction.

We solved this problem by adding a type to the constructor of our global
values. This allows us to type-check $\high G$ using two passes: The first pass
is used to generate $\Psi_{\mathrm{g}}$, since each global value can only have a
single unique type. The second pass then checks if the global values are
actually well-typed when using $\Psi_{\mathrm{g}}$ as the context. Finally we
can return the output of this step, which is either:

\begin{itemize}
\item A value for $\Psi_{\mathrm{g}}$ such that
  $\ofglobs{\high G}{\Psi_{\mathrm{g}}}$ is derivable.
\item A proof that no value of $\Psi_{\mathrm{g}}$ can make
  $\ofglobs{\high G}{\Psi_{\mathrm{g}}}$ derivable, in which case the entire
  type-checker fails.
\end{itemize}

\paragraph{Step 2, type-check the heap.} A similar two-pass approach is used to
check the heap. Here we have a slight complication, since heap-values does not
have a single unique type; they can be casted to less specific values. Instead
we find the most specific type for each heap value. This works, because
\cref{lemma:heap-casting} states that types are preserved even if we choose a
more specific heap. This means that the output from this step is either:

\begin{itemize}
\item A value for $\Psi_{\mathrm{h}}$ such that
  $\ofheap{\Psi_{\mathrm{g}}}{\high H}{\Psi_{\mathrm{h}}}$ is derivable. Furthermore we
  output a proof that any $\Psi_{\mathrm{h}}'$ making
  $\ofheap{\Psi_{\mathrm{g}}}{\high H}{\Psi_{\mathrm{h}}'}$ derivable will have
  $\subtype{}{\Psi_{\mathrm{h}}}{\Psi_{\mathrm{h}}'}$.
\item A proof that no value of $\Psi_{\mathrm{h}}$ can make
  $\ofheap{\Psi_{\mathrm{g}}}{\high H}{\Psi_{\mathrm{h}}}$ derivable, in which
  case the entire type-checker fails.
\end{itemize}

\paragraph{Step 3, type-check the register file.} We no longer have the
ambiguity associated with not knowing $\Psi_{\mathrm{g}}$ or
$\Psi_{\mathrm{h}}$. We will again find the most specific type
instead. Specifically this step will either either:

\begin{itemize}
\item A value for $\Gamma$ such that
  $\ofregister{\Psi_{\mathrm{g}},\Psi_{\mathrm{h}}}{\high R}{\Gamma}$ is
  derivable. Furthermore we output a proof that any $\Gamma'$ making
  $\ofregister{\Psi_{\mathrm{g}},\Psi_{\mathrm{h}}}{\high R}{\Gamma'}$ derivable
  will have $\subtype{}{\Gamma}{\Gamma'}$.
\item Proof that no value of $\Gamma$ can make
  $\ofregister{\Psi_{\mathrm{g}},\Psi_{\mathrm{h}}}{\high R}{\Gamma}$ derivable,
  in which case the entire type-checker fails.
\end{itemize}

\paragraph{Step 4, type-check the instruction sequence.} Compared to the
previous step, this step is somewhat forward: We have been given all of the
types, and simply need to check if
$\ofinstructions{\Psi_{\mathrm{g}}, \Delta}{\high I}{\Gamma}$ is derivable.

\paragraph{Combining it all.} If we ignore all the generated proofs, this can be
summed up using a very simple algorithm:

\begin{enumerate}
\item Scan $\high G$ superficially to find the only possible value of
  $\Psi_{\mathrm{g}}$.
\item Traverse $\high G$ in depth using the found value. This time to check if
  $\ofglobs{\high G}{\Psi_{\mathrm{g}}}$ is actually derivable. If not, then
  abort the algorithm.
\item Scan $\high H$ superficially to find the most specific value of
  $\Psi_{\mathrm{h}}$.
\item Traverse $\high H$ in depth using the found value. This time check if
  $\ofheap{\Psi_{\mathrm{g}}}{\high H}{\Psi_{\mathrm{h}}}$ is actually
  derivable. If not, then abort the algorithm.
\item Traverse $\high R$ in depth to check if
  $\ofregister{\Psi_{\mathrm{g}},\Psi_{\mathrm{h}}}{\high R}{\Gamma}$ is
  derivable for some value of $\Gamma$. If it is, find the most specific value
  of $\Gamma$, if not then abort the algorithm.
\item Traverse $\high I$ in depth see if
  $\ofinstructions{\Psi_{\mathrm{g}}, \Delta}{\high I}{\Gamma}$ is derivable.
\end{enumerate}
