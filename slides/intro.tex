\section{Introduction to Typed Assembly Language}
\begin{frame}
\frametitle{Table of Contents}
\tableofcontents[currentsection]
\end{frame}

\begin{frame}{Our goal}
  \begin{itemize}
  \item Fast
  \item \only<1-3>{Safe}\only<4->{\sout{Safe}\ \ Verifiably safe}
  \item Practical
  \item Low-level
  \end{itemize}

  \pause Use Rust...? \pause Not quite good enough.
\end{frame}

\begin{frame}{Programs}{}
  \begin{tikzpicture}[mystyle]
    \node[draw,circle,fill,red] (a) at (0,6.5) {};
    \node[right] at (0,6.5) {\footnotesize \ all programs};
    \draw[fill,red] (0,0) rectangle (10, 6);

    \pause
    \node[draw,circle,fill,blue] (b) at (2.5,6.5) {};
    \node[right,blue] at (2.5,6.5) {\footnotesize \ safe programs};
    \draw[fill,blue] (5,3) ellipse (4.0 and 2.2);
  \end{tikzpicture}
\end{frame}

\begin{frame}{Safety?}
  \textbf{Definition:} A safe program is one that does not access unmapped
  memory.

  \pause \textbf{Implications:}

  \begin{itemize}
  \pause\item Has no buffer overflows
  \pause\item Has no pointer corruption
  \pause\item Cannot execute arbitrary shellcode
  \pause\item Will allow coorporative sharing of a unified memory space
  \pause\item Will allow low-overhead sandboxes!
  \end{itemize}
\end{frame}

\begin{frame}[fragile]{First attempt}
  \textbf{Idea:} Code in a high-level Language!

  \pause\includegraphics[width=\textwidth]{attempt1_dot.pdf}
\end{frame}

\begin{frame}{Programs}{}
  \begin{tikzpicture}[mystyle]
    \node[draw,circle,fill,red] (a) at (0,6.7) {};
    \node[right] at (0,6.7) {\footnotesize \ all programs};
    \draw[fill,red] (0,0) rectangle (10, 6);

    \node[draw,circle,fill,blue] (b) at (0,6.3) {};
    \node[right] at (0,6.3) {\footnotesize \ safe programs};
    \draw[fill,blue] (5,3) ellipse (4.0 and 2.2);

    \pause
    \node[draw,circle,fill,yellow] (b) at (2.5,6.7) {};
    \node[right] at (2.5,6.7) {\footnotesize \ rust programs};
    \draw[fill,yellow] (7,3) circle (2.5);

    \node[draw,circle,fill,green] (b) at (2.5,6.3) {};
    \node[right] at (2.5,6.3) {\footnotesize \ rust programs without \texttt{unsafe}};
    \draw[fill,green] (7,2.65) circle (1.5);
  \end{tikzpicture}
\end{frame}

\begin{frame}[fragile]{First attempt}
  \textbf{Idea:} Code in a high-level Language!

  \includegraphics[width=\textwidth]{attempt1_dot.pdf}

  \pause How does the user know that it is safe?
\end{frame}

\begin{frame}[fragile]{Second attempt}
  \textbf{Idea:} Distribute programs in a high-level language.

  \pause \includegraphics[width=\textwidth]{attempt2_dot.pdf}
\end{frame}

\begin{frame}{Trusted Computing Base}
  \textbf{Definition:} The Trusted Computing Base (TCB) of a computer system is
  the set of software and hardware that is critical to its security.
\end{frame}

\begin{frame}[fragile]{Second attempt}
  \textbf{Idea:} Distribute programs in a high-level language.

  \includegraphics[width=\textwidth]{attempt2_dot.pdf}

  \pause This might work, \textbf{but}: The Rust compiler is huge; too big to be
  trusted. (It is also too slow to be practical).
\end{frame}

\begin{frame}{Programs}{}
  \begin{tikzpicture}[mystyle]
    \node[draw,circle,fill,red] (a) at (0,6.7) {};
    \node[right] at (0,6.7) {\footnotesize \ all programs};
    \draw[fill,red] (0,0) rectangle (10, 6);

    \node[draw,circle,fill,blue] (b) at (0,6.3) {};
    \node[right] at (0,6.3) {\footnotesize \ safe programs};
    \draw[fill,blue] (5,3) ellipse (4.0 and 2.2);

    \node[draw,circle,fill,yellow] (b) at (2.5,6.7) {};
    \node[right] at (2.5,6.7) {\footnotesize \ rust programs};
    \draw[fill,yellow] (7,3) circle (2.5);

    \node[draw,circle,fill,green] (b) at (2.5,6.3) {};
    \node[right] at (2.5,6.3) {\footnotesize \ rust programs without \texttt{unsafe}};
    \draw[fill,green] (7,2.65) circle (1.5);

    \pause
    \draw[->] (9,0.5) -- (7.9,1.3);
  \end{tikzpicture}
\end{frame}

\begin{frame}[fragile]{Third attempt}
  \textbf{Idea:} Distribute programs in bytecode.

  \pause \includegraphics[width=\textwidth]{attempt3_dot.pdf}

  \pause Better. TCB = JIT-compiler and standard-library.
\end{frame}

\begin{frame}[fragile]{Typed Assembly Language}
  \textbf{Idea:} Distribute a proof along with programs.

  \pause \includegraphics[width=\textwidth]{attempt4_dot.pdf}

  \pause TCB = Type-checker, finalizer and optionally a standard-library. But we
  can do better!
\end{frame}

\begin{frame}{Typed Assembly Language}{}
  \begin{tikzpicture}[mystyle]
    \node[draw,circle,fill,red] (a) at (0,7.0) {};
    \node[right] at (0,7.0) {\footnotesize \ all programs};

    \node[draw,circle,fill,green] (a) at (2.5,7.0) {};
    \node[right] at (2.5,7.0) {\footnotesize \ safe programs};

    \node[draw,circle,fill,blue] (a) at (5.0,7.0) {};
    \node[right] at (5.0,7.0) {\footnotesize \ well-typed programs};

    \node at (1.4, 6.2) {$ATAL$};
    \node at (4.4, 6.2) {$ATAL_e$};
    \node at (7.4, 6.2) {native};

    \draw[fill,red] (0,0) rectangle (2.8, 6);
    \draw[fill,red] (3,0) rectangle (5.8, 6);
    \draw[fill,red] (6,0) rectangle (8.8, 6);

    \draw[fill,green] (1.4,3) ellipse (1.3 and 2.9);
    \draw[fill,green] (4.4,3) ellipse (1.3 and 2.9);
    \draw[fill,green] (7.4,3) ellipse (1.3 and 2.9);

    \draw[fill,blue] (1.6,3.1) ellipse (1 and 2.4);

    \draw[->] (2.5,6.1) to [out=45, in=135] (3.5,6.1);
    \draw[->] (5.5,6.1) to [out=45, in=135] (6.5,6.1);
    \node at (3.0,6.5) {erasure};
    \node at (6.0,6.55) {assembler};
  \end{tikzpicture}
\end{frame}

\begin{frame}{Typed Assembly Language}
  \textbf{Naive:} TCB = type-checker, erasure, assembler.

  \pause
  \textbf{Smarter:}

  \begin{itemize}
  \item Implement type-checker and erasure in Agda.
  \item Prove that type-checker is correct.
  \item Prove that erasure is correct.
  \end{itemize}

  TCB = Agda, assembler.
\end{frame}
