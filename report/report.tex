\documentclass[a4paper]{report}
\usepackage[utf8]{inputenc}
\usepackage[english]{babel}
\usepackage{amsmath, amssymb, amsthm}
\usepackage{stmaryrd}
\usepackage{mathpartir}
\usepackage{semantic}
\usepackage{titling}
\usepackage[pdftex,dvipsnames]{xcolor}
\usepackage[lang=en,grid]{kufront}

\usepackage[colorinlistoftodos,prependcaption,textsize=tiny]{todonotes}
\newcommand{\later}[2][1=]{\todo[linecolor=Plum,backgroundcolor=Plum!25,bordercolor=Plum,#1]{#2}}

\author{\Large Mathias Svensson \quad {\ttfamily\large tpx783@alumni.ku.dk}}
\date{\today}
\title{A Rust-like Typed Assembly Language}
\project{\mdseries\LARGE Master's Thesis}
\supervisor{Supervisor: {\large Ken Friis Larsen} \texttt{kflarsen@di.ku.dk} \\[0.5cm]
            Co-supervisor: {\large Andrzej Filinski} \texttt{andrzej@di.ku.dk}}

\usepackage[pdftitle={\thetitle}, pdfauthor={\theauthor}, linkbordercolor={0.8 0.8 0.8}]{hyperref}

\newtheorem{theorem}{Theorem}[section]
\newtheorem{definition}[theorem]{Definition}
\newtheorem{lemma}[theorem]{Lemma}
\newtheorem{corollary}[theorem]{Corollary}

\mathlig{|-}{\vdash}

\newcommand \valid[2] {{#1} |- {#2}\ \mathbf{Valid}}
\newcommand \subtype[3] {{#1} |- {#2} \le {#3}}


\begin{document}

\begin{titlepage}
\maketitle
\end{titlepage}
\tableofcontents

\begin{abstract}
  Typed Assembly Language (TAL) is an interesting technique, as it allows
  provably safe distribution and execution of untrusted code, without assuming
  the presence of a trusted third-party. In this thesis we give a thorough
  introduction to TAL including how it is built ``on top'' of a real-world
  language. We then formalize a version of TAL that is mostly compatible with
  the one presented in the paper ``Stack-Based Typed Assembly Language''.

  Our contributions in this thesis is two-fold:

  \begin{itemize}
  \item We formalize the metatheory of our language. Specifically we prove
    soundness, decidable type-checking and a erasure theorem using Agda.
  \item While creating the Agda-code several challenges were encountered and
    overcome. This gave rise to a number of smaller, but nonetheless interesting
    techniques that could likely be relevant outside Typed Assembly
    Language. These ``lessons learned'' have been included in a chapter meant to
    be readable independently of the rest of the report.
  \end{itemize}
\end{abstract}


\chapter{Introduction}
\section{Motivation}

Let us say, that you are a software developer trying to develop er piece of code
for a client client. You are considering in what format to distribute your code,
and as the choices seem overwhelming you decide to list a few properties you
would like your code to have before you distribute it.

\paragraph{Fast startup:} You would like the code to be executable as fast as
possible after your client receives it. You might also want to distribute the
source code through a different channel, but that is not one of the current goal
criterias.

\paragraph{Low overhead:} You would like your code to use as few resources as
possible and run as fast as possible.

\paragraph{Reuseability:} You would ideally want a solution that is reusable for
other projects. Thus you would prefer solutions that allows general-purpose
code.

\paragraph{Security:} Ideally you would like to convince your client beyond any
doubt, that the code serves the intended function. However your client have had
problems even giving you a coherent design specification, much less a formalized
one; so you decide to settle for a less ambiguous goal, namely certainty that it
is not \emph{malicious}.

\paragraph{}
The when you combine the first two requirements, you are left with very few
options:

\begin{enumerate}
\item You could distribute raw machine code (inside a suitable container format
  such as \texttt{PE} or \texttt{ELF}).
\item You could distribute a bytecode format intended to be run through a fast
  interpreter or JIT-compiler.
\item You write a custom interpreter, only able to run specialized code, however
  not without compromising on \textbf{reuseability}.
\end{enumerate}

You are now left with the problem of convincing your client that your code is in
fact not malicious. You decide to do a small real-world survey to see how other
people have solved this problem:

\begin{itemize}
\item Do nothing! Use the fact most people are trustful of code they receive,
  even though they should not be.
\item Expect the client to run anti-virus software to detect malicious code.
\item Buy a code-signing certificate, and sign your code. The security of this
  system is based on the fact that a code-signing certificates are
  semi-expensive and only issued to entities verified to be non-malicious by a
  trusted certificate authority.
\item Have the client run the code inside a sandbox environment, by using
  e.g. the Java VM, Docker or Native Client from the Chromium project.
\item Publish your code as open source along with a method for compiling that
  source into a bitwise identical result.
\item Some combination of the above.
\end{itemize}

You notice that none of the real-world solutions are perfect!

\begin{itemize}
\item Rice's Theorem states that the anti-virus model is inherently flawed. In
  fact, it is a quite common occurrence for real-world anti-virus software to
  produce both false positives and false negatives.
\item It is way too easy to either steal a code-signing certificate, create a
  fake identity to fool the trusted certificate authority, or straight out
  convince them to give you a certificate in any case.
\item Every sandbox in existence has either had methods of sandbox escape, or
  have unreasonable demands on the other properties.
\item Even open source software with security audits sometimes have bugs or even
  backdoors.
\end{itemize}

\later[inline]{A lot of citations.}

Desperately you turn to research: Is there really no solution in sight? It turns
out there is: Proof-carrying code.

\clearpage
\section{Proof Carrying Code}

\later[inline]{Definition of proof-carrying code}

\later[inline]{Informal definition of typed assembly language}

\clearpage
\section{Problem Statement}

\later[inline]{Problem statement}


\chapter{Typed Assembly Language}
This chapter has three parts. First we will formally introduce the concepts of
Typed Assembly Language in an informal setting. We will then give a formal
definition of the same concepts. Finally we will introduce the specific instance
of Typed Assembly Language used in \cite{STAL}.

Note that I have chosen a specific way of presenting Typed Assembly Language,
that lends itself well to formalization. While this approach is not completely
similar to the approach used in \cite{STAL}, I expect that the concepts should
be fairly recognizable.

\section{Informal definition}
\paragraph{The native language} is the language on which the rest of the work is
based, typically a real-world assembler language. It comes with a notion of a
\textbf{native semantics}, defined by an external party. This means that we can
model the language as a non-deterministic state machine. Note that in this
semantics, a program can \emph{always} progress, however sometimes this progress
results termination for running invalid code.

In practice this native language can be a lot of different things. Some of these
are:

\begin{itemize}
\item The set of possible states a specific Pentium 4 can be in. The semantics
  are mostly those defined in the Intel Specification\cite{intelsys}, however
  the specific CPU might deviate slightly from this specification because of
  bugs or unspecified behavior.
\item The set of possible states of a process running on Linux version 4.2.0-18
  on a specific Mips-CPU. While the semantics will obviously be mostly
  equivalent to those defined in the Mips Specification\cite{mipssys} and in the
  Posix Standard\cite{posix}, there are a lot of undefined Linux-specific
  behavior that will depend heavily on the kernel version.
\end{itemize}


\paragraph{The simplified language} is a simplification of the raw language
along with \emph{simplified semantics} to match it. While the native language
typically deal with a complicated memory model, where there is no real
difference between code and data, pointers and numbers, this is not the case in
the simplified semantics. This simplification serves multiple purposes:

\begin{itemize}
\item The native semantics are typically not written as a formal
  specification. This means that they might contain both under-specified
  behavior in certain corner cases and very complicated behavior in others.

  By simplifying we avoid both of these problems.
\item Similarly we might be interested in talking about multiple different
  native languages, that only differ very slightly in their behavior. By
  simplifying the semantics, we are able to talk about multiple similarly
  machines are the same time.
\item The native semantics might contain semantics that are undesirable. For
  instance it might halt or crash when certain actions occur. In our
  simplification we purposefully avoid undesirable in our semantics, and instead
  leave these behaviors undefined.

  \textbf{Note:} This means that the semantic relation is no longer a total
  left-total.
\end{itemize}

\paragraph{The typed language} is an extension of this simplified language, with
the purpose of introducing typing judgments. Specifically it differs from the
simplified semantics in the following ways:

\begin{itemize}
\item The typed language contains annotations not present in this simplified
  language. If these annotations are thrown away, then we arrive at the
  simplified language.

\item The typed semantics might be more restrictive than the simplified
  semantics, however they are still in a certain sense ``compatible'' with the
  simplified semantics.

\item The typed language also comes with a number of typing judgments.
\end{itemize}

\section{Formal Definition}

\begin{definition}
  A \textbf{simplfied assembly language} is a tuple

  $L = (\mathcal{G}, \mathcal{H}, \mathcal{R}, \mathcal{I}, \mathcal{T}, R_\to)$
  such that:

  \begin{itemize}
  \item $\mathcal{G}$ a set, called the \emph{set of possible global states}.
  \item $\mathcal{H}$ a set, called the \emph{set of possible heap states}.
  \item $\mathcal{R}$ a set, called the \emph{set of possible register files}.
  \item $\mathcal{I}$ a set, called the \emph{set of possible instruction
      sequences}.
  \item $\mathcal{T}$ a set, called the \emph{set of possible termination
      states}.
  \item Write as $\mathcal{S}$ short hand for
    $\mathcal{H} \times \mathcal{R} \times \mathcal{I}$ and $\mathcal{S}^T$ as
    shorthand for $\mathcal{S} \cup \mathcal{T}$.
  \item $\mathcal{S} \cap \mathcal{T} = \emptyset$.
  \item $R_\to$ is a partial function
    $\mathcal{G} \times \mathcal{S} \to \mathcal{S}^T$,
    called the \emph{transition relationtion}.
  \item Write $R_\to^\ast$ for the transitive closure of $R_\to$.
  \end{itemize}

  Assume $S_1 \in \mathcal{S}$ and $S_2 \in \mathcal{S}^T$. We will write
  $G |- S_1 \to_L S_2$ as shorthand for $(G, S_1, S_2) \in R_\to$.  When $L$ can
  be inferred without ambiguity, we will leave it out. We will use similar
  notation with $\to^\ast$ and $R_\to^\ast$.
\end{definition}

\begin{definition}
  Let $L$ be a simplified assembly language. We say that
  $(\mathbf{\Psi}_g, \mathbf{\Psi}_h, \mathbf{\Gamma})$ is a \emph{typing
    scheme} for $L$ iff:

  \begin{itemize}
  \item $\mathbf{\Psi}_g$ is a set, called the \emph{set of global state
      types}. Let $\Psi_g$ be the typical member of this set.
  \item $\mathbf{\Psi}_h$ is a set, called the \emph{set of heap state
      types}. Let $\Psi_h$ be the typical member of this set.
  \item $\mathbf{\Gamma}$ is a set, called the \emph{set of register types}. Let
    $\Gamma$ be the typical member of this set.
  \item There is a judgment of the form $|- G : \Psi_g$.
  \item There is a judgment of the form $\Psi_g |- H : \Psi_h$.
  \item There is a judgment of the form $\Psi_g , \Psi_h |- R : \Gamma$.
  \item There is a judgment of the form
    $\Psi_g , \Psi_h , \Gamma |- I\ \mathbf{Valid}$.
  \end{itemize}

  We define $G |- S\ \mathbf{Valid}$ as a judgment only one rule:
  \begin{mathpar}
    \infer{
      |- G : \Psi_g \and
      \Psi_g |- H : \Psi_h \and
      \Psi_g , \Psi_h |- R : \Gamma \and
      \Psi_g , \Psi_h , \Gamma |- I\ \mathbf{Valid}
    }{
      G |- (H, R, I)\ \mathbf{Valid}
    }
  \end{mathpar}

  We call $L$ for a \emph{typed assembly language}, if it has a typing scheme.
\end{definition}

\begin{definition}
  Let $L$ be a typed assembly language. We say that it is \emph{progressing} iff
  for any given $G \in \mathcal{G}, S \in \mathcal{S}$ with

  \begin{itemize}
  \item $G |- S\ \mathbf{Valid}$
  \end{itemize}

  we can conclude that there is some $S' \in \mathcal{S}^T$ such that

  \begin{itemize}
  \item $G |- S \to S'$.
  \end{itemize}
\end{definition}

\begin{definition}
  Let $L$ be a typed assembly language. We say that it is \emph{reducing} iff
  for any given $G \in \mathcal{G}, S, S' \in \mathcal{S}$ with

  \begin{itemize}
  \item $G |- S\ \mathbf{Valid}$
  \item $G |- S \to S'$.
  \end{itemize}

  we can conclude that:

  \begin{itemize}
  \item $G |- S'\ \mathbf{Valid}$
  \end{itemize}
\end{definition}

\begin{definition}
  Let $L$ be a typed assembly language. We say that it is \emph{sound} iff it is
  both progressing and reducing.
\end{definition}

\begin{theorem}[Soundness]
  Let $L$ be a sound typed assembly language. Let $G \in \mathcal{G}$ and
  $S, S' \in \mathcal{S}$ and assume $G |- S\ \mathbf{Valid}$ along with
  $G |- S \to\^ast S'$. We now conclude that there exists some
  $S'' \in \mathcal{S}^T$ with $G |- S' \to S''$.
\end{theorem}

\begin{proof}
  We first conclude that $G |- S'\ \mathbf{Valid}$. This can be shown by
  induction over the derivation of $G |- S \to^\ast S'$ along with the fact that
  $L$ is reducing.

  Since $L$ is also progressing, this implies the result.
\end{proof}

\subsection{Bisimulation}

\begin{definition}
  Assume the following:
  \begin{itemize}
  \item
    $L_1 = (\mathcal{G}_1, \mathcal{H}_1, \mathcal{R}_1, \mathcal{I}_1,
    \mathcal{T}, R_\to)$.
  \item
    $L_2 = (\mathcal{G}_2, \mathcal{H}_2, \mathcal{R}_2, \mathcal{I}_2,
    \mathcal{T}, R_\to)$.
  \item Let $E = (f_G, f_H, f_R, f_I, f_T)$ be tuple of total functions. The
    functions should be defined over $\mathcal{G}_1 \to \mathcal{G}_2$,
    $\mathcal{H}_1 \to \mathcal{H}_2$, $\mathcal{R}_1 \to \mathcal{R}_2$,
    $\mathcal{I}_1 \to \mathcal{I}_2$ and $\mathcal{T}_1 \to \mathcal{T}_2$
    respectively.
  \end{itemize}

  Define $f_S : \mathcal{S}_1 \to \mathcal{S}_2$ and
  $f_{S^T} : \mathcal{S}_1^T \to \mathcal{S}_1^T$ as:

  \begin{align*}
    f_S(H, R, I) &= (f_H(H), f_R(R), f_I(I)) \\
    f_{S^T}(v) &=
                 \begin{cases}
                   f_S(v) & \text{if $v \in \mathcal{S}_1$} \\
                   f_T(v) & \text{if $v \in \mathcal{T}_1$} \\
                 \end{cases}
  \end{align*}

  We say that $E$ is a bisimulation between $L_1$ and $L_2$ iff the following
  holds:

  \begin{itemize}
  \item Assume
    $G \in \mathcal{G}, S_1 \in \mathcal{S}_1, S_1' \in \mathcal{S}_1^T, S_2 \in
    \mathcal{S}_2, S_2' \in \mathcal{S}_2^T$.
  \item Assume we have a derivationIf $\alpha$ is a derivation $\alpha$ of $G |- S_1 \to S_1'$, then we can
    find a derivation $\beta$ of $f_G(G) |- S_2 \to S_2'$ such that the diagram
    below commutes.
  \item The converse also holds: If $\beta$ is a derivation of
    $f_G(G) |- S_2 \to S_2'$, then we can find a derivation $\alpha$ of
    $G |- S_1 \to S_1'$ such that the diagram below commutes.
  \end{itemize}
\end{definition}

% Define $f : \mathcal{S}^T \to \mathcal{S}
% \begin{theorem}[Soundness of embeddings]
%   Let $L_1$ by a sound typed assembly language and $L_2$ be a simplified
%   assembly language. Assume there

\section{STAL}

% In this chapter we will introduce the concept of a Typed Assembly Language,
% specifically how it relates to a pure assembly language, how one designs it, and
% what it achieves.

% Throughout the section, we will assume that we have been given the task of
% designing a type-system for the real-world assembly language similar to
% MIPS. The details of the language does not matter, much and indeed most
% real-world assembler languages (or even some bytecode languages) could have been
% chosen instead. We choose MIPS primarily to have an easier comparison with existing

%   choice of language does not matter much, as the same techniques
% should be applicable to other hardware languages, and likely even to many
% bytecode languages. However using the choice of MIPS is advantagous, as it is
% very close to the language used in the next section.

% Before beginning, let us take a step back and look at what we have and we are trying to
% accomplish. We have been

% As stated previously\later[]{Put in reference}, we want to create a type system
% for a low-level language.

% In this section we will assume that we are trying to design a type system for a
% variant of MIPS, specified by an external party. This was chosen, because

%  as a base
% for our study. The actual chosen language does not matter much for the
% discussion at hand, as the same techniques should be applicable to other
% assembler language and likely even to some bytecode languages as well.

% Before we even begin to consider what it would mean to introduce types to an
% assembly languages, let us take a step back and look at the big-picture view.

% What we have is as design specification for the CPU in question. This design
% specification consist roughly of the following:

% \begin{itemize}
% \item Some description of how the state present in the CPU at any given
%   time. For most CPUs this will include at least the RAM, the register file and
%   the program counter.
% \item A partial function mapping the current state to an instruction $\iota$. For most CPUs
% \item A partial function mapping

% \begin{tabular}{lrcl}
% $instructions$ & $\iota$ & ::= & $\mathtt{add}\ \mathtt{r}_d, \mathtt{r}_s, v \mid \mathtt{sub}\ \mathtt{r}_d, \mathtt{r}_s, v \mid \dots$ \\
% \end{tabular}

% \later[inline]{Complete the above and add judgments too}

% It turns out that while the above semantics are very general in


\chapter{Stack-Based Typed Assembly Language}
This chapter will define the two languages $STAL$ and $STAL_S$. The design bla
bla previous section.

For the most part $STAL$ is equivalent to the one defined in \cite{STAL},
however a few changes have been made to simplify the proofs. It is our opinion
that the resulting language is morally equivalent to the original one.

\section{Base language}

In the following $r_{max}$ and $w$ ares a global constants, representing
respectively the number of registers on the machine and the number of bits in a
machine-word. While it is a constant, its concrete value is never used and it
can thus vary between implementation.
\\

\subsection{Grammar}

\begin{tabular}{lrcl}
Variables: \\
\textit{global pointers}    & $\ell_g$ \\
\textit{heap pointers}      & $\ell_h$ \\
\\
\textit{integer}            & $n$ & ::= & $0, 1, \dots$ \\
\textit{machine integers}   & $i$ & ::= & $0, 1, \dots, 2^{w}-1$ \\
\textit{registers}          & $r$ & ::= & $\mathtt{r}_1 \mid \dots \mid \mathtt{r}_{r_{max}}$ \\
\textit{word values}        & $w$ & ::= & $\mathtt{globval}\ \ell_g \mid \mathtt{heapval}\ \ell_h \mid \mathtt{int}\ i \mid \mathtt{ns} \mid \mathtt{uninit}$ \\
\textit{small values}       & $v$ & ::= & $\mathtt{globval}\ \ell_g \mid \mathtt{reg}\ r \mid \mathtt{int}\ i \mid \mathtt{ns} \mid \mathtt{uninit}$ \\
\textit{global values}      & $g$ & ::= & $\mathtt{code}\ I$ \\
\textit{global collections} & $G$ & ::= & $\{\ell_g |-> g, \dots, \ell_g |-> g\}$ \\
\textit{heap values}        & $h$ & ::= & $\langle w, \dots, w \rangle$ \\
\textit{heap collections}   & $H$ & ::= & $\{\ell_h |-> h, \dots, \ell_h |-> h\}$ \\
\textit{stacks}             & $S$ & ::= & $\mathtt{nil} \mid w :: S$ \\
\textit{register files}     & $R$ & ::= & $\{\mathtt{sp} |-> S, \mathtt{r}_1 |-> w_1, \dots, \mathtt{r}_{r_{max}} |-> w_{r_{max}}\}$ \\
\\
\textit{instructions} & $\iota$ & ::= & $\mathtt{add}\ r_d, r_s, v \mid \mathtt{sub}\ r_d, r_s, v \mid$ \\
        &&& $\mathtt{salloc}\ n \mid \mathtt{sfree}\ n \mid$ \\
        &&& $\mathtt{ld}\ r_d, \mathtt{sp}(n) \mid \mathtt{st}\ \mathtt{sp}(n), r_s \mid$\\
        &&& $\mathtt{ld}\ r_d, r_s(n) \mid \mathtt{st}\ r_d(n), r_s \mid$\\
        &&& $\mathtt{malloc}\ r_d,\ n \mid $ \\
        &&& $\mathtt{mov}\ r_d, r_s \mid \mathtt{beq}\ r, v$ \\
\textit{instruction sequences} & $I$ & ::= & $\iota ; I \mid \mathtt{jmp}\ v$ \\
\textit{programs} & $P$ & ::= & $(G, H, R, I)$ \\
\end{tabular}

\paragraph{}
The mappings of the form $\{x |-> y, \dots\}$ are unordered and do not have
repeating keys. In the Agda-implementation, these maps are mostly implemented as
ordered lists with implicit keys, however we can ignore this fact when using
paper-notation.

We will occasionally refer to a specific heap collection as a \textbf{heap} or
perhaps \textbf{the heap} for a specific program state. Similarly a specific
global collection might be referred to as \textbf{the globals} of a program. We
will however not use the term ``global'' without qualification, to avoid
ambiguity between global collections and global values.

The fundamental assumption of this language is that the state of a program is
totally defined by a collection of immutable global values, the current state of
the heap, the current stack, the current registers and the current instruction
pointer. This is represented in our language as:

\begin{itemize}
\item The immutable global values are places in a global collection
  consisting of individual global items.
\item The mutable global values are places in a heap collection
  consisting of individual heap items.
\item The current stack is encoded as a stack consisting and placed inside a
  register file.
\item The value of the current registers are encoded in a register file together
  with the stack.
\item The instruction pointer is not encoded directly as a pointer, it is
  instead represented as the instructions to be run until the next block.
\end{itemize}

While this representation is enough for many programs, it also excludes many
programs by design. For instance it excludes mutable global variables not stored
on the heap and self-modifying code (such as just-in-time compilers).

It should be possible to expand the language to allow more programs, however
some sacrifice will always be needed, as we are not able to exclude malicious
programs from our semantics without also excluding some valid programs.

\subsection{Related judgments}

\todo[inline]{This is a few syntax definitions. While they might be introduced
  here, they should be defined more formally elsewhere.}

\subsubsection{Dictionaries}
A \textbf{dictionary} is an unordered set of the form
$\{x_1 |-> y_1, \dots, x_n |-> y_n\}$ with the restriction that $x_i \neq x_j$
whenever $i \neq j$.

Assume that $D_1$ and $D_2$ are arbitrary dictionaries of the form
$\{x_1 |-> y_1, \dots, x_n |-> y_n\}$ and
$\{x_1' |-> y_1', \dots, x_m' |-> y_m'\}$ respectively.

Define $\mathbf{Keys}(D_1)$ to be the set $\{x_1, \dots, x_n\}$. Whenever
$\mathbf{Keys}(D_1) \cap \mathbf{Keys}(D_2) = \emptyset$, then the expression
$D_1 \cup D_2$ denotes the the dictionary
$\{x_1 |-> y_1, \dots, x_n |-> y_n, x_1' |-> y_1', \dots, x_m' |-> y_m'\}$.

We define $D_1\{x |-> y\}$ whenever $x \in \mathbf{Keys}(D_1)$. This expression
is the dictionary $\{x_1 |-> y_1', \dots, x_n |-> y_n'\}$ subject to the
restriction that $y_k' = y$ whenever $x_k = x$ and $y_k' = y_k$ otherwise.

By $(x |-> y) \in D$ we mean a judgment asserting that $D$ is of the form
$\{x_1 |-> y_1, \dots, x_n |-> y_n\}$ and that there exists some $k$ such that
$x = x_k$ and $y = y_k$. This syntax is meaning for e.g. global collections and
register files.

\subsubsection{List-like objects}
A set $S$ is \textbf{list-like} iff there is a meaningful way to interpret the
elements of $S$ as ordered lists of elements drawn from a base-set $S_B$. In
other words, there should exist a injective function
$f : S \to \mathbf{List}\ S_B$, where $f$ may only depend on the superficial
syntax of the elements.

For instance heap values and stacks are list-like, as they are written
$\langle w_1, \dots, w_n \rangle$ and $w_1 :: \dots :: w_n :: \mathtt{nil}$
respectively.

For the remainder of this subsection, assume $L_1$ and $L_2$ are typical
elements of the same list-like set $S$. Further assume that
$f(L_1) = [x_1, \dots, x_n]$ and $f(L_2) = [y_1, \dots, y_m]$.

Define $\mathbf{Length}(L_1) = n$.

Define $L_1 >> L_2$ be the element $L_3 \in S$ such that
$f(L_3) = [x_1, \dots, x_n, y_1, \dots, y_m]$, if such an element exists.

By $L_1 ! k => v$ we mean a judgment asserting $x_k = v$.

By $L_1[k]<- v => L_2$ we mean a judgment asserting that $n=m$, $y_k=v$ and
$x_i = y_i$ if $i \neq k$.

\subsection{Semantics}

We are now able to implement the semantics for our language: \\

\fbox{$\evalbig{R}{v}{w}$}
\begin{mathpar}
\infer{ }{
  \evalbig{R}{\mathtt{globval}\ \ell_g}{\mathtt{globval}\ \ell_g}
} \and
\infer{
  (r \mapsto w) \in R
}{
  \evalbig{R}{\mathtt{reg}\ r}{w}
} \and
\infer{ }{
  \evalbig{R}{\mathtt{int}\ i}{\mathtt{int}\ i}
} \and
\infer{ }{
  \evalbig{R}{\mathtt{ns}}{\mathtt{ns}}
} \and
\infer{ }{
  \evalbig{R}{\mathtt{uninit}}{\mathtt{uninit}}
}
\end{mathpar}

\fbox{$\execinstruction{G}{H, R, I}{H', R', I'}$}
\begin{mathpar}
\infer{
  \evalbig{R}{v}{\mathtt{int}\ i_1} \and
  (\mathtt{r}_s \mapsto \mathtt{int}\ i_2) \in R
}{
  \execinstruction{G}
    {H, R, (\mathtt{add}\ r_d, r_s, v) ; I}
    {H, R\{r_d \mapsto \mathtt{int}\ (i_1 + i_2)\}, I}
} \and
\infer{
  \evalbig{R}{v}{\mathtt{int}\ i_1} \and
  (r_s \mapsto \mathtt{int}\ i_2) \in R
}{
  \execinstruction{G}
    {H, R, (\mathtt{sub}\ r_d, r_s, v) ; I}
    {H, R\{r_d |-> \mathtt{int}\ (i_1 - i_2)\}, I}
} \and
\infer{
  (\mathtt{sp} |-> S) \in R \and
  S' = \overbrace{\mathtt{ns} :: \dots :: \mathtt{ns}}^n :: S
}{
  \execinstruction{G}
    {H, R, \mathtt{salloc}\ n ; I}
    {H, R\{\mathtt{sp} |-> S'\}, I}
} \and
\infer{
  (\mathtt{sp} |-> S) \in R \and
  S = \overbrace{w_1 :: \dots :: w_n}^n :: S'
}{
  \execinstruction{G}
    {H, R, \mathtt{sfree}\ n ; I}
    {H, R\{\mathtt{sp} |-> S'\}, I}
} \and
\infer{
  (\mathtt{sp} |-> S) \in R \and
  S ! n => w
}{
  \execinstruction{G}
    {H, R, (\mathtt{ld}\ r_d, \mathtt{sp}(n)) ; I}
    {H, R\{r_d |-> w\}, I}
} \and
\infer{
  (r_s |-> \mathtt{heapval}\ \ell_h) \in R \and
  (\ell_h |-> h) \in H \and
  h ! n => w
}{
  \execinstruction{G}
    {H, R, (\mathtt{ld}\ r_d, r_s(n)) ; I}
    {H, R\{r_d |-> w\}, I}
} \and
\infer{
  (r_s |-> w) \in R \and
  (\mathtt{sp} |-> S) \in R \and
  S[n] <- w => S'
}{
  \execinstruction{G}
    {H, R, (\mathtt{st}\ \mathtt{sp}(n), r_s) ; I}
    {H, R\{\mathtt{sp} |-> S'\}, I}
} \and
\infer{
  (r_s |-> w) \in R \and
  (r_d |-> \mathtt{heapval}\ \ell_h) \in R \and
  (\ell_h |-> h) \in H \and
  h[n] <- w => h'
}{
  \execinstruction{G}
    {H, R, (\mathtt{st}\ r_d(n), r_s) ; I}
    {H\{\ell_h |-> h'\}, R, I}
} \and
\infer{
  \ell_h \not\in \mathbf{Keys}(H) \and
  h = \overbrace{\langle \mathtt{uninit}, \dots, \mathtt{uninit} \rangle}^n
}{
  \execinstruction{G}
    {H, R, (\mathtt{malloc}\ r_d, n) ; I}
    {H \cup \{\ell_h |-> h\}, R\{r_d |-> \mathtt{heapval}\ \ell_h\}, I}
} \and
\infer{
  \evalbig{R}{v}{w}
}{
  \execinstruction{G}
    {H, R, (\mathtt{mov}\ r_d, v) ; I}
    {H, R\{r_d |-> w\}, I}
} \and
\infer{
  (r |-> \mathtt{int}\ 0) \in R \and
  \evalbig{R}{v}{\mathtt{code}\ I'}
}{
  \execinstruction{G}
    {H, R, (\mathtt{beq}\ r, v) ; I}
    {H, R, I'}
} \and
\infer{
  (r |-> \mathtt{int}\ i) \in R \and
  i \neq 0
}{
  \execinstruction{G}
    {H, R, (\mathtt{beq}\ r, v) ; I}
    {H, R, I}
} \and
\infer{
  \evalbig{R}{v}{\mathtt{code}\ I'}
}{
  \execinstruction{G}
    {H, R, \mathtt{jmp}\ v}
    {H, R, I'}
}
\end{mathpar}

\section{Annotated language}

\input{stal/typedgrammar}
\subsection{Semantics}

\paragraph{}
\fbox{$\evalrega{R}{v}{w}$}
\begin{mathpar}
\infer{ }{
  \evalrega{R}{\mathtt{globval}\ \ell_g}{\mathtt{globval}\ \ell_g}
} \and
\infer{
  (r \mapsto w) \in R
}{
  \evalrega{R}{\mathtt{reg}\ r}{w}
} \and
\infer{ }{
  \evalrega{R}{\mathtt{int}\ i}{\mathtt{int}\ i}
} \and
\infer{ }{
  \evalrega{R}{\mathtt{ns}}{\mathtt{ns}}
} \and
\infer{ }{
  \evalrega{R}{\mathtt{uninit}\ \tau}{\mathtt{uninit}\ \tau}
} \and
\infer{
  \evalrega{R}{v}{w}
}{
  \evalrega{R}{\Lambda\ \Delta.v[i_1, \dots, i_n]}{\Lambda\ \Delta.w[i_1, \dots, i_n]}
}
\end{mathpar}

\fbox{$\evalcodea{G}{w}{I}$}
\begin{mathpar}
\infer{
  \lookup{G}{\ell_g}{\mathtt{code}\ \forall[\Delta]\Gamma \cdot I}
}{
  \evalcodea{G}{\mathtt{globval}\ \ell_g}{I}
} \and
\infer{
  \evalcodea{G}{w}{I} \and
  I_1 = I[v_1 / x_1] \and
  \dots \and
  I_n = I_{n-1}[v_n / x_n]
}{
  \evalcodea{G}{\Lambda\ \Delta . w[x_1 / y_1, \dots, x_n / y_n]}{I_n}
}
\end{mathpar}

\fbox{$\execia{G}{C}{C}$}
\begin{mathpar}
\infer{
  \evalrega{R}{v}{\mathtt{int}\ i_1} \and
  (\mathtt{r}_s \mapsto \mathtt{int}\ i_2) \in R
}{
  \execia{G}
    {H, R, (\mathtt{add}\ r_d, r_s, v) ; I}
    {H, R\{r_d \mapsto \mathtt{int}\ (i_1 + i_2)\}, I}
} \and
\infer{
  \evalrega{R}{v}{\mathtt{int}\ i_1} \and
  (r_s \mapsto \mathtt{int}\ i_2) \in R
}{
  \execia{G}
    {H, R, (\mathtt{sub}\ r_d, r_s, v) ; I}
    {H, R\{r_d |-> \mathtt{int}\ (i_1 - i_2)\}, I}
} \and
\infer{
  (\mathtt{sp} |-> S) \in R \and
  S' = \overbrace{\mathtt{ns} :: \dots :: \mathtt{ns}}^n :: S
}{
  \execia{G}
    {H, R, \mathtt{salloc}\ n ; I}
    {H, R\{\mathtt{sp} |-> S'\}, I}
} \and
\infer{
  (\mathtt{sp} |-> S) \in R \and
  S = \overbrace{w_1 :: \dots :: w_n}^n :: S'
}{
  \execia{G}
    {H, R, \mathtt{sfree}\ n ; I}
    {H, R\{\mathtt{sp} |-> S'\}, I}
} \and
\infer{
  (\mathtt{sp} |-> S) \in R \and
  \lookup{S}{n}{w}
}{
  \execia{G}
    {H, R, (\mathtt{ld}\ r_d, \mathtt{sp}(n)) ; I}
    {H, R\{r_d |-> w\}, I}
} \and
\infer{
  (r_s |-> \mathtt{heapval}\ \ell_h) \in R \and
  (\ell_h |-> h) \in H \and
  \lookup{h}{n}{w}
}{
  \execia{G}
    {H, R, (\mathtt{ld}\ r_d, r_s(n)) ; I}
    {H, R\{r_d |-> w\}, I}
} \and
\infer{
  (r_s |-> w) \in R \and
  (\mathtt{sp} |-> S) \in R \and
  \update{S}{n}{w}{S'}
}{
  \execia{G}
    {H, R, (\mathtt{st}\ \mathtt{sp}(n), r_s) ; I}
    {H, R\{\mathtt{sp} |-> S'\}, I}
} \and
\infer{
  (r_s |-> w) \in R \and
  (r_d |-> \mathtt{heapval}\ \ell_h) \in R \and
  (\ell_h |-> h) \in H \and
  \update{h}{n}{w}{h'}
}{
  \execia{G}
    {H, R, (\mathtt{st}\ r_d(n), r_s) ; I}
    {H\{\ell_h |-> h'\}, R, I}
} \and
\infer{
  \ell_h \not\in \mathbf{Keys}(H) \and
  h = \overbrace{\langle \mathtt{uninit}\ \tau_1, \dots, \mathtt{uninit}\ \tau_n \rangle}^n
}{
  \execia{G}
    {H, R, (\mathtt{malloc}\ r_d, \langle \tau_1, \dots, \tau_n \rangle) ; I}
    {H \cup \{\ell_h |-> h\}, R\{r_d |-> \mathtt{heapval}\ \ell_h\}, I}
} \and
\infer{
  \evalrega{R}{v}{w}
}{
  \execia{G}
    {H, R, (\mathtt{mov}\ r_d, v) ; I}
    {H, R\{r_d |-> w\}, I}
} \and
\infer{
  (r |-> \mathtt{int}\ 0) \in R \and
  \evalrega{R}{v}{w} \and
  \evalcodea{G}{w}{I'}
}{
  \execia{G}
    {H, R, (\mathtt{beq}\ r, v) ; I}
    {H, R, I'}
} \and
\infer{
  (r |-> \mathtt{int}\ i) \in R \and
  i \neq 0
}{
  \execia{G}
    {H, R, (\mathtt{beq}\ r, v) ; I}
    {H, R, I}
} \and
\infer{
  \evalrega{R}{v}{w} \and
  \evalcodea{G}{w}{I'}
}{
  \execia{G}
    {H, R, \mathtt{jmp}\ v}
    {H, R, I'}
}
\end{mathpar}

\fbox{$\stepa{P}{P'}$}
\begin{mathpar}
\infer{
  \execia{G}{H, R, I}{H', R', I'}
}{
  (G, H, R, I) -> (G, H', R', I')
}
\end{mathpar}

\fbox{$\donea{P}$}
\begin{mathpar}
\infer{ }{
  \donea{(G, H, R, \mathtt{halt})}
}
\end{mathpar}



\chapter{Lessons Learned: Tips for doing formalizations in Agda}
\section{Decidable Equality in Agda}
\begin{frame}
\frametitle{Table of Contents}
\tableofcontents[currentsection]
\end{frame}

\begin{frame}{What is Agda?}
  \begin{itemize}
  \item Programming Language similar to Haskell.
  \item Proof system with features similar to Coq.
  \end{itemize}
\end{frame}

\begin{frame}{Propositions as Types}
  We identify a proposition $P$ by a type.

  \begin{itemize}
  \item We represent logical implication $P => Q$ as a function $P -> Q$.
  \item We represent logical conjunction $P \land Q$ by a tuple $(P, Q)$.
  \item We represent $P \lor Q$ by the Haskell-type $Either\ P\ Q$.
  \item We represent logical negation $\neg P$ by a function $P -> \bot$, where
    $\bot$ is a type without any constructors.
  \end{itemize}
\end{frame}

\begin{frame}{Decidability in Agda (1/2)}
  In classical logic, we can simply assume $P \lor \neg P$. We cannot do this in
  Agda:

  \begin{itemize}
  \item Assume $P = NP \lor P \neq NP$.
  \item Pattern match on the result.
  \item ???
  \item PROFIT!
  \end{itemize}
\end{frame}

\begin{frame}{Decidibility in Agda (2/2)}
  \textbf{Definition:} A proposition is decidable if there is a procedure for
  concluding either $P$ or $\neg P$.

  \pause If $P$ decidable in Agda, we write $Dec\ P$.
\end{frame}

\begin{frame}[fragile]{Equality in Programming}
  Equality is very common in programming:

\begin{lstlisting}
def gcd(a, b):
    while b != 0:
        a, b = b, a%b
    return a
    while x != 0:

class Point:
   ...

   def __eq__(self, other):
       return self.x == other.x and self.y == other.y
\end{lstlisting}
\end{frame}

\begin{frame}[fragile]{Equality in Programming}
  Equality is very common in programming:

\begin{lstlisting}
data Person = Person { firstName :: String
                     , lastName :: String
                     , age :: Int
                     } deriving (Eq)

main = print (dropWhile (==john) persons)
\end{lstlisting}
\end{frame}

\begin{frame}{Equality in Agda}
  Demo time!
\end{frame}

\section{Extensions to Equality Reasoning}
Not that smart or interesting. One would rather try to avoid equality reasoning,
but if that is not possible, one might as well have as strong a hacksaw as
possible.

\later[inline]{Write about equality reasoning}

\section{Substitutions}
I think I should be able to same something semi-smart about this.

\later[inline]{Write about substitutions}


% \chapter{Judgments}
% \section{Grammar}

\begin{tabular}{lrcl}
$types$ & $\tau$ & ::= & $\alpha \mid \mathtt{int} \mid \mathtt{ns} \mid \mathtt\forall[ \Delta ] \Gamma \mid \langle\tau_1^{\phi_1},\dots,\tau_n^{\phi_n}\rangle$ \\
$stack\ types$ & $\sigma$ & ::= & $\rho \mid \mathtt{nil} \mid \tau :: \sigma$ \\
$initialization\ flags$ & $\phi$ & ::= & $\mathtt{init} \mid \mathtt{uninit}$ \\
$type\ assignments$ & $\Delta$ & ::= & $\mathtt{nil} \mid a :: \Delta$ \\
$type\ assignment\ value$ & $a$ & ::= & $\alpha \mid \rho$ \\
$global\ label\ assignments$ & $\Psi_g$ & ::= & $\{\ell_{g,1}: \tau_1,\dots,\ell_{g,n}:\tau_n\}$ \\
$heap\ label\ assignmentss$ & $\Psi_h$ & ::= & $\{\ell_{h,1}: \tau_1,\dots,\ell_{h,n}:\tau_n\}$ \\
$label\ assignments$ & $\Psi$ & ::= & $(\Psi_g , \Psi_h)$ \\
$register\ assignments$ & $\Gamma$ & ::= & $\{\mathtt{sp} : \sigma, \mathtt{r}_1: \tau_1, \dots, \mathtt{r}_{regs}: \tau_{regs}\}$ \\
\\
$registers$ & $r$ & ::= & $\mathtt{r}_1 \mid \dots \mid \mathtt{r}_{regs}$ \\
$casts$ & $c$ & ::= & $\mathtt{\alpha}\Rightarrow\tau \mid \rho\Rightarrow\sigma \mid \mathtt{weaken}\ \Delta$ \\
$word\ values$ & $w$ & ::= & $\mathtt{globval}\ \ell_g \mid \mathtt{heapval}\ \ell_h \mid \mathtt{int}\ i \mid \mathtt{ns} \mid \mathtt{uninit}\ \tau \mid w[c]$ \\
$small\ values$ & $w$ & ::= & $\mathtt{reg}\ r \mid \mathtt{word}\ w \mid v[c]$ \\
$global\ values$ & $g$ & ::= & $\mathtt{code}[\Delta]\Gamma.I$ \\
$globals$ & $G$ & ::= & $\{l_{g,1}\mapsto g_1, \dots, l_{g,n} \mapsto g_n\}$ \\
$heap\ values$ & $h$ & ::= & $\langle w_1, \dots, w_n \rangle$ \\
$heaps$ & $H$ & ::= & $\{l_{h,1}\mapsto h_1, \dots, l_{h,n} \mapsto h_n\}$ \\
$register\ files$ & $R$ & ::= & $\{sp \mapsto S, r_1 \mapsto w_1, \dots, r_{regs} \mapsto w_{regs}\}$ \\
\\
$instructions$ & $\iota$ & ::= & $\mathtt{add}\ r_d, r_s, v \mid \mathtt{sub}\ r_d, r_s, v \mid$ \\
        &&& $\mathtt{push}\ v \mid \mathtt{pop} \mid$ \\
        &&& $\mathtt{ld}\ r_d, \mathtt{sp}, i \mid \mathtt{st}\ \mathtt{sp}, i, r_s \mid$\\
        &&& $\mathtt{ld}\ r_d, r_s, i \mid \mathtt{st}\ r_d, i, r_s \mid$\\
        &&& $\mathtt{malloc}\ \langle \tau_1, \dots, \tau_n \rangle \mid $ \\
        &&& $\mathtt{mov}\ r_d, r_s \mid \mathtt{beq}\ r, v$ \\
$instruction\ sequences$ & $I$ & ::= & $\iota ; I \mid \mathtt{jmp}\ v$ \\
$program\ states$ & $P$ & ::= & $(H, R, I)$ \\
\end{tabular}

\subsection{Comments}
In the previous, $regs$ is a constant. Its value of the number of registers on
the machine in question. The concrete value of $regs$ does not matter for any of
the proofs, but in my code, I set the value to $4$.

The references $\alpha$ and $\rho$ are implemented using De Bruijn indexes. This
is relevant both in $types$, $stack\ types$ and in $casts$, where they are
used. It is also relevant in the $\mathtt{weaken}$ case of $casts$, as we need
to know \emph{where} to weaken.

Note also that for technical reasons, a lot of the constructors in Agda does not
actually look like this. For instance we use $\iota \sim> I$ instead of
$\iota ; I$ and $v \llbracket c \rrbracket$ instead of $v [ c ]$. We also use
$\mathtt{sld}\ r_d, i$ and $\mathtt{sst}\ i, r_s$ instead of
$\mathtt{ld}\ r_d, \mathtt{sp}, i$ and $\mathtt{st}\ \mathtt{sp}, i, r_s$.

Note also that in the Agda-code, the $\mathtt{globval}\ l_{g,k}$ of $w$ also
includes the number of assumptions used in the corresponding code global. This
would ideally be removed later.

\subsection{Differences from STAL}

\begin{itemize}
\item No existential types
\item No stack-pointer types
\item No compound stacks
\item The original paper has an unlimited number of registers, with only a
  finite amount used at any time. We have a fixed number.
\item The heap has been split up into the \texttt{Globals}, containing immutable
  code and, and \texttt{Heap} containing only mutable tuples.
\item The instructions are slightly different -- for instance there is no \texttt{salloc}/\texttt{sfree}, only \texttt{push}/\texttt{pop}.
\item The small step semanics will not update the globals.
\end{itemize}

% \chapter{Type System for \ATAL}
\label{chap:types}

\section{Judgments about types}

\begin{tabular}{|c|p{7.5 cm}|}
  \hline
  Judgment & \multicolumn{1}{|c|}{Meaning} \\
  \hline

  $\valid{\Delta}{\tau}$ & $\tau$ is a well-formed type (i.e., does not contain invalid variables) \\
  $\valid{\Delta}{\sigma}$ & $\sigma$ is a well-formed stack type \\
  $\valid{\Delta}{\Gamma}$ & $\Gamma$ is a well-formed register assignment \\
  $\valid{\Delta}{\Psi_{\mathrm{h}}}$ & $\Psi_{\mathrm{h}}$ is a well-formed heap label assignments \\
  \hline

  $\subtype{\Delta}{\tau_1}{\tau_2}$ & $\tau_1$ is a subtype of $\tau_2$ \\
  $\subtype{\Delta}{\phi_1}{\phi_2}$ & $\phi_1$ is a subtype of $\phi_2$ \\
  $\subtype{\Delta}{\sigma_1}{\sigma_2}$ & $\sigma_1$ is a subtype of $\sigma_2$ \\
  $\subtype{\Delta}{\Gamma_1}{\Gamma_2}$ & $\Gamma_1$ is a subtype of $\Gamma_2$ \\
  $\subtype{\Delta}{\Psi_{\mathrm{h},1}}{\Psi_{\mathrm{h}_2}}$ & $\Psi_{\mathrm{h},1}$ is a subtype of $\Psi_{\mathrm{h},2}$ \\
  \hline

  $\Delta |- \theta : a$ & $\theta$ is a valid instantiation of the variable $a$ \\
  $\Delta |- \Theta : \Delta'$ & $\Theta$ are valid instantiations of the variables $\Delta'$ \\
  \hline
\end{tabular}


\subsection{Valid types}
\fbox{$\valid{\Delta}{\tau}$}
\begin{mathpar}
\infer{\alpha \in \Delta}{\valid{\Delta}{\alpha}} \and
\infer{ }{\valid{\Delta}{\mathtt{int}}} \and
\infer{ }{\valid{\Delta}{\mathtt{uninit}}} \and
\infer{\valid{\Delta' ++ \Delta}{\Gamma}}{\valid{\Delta}{\forall[ \Delta' ] \Gamma}} \and
\infer{\valid{\Delta}{\tau_1} \and \dots \and \valid{\Delta}{\tau_n}}
      {\valid{\Delta}{\langle \tau_1^{\phi_1}, \dots, \tau_n^{\phi_n} \rangle}}
\end{mathpar}

\fbox{$\valid{\Delta}{\sigma}$}
\begin{mathpar}
\infer{\rho \in \Delta}{\valid{\Delta}{\rho}} \and
\infer{ }{\valid{\Delta}{\nil}} \and
\infer{
  \valid{\Delta}{\tau} \and \valid{\Delta}{\sigma}
}{
  \valid{\Delta}{\tau :: \sigma}
}
\end{mathpar}

\fbox{$\valid{\Delta}{\Gamma}$}
\begin{mathpar}
\infer{
  \valid{\Delta}{\sigma} \and
  \valid{\Delta}{\tau_1} \and
  \dots \and
  \valid{\Delta}{\tau_{\mathrm{\mathrm{max}}}}
}{
  \valid{\Delta}{\{\mathtt{sp} |-> \sigma, \mathtt{r}_1 |-> \tau_1, \dots, \mathtt{r}_{\mathrm{\mathrm{max}}} |-> \tau_{\mathrm{\mathrm{max}}}\}}
}
\end{mathpar}

\fbox{$\valid{\Delta}{\Psi_{\mathrm{h}}}$}
\begin{mathpar}
\infer{
  \valid{\Delta}{\tau_1} \and
  \dots \and
  \valid{\Delta}{\tau_{n}}
}{
  \valid{\Delta}{\{\ell_{\mathrm{h},1} |-> \tau_1, \dots, \ell_{\mathrm{h},n} |-> \tau_{n}\}}
}
\end{mathpar}

\subsection{Subtypes}
\fbox{$\subtype{\Delta}{\tau_1}{\tau_2}$}
\begin{mathpar}
\infer{\alpha \in \Delta}{\subtype{\Delta}{\alpha}{\alpha}} \and
\infer{ }{\subtype{\Delta}{\mathtt{int}}{\mathtt{int}}} \and
\infer{ }{\subtype{\Delta}{\mathtt{uninit}}{\mathtt{uninit}}} \and
\infer{
  \subtype{\Delta' ++ \Delta}{\Gamma_2}{\Gamma_1}
}{
  \subtype{\Delta}{\forall[\Delta'] \Gamma_1}{\forall[\Delta'] \Gamma_2}
} \and
\infer{
  \subtype{}{\phi_1}{\phi_1'} \and
  \dots \and
  \subtype{}{\phi_n}{\phi_n'}
}{
  \subtype{\Delta}
          {\langle \tau_1^{\phi_1}, \dots, \tau_n^{\phi_n} \rangle}
          {\langle \tau_1^{\phi'_1}, \dots, \tau_n^{\phi'_n} \rangle}
}
\end{mathpar}

\fbox{$\subtype{}{\phi_1}{\phi_2}$}
\begin{mathpar}
\infer{ }{\subtype{}{\mathtt{init}}{\phi}} \and
\infer{ }{\subtype{}{\mathtt{uninit}}{\mathtt{uninit}}}
\end{mathpar}

\fbox{$\subtype{\Delta}{\sigma}{\sigma}$}
\begin{mathpar}
\infer{\rho \in \Delta}{\subtype{\Delta}{\rho}{\rho}} \and
\infer{ }{\subtype{\Delta}{\nil}{\nil}} \and
\infer{
  \subtype{\Delta}{\tau}{\tau'} \and
  \subtype{\Delta}{\sigma}{\sigma'} \and
}{
  \subtype{\Delta}{\tau :: \sigma}{\tau' :: \sigma'}
}
\end{mathpar}

\fbox{$\subtype{\Delta}{\Gamma}{\Gamma}$}
\begin{mathpar}
\infer{
  \subtype{\Delta}{\sigma}{\sigma'} \and
  \subtype{\Delta}{\tau_1}{\tau_1'} \and
  \dots \and
  \subtype{\Delta}{\tau_{\mathrm{\mathrm{max}}}}{\tau_{\mathrm{\mathrm{max}}}'}
}{
  \subtype{\Delta}
          {\{\mathtt{sp} |-> \sigma, \mathtt{r}_1 |-> \tau_1, \dots, \mathtt{r}_{\mathrm{\mathrm{max}}} |-> \tau_{\mathrm{\mathrm{max}}}\}}
          {\{\mathtt{sp} |-> \sigma', \mathtt{r}_1 |-> \tau_1', \dots, \mathtt{r}_{\mathrm{\mathrm{max}}} |-> \tau_{\mathrm{\mathrm{max}}}'\}}
}
\end{mathpar}

\fbox{$\subtype{\Delta}{\Psi_{\mathrm{h}}}{\Psi_{\mathrm{h}}}$}
\begin{mathpar}
\infer{
  \subtype{\Delta}{\tau_1}{\tau_1'} \and
  \dots \and
  \subtype{\Delta}{\tau_{n}}{\tau_{n}'}
}{
  \subtype{\Delta}
          {\{\ell_{\mathrm{h},1} |-> \tau_1, \dots, \ell_{\mathrm{h},n} |-> \tau_{n}\}}
          {\{\ell_{\mathrm{h},1} |-> \tau_1', \dots, \ell_{\mathrm{h},n} |-> \tau_{n}'\}}
}
\end{mathpar}

\subsection{Instantiations}

\fbox{$\ofinstantiation{\Delta}{\theta}{a}$}
\begin{mathpar}
\infer{
  \alpha \notin \Delta \and
  \valid{\Delta}{\tau}
}{
  \ofinstantiation{\Delta}{\tau / \alpha}{\alpha}
} \and
\infer{
  \rho \notin \Delta \and
  \valid{\Delta}{\sigma}
}{
  \ofinstantiation{\Delta}{\sigma / \rho}{\rho}
}
\end{mathpar}

\fbox{$\ofinstantiations{\Delta}{\Theta}{\Delta}$}
\begin{mathpar}
\infer{ }{
  \ofinstantiations{\Delta}{\nil}{\nil}
} \and
\infer{
  \ofinstantiation{\Delta' ++ \Delta}{\theta}{a} \and
  \ofinstantiations{\Delta}{\Theta}{\Delta'}
}{
  \ofinstantiations{\Delta}{(\theta :: \Theta)}{(a :: \Delta')}
}
\end{mathpar}

\subsection{Notes on the Agda implementation}
\texttt{Judgments/Types.agda} and some of \texttt{Judgments/Terms.agda}.

\subsection{Lemmas}

\texttt{Lemmas/Types.agda} and \texttt{Lemmas/TypeSubstitution.agda}.

\begin{lemma}
  \label{lemma:typdec}
  Validity and subtyping is decidable. So are the judgments
  $\ofinstantiation{\Delta}{\theta}{a}$ and
  $\ofinstantiations{\Delta}{\Theta}{\Delta'}$.
\end{lemma}

\begin{lemma}
  \label{lemma:typeq}
  Validity implies subtyping and vice versa.
\end{lemma}

\begin{lemma}
  \label{lemma:transitive}
  Subtyping is transitive.
\end{lemma}

\begin{lemma}
  \label{lemma:typ-context}
  Validity and subtyping is preserved by weakening or instantiation variables
  from the context.\footnote{Remember that substitution is a partial function in
    the Agda-code. In the Agda-code this lemma also implies totality of
    substitution given certain additional assumptions.}
\end{lemma}

\section{Judgments about values}
\begin{tabular}{|c|p{7.5 cm}|}
  \hline
  Judgment & \multicolumn{1}{|c|}{Meaning} \\
  \hline

  $\Psi_{\mathrm{g}}, \Delta |- \high v : \Gamma => \tau$ & $\high v$ is a well-formed small value evaluating register files of type $\Gamma$ to word values of type $\tau$ \\
  $\Psi_{\mathrm{g}}, \Delta |- \high \iota : \Gamma_1 => \Gamma_2$ & $\high \iota$ is a well-formed instruction which evaluating registers of type $\Gamma_1$ to registers of type $\Gamma_2$ \\
  $\Psi_{\mathrm{g}}, \Delta |- \high I : \Gamma$ & $\high I$ is a well-formed instruction sequence which will correctly evaluate registers of type $\Gamma$. \\
  \hline

  $\Psi_{\mathrm{g}}, \Psi_{\mathrm{h}} |- \high w : \tau$ & $\high w$ is a well-formed word value of type $\tau$ \\
  $\Psi_{\mathrm{g}}, \Psi_{\mathrm{h}} |- \high w : \tau^{\phi}$ & $\high w$ is a well-formed (and possibly uninitialized) word value of type $\tau^\phi$ (i.e., either $\high w : \tau$ or $\high w = \mathtt{uninit}$ and $\phi = \mathtt{uninit}$) \\
  $\Psi_{\mathrm{g}}, \Psi_{\mathrm{h}} |- \high S : \sigma$ & $\high S$ is a well-formed stack of type $\sigma$ \\
  $\Psi_{\mathrm{g}}, \Psi_{\mathrm{h}} |- \high R : \Gamma$ & $\high R$ is a well-formed register file of type $\Gamma$ \\
  $\Psi_{\mathrm{g}}, \Psi_{\mathrm{h}} |- \high h : \tau$ & $\high h$ is a well-formed heap value of type $\tau$ \\
  $\Psi_{\mathrm{g}} |- \high g : \tau$ & $\high g$ is a well-formed global value of type $\tau$. \\
  $\Psi_{\mathrm{g}} |- \high H : \Psi_{\mathrm{h}}$ & $\high H$ is a well-formed heap collection of type $\Psi_{\mathrm{h}}$ \\
  $|- \high G : \Psi_{\mathrm{g}}$ & $\high G$ is a well-formed global collection of type $\Psi_{\mathrm{g}}$. \\
  \hline

  $\Psi_{\mathrm{g}} |- (\high H, \high R, \high I) : (\Psi_{\mathrm{h}}, \Gamma)$ & $(\high H, \high R, \high I)$ are all valid while using types $\Psi_{\mathrm{h}}$ and $\Gamma$ for the heap and registers. \\
  $\valid{}{\high P}$ & $\high P$ is a well-formed and well-typed program. \\
  \hline
\end{tabular}

\subsection{Evaluation types}

\fbox{$\ofvval{\Psi_{\mathrm{g}},\Delta}{\high v}{\Gamma => \tau}$}
\begin{mathpar}
\infer{ }{
  \ofvval{\Psi_{\mathrm{g}},\Delta}{\mathtt{reg}\ r}{\Gamma => \Gamma[r]}
}\and
\infer{
}{
  \ofvval{\Psi_{\mathrm{g}},\Delta}{\mathtt{globval}\ \ell_{\mathrm{g}}}{\Gamma => \Psi_{\mathrm{g}}[\ell_{\mathrm{g}}]}
}\and
\infer{ }{\ofvval{\Psi_{\mathrm{g}},\Delta}{\mathtt{int}\ i}{\Gamma => \mathtt{int}}} \and
\infer{
  \ofvval{\Psi_{\mathrm{g}},\Delta}{\high v}{\Gamma => \forall[\Delta_1]\Gamma_1} \and
  \ofinstantiations{\Delta_2 ++ \Delta}{\Theta}{\Delta_1} \and
  \Gamma_2 = \Gamma_1[\Theta] \\
}{
  \ofvval{\Psi_{\mathrm{g}},\Delta}{\Lambda\ \Delta \cdot \high v[\Theta]}{\Gamma => \forall[\Delta_2]\Gamma_2}
}
\end{mathpar}

\fbox{$\ofinstruction{\Psi_{\mathrm{g}}, \Delta}{\high \iota}{\Gamma_1 => \Gamma_2}$}
\begin{mathpar}
\infer{
  \Gamma[r_b] = \mathtt{int} \and
  \ofvval{\Psi_{\mathrm{g}}, \Delta}{\high v}{\Gamma => \mathtt{int}}
}{
  \ofinstruction{\Psi_{\mathrm{g}}, \Delta}
    {\mathtt{add}\ r_a, r_b, \high v}
    {\Gamma => \Gamma[r_a |-> \mathtt{int}]}
} \and
\infer{
  \Gamma[r_b] = \mathtt{int} \and
  \ofvval{\Psi_{\mathrm{g}}, \Delta}{\high v}{\Gamma => \mathtt{int}}
}{
  \ofinstruction{\Psi_{\mathrm{g}}, \Delta}
    {\mathtt{sub}\ r_a, r_b, \high v}
    {\Gamma => \Gamma[r_a |-> \mathtt{int}]}
} \\
\infer{
  \sigma = \overbrace{\mathtt{uninit} :: \dots :: \mathtt{uninit}}^n :: \Gamma[\mathtt{sp}]
}{
  \ofinstruction{\Psi_{\mathrm{g}}, \Delta}
    {\mathtt{salloc}\ n}
    {\Gamma => \Gamma[\mathtt{sp} |-> \sigma]}
} \and
\infer{
  \Gamma[\mathtt{sp}] = \overbrace{\tau_1 :: \dots :: \tau_n}^n :: \sigma
}{
  \ofinstruction{\Psi_{\mathrm{g}}, \Delta}
    {\mathtt{sfree}\ n}
    {\Gamma => \Gamma[\mathtt{sp} |-> \sigma]}
} \and
\infer{
  \Gamma[\mathtt{sp}] = \sigma
}{
  \ofinstruction{\Psi_{\mathrm{g}}, \Delta}
    {\mathtt{ld}\ r, \mathtt{sp}(n)}
    {\Gamma => \Gamma[r |-> \sigma[n]]}
} \and
\infer{
  \Gamma[r_b] = \langle \tau_1^{\phi_1}, \dots, \tau_n^{\mathtt{init}}, \dots\rangle \and
}{
  \ofinstruction{\Psi_{\mathrm{g}}, \Delta}
    {\mathtt{ld}\ r_a, r_b(n)}
    {\Gamma => \Gamma[r_a |-> \tau_n]}
} \and
\infer{
  \Gamma[r] = \tau \and
  \Gamma[\mathtt{sp}] = \sigma
}{
  \ofinstruction{\Psi_{\mathrm{g}}, \Delta}
    {\mathtt{st}\ \mathtt{sp}(n), r}
    {\Gamma => \Gamma[\mathtt{sp} |-> \sigma[n |-> \tau]]}
} \and
\infer{
  \Gamma[r_a] = \langle \tau_1^{\phi_1}, \dots, \tau_n^{\phi_i}, \dots \rangle \and
  \Gamma[r_b] = \tau_n' \and
  \subtype{\Delta}{\tau_n'}{\tau_n}
}{
  \ofinstruction{\Psi_{\mathrm{g}}, \Delta}
    {\mathtt{st}\ r_a(n), r_b}
    {\Gamma => \Gamma[r_a |-> \langle \tau_1^{\phi_1}, \dots, \tau_n^{\mathtt{init}}, \dots, \rangle]}
} \and
\infer{
  \valid{\Delta}{\tau_1} \and
  \dots \and
  \valid{\Delta}{\tau_n}
}{
  \ofinstruction{\Psi_{\mathrm{g}}, \Delta}
    {\mathtt{malloc}\ r, \langle \tau_1, \dots, \tau_n \rangle}
    {\Gamma => \Gamma[r |-> \langle \tau_1^{\mathtt{uninit}}, \dots, \tau_n^{\mathtt{uninit}}\rangle]}
} \and
\infer{
  \ofvval{\Psi_{\mathrm{g}}, \Delta}{\high v}{\Gamma => \tau}
}{
  \ofinstruction{\Psi_{\mathrm{g}}, \Delta}
    {\mathtt{mov}\ r, \high v}
    {\Gamma => \Gamma[r |-> \tau]}
} \and
\infer{
  \Gamma[r] = \mathtt{int} \and
  \ofvval{\Psi_{\mathrm{g}}, \Delta}{\high v}{\Gamma => \forall[ \nil ] \Gamma'} \and
  \subtype{\Delta}{\Gamma}{\Gamma'}
}{
  \ofinstruction{\Psi_{\mathrm{g}}, \Delta}
    {\mathtt{beq}\ r, \high v}
    {\Gamma => \Gamma}
}
\end{mathpar}


\fbox{$\ofinstructions{\Psi_{\mathrm{g}}, \Delta}{\high I}{\Gamma}$}
\begin{mathpar}
\infer{
  \ofinstruction{\Psi_{\mathrm{g}}, \Delta}{\high \iota}{\Gamma => \Gamma'} \and
  \ofinstructions{\Psi_{\mathrm{g}}, \Delta}{\high I}{\Gamma'}
}{
  \ofinstructions{\Psi_{\mathrm{g}}, \Delta}{\high \iota ; \high I}{\Gamma}
} \and
\infer{
  \ofvval{\Psi_1, \Delta}{\high v}{\Gamma => \forall[ \nil ]\Gamma'} \and
  \subtype{\Delta}{\Gamma}{\Gamma'}
}{
  \ofinstructions{\Psi_{\mathrm{g}}, \Delta}{\mathtt{jmp}\ \high v}{\Gamma}
} \and
\infer{
}{
  \ofinstructions{\Psi_{\mathrm{g}}, \Delta}{\mathtt{halt}}{\Gamma}
}
\end{mathpar}

\subsection{Memory constructs}

\fbox{$\ofwval{\Psi_{\mathrm{g}},\Psi_{\mathrm{h}}}{\high w}{\tau}$}
\begin{mathpar}
\infer{
  \Psi_{\mathrm{g}}[\ell_{\mathrm{g}}] = \tau_1 \and
  \subtype{\nil}{\tau_1}{\tau_2}
}{
  \ofwval{\Psi_{\mathrm{g}},\Psi_{\mathrm{h}}}{\mathtt{globval}\ \ell_{\mathrm{g}}}{\tau_2}
}\and
\infer{
  \Psi_{\mathrm{h}}[\ell_{\mathrm{h}}] = \tau_1 \and
  \subtype{\nil}{\tau_1}{\tau_2}
}{
  \ofwval{\Psi_{\mathrm{g}},\Psi_{\mathrm{h}}}{\mathtt{heapval}\ \ell_{\mathrm{h}}}{\tau_2}
}\and
\infer{ }{\ofwval{\Psi_{\mathrm{g}},\Psi_{\mathrm{h}}}{\mathtt{int}\ i}{\mathtt{int}}} \and
\infer{ }{\ofwval{\Psi_{\mathrm{g}},\Psi_{\mathrm{h}}}{\mathtt{uninit}}{\mathtt{uninit}}} \and
\infer{
  \ofwval{\Psi_{\mathrm{g}},\Psi_{\mathrm{h}}}{\high w}{\forall[\Delta_1]\Gamma_1} \and
  \ofinstantiations{\Delta_2}{\Theta}{\Delta_1} \and
  \subtype{\Delta_2}{\Gamma_2}{\Gamma_1[\Theta]} \\
}{
  \ofwval{\Psi_{\mathrm{g}},\Psi_{\mathrm{h}}}{\Lambda\ \Delta_2 \cdot \high w[\Theta]}{\forall[\Delta_2]\Gamma_2}
}
\end{mathpar}

\fbox{$\ofwvaln{\Psi_{\mathrm{g}},\Psi_{\mathrm{h}}}{\high w}{\tau^\phi}$}
\begin{mathpar}
\infer{
  \valid{\nil}{\tau}
}{
  \ofwvaln{\Psi_{\mathrm{g}},\Psi_{\mathrm{h}}}{\mathtt{uninit}}{\tau^{\mathtt{uninit}}}
} \and
\infer{
  \ofwval{\Psi_{\mathrm{g}},\Psi_{\mathrm{h}}}{\high w}{\tau}
}{
  \ofwvaln{\Psi_{\mathrm{g}},\Psi_{\mathrm{h}}}{\high w}{\tau^{\mathtt{init}}}
}
\end{mathpar}

\fbox{$\ofstack{\Psi_{\mathrm{g}},\Psi_{\mathrm{h}}}{\high S}{\sigma}$}
\begin{mathpar}
\infer{ }{\ofstack{\Psi_{\mathrm{g}},\Psi_{\mathrm{h}}}{\nil}{\nil}} \and
\infer{
  \ofword{\Psi_{\mathrm{g}},\Psi_{\mathrm{h}}}{\high w}{\tau} \and
  \ofstack{\Psi_{\mathrm{g}},\Psi_{\mathrm{h}}}{\high S}{\sigma}
}{
  \ofstack{\Psi_{\mathrm{g}},\Psi_{\mathrm{h}}}{\high w :: \high S}{\tau :: \sigma}
}
\end{mathpar}

\fbox{$\ofregister{\Psi_{\mathrm{g}},\Psi_{\mathrm{h}}}{\high R}{\Gamma}$}
\begin{mathpar}
\infer{
  \ofstack{\Psi_{\mathrm{g}},\Psi_{\mathrm{h}}}{\high S}{\sigma} \and
  \ofwval{\Psi_{\mathrm{g}},\Psi_{\mathrm{h}}}{\high w_1}{\tau_1} \and
  \dots \and
  \ofwval{\Psi_{\mathrm{g}},\Psi_{\mathrm{h}}}{\high w_{\mathrm{max}}}{\tau_{\mathrm{max}}} \and
}{
  \ofregister{\Psi_{\mathrm{g}},\Psi_{\mathrm{h}}}{\{\mathtt{sp} |->  \high S, \mathtt{r}_1 |->  \high w_1, \dots, \mathtt{r}_{\mathrm{max}} |->  \high w_{\mathrm{max}}\}}{\{\mathtt{sp} |-> \sigma, \mathtt{r}_1 |-> \tau_1, \dots, \mathtt{r}_{\mathrm{max}} |-> \tau_{\mathrm{max}}\}}
}
\end{mathpar}

\fbox{$\ofhval{\Psi_{\mathrm{g}},\Psi_{\mathrm{h}}}{\high h}{\tau}$}
\begin{mathpar}
\infer{
  \ofwvaln{\Psi_{\mathrm{g}},\Psi_{\mathrm{h}}}{\high w_1}{\tau_1^{\phi_1}} \and
  \dots \and
  \ofwvaln{\Psi_{\mathrm{g}},\Psi_{\mathrm{h}}}{\high w_n}{\tau_n^{\phi_n}}
}{
  \ofhval{\Psi_{\mathrm{g}},\Psi_{\mathrm{h}}}{\mathtt{tuple}\ \langle \high \tau_1, \dots, \high \tau_n \rangle\ \langle \high w_1, \dots, \high w_n \rangle}{\langle \tau_1^{\phi_1}, \dots, \tau_n^{\phi_n}\rangle}
}
\end{mathpar}

\fbox{$\ofgval{\Psi_{\mathrm{g}}}{\high g}{\tau}$}
\begin{mathpar}
\infer{
  \valid{\Delta}{\Gamma} \and
  \ofinstructions{\Psi_{\mathrm{g}}, \Delta}{\high I}{\Gamma}
}{
  \ofgval{\Psi_{\mathrm{g}}}{\mathtt{code}[\Delta]\Gamma \cdot \high I}{\forall[\Delta]\Gamma}
}
\end{mathpar}

\fbox{$\ofheap{\Psi_{\mathrm{g}}}{\high H}{\Psi_{\mathrm{h}}}$}
\begin{mathpar}
\infer{
  \ofhval{\Psi_{\mathrm{g}},\Psi_{\mathrm{h}}}{\high h_1}{\tau_1} \and
  \dots \and
  \ofhval{\Psi_{\mathrm{g}},\Psi_{\mathrm{h}}}{\high h_n}{\tau_n} \and
  \Psi_{\mathrm{h}} = \{\ell_{h,1} |-> \tau_1, \dots, \ell_{h,n} |-> \tau_n\}
}{
  \ofheap{\Psi_{\mathrm{g}}}{\{\ell_{h,1} |->  \high h_1, \dots, \ell_{h,n}  |->  \high h_n\}}{\Psi_{\mathrm{h}}}
}
\end{mathpar}

\fbox{$\ofglobs{\high G}{\Psi_{\mathrm{g}}}$}
\begin{mathpar}
\infer{
  \ofgval{\Psi_{\mathrm{g}}}{\high g_1}{\tau_1} \and
  \dots \and
  \ofgval{\Psi_{\mathrm{g}}}{\high g_n}{\tau_n} \and
  \Psi_{\mathrm{g}} = \{\ell_{g,1} |-> \tau_1, \dots, \ell_{g,n} |-> \tau_n\}
}{
  \ofglobs{\{\ell_{g,1} |->  \high g_1, \dots, \ell_{g,n}  |->  \high g_n\}}{\Psi_{\mathrm{g}}}
}
\end{mathpar}

\subsection{Program states}

\fbox{$\ofprogramstate{\Psi_{\mathrm{g}}}{(\high H, \high R, \high I)}{(\Psi_{\mathrm{h}},\Gamma)}$}
\begin{mathpar}
\infer{
  \ofheap{\Psi_{\mathrm{g}}}{\high H}{\Psi_{\mathrm{h}}} \\
  \ofregister{\Psi_{\mathrm{g}},\Psi_{\mathrm{h}}}{\high R}{\Gamma} \and
  \ofinstructions{\Psi_{\mathrm{g}},\nil}{\high I}{\Gamma}
}{
  \ofprogramstate{\Psi_{\mathrm{g}}}{(\high H,\high R,\high I)}{(\Psi_{\mathrm{h}},\Gamma)}
}
\end{mathpar}

\fbox{$\valid{}{\high P}$}
\begin{mathpar}
\infer{
  \ofglobs{\high G}{\Psi_{\mathrm{g}}} \and
  \ofprogramstate{\Psi_{\mathrm{g}}}{(\high H, \high R, \high I)}{(\Psi_{\mathrm{h}}, \Gamma)}
}{
  \valid{}{(\high G,\high H, \high R, \high I)}
}
\end{mathpar}

\subsection{Notes on the Agda implementation}
\texttt{Judgments/Terms.agda}

\subsection{Lemmas}

\texttt{Lemmas/TermWeaken.agda} and \texttt{Lemmas/TermSubstitution.agda}.

\begin{lemma}
  \label{lemma:instructions-context}
  The judgment $\ofinstructions{\Psi_{\mathrm{g}}, \Delta}{\high I}{\Gamma}$ is
  preserved by weakening or instantiation variables
  from the context.\footnote{Remember that substitution is a partial function in
    the Agda-code. In the Agda-code this lemma also implies totality of
    substitution given certain additional assumptions.}
\end{lemma}

\texttt{Lemmas/TermCast.agda}
\begin{lemma}
  \label{lemma:evaluatin-casting}
  \begin{itemize}
  \item Assume $\ofvval{\Psi_{\mathrm{g}},\Delta}{\high v}{\Gamma => \tau}$
    and $\subtype{\Delta}{\Gamma'}{\Gamma}$. In that case there is some
    $\tau'$ such that
    $\ofvval{\Psi_{\mathrm{g}},\Delta}{\high v}{\Gamma' => \tau'}$.

  \item Assume $\ofinstruction{\Psi_{\mathrm{g}},\Delta}{\high \iota}{\Gamma_1 => \Gamma_2}$
    and $\subtype{\Delta}{\Gamma_1'}{\Gamma_1}$. In that case there is some
    $\Gamma_2'$ such that
    $\ofinstruction{\Psi_{\mathrm{g}},\Delta}{\high \iota}{\Gamma_1' => \Gamma_2'}$.

  \item Assume $\ofinstructions{\Psi_{\mathrm{g}},\Delta}{\high I}{\Gamma}$ and
    $\subtype{\Delta}{\Gamma'}{\Gamma}$. In that case
    $\ofinstructions{\Psi_{\mathrm{g}},\Delta}{\high I}{\Gamma'}$.
  \end{itemize}
\end{lemma}

\begin{lemma}
  \label{lemma:value-casting}
  Assume $\ofwval{\Psi_{\mathrm{g}},\Psi_{\mathrm{h}}}{\high w}{\tau}$ and that
  $\subtype{\nil}{\tau}{\tau'}$. In that case
  $\ofwval{\Psi_{\mathrm{g}},\Psi_{\mathrm{h}}}{\high w}{\tau'}$. Similar
  statements hold for stacks, register files and heap values.
\end{lemma}

\texttt{Lemmas/TermHeapCast.agda}
\begin{lemma}
  \label{lemma:heap-casting}
  Assume $\ofwval{\Psi_{\mathrm{g}},\Psi_{\mathrm{h}}}{\high w}{\tau}$ and that
  $\subtype{\nil}{\Psi_{\mathrm{h}}'}{\Psi_{\mathrm{h}}}$. In that case
  $\ofwval{\Psi_{\mathrm{g}},\Psi_{\mathrm{h}}'}{\high w}{\tau}$. Similar
  statements hold for stacks, register files and heap values.
\end{lemma}

\section{Soundness proof}

This section will give a very broad outline of the proof of soundness. The
actual proof is implemented in \texttt{Lemmas/Soundness.agda} and has the
type-signature:

\begin{code}
\>\AgdaFunction{steps-soundness} \AgdaSymbol{:} \AgdaSymbol{∀} \AgdaSymbol{\{}\AgdaBound{n} \AgdaBound{P₁} \AgdaBound{P₂}\AgdaSymbol{\}} \AgdaSymbol{→}\<%
\\
\>[2]\AgdaIndent{20}{}\<[20]%
\>[20]\AgdaDatatype{⊢} \AgdaBound{P₁} \AgdaDatatype{programstate} \AgdaSymbol{→}\<%
\\
\>[2]\AgdaIndent{20}{}\<[20]%
\>[20]\AgdaFunction{⊢} \AgdaBound{P₁} \AgdaFunction{⇒ₙ} \AgdaBound{n} \AgdaFunction{/} \AgdaBound{P₂} \AgdaSymbol{→}\<%
\\
\>[2]\AgdaIndent{20}{}\<[20]%
\>[20]\AgdaFunction{GoodState} \AgdaBound{P₂}\<%
\end{code}

For the actual proof of this statement, we will refer to the file. In
\cref{sec:soundness-proof} you will also find a proof sketch that follows the
structure of the this proof. Here we will instead discuss the very broad lines
included how the language was designed to permit this result.

\todo{Discuss why this is possible.}

\section{Decidability}

In this section we introduce our proof that the type system is decidable. In
particular the file \texttt{Lemmas/TermDec.agda} contains the function:

\begin{code}
\>\AgdaFunction{programstate-dec} \AgdaSymbol{:} \AgdaSymbol{∀} \AgdaBound{P} \AgdaSymbol{→} \AgdaDatatype{Dec} \AgdaSymbol{(}\AgdaDatatype{⊢} \AgdaBound{P} \AgdaDatatype{programstate}\AgdaSymbol{)}\<%
\end{code}

The actual proof is not particularly interesting, though we have given a sketch
of it in \cref{sec:decproof}. What is more interesting to discuss is how our
language was designed to allow this result to be achieved.

\todo{Discuss why this is possible.}

% \section{Subtyping judgments}

\fbox{$\subtype{\Delta}{\tau_1}{\tau_2}$}
\begin{mathpar}
\infer{\alpha \in \Delta}{\subtype{\Delta}{\alpha}{\alpha}} \and
\infer{ }{\subtype{\Delta}{\mathtt{int}}{\mathtt{int}}} \and
\infer{ }{\subtype{\Delta}{\mathtt{ns}}{\mathtt{ns}}} \and
\infer
  {\valid{\Delta}{\Delta'} \and
    \subtype{\Delta', \Delta}{\Gamma_1}{\Gamma_2}}
  {\subtype{\Delta}{\forall[ \Delta' ]\Gamma_1}{\forall[ \Delta' ]\Gamma_2}} \and
\infer
  {\subtype{\Delta}{\tau_1^{\phi_1}}{\tau'_1{}^{\phi'_1}} \and
    \dots \and
    \subtype{\Delta}{\tau_n^{\phi_n}}{\tau'_n{}^{\phi'_n}}}
  {\subtype{\Delta}{\langle \tau'_1{}^{\phi_1}, \dots, \tau_n{}^{\phi_n} \rangle}
                   {\langle \tau'_1{}^{\phi'_1}, \dots, \tau'_n{}^{\phi'_n} \rangle}}
\end{mathpar}

\fbox{$\subtype{}{\phi_1}{\phi_2}$}
\begin{mathpar}
\infer{ }{\subtype{}{\mathtt{init}}{\mathtt{init}}} \and
\infer{ }{\subtype{}{\mathtt{\phi}}{\mathtt{uninit}}}
\end{mathpar}

\fbox{$\subtype{\Delta}{\tau_1^{\phi_1}}{\tau_2^{\phi_2}}$}
\begin{mathpar}
\infer{\valid{\Delta}{\tau} \and \subtype{}{\phi_1}{\phi_2}}{\subtype{\Delta}{\tau^{\phi_1}}{\tau^{\phi_2}}}
\end{mathpar}

% \section{Substitution judgments}

\fbox{$\substitution{\tau_1}{c}{\tau_2}$}
\begin{mathpar}
  \infer{ }{\substitution{\alpha}{\alpha => \tau}{\tau}} \and
  \infer{\alpha_1 \neq \alpha_2}{\substitution{\alpha_1}{\alpha_2 => \tau}{\alpha_1}} \and
  \infer{ }{\substitution{\alpha}{\rho => \sigma}{\alpha}} \and
  \infer{ }{\substitution{\alpha}{\mathtt{weaken}\ \Delta^{+}}{\alpha}}
\end{mathpar}

(This also includes obvious, mutually recursive definition.)\\\\
\fbox{$\substitution{\tau_1^{\phi_1}}{c}{\tau_2^{\phi_2}}$}\\

(The obvious, mutually recursive definitions.)\\\\
\fbox{$\substitution{\sigma_1}{c}{\sigma_2}$}
\begin{mathpar}
  \infer{ }{\substitution{\rho}{\rho => \sigma}{\sigma}} \and
  \infer{\rho_1 \neq \rho_2}{\substitution{\rho_1}{\rho_2 => \sigma}{\rho_1}} \and
  \infer{ }{\substitution{\rho}{\alpha => \tau}{\rho}} \and
  \infer{ }{\substitution{\rho}{\mathtt{weaken}\ \Delta^{+}}{\rho}}
\end{mathpar}

(This also includes obvious, mutually recursive definition.)\\\\
\fbox{$\substitution{\Delta_1}{c}{\Delta_2}$}\\

(The obvious, mutually recursive definitions.)\\\\
\fbox{$\substitution{a_1}{c}{a_2}$}\\

(The obvious, mutually recursive definitions.)\\\\
\fbox{$\substitution{\Gamma_1}{c}{\Gamma_2}$}\\
\begin{mathpar}
  \infer{ }{\substitution{\mathtt{nil}}{c}{\mathtt{nil}}} \and
  \infer{
    \substitution{a_1}{c}{a_2} \and
    \substitution{\Delta_1}{c}{\Delta_2}
  }{
    \substitution{(a_1 :: \Delta_1)}{c}{a_2 :: \Delta_2}
  }
\end{mathpar}\\
\fbox{$\substitution{c_1}{c}{c_2}$}\\

(The obvious, mutually recursive definitions.)\\\\
\fbox{$\substitution{w_1}{c}{w_2}$}\\

(The obvious, mutually recursive definitions.)\\\\
\fbox{$\substitution{v_1}{c}{v_2}$}\\

(The obvious, mutually recursive definitions.)\\\\
\fbox{$\substitution{\iota_1}{c}{\iota_2}$}\\

(The obvious, mutually recursive definitions.)\\\\
\fbox{$\substitution{I_1}{c}{I_2}$}\\

(The obvious, mutually recursive definitions.)

\subsection{Comments}

I write that the rules are obvious -- and they are, as long as you keep yourself
to using paper-notation. If you want to use De Bruijn indices they are not
nearly as obvious.

Note that the substitution for $\Delta$ assumes that the actual substitution is
to be done elsewhere -- it merely updates all references inside the $\Delta$. In
the current grammar there \emph{are} no references inside the assumptions, so we
get that $\substitution{\Delta}{c}{\Delta}$ for all $\Delta$. This would ideally
not hold in the final version.

For the substitution that actually updates the assumption list, see the next
section.

\subsection{Differences from STAL}

STAL handwaves even more that I do here. Rest assured that my Agda-code is
precise enough for a machine-verified proof.

% \section{Run Lemmas}

\begin{lemma}
  Assume we have $\Delta, \Delta_1, \Delta_2$ and $c$ such that
  $\run{\Delta}{c}{\Delta_1}$ and $\run{\Delta}{c}{\Delta_2}$. We can then
  conclude that $\Delta_1 = \Delta_2$.
\end{lemma}

\begin{proof}
  By induction over the derivations of $\run{\Delta}{c}{\Delta_1}$ and
  $\run{\Delta}{c}{\Delta_2}$.
\end{proof}

\begin{lemma}
  Assume we have $\Delta$ and $c$. It is now decidable whether there exists some
  $\Delta'$ such that $\run{\Delta}{c}{\Delta'}$.
\end{lemma}

\begin{proof}
  By structural induction over $\Delta$, using \autoref{dec:substitution}.
\end{proof}

\begin{lemma}
  Assume we have some $\Delta$ and $c$, such that $\valid{\mathtt{nil}}{\Delta}$
  and $\valid{\Delta}{c}$. We can then find $\Delta'$ such that
  $\run{\Delta}{c}{\Delta'}$.
\end{lemma}

\begin{proof}
  Long proof using previous lemmas.
\end{proof}

% \section{Term judgments}

Judgments that values are valid or of a specific type.

\fbox{$\ofglobs{G}{\Psi_g}$}
\begin{mathpar}
\infer{
  \ofgval{\Psi_g}{g_1}{\tau_1} \and
  \dots \and
  \ofgval{\Psi_g}{g_n}{\tau_n} \and
  \Psi_g = \{\ell_{g,1}:\tau_1, \dots, \ell_{g,n}:\tau_n\}
}{
  \ofglobs{\{\ell_{g,1}\mapsto g_1, \dots, \ell_{g,n} \mapsto g_n\}}{\Psi_g}
}
\end{mathpar}

\fbox{$\ofheap{\Psi_g}{H}{\Psi_h}$}
\begin{mathpar}
\infer{
  \ofhval{\Psi_g,\Psi_h}{h_1}{\tau_1} \and
  \dots \and
  \ofhval{\Psi_g,\Psi_h}{h_n}{\tau_n} \and
  \Psi_h = \{\ell_{h,1}:\tau_1, \dots, \ell_{h,n}:\tau_n\}
}{
  \ofheap{\Psi_g}{\{\ell_{h,1}\mapsto h_1, \dots, \ell_{h,n} \mapsto h_n\}}{\Psi_h}
}
\end{mathpar}

\fbox{$\ofstack{\Psi_g,\Psi_h}{S}{\sigma}$}
\begin{mathpar}
\infer{ }{\ofstack{\Psi_g,\Psi_h}{\mathtt{nil}}{\mathtt{nil}}} \and
\infer{
  \ofword{\Psi_g,\Psi_h}{w}{\tau} \and
  \ofstack{\Psi_g,\Psi_h}{S}{\sigma}
}{
  \ofstack{\Psi_g,\Psi_h}{w :: S}{\tau :: \sigma}
}
\end{mathpar}

\fbox{$\ofregister{\Psi}{R}{\Gamma}$}
\begin{mathpar}
\infer{
  \ofstack{\Psi_g,\Psi_h}{S}{\sigma} \and
  \ofwval{\Psi_g,\Psi_h,\mathtt{nil}}{w_1}{\tau_1} \and
  \dots
  \ofwval{\Psi_g,\Psi_h,\mathtt{nil}}{w_{regs}}{\tau_{regs}} \and
}{
  \ofregister{\Psi_g,\Psi_h}{\{\mathtt{sp}\mapsto S, \mathtt{r}_1\mapsto w_1, \dots, \mathtt{r}_{regs}\mapsto w_{regs}\}}{\{\mathtt{sp}:\sigma, \mathtt{r}_1:\tau_1, \dots, \mathtt{r}_{regs}:\tau_{regs}\}}
}
\end{mathpar}

\fbox{$\ofgval{\Psi_g}{g}{\tau}$}
\begin{mathpar}
\infer{
  \valid{\mathtt{nil}}{\Delta} \and
  \valid{\Delta}{\Gamma} \and
  \ofinstructions{\Psi_g, \Delta, \Gamma}{I}
}{
  \ofgval{\Psi_g}{\mathtt{code}[\Delta]\Gamma.I}{\forall[\Delta]\Gamma}
}
\end{mathpar}

\fbox{$\ofhval{\Psi_g,\Psi_h}{h}{\tau}$}
\begin{mathpar}
\infer{
  \ofwvaln{\Psi_g,\Psi_h}{w_1}{\tau_1^{\phi_1}} \and
  \dots \and
  \ofwvaln{\Psi_g,\Psi_h}{w_n}{\tau_n^{\phi_n}}
}{
  \ofgval{\Psi_g,\Psi_h}{\langle w_1, \dots, w_n \rangle}{\langle \tau_1^{\phi_1}, \dots, \tau_n^{\phi_n}\rangle}
}
\end{mathpar}

\fbox{$\ofwval{\Psi_g,\Psi_h,\Delta}{w}{\tau}$}
\begin{mathpar}
\infer{
  (\ell_g: \tau_1) \in \Psi_g \and
  \subtype{\mathtt{nil}}{\tau_1}{\tau_2}
}{
  \ofwval{\Psi_g,\Psi_h,\Delta}{\mathtt{globval}\ \ell_g}{\tau_2}
}\and
\infer{
  (\ell_h: \tau_1) \in \Psi_h \and
  \subtype{\mathtt{nil}}{\tau_1}{\tau_2}
}{
  \ofwval{\Psi_g,\Psi_h,\Delta}{\mathtt{heapval}\ \ell_h}{\tau_2}
}\and
\infer{ }{\ofwval{\Psi_g,\Psi_h,\Delta}{\mathtt{int}\ i}{\mathtt{int}}} \and
\infer{ }{\ofwval{\Psi_g,\Psi_h,\Delta}{\mathtt{ns}}{\mathtt{ns}}} \and
\infer{
  \ofwval{\Psi_g,\Psi_h,\Delta}{w}{\forall[\Delta_1]\Gamma_1} \and
  \valid{\Delta_1,\Delta}{c} \and
  \run{\Delta_1}{c}{\Delta_2} \\
  \substitution{\Gamma_1}{c}{\Gamma_2} \and
  \subtype{\Delta_2,\Delta}{\Gamma_2}{\Gamma_3}
}{
  \ofwval{\Psi_g,\Psi_h,\Delta}{w \llbracket c \rrbracket}{\forall[\Delta_2]\Gamma_3}
}
\end{mathpar}

\fbox{$\ofwvaln{\Psi_g,\Psi_h}{w}{\tau^\phi}$}
\begin{mathpar}
\infer{
  \valid{\mathtt{nil}}{\tau}
}{
  \ofwvaln{\Psi_g,\Psi_h}{\mathtt{uninit}\ \tau}{\tau^{\mathtt{uninit}}}
} \and
\infer{
  \ofwval{\Psi_g,\Psi_h,\mathtt{nil}}{w}{\tau}
}{
  \ofwvaln{\Psi_g,\Psi_h}{w}{\tau^{\mathtt{init}}}
}
\end{mathpar}

\fbox{$\ofvval{\Psi_g,\Delta,\Gamma}{v}{\tau}$}
\begin{mathpar}
\infer{
  (\mathtt{r}_k:\tau_k) \in \Gamma \and
  \subtype{\Delta}{\tau_k}{\tau'_k}
}{
  \ofvval{\Psi_g,\Delta,\Gamma}{\mathtt{reg}\ \mathtt{r}_k}{\tau'_k}
} \and
\infer{
  \ofwval{\Psi_g,\{\},\Delta}{w}{\tau}
}{
  \ofvval{\Psi_g,\Delta,\Gamma}{\mathtt{word}\ w}{\tau}
} \and
\infer{
  \ofvval{\Psi_g,\Delta,\Gamma}{w}{\forall[\Delta_1]\Gamma_1} \and
  \valid{\Delta_1,\Delta}{c} \and
  \run{\Delta_1}{c}{\Delta_2} \\
  \substitution{\Gamma_1}{c}{\Gamma_2} \and
  \subtype{\Delta_2,\Delta}{\Gamma_2}{\Gamma_3}
}{
  \ofvval{\Psi_g,\Delta,\Gamma}{w \llbracket c \rrbracket}{\forall[\Delta_2]\Gamma_3}
}
\end{mathpar}

\fbox{$\ofinstructions{\Psi_g, \Delta, \Gamma}{I}$}
\begin{mathpar}
\infer{
  \ofinstruction{\Psi_g, \Delta, \Gamma}{\iota}{\Gamma'} \and
  \ofinstructions{\Psi_g, \Delta, \Gamma'}{I}
}{
  \ofinstructions{\Psi_g, \Delta, \Gamma}{\iota ; I}
} \and
\infer{
  \ofvval{\Psi_1, \Delta, \Gamma}{v}{\forall[ \mathtt{nil} ]\Gamma'} \and
  \subtype{\Delta}{\Gamma}{\Gamma'}
}{
  \ofinstructions{\Psi_g, \Delta, \Gamma}{\mathtt{jmp}\ v}
}
\end{mathpar}

\fbox{$\ofinstruction{\Psi_g, \Delta, \Gamma}{\iota}{\Gamma'}$}
\begin{mathpar}
\infer{
  (\mathtt{r}_s : \mathtt{int}) \in \Gamma \and
  \ofvval{\Psi_g, \Delta, \Gamma}{v}{\mathtt{int}}
}{
  \ofinstruction{\Psi_g, \Delta, \Gamma}
    {\mathtt{add}\ \mathtt{r}_d, \mathtt{r}_s, v}
    {\Gamma\{\mathtt{r}_d:\mathtt{int}\}}
} \and
\infer{
  (\mathtt{r}_s : \mathtt{int}) \in \Gamma \and
  \ofvval{\Psi_g, \Delta, \Gamma}{v}{\mathtt{int}}
}{
  \ofinstruction{\Psi_g, \Delta, \Gamma}
    {\mathtt{sub}\ \mathtt{r}_d, \mathtt{r}_s, v}
    {\Gamma\{\mathtt{r}_d:\mathtt{int}\}}
} \and
\infer{
  \ofvval{\Psi_g, \Delta, \Gamma}{v}{\tau} \and
  (\mathtt{sp}:\sigma) \in \Gamma
}{
  \ofinstruction{\Psi_g, \Delta, \Gamma}
    {\mathtt{push}\ v}
    {\Gamma\{\mathtt{sp}: \tau :: \sigma\}}
} \and
\infer{
  (\mathtt{sp}: \tau::\sigma) \in \Gamma
}{
  \ofinstruction{\Psi_g, \Delta, \Gamma}
    {\mathtt{pop}}
    {\Gamma\{\mathtt{sp}: \sigma\}}
} \and
\infer{
  (\mathtt{sp}: \tau_1 :: \tau_2 :: \dots :: \tau_i :: \sigma) \in \Gamma
}{
  \ofinstruction{\Psi_g, \Delta, \Gamma}
    {\mathtt{ld}\ \mathtt{r}_d, \mathtt{sp}, i}
    {\Gamma\{\mathtt{r}_d: \tau_i\}}
} \and
\infer{
  (\mathtt{sp}: \tau_1 :: \tau_2 :: \dots :: \tau_i :: \sigma) \in \Gamma \and
  (\mathtt{r}_s: \tau_i') \in \Gamma
}{
  \ofinstruction{\Psi_g, \Delta, \Gamma}
    {\mathtt{st}\ \mathtt{sp}, i, \mathtt{r}_s}
    {\Gamma\{\mathtt{sp}: \tau_1 :: \tau_2 :: \dots :: \tau_i' :: \sigma\}}
} \and
\infer{
  (\mathtt{r}_s: \langle \tau_1^{\phi_1}, \dots, \tau_i^{\mathtt{init}}, \dots, \tau_n^{\phi_n}\rangle) \in \Gamma
}{
  \ofinstruction{\Psi_g, \Delta, \Gamma}
    {\mathtt{ld}\ \mathtt{r}_d, \mathtt{r}_s, i}
    {\Gamma\{\mathtt{r}_d: \tau_i\}}
} \and
\infer{
  (\mathtt{r}_d: \langle \tau_1^{\phi_1}, \dots, \tau_i^{\phi_i}, \dots, \tau_n^{\phi_n}\rangle) \in \Gamma \and
  (\mathtt{r}_s: \tau_i') \in \Gamma \and
  \subtype{\Delta}{\tau_i'}{\tau_i}
}{
  \ofinstruction{\Psi_g, \Delta, \Gamma}
    {\mathtt{st}\ \mathtt{r}_d, i, \mathtt{r}_s}
    {\Gamma\{\mathtt{r}_d: \langle \tau_1^{\phi_1}, \dots, \tau_i^{\mathtt{init}}, \dots, \tau_n^{\phi_n}\}}
} \and
\infer{
  \valid{\Delta}{\tau_1} \and
  \dots \and
  \valid{\Delta}{\tau_n}
}{
  \ofinstruction{\Psi_g, \Delta, \Gamma}
    {\mathtt{malloc}\ \mathtt{r}_d, \langle \tau_1, \dots, \tau_n \rangle}
    {\Gamma\{\mathtt{r}_d: \langle \tau_1^{\mathtt{uninit}}, \dots, \tau_n^{\mathtt{uninit}}\rangle\}}
} \and
\infer{
  \ofvval{\Psi_g, \Delta, \Gamma}{v}{\tau}
}{
  \ofinstruction{\Psi_g, \Delta, \Gamma}
    {\mathtt{mov}\ \mathtt{r}_d, v}
    {\Gamma\{\mathtt{r}_d: \tau\}}
} \and
\infer{
  (\mathtt{r}_k: \mathtt{int}) \in \Gamma \and
  \ofvval{\Psi_g, \Delta, \Gamma}{v}{\forall[ \mathtt{nil} ] \Gamma'} \and
  \subtype{\Delta}{\Gamma}{\Gamma'}
}{
  \ofinstruction{\Psi_g, \Delta, \Gamma}
    {\mathtt{beq}\ \mathtt{r}_k, v}
    {\Gamma}
}
\end{mathpar}

\fbox{$\ofinstruction{\Psi_g, \Delta, \Gamma}{I}{\Gamma'}$}
\begin{mathpar}
\infer{
  \ofinstruction{\Psi_g, \Delta, \Gamma}{\iota}{\Gamma'} \and
  \ofinstructions{\Psi_g, \Delta, \Gamma'}{I}
}{
  \ofinstructions{\Psi_g, \Delta, \Gamma}{\iota ; I}
}
\end{mathpar}

\fbox{$\ofprogram{G}{P}$}
\begin{mathpar}
\infer{
  \ofglobs{G}{\Psi_g} \and
  \ofheap{\Psi_g}{H}{\Psi_h} \\
  \ofregister{\Psi_g,\Psi_h}{R}{\Gamma} \and
  \ofinstructions{\Psi_g,\mathtt{nil},\Gamma}{I}
}{
  \ofprogram{G}{(H,R,I)}
}
\end{mathpar}

\subsection{Comments}

TODO

\subsection{Differences from STAL}

TODO

% \section{Semantics}

Finally: The semantics!\\

\fbox{$\evalsmall{R}{v}{w}$}
\begin{mathpar}
\infer{
  (\mathbf{r}_k \mapsto w) \in R
}{
  \evalsmall{R}{\mathtt{reg}_k}{w}
} \and
\infer{ }{\evalsmall{R}{\mathtt{word}\ w}{w}} \and
\infer{
  \evalsmall{R}{v}{w}
}{
  \evalsmall{R}{v \llbracket c \rrbracket}{w \llbracket c \rrbracket}
}
\end{mathpar}

\fbox{$\evalword{G}{w}{I}$}
\begin{mathpar}
\infer{
  (\ell_g \mapsto \mathtt{code}[\Delta]\Gamma.I) \in G
}{
  \evalword{G}{\mathtt{globval}\ \ell_g}{I}
} \and
\infer{
  \evalword{G}{w}{I} \and
  \substitution{I}{c}{I'}
}{
  \evalword{G}{w \llbracket c \rrbracket}{I'}
}
\end{mathpar}

\fbox{$\execinstruction{G}{P}{P'}$}
\begin{mathpar}
\infer{
  \evalsmall{R}{v}{\mathtt{int}\ n_1} \and
  (\mathtt{r}_s \mapsto \mathtt{int}\ n_2) \in R
}{
  \execinstruction{G}
    {H, R, (\mathtt{add}\ \mathtt{r}_d, \mathtt{r}_s, v) ; I}
    {(H, R\{\mathtt{r}_d \mapsto \mathtt{int}\ (n_1 + n_2)\}, I)}
} \and
\infer{
  \evalsmall{R}{v}{\mathtt{int}\ n_1} \and
  (\mathtt{r}_s \mapsto \mathtt{int}\ n_2) \in R
}{
  \execinstruction{G}
    {H, R, (\mathtt{sub}\ \mathtt{r}_d, \mathtt{r}_s, v) ; I}
    {(H, R\{\mathtt{r}_d \mapsto \mathtt{int}\ (n_1 - n_2)\}, I)}
} \and
\infer{
  \evalsmall{R}{v}{w} \and
  (\mathtt{sp} \mapsto S) \in R
}{
  \execinstruction{G}
    {H, R, \mathtt{push}\ v ; I}
    {(H, R\{\mathtt{sp} \mapsto w :: S\}, I)}
} \and
\infer{
  (\mathtt{sp} \mapsto w :: S) \in R
}{
  \execinstruction{G}
    {H, R, \mathtt{pop} ; I}
    {(H, R\{\mathtt{sp} \mapsto S\}, I)}
} \and
\infer{
  (\mathtt{sp} \mapsto w_1 :: w_2 :: \dots :: w_i :: S) \in R
}{
  \execinstruction{G}
    {H, R, (\mathtt{ld}\ \mathtt{r}_d, \mathtt{sp}, i) ; I}
    {(H, R\{\mathtt{r}_d \mapsto w_i\}, I)}
} \and
\infer{
  (\mathtt{sp} \mapsto w_1 :: w_2 :: \dots :: w_i :: S) \in R \and
  (\mathtt{r}_s \mapsto w_i') \in R
}{
  \execinstruction{G}
    {H, R, (\mathtt{st}\ \mathtt{sp}, i, \mathtt{r}_s) ; I}
    {(H, R\{\mathtt{sp} \mapsto w_1 :: w_2 \dots :: w_i' :: S\}, I)}
} \and
\infer{
  (\mathtt{r}_s \mapsto \mathtt{heapval}\ \ell_h) \in R \and
  (\ell_h \mapsto \langle w_1, \dots, w_i, \dots, w_n \rangle) \in H
}{
  \execinstruction{G}
    {H, R, (\mathtt{ld}\ \mathtt{r}_d, \mathtt{r}_s, i) ; I}
    {(H, R\{\mathtt{r}_d \mapsto w_i\}, I)}
} \and
\infer{
  (\mathtt{r}_d \mapsto \ell_h) \in R \and
  (\ell_h \mapsto \langle w_1, \dots, w_i, w_n \rangle) \in H \and
  (\mathtt{r}_s \mapsto w_i') \in R
}{
  \execinstruction{G}
    {H, R, (\mathtt{st}\ \mathtt{r}_d, i, \mathtt{r}_s) ; I}
    {(H\{\ell_h \mapsto \langle w_1, \dots, w_i', \dots, w_n \rangle\}, R, I)}
} \and
\infer{
  \text{$\ell_h$ is not present in $H$}
}{
  G |- \mathtt{eval} (H, R, (\mathtt{malloc}\ \mathtt{r}_d, \langle \tau_1, \dots, \tau_n \rangle) ; I) = \\
  (H\{\ell_h \mapsto \langle \mathtt{uninit}\ \tau_1, \dots, \mathtt{uninit}\ \tau_n \rangle\}, R\{\mathtt{r}_d \mapsto \mathtt{heapval}\ \ell_h\}, I)
} \and
\infer{
  \evalsmall{R}{v}{w}
}{
  \execinstruction{G}
    {H, R, (\mathtt{mov}\ \mathtt{r}_d, v) ; I}
    {(H, R\{\mathtt{r}_d \mapsto w\}, I)}
} \and
\infer{
  (\mathtt{r}_k \mapsto \mathtt{int}\ 0) \in R \and
  \evalsmall{R}{v}{w} \and
  \evalword{G}{w}{I'}
}{
  \execinstruction{G}
    {H, R, (\mathtt{beq}\ \mathtt{r}_k, v) ; I}
    {(H, R, I')}
} \and
\infer{
  (\mathtt{r}_k \mapsto \mathtt{int}\ i) \in R \and
  i \neq 0
}{
  \execinstruction{G}
    {H, R, (\mathtt{beq}\ \mathtt{r}_k, v) ; I}
    {(H, R, I)}
} \and
\infer{
  \evalsmall{R}{v}{w} \and
  \evalword{G}{w}{I'}
}{
  \execinstruction{G}
    {H, R, \mathtt{jmp}\ v}
    {(H, R, I')}
}
\end{mathpar}


% \chapter{Lemmas}
% \section{Miscellaneous small lemmas}

There are more small lemmas in the Agda file \texttt{Misc.agda}. I have only
included those that are not deemed trivial.

\begin{lemma}
  If we have $\substitution{\Delta_1}{c}{\Delta_1'}$ and
  $\substitution{\Delta_2}{c}{\Delta_2'}$, then we also have
  $\substitution{(\Delta_1,\Delta_2)}{c}{(\Delta_1',\Delta_2')}$.
\end{lemma}
\begin{proof}
  By induction over the derivation of $\substitution{\Delta_1}{c}{\Delta_1'}$.
\end{proof}

\begin{lemma}
  If we have $\substitution{\Delta_1}{c}{\Delta_1'}$ and
  $\run{\Delta_2}{c}{\Delta_2'}$, then we also have
  $\run{(\Delta_1,\Delta_2)}{c}{(\Delta_1',\Delta_2')}$.
\end{lemma}
\begin{proof}
  By induction over the derivation of $\substitution{\Delta_1}{c}{\Delta_1'}$.
\end{proof}

\begin{lemma}
  If we have $\run{\Delta_1}{c}{\Delta_1'}$, then we also have
  $\run{(\Delta_1,\Delta_2)}{c}{(\Delta_1',\Delta_2)}$.
\end{lemma}
\begin{proof}
  By induction over the derivation of $\run{\Delta_1}{c}{\Delta_1'}$.
\end{proof}

% \section{Decidable Equality}

\begin{definition}
  Let $\mathbb{T}$ be the set of all rooted trees where each node has been
  assigned a single value from $\mathbb{N}$.
\end{definition}

\begin{lemma}
  Let $t_1, t_2 \in \mathbb{T}$. Then it is decidable if $t_1 = t_2$.
\end{lemma}
\label{eq:tree}
\begin{proof}
  By induction over the structure of $t_1$ and $t_2$.
\end{proof}

\begin{lemma}
  Let $T$ be one of the core datatypes from \autoref{sec:grammar}. Then there
  exists a injective mapping from $T \to \mathbb{T}$.
\end{lemma}
\label{eq:inj}
\begin{proof}
  The mapping is obvious, since all of the definitions are finite, inductive
  datatypes.
\end{proof}

\begin{corollary}
  Let $T$ be one of the datatypes from \autoref{sec:grammar} and let
  $a, b \in T$. Then it is decidable if whether $a=b$ or $a \neq b$.
\end{corollary}
\label{dec:eq}
\begin{proof}
  Let $f$ be the function from \autoref{eq:inj}. Since
  $f(a), f(b) \in \mathbb{T}$ then by \autoref{eq:tree} we can decide if
  $f(a) = f(b)$. Because $f$ is injective, we have $a = b \iff f(a) = f(b)$, so
  we are done.
\end{proof}

% \chapter{Type System for \ATAL}
\label{chap:types}

\section{Judgments about types}

\begin{tabular}{|c|p{7.5 cm}|}
  \hline
  Judgment & \multicolumn{1}{|c|}{Meaning} \\
  \hline

  $\valid{\Delta}{\tau}$ & $\tau$ is a well-formed type (i.e., does not contain invalid variables) \\
  $\valid{\Delta}{\sigma}$ & $\sigma$ is a well-formed stack type \\
  $\valid{\Delta}{\Gamma}$ & $\Gamma$ is a well-formed register assignment \\
  $\valid{\Delta}{\Psi_{\mathrm{h}}}$ & $\Psi_{\mathrm{h}}$ is a well-formed heap label assignments \\
  \hline

  $\subtype{\Delta}{\tau_1}{\tau_2}$ & $\tau_1$ is a subtype of $\tau_2$ \\
  $\subtype{\Delta}{\phi_1}{\phi_2}$ & $\phi_1$ is a subtype of $\phi_2$ \\
  $\subtype{\Delta}{\sigma_1}{\sigma_2}$ & $\sigma_1$ is a subtype of $\sigma_2$ \\
  $\subtype{\Delta}{\Gamma_1}{\Gamma_2}$ & $\Gamma_1$ is a subtype of $\Gamma_2$ \\
  $\subtype{\Delta}{\Psi_{\mathrm{h},1}}{\Psi_{\mathrm{h}_2}}$ & $\Psi_{\mathrm{h},1}$ is a subtype of $\Psi_{\mathrm{h},2}$ \\
  \hline

  $\Delta |- \theta : a$ & $\theta$ is a valid instantiation of the variable $a$ \\
  $\Delta |- \Theta : \Delta'$ & $\Theta$ are valid instantiations of the variables $\Delta'$ \\
  \hline
\end{tabular}


\subsection{Valid types}
\fbox{$\valid{\Delta}{\tau}$}
\begin{mathpar}
\infer{\alpha \in \Delta}{\valid{\Delta}{\alpha}} \and
\infer{ }{\valid{\Delta}{\mathtt{int}}} \and
\infer{ }{\valid{\Delta}{\mathtt{uninit}}} \and
\infer{\valid{\Delta' ++ \Delta}{\Gamma}}{\valid{\Delta}{\forall[ \Delta' ] \Gamma}} \and
\infer{\valid{\Delta}{\tau_1} \and \dots \and \valid{\Delta}{\tau_n}}
      {\valid{\Delta}{\langle \tau_1^{\phi_1}, \dots, \tau_n^{\phi_n} \rangle}}
\end{mathpar}

\fbox{$\valid{\Delta}{\sigma}$}
\begin{mathpar}
\infer{\rho \in \Delta}{\valid{\Delta}{\rho}} \and
\infer{ }{\valid{\Delta}{\nil}} \and
\infer{
  \valid{\Delta}{\tau} \and \valid{\Delta}{\sigma}
}{
  \valid{\Delta}{\tau :: \sigma}
}
\end{mathpar}

\fbox{$\valid{\Delta}{\Gamma}$}
\begin{mathpar}
\infer{
  \valid{\Delta}{\sigma} \and
  \valid{\Delta}{\tau_1} \and
  \dots \and
  \valid{\Delta}{\tau_{\mathrm{\mathrm{max}}}}
}{
  \valid{\Delta}{\{\mathtt{sp} |-> \sigma, \mathtt{r}_1 |-> \tau_1, \dots, \mathtt{r}_{\mathrm{\mathrm{max}}} |-> \tau_{\mathrm{\mathrm{max}}}\}}
}
\end{mathpar}

\fbox{$\valid{\Delta}{\Psi_{\mathrm{h}}}$}
\begin{mathpar}
\infer{
  \valid{\Delta}{\tau_1} \and
  \dots \and
  \valid{\Delta}{\tau_{n}}
}{
  \valid{\Delta}{\{\ell_{\mathrm{h},1} |-> \tau_1, \dots, \ell_{\mathrm{h},n} |-> \tau_{n}\}}
}
\end{mathpar}

\subsection{Subtypes}
\fbox{$\subtype{\Delta}{\tau_1}{\tau_2}$}
\begin{mathpar}
\infer{\alpha \in \Delta}{\subtype{\Delta}{\alpha}{\alpha}} \and
\infer{ }{\subtype{\Delta}{\mathtt{int}}{\mathtt{int}}} \and
\infer{ }{\subtype{\Delta}{\mathtt{uninit}}{\mathtt{uninit}}} \and
\infer{
  \subtype{\Delta' ++ \Delta}{\Gamma_2}{\Gamma_1}
}{
  \subtype{\Delta}{\forall[\Delta'] \Gamma_1}{\forall[\Delta'] \Gamma_2}
} \and
\infer{
  \subtype{}{\phi_1}{\phi_1'} \and
  \dots \and
  \subtype{}{\phi_n}{\phi_n'}
}{
  \subtype{\Delta}
          {\langle \tau_1^{\phi_1}, \dots, \tau_n^{\phi_n} \rangle}
          {\langle \tau_1^{\phi'_1}, \dots, \tau_n^{\phi'_n} \rangle}
}
\end{mathpar}

\fbox{$\subtype{}{\phi_1}{\phi_2}$}
\begin{mathpar}
\infer{ }{\subtype{}{\mathtt{init}}{\phi}} \and
\infer{ }{\subtype{}{\mathtt{uninit}}{\mathtt{uninit}}}
\end{mathpar}

\fbox{$\subtype{\Delta}{\sigma}{\sigma}$}
\begin{mathpar}
\infer{\rho \in \Delta}{\subtype{\Delta}{\rho}{\rho}} \and
\infer{ }{\subtype{\Delta}{\nil}{\nil}} \and
\infer{
  \subtype{\Delta}{\tau}{\tau'} \and
  \subtype{\Delta}{\sigma}{\sigma'} \and
}{
  \subtype{\Delta}{\tau :: \sigma}{\tau' :: \sigma'}
}
\end{mathpar}

\fbox{$\subtype{\Delta}{\Gamma}{\Gamma}$}
\begin{mathpar}
\infer{
  \subtype{\Delta}{\sigma}{\sigma'} \and
  \subtype{\Delta}{\tau_1}{\tau_1'} \and
  \dots \and
  \subtype{\Delta}{\tau_{\mathrm{\mathrm{max}}}}{\tau_{\mathrm{\mathrm{max}}}'}
}{
  \subtype{\Delta}
          {\{\mathtt{sp} |-> \sigma, \mathtt{r}_1 |-> \tau_1, \dots, \mathtt{r}_{\mathrm{\mathrm{max}}} |-> \tau_{\mathrm{\mathrm{max}}}\}}
          {\{\mathtt{sp} |-> \sigma', \mathtt{r}_1 |-> \tau_1', \dots, \mathtt{r}_{\mathrm{\mathrm{max}}} |-> \tau_{\mathrm{\mathrm{max}}}'\}}
}
\end{mathpar}

\fbox{$\subtype{\Delta}{\Psi_{\mathrm{h}}}{\Psi_{\mathrm{h}}}$}
\begin{mathpar}
\infer{
  \subtype{\Delta}{\tau_1}{\tau_1'} \and
  \dots \and
  \subtype{\Delta}{\tau_{n}}{\tau_{n}'}
}{
  \subtype{\Delta}
          {\{\ell_{\mathrm{h},1} |-> \tau_1, \dots, \ell_{\mathrm{h},n} |-> \tau_{n}\}}
          {\{\ell_{\mathrm{h},1} |-> \tau_1', \dots, \ell_{\mathrm{h},n} |-> \tau_{n}'\}}
}
\end{mathpar}

\subsection{Instantiations}

\fbox{$\ofinstantiation{\Delta}{\theta}{a}$}
\begin{mathpar}
\infer{
  \alpha \notin \Delta \and
  \valid{\Delta}{\tau}
}{
  \ofinstantiation{\Delta}{\tau / \alpha}{\alpha}
} \and
\infer{
  \rho \notin \Delta \and
  \valid{\Delta}{\sigma}
}{
  \ofinstantiation{\Delta}{\sigma / \rho}{\rho}
}
\end{mathpar}

\fbox{$\ofinstantiations{\Delta}{\Theta}{\Delta}$}
\begin{mathpar}
\infer{ }{
  \ofinstantiations{\Delta}{\nil}{\nil}
} \and
\infer{
  \ofinstantiation{\Delta' ++ \Delta}{\theta}{a} \and
  \ofinstantiations{\Delta}{\Theta}{\Delta'}
}{
  \ofinstantiations{\Delta}{(\theta :: \Theta)}{(a :: \Delta')}
}
\end{mathpar}

\subsection{Notes on the Agda implementation}
\texttt{Judgments/Types.agda} and some of \texttt{Judgments/Terms.agda}.

\subsection{Lemmas}

\texttt{Lemmas/Types.agda} and \texttt{Lemmas/TypeSubstitution.agda}.

\begin{lemma}
  \label{lemma:typdec}
  Validity and subtyping is decidable. So are the judgments
  $\ofinstantiation{\Delta}{\theta}{a}$ and
  $\ofinstantiations{\Delta}{\Theta}{\Delta'}$.
\end{lemma}

\begin{lemma}
  \label{lemma:typeq}
  Validity implies subtyping and vice versa.
\end{lemma}

\begin{lemma}
  \label{lemma:transitive}
  Subtyping is transitive.
\end{lemma}

\begin{lemma}
  \label{lemma:typ-context}
  Validity and subtyping is preserved by weakening or instantiation variables
  from the context.\footnote{Remember that substitution is a partial function in
    the Agda-code. In the Agda-code this lemma also implies totality of
    substitution given certain additional assumptions.}
\end{lemma}

\section{Judgments about values}
\begin{tabular}{|c|p{7.5 cm}|}
  \hline
  Judgment & \multicolumn{1}{|c|}{Meaning} \\
  \hline

  $\Psi_{\mathrm{g}}, \Delta |- \high v : \Gamma => \tau$ & $\high v$ is a well-formed small value evaluating register files of type $\Gamma$ to word values of type $\tau$ \\
  $\Psi_{\mathrm{g}}, \Delta |- \high \iota : \Gamma_1 => \Gamma_2$ & $\high \iota$ is a well-formed instruction which evaluating registers of type $\Gamma_1$ to registers of type $\Gamma_2$ \\
  $\Psi_{\mathrm{g}}, \Delta |- \high I : \Gamma$ & $\high I$ is a well-formed instruction sequence which will correctly evaluate registers of type $\Gamma$. \\
  \hline

  $\Psi_{\mathrm{g}}, \Psi_{\mathrm{h}} |- \high w : \tau$ & $\high w$ is a well-formed word value of type $\tau$ \\
  $\Psi_{\mathrm{g}}, \Psi_{\mathrm{h}} |- \high w : \tau^{\phi}$ & $\high w$ is a well-formed (and possibly uninitialized) word value of type $\tau^\phi$ (i.e., either $\high w : \tau$ or $\high w = \mathtt{uninit}$ and $\phi = \mathtt{uninit}$) \\
  $\Psi_{\mathrm{g}}, \Psi_{\mathrm{h}} |- \high S : \sigma$ & $\high S$ is a well-formed stack of type $\sigma$ \\
  $\Psi_{\mathrm{g}}, \Psi_{\mathrm{h}} |- \high R : \Gamma$ & $\high R$ is a well-formed register file of type $\Gamma$ \\
  $\Psi_{\mathrm{g}}, \Psi_{\mathrm{h}} |- \high h : \tau$ & $\high h$ is a well-formed heap value of type $\tau$ \\
  $\Psi_{\mathrm{g}} |- \high g : \tau$ & $\high g$ is a well-formed global value of type $\tau$. \\
  $\Psi_{\mathrm{g}} |- \high H : \Psi_{\mathrm{h}}$ & $\high H$ is a well-formed heap collection of type $\Psi_{\mathrm{h}}$ \\
  $|- \high G : \Psi_{\mathrm{g}}$ & $\high G$ is a well-formed global collection of type $\Psi_{\mathrm{g}}$. \\
  \hline

  $\Psi_{\mathrm{g}} |- (\high H, \high R, \high I) : (\Psi_{\mathrm{h}}, \Gamma)$ & $(\high H, \high R, \high I)$ are all valid while using types $\Psi_{\mathrm{h}}$ and $\Gamma$ for the heap and registers. \\
  $\valid{}{\high P}$ & $\high P$ is a well-formed and well-typed program. \\
  \hline
\end{tabular}

\subsection{Evaluation types}

\fbox{$\ofvval{\Psi_{\mathrm{g}},\Delta}{\high v}{\Gamma => \tau}$}
\begin{mathpar}
\infer{ }{
  \ofvval{\Psi_{\mathrm{g}},\Delta}{\mathtt{reg}\ r}{\Gamma => \Gamma[r]}
}\and
\infer{
}{
  \ofvval{\Psi_{\mathrm{g}},\Delta}{\mathtt{globval}\ \ell_{\mathrm{g}}}{\Gamma => \Psi_{\mathrm{g}}[\ell_{\mathrm{g}}]}
}\and
\infer{ }{\ofvval{\Psi_{\mathrm{g}},\Delta}{\mathtt{int}\ i}{\Gamma => \mathtt{int}}} \and
\infer{
  \ofvval{\Psi_{\mathrm{g}},\Delta}{\high v}{\Gamma => \forall[\Delta_1]\Gamma_1} \and
  \ofinstantiations{\Delta_2 ++ \Delta}{\Theta}{\Delta_1} \and
  \Gamma_2 = \Gamma_1[\Theta] \\
}{
  \ofvval{\Psi_{\mathrm{g}},\Delta}{\Lambda\ \Delta \cdot \high v[\Theta]}{\Gamma => \forall[\Delta_2]\Gamma_2}
}
\end{mathpar}

\fbox{$\ofinstruction{\Psi_{\mathrm{g}}, \Delta}{\high \iota}{\Gamma_1 => \Gamma_2}$}
\begin{mathpar}
\infer{
  \Gamma[r_b] = \mathtt{int} \and
  \ofvval{\Psi_{\mathrm{g}}, \Delta}{\high v}{\Gamma => \mathtt{int}}
}{
  \ofinstruction{\Psi_{\mathrm{g}}, \Delta}
    {\mathtt{add}\ r_a, r_b, \high v}
    {\Gamma => \Gamma[r_a |-> \mathtt{int}]}
} \and
\infer{
  \Gamma[r_b] = \mathtt{int} \and
  \ofvval{\Psi_{\mathrm{g}}, \Delta}{\high v}{\Gamma => \mathtt{int}}
}{
  \ofinstruction{\Psi_{\mathrm{g}}, \Delta}
    {\mathtt{sub}\ r_a, r_b, \high v}
    {\Gamma => \Gamma[r_a |-> \mathtt{int}]}
} \\
\infer{
  \sigma = \overbrace{\mathtt{uninit} :: \dots :: \mathtt{uninit}}^n :: \Gamma[\mathtt{sp}]
}{
  \ofinstruction{\Psi_{\mathrm{g}}, \Delta}
    {\mathtt{salloc}\ n}
    {\Gamma => \Gamma[\mathtt{sp} |-> \sigma]}
} \and
\infer{
  \Gamma[\mathtt{sp}] = \overbrace{\tau_1 :: \dots :: \tau_n}^n :: \sigma
}{
  \ofinstruction{\Psi_{\mathrm{g}}, \Delta}
    {\mathtt{sfree}\ n}
    {\Gamma => \Gamma[\mathtt{sp} |-> \sigma]}
} \and
\infer{
  \Gamma[\mathtt{sp}] = \sigma
}{
  \ofinstruction{\Psi_{\mathrm{g}}, \Delta}
    {\mathtt{ld}\ r, \mathtt{sp}(n)}
    {\Gamma => \Gamma[r |-> \sigma[n]]}
} \and
\infer{
  \Gamma[r_b] = \langle \tau_1^{\phi_1}, \dots, \tau_n^{\mathtt{init}}, \dots\rangle \and
}{
  \ofinstruction{\Psi_{\mathrm{g}}, \Delta}
    {\mathtt{ld}\ r_a, r_b(n)}
    {\Gamma => \Gamma[r_a |-> \tau_n]}
} \and
\infer{
  \Gamma[r] = \tau \and
  \Gamma[\mathtt{sp}] = \sigma
}{
  \ofinstruction{\Psi_{\mathrm{g}}, \Delta}
    {\mathtt{st}\ \mathtt{sp}(n), r}
    {\Gamma => \Gamma[\mathtt{sp} |-> \sigma[n |-> \tau]]}
} \and
\infer{
  \Gamma[r_a] = \langle \tau_1^{\phi_1}, \dots, \tau_n^{\phi_i}, \dots \rangle \and
  \Gamma[r_b] = \tau_n' \and
  \subtype{\Delta}{\tau_n'}{\tau_n}
}{
  \ofinstruction{\Psi_{\mathrm{g}}, \Delta}
    {\mathtt{st}\ r_a(n), r_b}
    {\Gamma => \Gamma[r_a |-> \langle \tau_1^{\phi_1}, \dots, \tau_n^{\mathtt{init}}, \dots, \rangle]}
} \and
\infer{
  \valid{\Delta}{\tau_1} \and
  \dots \and
  \valid{\Delta}{\tau_n}
}{
  \ofinstruction{\Psi_{\mathrm{g}}, \Delta}
    {\mathtt{malloc}\ r, \langle \tau_1, \dots, \tau_n \rangle}
    {\Gamma => \Gamma[r |-> \langle \tau_1^{\mathtt{uninit}}, \dots, \tau_n^{\mathtt{uninit}}\rangle]}
} \and
\infer{
  \ofvval{\Psi_{\mathrm{g}}, \Delta}{\high v}{\Gamma => \tau}
}{
  \ofinstruction{\Psi_{\mathrm{g}}, \Delta}
    {\mathtt{mov}\ r, \high v}
    {\Gamma => \Gamma[r |-> \tau]}
} \and
\infer{
  \Gamma[r] = \mathtt{int} \and
  \ofvval{\Psi_{\mathrm{g}}, \Delta}{\high v}{\Gamma => \forall[ \nil ] \Gamma'} \and
  \subtype{\Delta}{\Gamma}{\Gamma'}
}{
  \ofinstruction{\Psi_{\mathrm{g}}, \Delta}
    {\mathtt{beq}\ r, \high v}
    {\Gamma => \Gamma}
}
\end{mathpar}


\fbox{$\ofinstructions{\Psi_{\mathrm{g}}, \Delta}{\high I}{\Gamma}$}
\begin{mathpar}
\infer{
  \ofinstruction{\Psi_{\mathrm{g}}, \Delta}{\high \iota}{\Gamma => \Gamma'} \and
  \ofinstructions{\Psi_{\mathrm{g}}, \Delta}{\high I}{\Gamma'}
}{
  \ofinstructions{\Psi_{\mathrm{g}}, \Delta}{\high \iota ; \high I}{\Gamma}
} \and
\infer{
  \ofvval{\Psi_1, \Delta}{\high v}{\Gamma => \forall[ \nil ]\Gamma'} \and
  \subtype{\Delta}{\Gamma}{\Gamma'}
}{
  \ofinstructions{\Psi_{\mathrm{g}}, \Delta}{\mathtt{jmp}\ \high v}{\Gamma}
} \and
\infer{
}{
  \ofinstructions{\Psi_{\mathrm{g}}, \Delta}{\mathtt{halt}}{\Gamma}
}
\end{mathpar}

\subsection{Memory constructs}

\fbox{$\ofwval{\Psi_{\mathrm{g}},\Psi_{\mathrm{h}}}{\high w}{\tau}$}
\begin{mathpar}
\infer{
  \Psi_{\mathrm{g}}[\ell_{\mathrm{g}}] = \tau_1 \and
  \subtype{\nil}{\tau_1}{\tau_2}
}{
  \ofwval{\Psi_{\mathrm{g}},\Psi_{\mathrm{h}}}{\mathtt{globval}\ \ell_{\mathrm{g}}}{\tau_2}
}\and
\infer{
  \Psi_{\mathrm{h}}[\ell_{\mathrm{h}}] = \tau_1 \and
  \subtype{\nil}{\tau_1}{\tau_2}
}{
  \ofwval{\Psi_{\mathrm{g}},\Psi_{\mathrm{h}}}{\mathtt{heapval}\ \ell_{\mathrm{h}}}{\tau_2}
}\and
\infer{ }{\ofwval{\Psi_{\mathrm{g}},\Psi_{\mathrm{h}}}{\mathtt{int}\ i}{\mathtt{int}}} \and
\infer{ }{\ofwval{\Psi_{\mathrm{g}},\Psi_{\mathrm{h}}}{\mathtt{uninit}}{\mathtt{uninit}}} \and
\infer{
  \ofwval{\Psi_{\mathrm{g}},\Psi_{\mathrm{h}}}{\high w}{\forall[\Delta_1]\Gamma_1} \and
  \ofinstantiations{\Delta_2}{\Theta}{\Delta_1} \and
  \subtype{\Delta_2}{\Gamma_2}{\Gamma_1[\Theta]} \\
}{
  \ofwval{\Psi_{\mathrm{g}},\Psi_{\mathrm{h}}}{\Lambda\ \Delta_2 \cdot \high w[\Theta]}{\forall[\Delta_2]\Gamma_2}
}
\end{mathpar}

\fbox{$\ofwvaln{\Psi_{\mathrm{g}},\Psi_{\mathrm{h}}}{\high w}{\tau^\phi}$}
\begin{mathpar}
\infer{
  \valid{\nil}{\tau}
}{
  \ofwvaln{\Psi_{\mathrm{g}},\Psi_{\mathrm{h}}}{\mathtt{uninit}}{\tau^{\mathtt{uninit}}}
} \and
\infer{
  \ofwval{\Psi_{\mathrm{g}},\Psi_{\mathrm{h}}}{\high w}{\tau}
}{
  \ofwvaln{\Psi_{\mathrm{g}},\Psi_{\mathrm{h}}}{\high w}{\tau^{\mathtt{init}}}
}
\end{mathpar}

\fbox{$\ofstack{\Psi_{\mathrm{g}},\Psi_{\mathrm{h}}}{\high S}{\sigma}$}
\begin{mathpar}
\infer{ }{\ofstack{\Psi_{\mathrm{g}},\Psi_{\mathrm{h}}}{\nil}{\nil}} \and
\infer{
  \ofword{\Psi_{\mathrm{g}},\Psi_{\mathrm{h}}}{\high w}{\tau} \and
  \ofstack{\Psi_{\mathrm{g}},\Psi_{\mathrm{h}}}{\high S}{\sigma}
}{
  \ofstack{\Psi_{\mathrm{g}},\Psi_{\mathrm{h}}}{\high w :: \high S}{\tau :: \sigma}
}
\end{mathpar}

\fbox{$\ofregister{\Psi_{\mathrm{g}},\Psi_{\mathrm{h}}}{\high R}{\Gamma}$}
\begin{mathpar}
\infer{
  \ofstack{\Psi_{\mathrm{g}},\Psi_{\mathrm{h}}}{\high S}{\sigma} \and
  \ofwval{\Psi_{\mathrm{g}},\Psi_{\mathrm{h}}}{\high w_1}{\tau_1} \and
  \dots \and
  \ofwval{\Psi_{\mathrm{g}},\Psi_{\mathrm{h}}}{\high w_{\mathrm{max}}}{\tau_{\mathrm{max}}} \and
}{
  \ofregister{\Psi_{\mathrm{g}},\Psi_{\mathrm{h}}}{\{\mathtt{sp} |->  \high S, \mathtt{r}_1 |->  \high w_1, \dots, \mathtt{r}_{\mathrm{max}} |->  \high w_{\mathrm{max}}\}}{\{\mathtt{sp} |-> \sigma, \mathtt{r}_1 |-> \tau_1, \dots, \mathtt{r}_{\mathrm{max}} |-> \tau_{\mathrm{max}}\}}
}
\end{mathpar}

\fbox{$\ofhval{\Psi_{\mathrm{g}},\Psi_{\mathrm{h}}}{\high h}{\tau}$}
\begin{mathpar}
\infer{
  \ofwvaln{\Psi_{\mathrm{g}},\Psi_{\mathrm{h}}}{\high w_1}{\tau_1^{\phi_1}} \and
  \dots \and
  \ofwvaln{\Psi_{\mathrm{g}},\Psi_{\mathrm{h}}}{\high w_n}{\tau_n^{\phi_n}}
}{
  \ofhval{\Psi_{\mathrm{g}},\Psi_{\mathrm{h}}}{\mathtt{tuple}\ \langle \high \tau_1, \dots, \high \tau_n \rangle\ \langle \high w_1, \dots, \high w_n \rangle}{\langle \tau_1^{\phi_1}, \dots, \tau_n^{\phi_n}\rangle}
}
\end{mathpar}

\fbox{$\ofgval{\Psi_{\mathrm{g}}}{\high g}{\tau}$}
\begin{mathpar}
\infer{
  \valid{\Delta}{\Gamma} \and
  \ofinstructions{\Psi_{\mathrm{g}}, \Delta}{\high I}{\Gamma}
}{
  \ofgval{\Psi_{\mathrm{g}}}{\mathtt{code}[\Delta]\Gamma \cdot \high I}{\forall[\Delta]\Gamma}
}
\end{mathpar}

\fbox{$\ofheap{\Psi_{\mathrm{g}}}{\high H}{\Psi_{\mathrm{h}}}$}
\begin{mathpar}
\infer{
  \ofhval{\Psi_{\mathrm{g}},\Psi_{\mathrm{h}}}{\high h_1}{\tau_1} \and
  \dots \and
  \ofhval{\Psi_{\mathrm{g}},\Psi_{\mathrm{h}}}{\high h_n}{\tau_n} \and
  \Psi_{\mathrm{h}} = \{\ell_{h,1} |-> \tau_1, \dots, \ell_{h,n} |-> \tau_n\}
}{
  \ofheap{\Psi_{\mathrm{g}}}{\{\ell_{h,1} |->  \high h_1, \dots, \ell_{h,n}  |->  \high h_n\}}{\Psi_{\mathrm{h}}}
}
\end{mathpar}

\fbox{$\ofglobs{\high G}{\Psi_{\mathrm{g}}}$}
\begin{mathpar}
\infer{
  \ofgval{\Psi_{\mathrm{g}}}{\high g_1}{\tau_1} \and
  \dots \and
  \ofgval{\Psi_{\mathrm{g}}}{\high g_n}{\tau_n} \and
  \Psi_{\mathrm{g}} = \{\ell_{g,1} |-> \tau_1, \dots, \ell_{g,n} |-> \tau_n\}
}{
  \ofglobs{\{\ell_{g,1} |->  \high g_1, \dots, \ell_{g,n}  |->  \high g_n\}}{\Psi_{\mathrm{g}}}
}
\end{mathpar}

\subsection{Program states}

\fbox{$\ofprogramstate{\Psi_{\mathrm{g}}}{(\high H, \high R, \high I)}{(\Psi_{\mathrm{h}},\Gamma)}$}
\begin{mathpar}
\infer{
  \ofheap{\Psi_{\mathrm{g}}}{\high H}{\Psi_{\mathrm{h}}} \\
  \ofregister{\Psi_{\mathrm{g}},\Psi_{\mathrm{h}}}{\high R}{\Gamma} \and
  \ofinstructions{\Psi_{\mathrm{g}},\nil}{\high I}{\Gamma}
}{
  \ofprogramstate{\Psi_{\mathrm{g}}}{(\high H,\high R,\high I)}{(\Psi_{\mathrm{h}},\Gamma)}
}
\end{mathpar}

\fbox{$\valid{}{\high P}$}
\begin{mathpar}
\infer{
  \ofglobs{\high G}{\Psi_{\mathrm{g}}} \and
  \ofprogramstate{\Psi_{\mathrm{g}}}{(\high H, \high R, \high I)}{(\Psi_{\mathrm{h}}, \Gamma)}
}{
  \valid{}{(\high G,\high H, \high R, \high I)}
}
\end{mathpar}

\subsection{Notes on the Agda implementation}
\texttt{Judgments/Terms.agda}

\subsection{Lemmas}

\texttt{Lemmas/TermWeaken.agda} and \texttt{Lemmas/TermSubstitution.agda}.

\begin{lemma}
  \label{lemma:instructions-context}
  The judgment $\ofinstructions{\Psi_{\mathrm{g}}, \Delta}{\high I}{\Gamma}$ is
  preserved by weakening or instantiation variables
  from the context.\footnote{Remember that substitution is a partial function in
    the Agda-code. In the Agda-code this lemma also implies totality of
    substitution given certain additional assumptions.}
\end{lemma}

\texttt{Lemmas/TermCast.agda}
\begin{lemma}
  \label{lemma:evaluatin-casting}
  \begin{itemize}
  \item Assume $\ofvval{\Psi_{\mathrm{g}},\Delta}{\high v}{\Gamma => \tau}$
    and $\subtype{\Delta}{\Gamma'}{\Gamma}$. In that case there is some
    $\tau'$ such that
    $\ofvval{\Psi_{\mathrm{g}},\Delta}{\high v}{\Gamma' => \tau'}$.

  \item Assume $\ofinstruction{\Psi_{\mathrm{g}},\Delta}{\high \iota}{\Gamma_1 => \Gamma_2}$
    and $\subtype{\Delta}{\Gamma_1'}{\Gamma_1}$. In that case there is some
    $\Gamma_2'$ such that
    $\ofinstruction{\Psi_{\mathrm{g}},\Delta}{\high \iota}{\Gamma_1' => \Gamma_2'}$.

  \item Assume $\ofinstructions{\Psi_{\mathrm{g}},\Delta}{\high I}{\Gamma}$ and
    $\subtype{\Delta}{\Gamma'}{\Gamma}$. In that case
    $\ofinstructions{\Psi_{\mathrm{g}},\Delta}{\high I}{\Gamma'}$.
  \end{itemize}
\end{lemma}

\begin{lemma}
  \label{lemma:value-casting}
  Assume $\ofwval{\Psi_{\mathrm{g}},\Psi_{\mathrm{h}}}{\high w}{\tau}$ and that
  $\subtype{\nil}{\tau}{\tau'}$. In that case
  $\ofwval{\Psi_{\mathrm{g}},\Psi_{\mathrm{h}}}{\high w}{\tau'}$. Similar
  statements hold for stacks, register files and heap values.
\end{lemma}

\texttt{Lemmas/TermHeapCast.agda}
\begin{lemma}
  \label{lemma:heap-casting}
  Assume $\ofwval{\Psi_{\mathrm{g}},\Psi_{\mathrm{h}}}{\high w}{\tau}$ and that
  $\subtype{\nil}{\Psi_{\mathrm{h}}'}{\Psi_{\mathrm{h}}}$. In that case
  $\ofwval{\Psi_{\mathrm{g}},\Psi_{\mathrm{h}}'}{\high w}{\tau}$. Similar
  statements hold for stacks, register files and heap values.
\end{lemma}

\section{Soundness proof}

This section will give a very broad outline of the proof of soundness. The
actual proof is implemented in \texttt{Lemmas/Soundness.agda} and has the
type-signature:

\begin{code}
\>\AgdaFunction{steps-soundness} \AgdaSymbol{:} \AgdaSymbol{∀} \AgdaSymbol{\{}\AgdaBound{n} \AgdaBound{P₁} \AgdaBound{P₂}\AgdaSymbol{\}} \AgdaSymbol{→}\<%
\\
\>[2]\AgdaIndent{20}{}\<[20]%
\>[20]\AgdaDatatype{⊢} \AgdaBound{P₁} \AgdaDatatype{programstate} \AgdaSymbol{→}\<%
\\
\>[2]\AgdaIndent{20}{}\<[20]%
\>[20]\AgdaFunction{⊢} \AgdaBound{P₁} \AgdaFunction{⇒ₙ} \AgdaBound{n} \AgdaFunction{/} \AgdaBound{P₂} \AgdaSymbol{→}\<%
\\
\>[2]\AgdaIndent{20}{}\<[20]%
\>[20]\AgdaFunction{GoodState} \AgdaBound{P₂}\<%
\end{code}

For the actual proof of this statement, we will refer to the file. In
\cref{sec:soundness-proof} you will also find a proof sketch that follows the
structure of the this proof. Here we will instead discuss the very broad lines
included how the language was designed to permit this result.

\todo{Discuss why this is possible.}

\section{Decidability}

In this section we introduce our proof that the type system is decidable. In
particular the file \texttt{Lemmas/TermDec.agda} contains the function:

\begin{code}
\>\AgdaFunction{programstate-dec} \AgdaSymbol{:} \AgdaSymbol{∀} \AgdaBound{P} \AgdaSymbol{→} \AgdaDatatype{Dec} \AgdaSymbol{(}\AgdaDatatype{⊢} \AgdaBound{P} \AgdaDatatype{programstate}\AgdaSymbol{)}\<%
\end{code}

The actual proof is not particularly interesting, though we have given a sketch
of it in \cref{sec:decproof}. What is more interesting to discuss is how our
language was designed to allow this result to be achieved.

\todo{Discuss why this is possible.}

% \section{Subtyping judgments}

\fbox{$\subtype{\Delta}{\tau_1}{\tau_2}$}
\begin{mathpar}
\infer{\alpha \in \Delta}{\subtype{\Delta}{\alpha}{\alpha}} \and
\infer{ }{\subtype{\Delta}{\mathtt{int}}{\mathtt{int}}} \and
\infer{ }{\subtype{\Delta}{\mathtt{ns}}{\mathtt{ns}}} \and
\infer
  {\valid{\Delta}{\Delta'} \and
    \subtype{\Delta', \Delta}{\Gamma_1}{\Gamma_2}}
  {\subtype{\Delta}{\forall[ \Delta' ]\Gamma_1}{\forall[ \Delta' ]\Gamma_2}} \and
\infer
  {\subtype{\Delta}{\tau_1^{\phi_1}}{\tau'_1{}^{\phi'_1}} \and
    \dots \and
    \subtype{\Delta}{\tau_n^{\phi_n}}{\tau'_n{}^{\phi'_n}}}
  {\subtype{\Delta}{\langle \tau'_1{}^{\phi_1}, \dots, \tau_n{}^{\phi_n} \rangle}
                   {\langle \tau'_1{}^{\phi'_1}, \dots, \tau'_n{}^{\phi'_n} \rangle}}
\end{mathpar}

\fbox{$\subtype{}{\phi_1}{\phi_2}$}
\begin{mathpar}
\infer{ }{\subtype{}{\mathtt{init}}{\mathtt{init}}} \and
\infer{ }{\subtype{}{\mathtt{\phi}}{\mathtt{uninit}}}
\end{mathpar}

\fbox{$\subtype{\Delta}{\tau_1^{\phi_1}}{\tau_2^{\phi_2}}$}
\begin{mathpar}
\infer{\valid{\Delta}{\tau} \and \subtype{}{\phi_1}{\phi_2}}{\subtype{\Delta}{\tau^{\phi_1}}{\tau^{\phi_2}}}
\end{mathpar}

% \section{Substitution judgments}

\fbox{$\substitution{\tau_1}{c}{\tau_2}$}
\begin{mathpar}
  \infer{ }{\substitution{\alpha}{\alpha => \tau}{\tau}} \and
  \infer{\alpha_1 \neq \alpha_2}{\substitution{\alpha_1}{\alpha_2 => \tau}{\alpha_1}} \and
  \infer{ }{\substitution{\alpha}{\rho => \sigma}{\alpha}} \and
  \infer{ }{\substitution{\alpha}{\mathtt{weaken}\ \Delta^{+}}{\alpha}}
\end{mathpar}

(This also includes obvious, mutually recursive definition.)\\\\
\fbox{$\substitution{\tau_1^{\phi_1}}{c}{\tau_2^{\phi_2}}$}\\

(The obvious, mutually recursive definitions.)\\\\
\fbox{$\substitution{\sigma_1}{c}{\sigma_2}$}
\begin{mathpar}
  \infer{ }{\substitution{\rho}{\rho => \sigma}{\sigma}} \and
  \infer{\rho_1 \neq \rho_2}{\substitution{\rho_1}{\rho_2 => \sigma}{\rho_1}} \and
  \infer{ }{\substitution{\rho}{\alpha => \tau}{\rho}} \and
  \infer{ }{\substitution{\rho}{\mathtt{weaken}\ \Delta^{+}}{\rho}}
\end{mathpar}

(This also includes obvious, mutually recursive definition.)\\\\
\fbox{$\substitution{\Delta_1}{c}{\Delta_2}$}\\

(The obvious, mutually recursive definitions.)\\\\
\fbox{$\substitution{a_1}{c}{a_2}$}\\

(The obvious, mutually recursive definitions.)\\\\
\fbox{$\substitution{\Gamma_1}{c}{\Gamma_2}$}\\
\begin{mathpar}
  \infer{ }{\substitution{\mathtt{nil}}{c}{\mathtt{nil}}} \and
  \infer{
    \substitution{a_1}{c}{a_2} \and
    \substitution{\Delta_1}{c}{\Delta_2}
  }{
    \substitution{(a_1 :: \Delta_1)}{c}{a_2 :: \Delta_2}
  }
\end{mathpar}\\
\fbox{$\substitution{c_1}{c}{c_2}$}\\

(The obvious, mutually recursive definitions.)\\\\
\fbox{$\substitution{w_1}{c}{w_2}$}\\

(The obvious, mutually recursive definitions.)\\\\
\fbox{$\substitution{v_1}{c}{v_2}$}\\

(The obvious, mutually recursive definitions.)\\\\
\fbox{$\substitution{\iota_1}{c}{\iota_2}$}\\

(The obvious, mutually recursive definitions.)\\\\
\fbox{$\substitution{I_1}{c}{I_2}$}\\

(The obvious, mutually recursive definitions.)

\subsection{Comments}

I write that the rules are obvious -- and they are, as long as you keep yourself
to using paper-notation. If you want to use De Bruijn indices they are not
nearly as obvious.

Note that the substitution for $\Delta$ assumes that the actual substitution is
to be done elsewhere -- it merely updates all references inside the $\Delta$. In
the current grammar there \emph{are} no references inside the assumptions, so we
get that $\substitution{\Delta}{c}{\Delta}$ for all $\Delta$. This would ideally
not hold in the final version.

For the substitution that actually updates the assumption list, see the next
section.

\subsection{Differences from STAL}

STAL handwaves even more that I do here. Rest assured that my Agda-code is
precise enough for a machine-verified proof.

% \section{Run Lemmas}

\begin{lemma}
  Assume we have $\Delta, \Delta_1, \Delta_2$ and $c$ such that
  $\run{\Delta}{c}{\Delta_1}$ and $\run{\Delta}{c}{\Delta_2}$. We can then
  conclude that $\Delta_1 = \Delta_2$.
\end{lemma}

\begin{proof}
  By induction over the derivations of $\run{\Delta}{c}{\Delta_1}$ and
  $\run{\Delta}{c}{\Delta_2}$.
\end{proof}

\begin{lemma}
  Assume we have $\Delta$ and $c$. It is now decidable whether there exists some
  $\Delta'$ such that $\run{\Delta}{c}{\Delta'}$.
\end{lemma}

\begin{proof}
  By structural induction over $\Delta$, using \autoref{dec:substitution}.
\end{proof}

\begin{lemma}
  Assume we have some $\Delta$ and $c$, such that $\valid{\mathtt{nil}}{\Delta}$
  and $\valid{\Delta}{c}$. We can then find $\Delta'$ such that
  $\run{\Delta}{c}{\Delta'}$.
\end{lemma}

\begin{proof}
  Long proof using previous lemmas.
\end{proof}


\listoftodos[Notes]

\bibliographystyle{plain}
\bibliography{report}
\end{document}
