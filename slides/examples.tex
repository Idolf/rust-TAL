
% \begin{frame}
%   \frametitle{Motivation}
%   \begin{itemize}
%   \item We want to add \emph{coroutines} to Janus. These are like
%   subroutines/functions, except their execution can be \emph{paused},
%   and be resumed by anybody who has a ``pointer'' to it.

%   \pause\item As a first class citizen of the language, they allow
%   \emph{the user} to implement many powerful mechanisms, such as
%   simple concurrent programming, ``unix pipe''-like data transfering,
%   and higher order functions (map, fold, filter).

%   \pause\item They also form the theoretical foundation of language features
%   such as generators from Python.
%   \end{itemize}
% \end{frame}

% \begin{frame}
%   \frametitle{Intuition}
%   When a coroutine is not active, we maintain the following state
%   about it:
%     \begin{itemize}
%     \pause\item The function which the coroutine is an instance of
%     \pause\item A store with variables specific to \emph{this} coruotine
%       (arguments, local variables)
%     \pause\item A unique label indicating where the coroutine should be run
%       from next time it is resumed
%       \pause\begin{itemize}
%       \item $\go$ if just created, or just finished
%       \item $\pausel_i$ when last suspended at unique label $i$
%       \end{itemize}
%     \end{itemize}
% \end{frame}

% \begin{frame}[fragile]
%   \frametitle{Example -- fibonacci}
%   \begin{lstlisting}
% procedure fib (int n, int x1, int x2)[int ret]
%   x1 += 1
%   x2 += 1
%   from x1 = x2
%   do n -= 1
%      ret += x1
%      yield
%   loop x1 += x2
%        x1 <=> x2
%   until n = 0

% procedure main()
%   int a,b,c,d,arg1,arg2,arg3
%   local x = nil; create x (4,0,0)
%   resume x (a); resume x (b); resume x (c);
%   resume x (d);
%   uncreate x (arg1,arg2,arg3); delocal x = nil
%   \end{lstlisting}
% $a=1,b=2,c=3,d=0 \qquad arg1=0, arg2=3, arg3=5$
% \end{frame}

% \begin{frame}
%   \frametitle{Informal semantics -- \texttt{create}}
%   \begin{align*}
%     &\<procedure> f(y_0, \dots, y_n)[z_0, \dots, z_k] \\
%     &\quad\quad\dots \\
%     &\<create> c\; f(x_0, x_1, \dots, x_n)
%   \end{align*}

%   \begin{itemize}
%     \pause\item Replaces the value of $x_0$: $v_0 |-> 0$.
%     \pause\item Replaces the value of $x_1$: $v_1 |-> 0$.
%     \pause\item \dots
%     \item Replaces the value of $x_n$: $v_n |-> 0$.
%     \pause\item Replaces the value of $c$: $\nil |-> Coroutine(f, [y_0=v_0, \dots, y_n=v_n],\start)$.
%   \end{itemize}
%   \pause \texttt{uncreate} does the opposite.
% \end{frame}


% \begin{frame}
%   \frametitle{Informal semantics -- \texttt{resume}}
%   \begin{align*}
%     &\<procedure> f(y_0, \dots, y_n)[z_0, \dots, z_k] \\
%     &\quad\quad\dots \\
%     &\<resume> c(x_0, x_1, \dots, x_k)
%   \end{align*}

%   \begin{itemize}
%     \pause\item $c = Coroutine(f, store,label)$
%     \pause\item $x_0 = v_0, \dots, x_k = v_k$.
%     \pause\item Run procedure $f$ starting from $label$. For local variables use $store ++ [z_0 = v_0, \dots, z_k = v_k]$.
%     \pause\item When it is done, split resulting store into $store' ++ [z_0 = v_0', \dots, z_k = v_k']$.
%     \pause\item Save $x_0 = v_0', \dots, x_k = v_k'$.
%     \pause\item Save $c = Coroutine(f, store', label')$.
%   \end{itemize}

%   \pause \texttt{unresume} does the opposite.
% \end{frame}

% \begin{frame}[fragile]
%   \frametitle{Example}
%   \begin{lstlisting}
% procedure map_func(v)[x]
%   x += v

% procedure fold_func(r)[x]
%   r += x

% procedure main
%   ...
%   create c map_func(5)
%   call map(c, my_stack)
%   ...
%   create c fold_func(0)
%   call fold(c, my_stack)
%   ...
%   \end{lstlisting}
% \end{frame}

% \begin{frame}[fragile]
%   \frametitle{Example}
%   \begin{lstlisting}
% procedure traverse(c, s)[]
%   if empty(s)
%   then skip
%   else
%     local x = 0
%       pop(x, s)
%       resume c(x)
%       call traverse(c, s)
%       push(x, s)
%     delocal x = 0
%   fi empty(s)
%   \end{lstlisting}
% \end{frame}

% \begin{frame}
%   \frametitle{The statements of Janus}
%   \begin{tabular}{rl}
%     $s ::=$
%     & $x *= e \mid x\texttt{[}e\texttt{]} *= e$\\
%     & $\<if> e \<then> s \<else> s \<fi> e$\\
%     & $\<from> e \<do> s \<loop> s \<until> e$\\
%     & $\<push>(x,x) \mid \<pop>(x,x)$\\
%     & $\<local> x = e \:\: s \:\: \<delocal> x = e$\\
%     & $\<skip> \mid s \: s$\\
%     & \only<1>{$\<call> f(x,\ldots,x) \mid \<uncall> f(x,\dots,x)$}
%       \only<2->{\sout{$\<call> f(x,\ldots,x) \mid \<uncall> f(x,\dots,x)$}}\\
%     & \only<3->{$\<create> x \; q(x,\ldots,x) \mid \<uncreate> x \; q(x,\ldots,x)$}\\
%     & \only<3->{$\<resume> x(x,\ldots,x) \mid \<unresume> x(x,\ldots,x)$}\\
%     & \only<3->{$\<yield>$}
%   \end{tabular}
%   \vspace{0.3cm}

%   \pause\pause\pause Procedures definitions extended to $\<procedure> f(x,\dots,x)[x,\dots,x]$.
% \end{frame}


% \begin{frame}
%   \frametitle{Formal semantics}

%   Judgments:
%   \begin{itemize}
%   \pause\item $\sigma |- e => v$
%   \pause\item $\sigma, S |- s => \sigma', S'$
%   \end{itemize}
% \end{frame}

% \begin{frame}
%   \frametitle{Formal semantics (example 1/4)}

% \begin{align*}
% \inference[]{
%   \sigma |-_{expr} e_1 /=> 0 &
%   \sigma, \go |-_{stmt} s_1 => \sigma', \go &
%   \sigma' |-_{expr} e_2 /=> 0
% }{
%   \sigma, \go |-_{stmt} \<if> e_1 \<then> s_1 \<else> s_2 \<fi> e_2 => \sigma', \go
% }
% \end{align*}
% \end{frame}

% \begin{frame}
%   \frametitle{Formal semantics (example 2/4)}

% \begin{align*}
% \inference[]{
%   \sigma |-_{expr} e_1 /=> 0 &
%   \sigma, \go |-_{stmt} s_1 => \sigma', \pausel_i
% }{
%   \sigma, \go |-_{stmt} \<if> e_1 \<then> s_1 \<else> s_2 \<fi> e_2 => \sigma', \pausel_i
% }
% \end{align*}
% \end{frame}

% \begin{frame}
%   \frametitle{Formal semantics (example 3/4)}

% \begin{align*}
% \inference[]{
%   \sigma, \pausel_i |-_{stmt} s_1 => \sigma', \go &
%   \sigma |-_{expr} e_2 /=> 0
% }{
%   \sigma, \pausel_i |-_{stmt} \<if> e_1 \<then> s_1 \<else> s_2 \<fi> e_2 => \sigma', \go
% }
% \end{align*}
% \end{frame}

% \begin{frame}[fragile]
%   \frametitle{Example -- finding the \texttt{yield}}
%   \begin{lstlisting}[escapechar=€]
% if cond1
% then ...
%      yield€$_i$€
% else ...
%      yield€$_j$€
% fi cond2
%   \end{lstlisting}
% \end{frame}

% \begin{frame}
%   \frametitle{Formal semantics (example 4/4)}

% \begin{align*}
% \inference[]{
%   \<yield>_i \in s_1 &
%   \sigma, \pausel_i |-_{stmt} s_1 => \sigma', \pausel_j &
% }{
%   \sigma, \pausel_i |-_{stmt} \<if> e_1 \<then> s_1 \<else> s_2 \<fi> e_2 => \sigma', \pausel_j
% }
% \end{align*}
% \end{frame}


% \begin{frame}[fragile]
%   \frametitle{Ensuring reversibility}

%   As with regular Janus, there are quite a few failure modes:

%   \pause
%   \begin{lstlisting}
%     x += x
%   \end{lstlisting}

%   \pause
%   \begin{lstlisting}
%   procedure f(x,y)
%     x += y
%   ...
%   call f(x,x)
%   \end{lstlisting}

%   \pause
%   \begin{lstlisting}
%   procedure f(x)
%     x += y
%   ...
%   procedure main
%     int y
%     call f(y)
%   \end{lstlisting}
% \end{frame}

% \begin{frame}[fragile]
%   \frametitle{Ensuring reversibility (cont.)}

%   However there is also a new variant:

%   \pause
%   \begin{lstlisting}
%   procedure f()[c]
%     local c2 = nil
%     c <=> c2
%     yield
%     ...

%   procedure main
%     ...
%     create c f()
%     resume c(c)
%   \end{lstlisting}
% \end{frame}

% \begin{frame}[fragile]
%   \frametitle{Ensuring reversibility (cont.)}

%   However there is also a new variant:

%   \begin{lstlisting}
%   procedure f()[]
%     local c2 = nil
%     c <=> c2
%     yield
%     ...

%   procedure main
%     coroutine c
%     create c f()
%     resume c()
%   \end{lstlisting}
% \end{frame}

% \begin{frame}
%   \frametitle{Ensuring reversibility (cont.)}

%   Our solution:

%   \begin{itemize}
%     \pause\item When doing $\<create>$, aliases are banned (they could be
%     allowed safely, but not with an obvious semantic).
%     \pause\item When doing $\<resume>$, aliases are banned and all variables
%     must be local.
%   \end{itemize}

%   \pause This is sufficient to ensure reversibility.
% \end{frame}

% \begin{frame}[fragile]
%   \frametitle{Implementing call (and other sugar)}
%   \verb|call f(x_1,...,x_n)| where \texttt{f} has no $\<yield>$ can
%   be translated to
%   \begin{lstlisting}
% local c = nil
% create c f(x0,x1,x2)
% resume c ()
% uncreate c(x0,x1,x2)
% delocal c = nil
%   \end{lstlisting}
% \end{frame}

% \begin{frame}[fragile]
%   \frametitle{Implementing call (and other sugar)}
%   \verb|call f(x_1,...,x_n)| where \texttt{f} \emph{has} a $\<yield>$ can
%   be translated to
%   \begin{lstlisting}
% local c = nil
% create c f(x0,x1,x2)

% from isgo(c)
% do resume c()
% loop yield
% until isgo(c)

% uncreate c(x0,x1,x2)
% delocal c = nil
%   \end{lstlisting}
% \end{frame}
