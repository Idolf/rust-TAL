\subsection{Soundness (annotated)}

\begin{lemma}[Progress]
  \label{thm:progress}
  Assume $P_1$ is an annotated program such that $|- P_1$.
  Then we can either get a derivation of $\donea{P_1}$ or there exists
  a $P_2$ such that $\stepa{P_1}{P_2}$.
\end{lemma}
\begin{proof}
  TODO
\end{proof}

\begin{lemma}[Preservation]
  \label{thm:reduction}
  Assume $P_1, P_2$ are annotated programs such that $|- P_1$ and
  $\stepa{P_1}{P_2}$. Then $|- P_2$.
\end{lemma}
\begin{proof}
  TODO
\end{proof}

\begin{lemma}[Soundness]
  \label{thm:soundness}
  Assume $P_1, \dots, P_n$ are annotated programs such that $|- P_1$ and
  $\stepa{P_1}{P_2}, \dots, \stepa{P_{n-1}}{P_n}$. Then we can either get a
  derivation of $\donea{P_n}$ or there exists a $P_{n+1}$ such that
  $\stepa{P_n}{P_{n+1}}$.
\end{lemma}
\begin{proof}
  Let $V(k)$ be the statement ``either $k > n$ or there exists a derivation of
  $|- P_k$''. We use natural induction to show $V(k)$:

  \begin{itemize}
  \item $V(1)$: We have the result as an assumption.
  \item $V(k) => V(k+1)$: If $k+1 > n$ we are done, so assume $k+1 \le n$. In
    that case $V(k)$ implies that we have a derivation of $|- P_k$. We also have
    a derivation of $\stepa{P_k}{P_{k+1}}$ by assumption. We can now use the
    preservation lemma to get a derivation of $|- P_k$.
  \end{itemize}

  We can thus conclude $|- P_n$. We use this on the progress lemma and conclude
  the desired result.
\end{proof}

\subsection{Soundness (simplified)}
\begin{lemma}[Bisimulation]
  \label{thm:bisimulation}
  Define:

  $$\mathcal{R}_B =\{(P, \drop{P}{P}) : \text{there exists a judgment of } |-
  P\}$$

  $\mathcal{R}_B$ is a bisimulation.
\end{lemma}
\begin{proof}
  We need to show that $\mathcal{R}_B$ and $\mathcal{R}_B^{-1}$ are
  simulations. First assume $(P_1, P^\circ_1) \in \mathcal{R}_B$, which implies
  that $P^\circ_1 = \drop{P}{P_1}$.

  For the first part, assume that $\stepa{P_1}{P_2}$ for some $P_2$. We need to
  find an $P^\circ_2$ such that $\steps{P^\circ_1}{P^\circ_2}$ and
  $(P_2, P^\circ_2) \in \mathcal{R}_B$. If we let $P^\circ_2 = \drop{P}{P_2}$,
  then it remains to be shown that $\steps{\drop{P}{P_1}}{\drop{P}{P_2}}$ and
  $|- P_2$. However these follow directly from \autoref{thm:exec-drop} and
  preservation lemma respectively.

  For the second part, assume that $\steps{P^\circ_1}{P^\circ_2}$ for some
  $P^\circ_2$. We find an $P_2$ such that $\stepa{P_1}{P_2}$ and
  $(P_2, P^\circ_2) \in \mathcal{R}_B$. Since
  $(P_1, P^\circ_1) \in \mathcal{R}_B$ we know $|- P_1$. We use the progress
  lemma on this, which results in two cases.

  We cannot have a derivation of $\donea{P_1}$, as this would imply a
  contradiction: We have $P_1 = (\dots, \mathtt{halt})$, which means that
  $P^\circ_1 = (\dots, \mathtt{halt})$. However this is not possible since
  $\steps{P^\circ_1}{P^\circ_2}$.

  So we instead conclude there is some $P_t$ such that $\stepa{P_1}{P_t}$. If we let
  $P_2 = P_t$, then it remains to be shown that $P^\circ_2 = \drop{P}{P_2}$ and
  $|- P_2$. The first part follows from \autoref{thm:exec-drop} and
  \autoref{thm:determinism}. The second part follows from the preservation
  lemma.
\end{proof}

\begin{theorem}[Soundness (simplified)]
  \label{thm:soundness-simple}
  Let $P_1$ be an annotated program with $|- P_1$ and let
  $P^\circ_1, \dots, P^\circ_n$ be simplified program such that
  $P^\circ_1 = \drop{P}{P_1}$ and
  $\steps{P^\circ_1}{P^\circ_2}, \dots, \steps{P^\circ_{n-1}}{P^\circ_n}$.
  Then we can either get a derivation of $\dones{P^\circ_n}$ or there is some
  $P^\circ_{n+1}$ such that $\steps{P^\circ_n}{P^\circ_{n+1}}$.
\end{theorem}
\begin{proof}
  We see that $(P_1, P^\circ_1) \in \mathcal{R}_B$ from the bisimulation
  lemma. By repeatedly using the fact that $\mathcal{R}_B$ is a bisimulation we
  get annotated programs $P_2, \dots, P_n$ where $\stepa{P_i}{P_{i+1}}$ and
  $(P_{i+1}, P^\circ_{i+1}) \in \mathcal{R}_B$ for all $1 \le i < n$.

  Specifically we have $P^\circ_n = \drop{P}{P_n}$ and $|- P_n$. We apply the
  progress lemma on $|- P_n$, which results in two cases:

  \begin{itemize}
  \item Either we have a derivation of $\donea{P_n}$, so
    $P_n = (\dots, \mathtt{halt})$. This implies
    $P^\circ_n = (\dots, \mathtt{halt})$, so we
    get a derivation of $\dones{P^\circ_n}$.
  \item Alternatively we have a derivation of $\stepa{P_n}{P_{n+1}}$ for some
    value of $P_{n+1}$. Since $\mathcal{R}_B$ is a bisimulation, this implies we
    have some $P^\circ_{n+1}$ such that $\steps{P^\circ_n}{P^\circ_{n+1}}$.
  \end{itemize}
\end{proof}
