\section{Step 1: Defining the languages}

\nativelang is MIPS. \simplelang and \highlang are defined inductively by
these grammars. In the definitions, we use $wordsize$ and $max$ as CPU-specific
constants. They are concrete, low constants such as 32, however their exact
values are irrelevant for the later discussions.

{\footnotesize
\begin{tabular}{lrcl}
Variables: \\
\textit{global pointers} & $\ell_g$ \\
\textit{heap pointers}   & $\ell_h$ \\
\textit{type variable}   & $\alpha$ \\
\textit{stack variable}  & $\rho$ \\\\

Common definitions: \\
\textit{integer}            & $n,k$ & ::= & $0, 1, \dots$ \\
\textit{machine integers}   & $i$   & ::= & $0, 1, \dots, 2^{wordsize}-1$ \\
\textit{registers}          & $r$   & ::= & $\mathtt{r}_1 \mid \dots \mid \mathtt{r}_{max}$ \\\\

The grammar for \simplelang: \\
\textit{word values}        & $\simple w$ & ::= & $\mathtt{globval}\ \ell_g \mid \mathtt{heapval}\ \ell_h \mid \mathtt{int}\ i \mid \mathtt{ns} \mid \mathtt{uninit}$ \\
\textit{small values}       & $\simple v$ & ::= & $\mathtt{globval}\ \ell_g \mid \mathtt{reg}\ r \mid \mathtt{int}\ i$ \\
\textit{global values}      & $\simple g$ & ::= & $\mathtt{code}\ \simple I$ \\
\textit{global collections} & $\simple G$ & ::= & $\{\ell_g |-> \simple{g}, \dots, \ell_g |-> \simple{g}\}$ \\
\textit{heap values}        & $\simple h$ & ::= & $\langle \simple w, \dots, \simple w \rangle$ \\
\textit{heap collections}   & $\simple H$ & ::= & $\{\ell_h |-> \simple{h}, \dots, \ell_h |-> \simple{h}\}$ \\
\textit{stacks}             & $\simple S$ & ::= & $\mathtt{nil} \mid \simple w :: \simple S$ \\
\textit{register files}     & $\simple R$ & ::= & $\{\mathtt{sp} |-> \simple{S}, \mathtt{r}_1 |-> \simple{w}, \dots, \mathtt{r}_{max} |-> \simple{w}\}$ \\\\

\textit{instructions} & $\simple \iota$ & ::= & $\mathtt{add}\ r, r, \simple v \mid \mathtt{sub}\ r, r, \simple v \mid$ \\
        &&& $\mathtt{salloc}\ n \mid \mathtt{sfree}\ n \mid$ \\
        &&& $\mathtt{ld}\ r, \mathtt{sp}(n) \mid \mathtt{ld}\ r, r(n) \mid$ \\
        &&& $\mathtt{st}\ \mathtt{sp}(n), r \mid \mathtt{st}\ r(n), r \mid$ \\
        &&& $\mathtt{malloc}\ r,\ n \mid $ \\
        &&& $\mathtt{mov}\ r, \simple v \mid \mathtt{beq}\ r, \simple v$ \\
\textit{instruction sequences} & $\simple I$ & ::= & $\simple\iota ; \simple I \mid \mathtt{jmp}\ \simple v \mid \mathtt{halt}$ \\
\textit{program states} & $\simple P$ & ::= & $(\simple H, \simple R, \simple I)$ \\
\textit{programs} & $\simplelang$ & ::= & $\simple\running (\simple G, \simple P) \mid \simple\halted$ \\\\

The type system for \highlang: \\
\textit{types}                    & $\tau$ & ::= & $\alpha \mid \mathtt{int} \mid \mathtt{ns} \mid \mathtt\forall[ \Delta ] \Gamma \mid \langle\tau^\phi,\dots,\tau^\phi\rangle$ \\
\textit{initialization flags}     & $\phi$ & ::= & $\mathtt{init} \mid \mathtt{uninit}$ \\
\textit{stack types}              & $\sigma$ & ::= & $\rho \mid \mathtt{nil} \mid \tau :: \sigma$ \\
\textit{type assignments}         & $\Delta$ & ::= & $\mathtt{nil} \mid a :: \Delta$ \\
\textit{type assignment value}    & $a$ & ::= & $\alpha \mid \rho$ \\
\textit{global label assignments} & $\Psi_g$ & ::= & $\{\ell_g |-> \tau, \dots, \ell_g |-> \tau\}$ \\
\textit{heap label assignmentss}  & $\Psi_h$ & ::= & $\{\ell_h |-> \tau, \dots, \ell_h |-> \tau\}$ \\
\textit{register assignments}     & $\Gamma$ & ::= & $\{\mathtt{sp} |-> \sigma, \mathtt{r}_1 |-> \tau, \dots, \mathtt{r}_{max} |-> \tau\}$ \\
\textit{instantiation}            & $\theta$ & ::= & $\tau/\alpha \mid \sigma/\rho$ \\\\

The grammar for \highlang: \\
\textit{word values}              & $\high w$ & ::= & $\mathtt{globval}\ \ell_g \mid \mathtt{heapval}\ \ell_h \mid \mathtt{int}\ i \mid \mathtt{ns} \mid \mathtt{uninit}\ \tau \mid \Lambda\ \Delta \cdot \high{w}[\theta, \dots, \theta]$ \\
\textit{small values}             & $\high v$ & ::= & $\mathtt{globval}\ \ell_g \mid \mathtt{reg}\ r \mid \mathtt{int}\ i \mid \Lambda\ \Delta \cdot \high{v}[\theta, \dots, \theta]$ \\
\textit{global values}            & $\high g$ & ::= & $\mathtt{code}[ \Delta ] \Gamma \cdot \high I$ \\
\textit{global collections}       & $\high G$ & ::= & $\{\ell_g |-> \high{g}, \dots, \ell_g |-> \high{g}\}$ \\
\textit{heap values}              & $\high h$ & ::= & $\langle \high w, \dots, \high w \rangle$ \\
\textit{heap collections}         & $\high H$ & ::= & $\{\ell_h |-> \high{h}, \dots, \ell_h |-> \high{h}\}$ \\
\textit{stacks}                   & $\high S$ & ::= & $\mathtt{nil} \mid \high w :: \high S$ \\
\textit{register files}           & $\high R$ & ::= & $\{\mathtt{sp} |-> \high{S}, \mathtt{r}_1 |-> \high{w}, \dots, \mathtt{r}_{max} |-> \high{w}\}$ \\\\

\textit{instructions} & $\high\iota$ & ::= & $\mathtt{add}\ r, r, \high v \mid \mathtt{sub}\ r, r, \high v \mid$ \\
        &&& $\mathtt{salloc}\ n \mid \mathtt{sfree}\ n \mid$ \\
        &&& $\mathtt{ld}\ r, \mathtt{sp}(n) \mid \mathtt{ld}\ r, r(n) \mid$ \\
        &&& $\mathtt{st}\ \mathtt{sp}(n), r \mid \mathtt{st}\ r(n), r \mid$ \\
        &&& $\mathtt{malloc}\ r,\ \langle \tau, \dots, \tau \rangle \mid $ \\
        &&& $\mathtt{mov}\ r, \high v \mid \mathtt{beq}\ r, \high v$ \\
\textit{instruction sequences} & $\high I$ & ::= & $\high\iota ; \high I \mid \mathtt{jmp}\ \high v \mid \mathtt{halt}$ \\
\textit{program states} & $\high P$ & ::= & $(\high H, \high R, \high I)$ \\
\textit{programs} & $\highlang$ & ::= & $\high\running (\high G, \high P) \mid \high\halted$ \\\\
\end{tabular}
}


We will occasionally refer to a specific heap collection as a \textbf{heap} or
perhaps \textbf{the heap} for a specific program state. Similarly a specific
global collection might be referred to as \textbf{the globals} of a program. We
will however not use the term ``global'' without qualification, to avoid
ambiguity between global collections and global values.

The fundamental assumption of this language is that the state of a program is
totally defined by a collection of immutable global values, the current state of
the heap, the current stack, the current registers and the current instruction
pointer. This is represented in our language as:

\begin{itemize}
\item The immutable global values are places in a global collection
  consisting of individual global items.
\item The mutable global values are places in a heap collection
  consisting of individual heap items.
\item The current stack is encoded as a stack consisting and placed inside a
  register file.
\item The value of the current registers are encoded in a register file together
  with the stack.
\item The instruction pointer is not encoded directly as a pointer, it is
  instead represented as the instructions to be run until the next block.
\end{itemize}

While this representation is enough for many programs, it also excludes many
programs by design. For instance it excludes mutable global variables not stored
on the heap and self-modifying code (such as just-in-time compilers).

It should be possible to expand the language to allow more programs, however
some sacrifice will always be needed, as we are not able to exclude malicious
programs from our semantics without also excluding some valid programs.

\subsection{Notes on the Agda implementation}

The Agda implementation is mostly a faithful representation of the grammar
presented here, though the notation used for it is slightly different in places.

The biggest difference is how we represent the variables (i.e.
$\ell_g, \ell_h, \alpha, \rho$) and the stores that use these variables
(i.e. $\Psi_g, \Psi_h, \Delta$). In this report we will mostly just assume that
we have an endless supply of unique variables (e.g.
$\alpha_1, \rho_1, \ell_{g,1}, \ell_{h,1}, \alpha_2, \dots$), however this is
not a practical solution in Agda.

\todo{Bad font for $\mathbb{\alpha}$}

For the $\alpha$ and $\rho$ we will use De Bruijn indices. The $\Delta$ values
will simply be lists composed over two elements $\mathbb{\alpha}$ and
$\mathtt{\rho}$. For an $\alpha$-variable to be valid in a given
$\Delta$-context we must have that that $\Delta[\alpha] = \mathtt{\alpha}$ (and
similarly for $\rho$). This means that while on paper we might simply allow
assume weakening or reordering of contexts as fundamental rules, this approach
will not work in Agda. In this particular case we will instead have to prove a
explicit weakening lemma, which depends on a renaming function.

For the heap and globals we will simply compose them into lists and have the
variables be indices into this list. This will not cause any significant
complications, as our semantics will never remove anything from either of the
lists.
