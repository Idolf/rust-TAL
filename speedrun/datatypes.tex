\section{Datatypes}

\subsubsection{Dictionaries}
A \textbf{dictionary} is an unordered set of the form
$\{x_1 |-> y_1, \dots, x_n |-> y_n\}$ such that $x_i \neq x_j$ whenever
$i \neq j$. By unordered we mean that e.g.
$\{x_1 |-> y_1, x_2 |-> y_2\} = \{x_2 |-> y_2, x_1 |-> y_1\}$.

Assume $D$ is some dictionary $\{x_1 |-> y_1, \dots, x_n |-> y_n\}$. In that
case $\mathbf{Keys}(D) = \{x_1, \dots, x_n\}$. We define as $D[x_k] = y_k$ and
$D[x_k |-> y_{new}] = \{x_1 |-> y_1, \dots x_k |-> y_{new}, \dots, x_n |->
y_n\}$.

Now let $D'$ be another dictionary $\{x_1' |-> y_1', \dots, x_m' |-> y_m'\}$
such that $\mathbf{Keys}(D_1) \cap \mathbf{Keys}(D_2) = \emptyset$. In this case
$D_1 \cup D_2$ is the dictionary
$\{x_1 |-> y_1, \dots, x_n |-> y_n, x_1' |-> y_1', \dots, x_m' |-> y_m'\}$.

\paragraph{}
Note: In the Agda-implementation, these maps are mostly implemented as ordered
lists with implicit keys, however we can ignore this fact when using
paper-notation.

\subsubsection{List-like objects}
A set $S$ is \textbf{list-like} iff there is a meaningful way to interpret the
elements of $S$ as ordered lists of elements drawn from a base-set $S_B$. In
other words iff there exist an injective function
$f : S \to \mathbf{List}\ S_B$, where $f$ may only depend on the superficial
syntax of the elements.

For instance heap values and stacks are list-like, as they are written
$\langle w_1, \dots, w_n \rangle$ and $w_1 :: \dots :: w_n :: \mathtt{nil}$
respectively.

Let $S$ be a list-like set and assume that $L$ is a typical member of this set
such that $f(L) = [x_1, \dots, x_n]$. In that case $\mathbf{Length}(L) = n$,
i.e. the number of elements in $L$.

We further define $L[k] = x_k$ and $L[k |-> x_{new}]$ to be the list-like member
of $S$ such that
$f(L[k |-> x_{new}]) = [x_1, \dots, x_{k-1}, x_{new}, \dots, x_n]$. We write
$x \in L$ iff $x = L[k]$ for some $k$.

Finally if $L'$ is another member of $S$ such that $f(L') = [y_1, \dots, y_m]$,
then $L ++ L'$ is the list such that
$f(L ++ L') = [x_1, \dots, x_n, y_1, \dots, y_m]$.
