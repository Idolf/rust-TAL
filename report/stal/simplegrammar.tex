\subsection{Grammar}

\begin{tabular}{lrcl}
Variables: \\
\textit{global pointers}    & $\ell_g$ \\
\textit{heap pointers}      & $\ell_h$ \\
\\
\textit{integer}            & $n$ & ::= & $0, 1, \dots$ \\
\textit{machine integers}   & $i$ & ::= & $0, 1, \dots, 2^{w}-1$ \\
\textit{registers}          & $r$ & ::= & $\mathtt{r}_1 \mid \dots \mid \mathtt{r}_{r_{max}}$ \\
\textit{word values}        & $w^\circ$ & ::= & $\mathtt{globval}\ \ell_g \mid \mathtt{heapval}\ \ell_h \mid \mathtt{int}\ i \mid \mathtt{ns} \mid \mathtt{uninit}$ \\
\textit{small values}       & $v^\circ$ & ::= & $\mathtt{globval}\ \ell_g \mid \mathtt{reg}\ r \mid \mathtt{int}\ i \mid \mathtt{ns} \mid \mathtt{uninit}$ \\
\textit{global values}      & $g^\circ$ & ::= & $\mathtt{code}\ I^\circ$ \\
\textit{global collections} & $G^\circ$ & ::= & $\{\ell_g |-> g^\circ, \dots, \ell_g |-> g^\circ\}$ \\
\textit{heap values}        & $h^\circ$ & ::= & $\langle w^\circ, \dots, w^\circ \rangle$ \\
\textit{heap collections}   & $H^\circ$ & ::= & $\{\ell_h |-> h^\circ, \dots, \ell_h |-> h^\circ\}$ \\
\textit{stacks}             & $S^\circ$ & ::= & $\mathtt{nil} \mid w^\circ :: S^\circ$ \\
\textit{register files}     & $R^\circ$ & ::= & $\{\mathtt{sp} |-> S^\circ, \mathtt{r}_1 |-> w^\circ, \dots, \mathtt{r}_{r_{max}} |-> w^\circ\}$ \\
\\
\textit{instructions} & $\iota^\circ$ & ::= & $\mathtt{add}\ r, r, v^\circ \mid \mathtt{sub}\ r, r, v^\circ \mid$ \\
        &&& $\mathtt{salloc}\ n \mid \mathtt{sfree}\ n \mid$ \\
        &&& $\mathtt{ld}\ r, \mathtt{sp}(n) \mid \mathtt{st}\ \mathtt{sp}(n), r \mid$\\
        &&& $\mathtt{ld}\ r, r(n) \mid \mathtt{st}\ r(n), r \mid$\\
        &&& $\mathtt{malloc}\ r,\ n \mid $ \\
        &&& $\mathtt{mov}\ r, r \mid \mathtt{beq}\ r, v^\circ$ \\
        &&& $\mathtt{halt}$ \\
\textit{instruction sequences} & $I^\circ$ & ::= & $\iota^\circ ; I^\circ \mid \mathtt{jmp}\ v^\circ$ \\
\textit{programs} & $P^\circ$ & ::= & $(G^\circ, H^\circ, R^\circ, I^\circ)$ \\
\end{tabular}

\paragraph{}
The mappings of the form $\{x |-> y, \dots\}$ are unordered and do not have
repeating keys. In the Agda-implementation, these maps are mostly implemented as
ordered lists with implicit keys, however we can ignore this fact when using
paper-notation.

We will occasionally refer to a specific heap collection as a \textbf{heap} or
perhaps \textbf{the heap} for a specific program state. Similarly a specific
global collection might be referred to as \textbf{the globals} of a program. We
will however not use the term ``global'' without qualification, to avoid
ambiguity between global collections and global values.

The fundamental assumption of this language is that the state of a program is
totally defined by a collection of immutable global values, the current state of
the heap, the current stack, the current registers and the current instruction
pointer. This is represented in our language as:

\begin{itemize}
\item The immutable global values are places in a global collection
  consisting of individual global items.
\item The mutable global values are places in a heap collection
  consisting of individual heap items.
\item The current stack is encoded as a stack consisting and placed inside a
  register file.
\item The value of the current registers are encoded in a register file together
  with the stack.
\item The instruction pointer is not encoded directly as a pointer, it is
  instead represented as the instructions to be run until the next block.
\end{itemize}

While this representation is enough for many programs, it also excludes many
programs by design. For instance it excludes mutable global variables not stored
on the heap and self-modifying code (such as just-in-time compilers).

It should be possible to expand the language to allow more programs, however
some sacrifice will always be needed, as we are not able to exclude malicious
programs from our semantics without also excluding some valid programs.
