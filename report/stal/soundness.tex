\begin{lemma}[Progress]
  \label{thm:progress}
  Assume $P_1$ is an annotated program such that $|- P_1$.
  Then we can either get a derivation of $\done{P_1}$ or there exists
  a $P_2$ such that $P_1 -> P_2$.
\end{lemma}

\begin{lemma}[Preservation]
  \label{thm:reduction}
  Assume $P_1, P_2$ are annotated programs such that $|- P_1$ and $P_1 ->
  P_2$. Then $|- P_2$.
\end{lemma}

\begin{lemma}[Soundness]
  \label{thm:soundness}
  Assume $P_1, \dots, P_n$ are annotated programs such that $|- P_1$ and
  $P_1 -> P_2, \dots, P_{n-1} -> P_n$. Then we can either get a derivation of
  $\done{P_n}$ or there exists a $P_{n+1}$ such that $P_n -> P_{n+1}$.
\end{lemma}
\begin{proof}
  Let $V(k)$ be the statement ``$k > n$ or there exists a derivation of
  $|- P_k$''. We use natural induction to show $V(k)$:

  \begin{itemize}
  \item $V(1)$: We have an assumption stating $|- P_1$, so we are done.
  \item $V(k) => V(k+1)$: If $k+1 > n$ we are done, so assume $k+1 \le n$. In
    that case $V(k)$ implies that we have a derivation of $|- P_k$. We also have
    a derivation of $P_k => P_{k+1}$ by assumption. We can now use the
    preservation lemma to get a derivation of $|- P_k$.
  \end{itemize}

  We can thus conclude that $|- P_n$. We use this on the progress lemma and are
  done.
\end{proof}

\begin{lemma}[Progress (simplified)]
  \label{thm:progress-simple}
  Let $P_1$ be an annotated program with $|- P_1$ and
  $P^\circ_1 = \drop{P}{P_1}$. Then we can either get a derivation of
  $\done{P^\circ_1}$ or there exists a $P^\circ_2$ such that
  $P^\circ_1 -> P^\circ_2$.
\end{lemma}
\begin{proof}
  We use the progress lemma on $|- P_1$, which results in two cases:

  \begin{itemize}
  \item Either we have a derivation of $\done{P_1}$. In that case
    $P_1 = (G, H, R, \mathtt{halt})$, so
    $P^\circ_1 = (\drop{G}{G}, \drop{H}{H}, \drop{R}{R}, \mathtt{halt})$. Thus
    we can construct a derivation of $\done{P^\circ_1}$.
  \item Alternatively we have a derivation of $P_1 -> P_2$ for some $P_2$. We
    see that $(P_1, P^\circ_1) = (P_1, \drop{P}{P_1}) \in \mathcal{R}$ from
    \autoref{thm:simulation}. Since $\mathcal{R}$ is a simulation, we have the
    desired result.
  \end{itemize}
\end{proof}

\begin{lemma}[Preservation (simplified)]
  \label{thm:reduction-simple}
  Let $P_1$ be an annotated program with $|- P_1$ and $P^\circ_1 = \drop{P}{P}$.
  Assume that $P^\circ_1 -> P^\circ_2$. Then there is some annotated program
  $P_2$ with $|- P_2$ and $P^\circ_2 = \drop{P}{P_2}$.
\end{lemma}
\begin{proof}
  We use the progress lemma on $|- P_1$, which results in two cases.

  We cannot have a derivation of $\done{P_1}$, as this would imply a
  contraction. The derivation would imply that $P_1 = (G, H, R, \mathtt{halt})$,
  which means that
  $P^\circ_1 = (\drop{G}{G}, \drop{H}{H}, \drop{R}{R}, \mathtt{halt})$. However
  this is not possible since $P^\circ_1 -> P^\circ_2$.

  So we conclude that there is some $P_t$ such that $P_1 -> P__t$. This means
  that $(P_1, P^\circ_1) = (P_1, \drop{P}{P_1}) \in \mathcal{R}$ from
  \autoref{thm:simulation}, by using $P_t$ as the witness. Since $\mathcal{R}$
  is a bisimulation, we there is a $P_2$ such that $P_1 -> P_2$ and
  $(P_2, P^\circ_2) \in \mathcal{R}$, which implies that
  $P^\circ_2 = \drop{P}{P_2}$.\footnote{It happens to be the case that
    $P_2 = P_t$ because both languages are deterministic, but our proof does not
    depend on this fact directly.}

  It just remains to be shown that $|- P_2$, however this follows directly from
  the preservation lemma.
  \end{itemize}
\end{proof}

\begin{theorem}[Soundness (simplified)]
  \label{thm:soundness-simple}
  Let $P_1$ be an annotated program with $|- P_1$ and let
  $P^\circ_1, \dots, P^\circ_n$ be simplified program such that
  $P^\circ_1 = \drop{P}{P_1}, P^\circ_1 -> P^\circ_2, \dots, P^\circ_{n-1} ->
  P^\circ_n$.
  Then we can either get a derivation of $\done{P^\circ_n}$ or there is some
  $P^\circ_{n+1}$ such that $P^\circ_n -> P^\circ_{n+1}$.
\end{theorem}
\begin{proof}
  Let $V(k)$ be the statement ``$k > n$ or there exists a $P_k$ such with
  $P^\circ_k = \drop{P}{P_k}$ and $|- P_k$''. We use natural induction to show
  $V(k)$:

  \begin{itemize}
  \item $V(1)$: Let $P = P_1$ and the result follows from the assumptions.
  \item $V(k) => V(k+1)$: If $k+1>n$ we are done, so assume $k \le n$. $V(k)$
    now implies that there is some $P_k$ such that $|- P_k$ and
    $P^\circ_k = \drop{P}{P_k}$. We also have a derivation of
    $P^\circ_k -> P^\circ_{k+1}$ by assumption. By applying the simplified
    preservation lemma, we get the result.
  \end{itemize}

  By $V(n)$ we can thus conclude that we have $P^\circ_n = \drop{P}{P_n}$ and
  $|- P_n$ for some $P_n$. We use this on the simplified progress lemma and get
  the desired result.
\end{proof}
