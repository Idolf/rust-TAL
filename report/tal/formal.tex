\section{Formal Definition}

\begin{definition}
  A \textbf{simplfied assembly language} is a tuple

  $L = (\mathcal{G}, \mathcal{H}, \mathcal{R}, \mathcal{I}, R_\to)$ such that:

  \begin{itemize}
  \item $\mathcal{G}$ a set, called the \emph{set of possible global
      states}. Let the typical member of this set be called $G$.
  \item $\mathcal{H}$ a set, called the \emph{set of possible heap states}. Let
    the typical member of this set be called $H$.
  \item $\mathcal{R}$ a set, called the \emph{set of possible register files}.
    Let the typical member of this set be called $R$.
  \item $\mathcal{I}$ a set, called the \emph{set of possible instruction
      sequences}. Let the typical member of this set be called $I$.
  \item $R_\to$ a subset of
    $\mathcal{G} \times (\mathcal{H} \times \mathcal{R} \times \mathcal{I})
    \times (\mathcal{H} \times \mathcal{R} \times \mathcal{I})$,
    called the \emph{transition relationtion}.  Write
    $G |- (H, R, I) \to_L (H', R', I')$ as notation for
    $(G, (H, R, I), (H', R', I')) \in R_\to$.  When $L$ can be inferred, we will
    write $G |- S_1 \to S_2$ to mean $G |- S_1 \to_L S_2$.
  \end{itemize}
\end{definition}

\begin{definition}
  Let $L = (\mathcal{G}, \mathcal{H}, \mathcal{R}, \mathcal{I}, R_\to)$ be a
  simplified assembly language. We say that
  $T = (\mathbf{\Psi}_g, \mathbf{\Psi}_h, \mathbf{\Gamma})$ is a typing scheme
  for $L$ iff:

  \begin{itemize}
  \item $\mathbf{\Psi}_g$ is a set, called the \emph{set of global state
      types}. Let $\Psi_g$ be the typical member of this set.
  \item $\mathbf{\Psi}_h$ is a set, called the \emph{set of heap state
      types}. Let $\Psi_h$ be the typical member of this set.
  \item $\mathbf{\Gamma}$ is a set, called the \emph{set of register types}. Let
    $\Gamma$ be the typical member of this set.
  \item There is a judgment of the form $|- G : \Psi_g$.
  \item There is a judgment of the form $\Psi_g |- H : \Psi_h$.
  \item There is a judgment of the form $\Psi_g , \Psi_h |- R : \Gamma$.
  \item There is a judgment of the form $\Psi_g , \Psi_h , \Gamma |- I\ \mathbf{Valid}$.
  \end{itemize}

  We call $(L, T)$ for a \emph{typed assembly language}.
\end{definition}

\begin{definition}
  Given the following:
  \begin{itemize}
  \item
    $L_1 = (\mathcal{G}_1, \mathcal{H}_1, \mathcal{R}_1, \mathcal{I}_1,
    R_{1\to})$ is a simplified assembly language.

  \item
    $L_2 = (\mathcal{G}_2, \mathcal{H}_2, \mathcal{R}_2, \mathcal{I}_2,
    R_{2\to})$ is a simplified assembly language.

  \item $E = (f_G, f_H, f_R, f_I)$ is a tuple of total functions. The functions
    should be defined for $\mathcal{G}_1 \to \mathcal{G}_2$,
    $\mathcal{H}_1 \to \mathcal{H}_2$, $\mathcal{R}_1 \to \mathcal{R}_2$ and
    $\mathcal{I}_1 \to \mathcal{I}_2$ respectively.
  \end{itemize}

  Let $E(H, R, I)$ be shorthand for $(f_H(H), f_R(R), f_I(I))$.

  We say that $E$ is an \emph{embedding} of $L_1$ into $L_2$ iff
  $G |- (H, R, I) \to (H', R', I')$ implies
  $f_G(G) |- E(H, R, I) \to E(H', R', I')$.
\end{definition}


% \begin{definition}
%   A \textbf{native language} is a tuple $L_N = (S_N, \to_N)$, where $S_N$ is the
%   set of possible states and
%   $\to_N\ \subseteq S_N \times (S_N \cup \{\mathbf{halt}\})$ is multi-valued
%   function called the \textbf{transition relation}. We write $p \to_N q$ as a
%   shorthand for $(p, q) \in\ \to_N$.
% \end{definition}

% \begin{definition}
%   Assume the following:

%   \begin{itemize}
%   \item
%   Let $L$ be a simplified assembly language and let $M$ be a machine. We say
%   that $L$ can be implemented on a machine $M$ iff  a \textbf{simplified language based on $M$} iff $\dots$.
%   \todo[inline]{Finish definition of machine and simplified language. Do we even
%     need it?}.
% \end{definition}
