\section{Introduction to Typed Assembly Language}
\begin{frame}
\frametitle{Table of Contents}
\tableofcontents[currentsection]
\end{frame}

\begin{frame}{Our goal}
  \begin{itemize}
  \item Fast
  \item \only<1-3>{Safe}\only<4->{\sout{Safe}\ \ Verifiably safe}
  \item Practical
  \item Low-level
  \end{itemize}

  \pause Use Rust...? \pause Not quite good enough.
\end{frame}

\begin{frame}{Programs}{}
  \begin{tikzpicture}[mystyle]
    \node[draw,circle,fill,red] (a) at (0,6.5) {};
    \node[right] at (0,6.5) {\footnotesize \ all programs};
    \draw[fill,red] (0,0) rectangle (10, 6);

    \pause
    \node[draw,circle,fill,blue] (b) at (2.5,6.5) {};
    \node[right,blue] at (2.5,6.5) {\footnotesize \ safe programs};
    \draw[fill,blue] (5,3) ellipse (4.0 and 2.2);
  \end{tikzpicture}
\end{frame}

\begin{frame}{Safety?}
  \textbf{Definition:} A safe program is one that does not access unmapped
  memory.

  \pause \textbf{Implications:}

  \begin{itemize}
  \pause\item Has no buffer overflows
  \pause\item Has no pointer corruption
  \pause\item Cannot execute arbitrary shellcode
  \pause\item Will allow coorporative sharing of a unified memory space
  \pause\item Will allow low-overhead sandboxes!
  \end{itemize}
\end{frame}

\begin{frame}{First attempt}
  \textbf{Idea:} Do not distribute raw executable programs. Distribute programs
  in a high-level language (such as Haskell or Rust). \pause

  \pause \includegraphics[width=0.8\textwidth]{attempt1_dot.pdf}
\end{frame}
