\section{Overview}

The purpose of this chapter is to introduce our typed assembly language along
with our proof of soundness. To give some context, we will first give a quick
run-through of our strategy.

\paragraph{Step 1:} Define three languages:

\begin{itemize}
\item \nativelang is a real-world language, e.g. MIPS as defined by
  \cite{mipssys}.
\item \simplelang is a simplification of \nativelang. The purpose of
  this simplification is to restrict the evaluation model. Our restricted model
  will only to only permit benevolent programs.
\item \highlang is an extension of \simplelang, which will
  introduce types.
\end{itemize}

To avoid dealing with halting semantics, we introduce the following states:
\begin{itemize}
\item $\malicious, \nhalt \in \nativelang$
\item $\shalt \in \simplelang$
\item $\hhalt \in \highlang$
\end{itemize}

$\halt$ represents every program that has halted in a non-malicious way (such as
intentional exiting or properly handled out-of-memory exceptions). $\malicious$
represents every any malicious state.

\paragraph{Step 2:} Define two translation functions and a relation:

\begin{itemize}
\item $\high\embed : \highlang -> \simplelang$. Intuition: Simply drop the type
  annotations.
\item $\simple\embed : \simplelang -> \nativelang$. Intuition: An assembler
  along with a code generator to construct the heap.
\item $\simple\embedp \subset \simplelang \times \nativelang$. Intuition:
  Multiple equivalent run-time equivalents of a single problem might exist. We
  use this relate multiple native programs to the same simplified program.
\end{itemize}

We observe that:

\begin{itemize}
\item $\high\embed(\hhalt) = \shalt$
\item $\simple\embedp(\shalt) = \{\nhalt\}$
\item $\malicious \notin \simple\embedp(p)$ for all p
\item $\simple\embed(p) \in \simple\embedp(p)$ for all p
\end{itemize}

\paragraph{Step 3:} For each of these languages, define a small-step relation:

\begin{itemize}
\item $\nativeo{->} \subset \nativelang \times \nativelang$.
\item $\simpleo{->} \subset \simplelang \times \simplelang$.
\item $\higho{->} \subset \highlang \times \highlang$.
\end{itemize}

Use $p_1 \nativeo{->} p_2 \nativeo{->} \dots \nativeo{->} p_{n-1} \nativeo{->} p_n$
as shorthand for $(p_1, p_2), \dots, (p_{n-1}, p_n) \in \nativeo{->}$. Similarly
for \simpleo{->} and \higho{->}.

We observe that:

\begin{itemize}
\item If $p \in \nativelang$ is halting for non-malicious reasons, then
  $p \nativeo{->} \nhalt$.
\item $\nhalt \nativeo{->} p \iff p = \nhalt$.
\item (Similarly for \simplelang and \highlang).
\item If $p \in \nativelang$ can transition in a malicious or undefined way,
  then $p \nativeo{->} \malicious$.
\item $\malicious \nativeo{->} p \iff p = \malicious$
\end{itemize}

\paragraph{Note:} This implies \nativelang will always progress (i.e. if
$p \in \nativelang$, then $p \nativeo{->} p'$ for some $p'$). This is \emph{not}
true for \simplelang or \highlang.

\paragraph{Step 4:} Define a typing judgment for \highlang and a sub-language:

\begin{itemize}
\item If $p \in \highlang$, then $\valid{}{p}$ is a judgment (for which there
  might or might not exist a derivation).
\item
  $\typedlang = \{p \in \highlang \mid \text{there exists a derivation of\ }
    \valid{}{p}\} \subset \highlang$.
\end{itemize}

\paragraph{Step 5:} Prove these theorems about \highlang:

\begin{itemize}
\item \textbf{Progress}: Assume $p \in \typedlang$. In that case there exists
  some $p' \in \highlang$ such that $p \higho{->} p'$.
\item \textbf{Preservation}: Assume $p \higho{->} p'$ and $p \in \typedlang$. In
  that case $p' \in \typedlang$.
\item \textbf{Soundness for \highlang}: If $p_1 \in \typedlang$ and
  $p_1 \higho{->} \dots \higho{->} p_n$, then there exists some $p_{n+1}$ such that
  $p_n \higho{->} p_{n+1}$. This follows directly from the two previous
  theorems.
\end{itemize}

\paragraph{Step 6:} Prove the following theorems about the relationship between
\highlang and \simplelang:

\begin{itemize}
\item \textbf{Bisimulation}: Define the following predicate on
  programs in \highlang:

  $$\Stuck(p) = (\exists p': p \simpleo{->} p' \land \Stuck(p')) \lor (\not\exists p': p \higho{->} p')$$

  Define the relation $R$ as:

  $$R = \{(p, \high\embed(p)) \mid \neg\Stuck(p)\} \subset \highlang \times
  \simplelang$$

  We see that $R$ is bisimulation between \higho{->} and \simpleo{->}. More
  specifically:\footnote{A more general definition of bisimulation exists,
    however introducing the generalization serves no real purpose in this
    instance.}
  \begin{itemize}
  \item Assume $(p_1, p_1') \in R$ and $p_1 \higho{->} p_2$. Then there exists
    some $p_2'$ such that $(p_2, p_2') \in R$ and $p_1' \simpleo{->} p_2'$.
  \item Assume $(p_1, p_1') \in R$ and $p_1' \simpleo{->} p_2'$. Then there
    exists some $p_2$ such that $(p_2, p_2') \in R$ and
    $p_1 \higho{->} p_2$.
  \end{itemize}
\item \textbf{Soundness for \simplelang}: Assume $p_1 \in \typedlang$ and
  $\high\embed(p_1) \simpleo{->} \dots \simpleo{->} p_2$. In that case there
  exists some $p_3$ such that $p_2 \simpleo{->} p_3$. This follows
  directly from the previous two theorems.
\end{itemize}

\paragraph{Step 7:} Prove the following theorem about the relationship between
\simplelang and \nativelang:

\begin{itemize}
\item \textbf{Stuttering bisimulation}: Define the following predicate on
  programs in \simplelang:

  $$\Stuck(p) = (\exists p': p \simpleo{->} p' \land \Stuck(p')) \lor (\not\exists p': p \simpleo{->} p')$$

  Note specifically that $\neg\Stuck(\shalt)$. Now define a relation $R$
  by:

  $$R = \{(p_1, p_2) \mid \neg\Stuck(p_1), p_2 \in \simple\embedp(p_1) \cup \{\nhalt\}\}$$

  This relation is a stuttering bisimulation of $\simpleo{->}$ and
  $\nativeo{->}$. More specifically there is a function
  $f : \simplelang \times \nativelang \to \mathbb{N}$ such that:\footnote{This is
    again a somewhat specialized definition of a much more general concept.}

  \begin{itemize}
  \item Assume $(p, p_1) \in R$ and $p \simpleo{->} p'$. Then there exists
    $1 \leq n \leq f(p, p_1)$ and a sequence
    $p_1 \nativeo{->} p_2 \nativeo{->} \dots \nativeo{->} p_n$ such that
    $(p', p_n) \in R$.
  \item Assume $(p, p_1) \in R$, assume $n \geq f(p, p_1)$ and that there is
    some sequence $p_1 \nativeo{->} \dots \nativeo{->} p_n$. Then there is some
    $1 \leq k \leq f(p, p_1)$ and a $p'$ such that $(p', p_k) \in R$.
  \end{itemize}

  (Note that since no programs are stuck in \nativelang, we can show a slightly
  different statement regarding sequences of any length.)
\end{itemize}

\paragraph{Step 8:} We finally prove the desired theorems:
\begin{itemize}
\item \textbf{Safety}: Assume $p \in \typedlang$,
  $p' \in \simple\embedp(\high\embed(p))$. There exists no path such that
  $p \nativeo{->} \dots \native{\malicious}$. In particular it is safe to
  execute $\simple\embed(\high\embed(p))$.

  This theorem follows directly from the two previous theorems.

\item \textbf{Correctness}: Assume $p_1 \in \typedlang$,
  $p_1' \in \simple\embedp(\high\embed(p_1))$ and assume
  $p_1 \higho{->} \dots \higho{->} p_n$. There now exists a finite sequence
  $p_1' \nativeo{->} \dots \nativeo{->} p_m'$ such that
  $p_m' \in \simple\embedp(\high\embed(p_n)) \cup \{\nhalt\}$.

\item \textbf{Decidability}: If $p \in \highlang$, then it is decidable to check
  if $p$ is a member of $\typedlang$.
\end{itemize}

\paragraph{Scoping limitations} To limit the scope of this thesis, none of the
parts related to \nativelang will be attempted formalized. At best, we will
discuss how future work might address these issues.

Besides this, the parts related to \simplelang have not been formalized in Agda,
and a few steps of the proofs related to \highlang are not yet formalized. We
will be sure to mention which parts as they come up.
