\chapter{Typed Assembly Language, Safety and Formalizations}
\label{chap:introduction}

\section{The security ecosystem}
You would be hard pressed to find more than a handful of system on the internet
that has never contained a vulnerability. Many computers are even vulnerable to
known remote exploits for months or years at a time. As a result a large
percentage of the computers on the internet are infected by some form of
malware. To try to combat these and similar problems, we have developed a number
of tools.

\begin{itemize}
\item Anti-virus software and intrusion detection systems try to catch malicious
  events in the making, by looking at e.g. network streams, files on disk and
  the behaviors of the programs running on a machine.
\item It is possible to use code-signing certificates and public-key
  infrastructure to prove that a program was written by a verified third-party.
\item Sandboxes and virtual machines two examples of systems that try to limit
  what a program is \emph{able} to do at run-time. They do this by limiting the
  actions available to programs, for instance for not exposing dangerous
  APIs. Other examples from this category are POSIX user permissions, memory
  protections and firewalls. We will collectively refer to these and similar
  approaches as ``run-time protections''.  \footnote{ While anti-virus and
    code-signing certificates principally belong to the category of run-time
    protections, they are sufficiently different from most other such
    protections that they deserve their own point.  }
\item Static analysis, fuzzing and type systems try to detect vulnerabilities in
  programs before they are deployed. Compiler modifcations such as stack-cookies
  or printf-hardening try to generate code that is hard to exploit even in the
  event of an exploit. By using a modern languages one might avoid whole classes
  of bugs. We will collectively refer to these and similar approaches as
  ``compile-time protections''.
\end{itemize}

Note, that while all of these are valuable tools in the current security
industry, none of them are perfect -- many are specifically designed to increase
the cost of attacks as much as possible, not preventing them altogether.

Anti-virus software is inherently limited by Rice's Theorem. This theorem
implies, that any attempt to classify programs as good or bad will have either
false positives (good programs classified as bad) or false negatives (bad
programs classified as good). In practice most real-world implementations have
both. In addition, anti-virus software is often so low quality that they are
effectively increasing the potential attack surface instead of decreasing
it.\todo{Insert citation}

The core idea of code-signing is to include a certificate with any distributed
programs. This certificate will prove that the program have been written by a
real-world entity verified by a trusted third-party using a public key
infrastructure (PKI). The purpose is to effectively increase the price of
running malicious programs on other people's machines; ideally the only attack
available would be to create a fake real-world identity (expensive!) and use it
to buy a code-signing certificate. However in practice there are multiple other
avenues available: One could steal a valid certificate, exploit another
vulnerable (but signed) program or try to bypass the validation check
altogether.

Run-time protections have a long history in computer science, and could arguably
be said to have had the biggest practical impact so far. One reason for their
effectiveness, is that it is often possible to deploy many unrelated protections
in parallel (a technique sometimes referred to as ``Defense in Depth''). It is
for instance uncommon to see end-user machines without at least memory
protections, file permissions and sandboxes for the most exposed
applications. Another reason is that this system is not (directly) limited by
Rice's Theorem.\footnote{Many implementations \emph{would} be limited by Rice's
  Theorem, but that is not an inherent property of run-time protections. For
  instance if one simply wanted to stop a system from ``firing the missiles'',
  then one could cut the wire to the missile system.} It is for instance
theoretically possible to design system where any potentially malicious actions
must be approved by a user before execution -- however such a system would not
be practical. In general one has to balance security with other priorities such
as performance, usability and ease of development.

Compile-time protections have also had huge influence in improving the quality
of the code generated and distributed in the world, though it is harder to
quantify exactly how much. With run-time protections one can often turn off a
specific protection (at least in principle) and look at how the change
influences the system. In contrast compile-time protections often work by
permanently improving code quality or are so integrated into the
compilation-process that they cannot be turned off. Many of these protections
are superficially similar to anti-virus software, as they work by analyzing code
and detecting potential problems, however they have an advantage in that they
are able to interact directly with the programmer. He can ignore false
positives, or he can improve the detection algorithms if he finds false
negatives. A disadvantage is that, even though this approach \emph{is} effective
at stopping problems, there is little possibility for users to \emph{verify}
that this is the case.

\paragraph{}
This thesis discusses a technique known Typed Assembly Language (TAL). While
this technique should probably be group with the run-time protections, it share
properties with all four previously discussed protection-classes:

\begin{itemize}
\item TAL needs compile-time support to be useful. The compiler must to
  type-check the source program and use these types when compiling the code.

\item The compiler uses the types to output a ``certificate'' along with the
  code. This certificate proves the program to be non-malicious. A difference is
  that this proof depends on a \emph{Trusted Computing Base}, not a
  \emph{Trusted Third Party}.

\item Before trying to run a program, it must first be classified as good or
  bad. This classification is limited by Rice's Theorem just like anti-virus
  software. Where anti-virus software will typically try to trade-off the false
  negatives with the false positives, this classifier will instead reject
  \emph{all} malicious programs (and some non-malicious ones). Another
  difference is that this classifier gets an additional input in the form of the
  ``certificate''.

\item Ultimately the system would have an API very similar to that of a
  sandbox. It would accept untrusted programs and run (or reject) them in a
  secure fashion. It would internally use the classifier, but ultimately this
  does not matter to an end-user.
\end{itemize}

\section{Typed Assembly Language}

Typed Assembly Language was introduced in the paper ``From System F to Typed
Assembly Language'' by Morrisett et al. The procedure set forth by this
technique is as follows:

\begin{itemize}
\item Start by choosing a subset of a real-world assembly language.

\item Define a small-step semantics for this subset. A small-step semantics is a
  model for how to execute programs in the subset. For instance if our model
  states that a machine in state $p_1$ can transition to a machine in state
  $p_2$, then we would write $p_1 -> p_2$. Furthermore this model will designate
  some states as \emph{halting} states, i.e. states that will halt immediately
  without error. If a program is neither transitioning nor in a valid halting
  state, then we say that it is \emph{stuck} and we considered that an error.

\item Define a type system for this subset. This type system should have a
  type-checker, i.e. a program that outputs true for well-typed programs and
  false for programs with type errors. The program is allowed to report some
  valid programs as invalid, but ideally it should not. If such a perfect
  type-checker exists, we say that the type system is \emph{decidable}.

\item Prove that this type-system is \emph{sound}. This means that is $p_1$ is
  well-typed and $p_1 -> \dots -> p_2$, then $p_2$ is not stuck.
\end{itemize}

It is okay for our small-step semantics to be a simplification of the real
machine, as long as it agrees with the real-world on some important
points:
\begin{itemize}
\item If our model states that $p$ can transition to one of $p_1, \dots, p_n$
  (with $n \geq 1$), then the real-world machine should ideally transition to
  one of these states too. If it cannot do this (for instance if it is out of
  memory), then it must instead abort the program in a non-harmful way.
\item If our model states that $p$ is halting, then the real world machine
  must halt without error.
\item If our model states that $p$ is stuck, then the real-world machine is
  allowed to do anything.
\item Our model should ideally be chosen such that the ``bad stuff'' is left
  out. How one could hope to achieve this is discussed in \Cref{chap:safesound}.
\end{itemize}

We can now implement the sandbox as follows:

\begin{enumerate}
\item Receive a program $p$.
\item Run the type-checker on $p$.
\item If the type-checker outputs true, then run $p$.
\end{enumerate}

If our soundness-proof and type-checker are correct and our model is valid, then
this is safe.

\section{The Goal of the Thesis}
\label{sec:goal}

The goal of this thesis is ultimately how to design a secure sandbox primitive
based on Typed Assembly Language with specific focus on how to minimize the
Trusted Computing Base. This minimization takes two forms: By formalizing the
meta-theory and by trying to simplify the components needed beyond this
formalization. As a result most of the work revolves around proofs (particularly
proof of soundness) how they can be mechanized.

This thesis is \emph{not} about:

\begin{itemize}
\item How to implement infrastructure for this system beyond the sandbox
  itself. In particular we will not discuss compiler implementations.
\item How to reduce the Trusted Computing Base beyond reducing code size and
  formalizing the metatheory. We will for example not talk about bugs in the
  hardware or CPU specification.
\item Any form of detailed analysis of the performance characteristics of the
  system.  We will specifically not do any thorough analysis of the algorithmic
  complexity of any algorithms, though we might informally argue why we believe
  that an algorithm will run within a certain bound.
\end{itemize}

\section{Outline of the Thesis}

The rest of this thesis is organized as follows:

\textbf{\Cref{chap:lang}} formally defines a variant of STAL, which we will
refer to as \ATAL (Agda-based TAL). \ATAL is mostly equivalent to STAL, however
support for exceptions and existential types have been removed, and a few
details have been simplified without loss of generality. While introducing
\ATAL, we also introduce a simplified version of the same language,
\ATALe. After introducing the languages, we will define small-step semantics for
both languages and prove that the two semantics are related by type
erasure. \textbf{\Cref{chap:types}} introduces the type system for \ATAL and
proves that it is sound. We will also prove that the type system is
decidable. \textbf{\Cref{chap:safesound}} will discuss the issue of safety and
how it relates to the proof of soundness. Specifically we will give a high-level
sketch of how one could implement a platform with a small trusted computing base
from the results in the previous chapters. \textbf{\Cref{chap:hindsight}} lists
a few lessons lessons from working doing a project of this size in Agda. This
includes both observations specific to implement a TAL-like language in Agda,
but also a few more general tricks for making Agda easier to work with. It is
mostly meant to be readable independently of the other
chapters. \textbf{\Cref{chap:future}} discusses how the work could be extended
and relates it to other work. Finally, \textbf{\Cref{chap:conclusion}} summaries
my contributions and concludes.
