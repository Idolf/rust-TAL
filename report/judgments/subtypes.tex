\section{Subtyping judgments}

\fbox{$\subtype{\Delta}{\tau_1}{\tau_2}$}
\begin{mathpar}
\infer{\alpha \in \Delta}{\subtype{\Delta}{\alpha}{\alpha}} \and
\infer{ }{\subtype{\Delta}{\mathtt{int}}{\mathtt{int}}} \and
\infer{ }{\subtype{\Delta}{\mathtt{ns}}{\mathtt{ns}}} \and
\infer
  {\valid{\Delta}{\Delta'} \and
    \subtype{\Delta', \Delta}{\Gamma_1}{\Gamma_2}}
  {\subtype{\Delta}{\forall[ \Delta' ]\Gamma_1}{\forall[ \Delta' ]\Gamma_2}} \and
\infer
  {\subtype{\Delta}{\tau_1^{\phi_1}}{\tau'_1{}^{\phi'_1}} \and
    \dots \and
    \subtype{\Delta}{\tau_n^{\phi_n}}{\tau'_n{}^{\phi'_n}}}
  {\subtype{\Delta}{\langle \tau'_1{}^{\phi_1}, \dots, \tau_n{}^{\phi_n} \rangle}
                   {\langle \tau'_1{}^{\phi'_1}, \dots, \tau'_n{}^{\phi'_n} \rangle}}
\end{mathpar}

\fbox{$\subtype{}{\phi_1}{\phi_2}$}
\begin{mathpar}
\infer{ }{\subtype{}{\mathtt{init}}{\mathtt{init}}} \and
\infer{ }{\subtype{}{\mathtt{\phi}}{\mathtt{uninit}}}
\end{mathpar}

\fbox{$\subtype{\Delta}{\tau_1^{\phi_1}}{\tau_2^{\phi_2}}$}
\begin{mathpar}
\infer{\valid{\Delta}{\tau} \and \subtype{}{\phi_1}{\phi_2}}{\subtype{\Delta}{\tau^{\phi_1}}{\tau^{\phi_2}}}
\end{mathpar}

\fbox{$\subtype{\Delta}{\Gamma_1}{\Gamma_2}$}
\begin{mathpar}
\infer
  {\subtype{\Delta}{\sigma}{\sigma'} \and
    \subtype{\Delta}{\tau_1}{\tau'_1} \and
    \dots \and
    \subtype{\Delta}{\tau_{regs}}{\tau'_{regs}}}
  {\subtype{\Delta}{\{\mathtt{sp}: \sigma, \mathtt{r}_1:\tau'_1, \dots, \mathtt{r}_{regs}:\tau_{regs}\}}
                   {\{\mathtt{sp}: \sigma', \mathtt{r}_1:\tau'_1, \dots, \mathtt{r}_{regs}:\tau'_{regs}\}}}
\end{mathpar}

\subsection{Comments}
This is pretty much what you would expect. A few things of note though:

\begin{itemize}
\item For code-types we require identical assumption lists.
\item For possibly uninitialized types, we require identical base types, not
  just that they also have subtyping.
\end{itemize}

The last one is a requirement, for soundness. (The exact reasons are somewhat complicated and will be included in the final report).

\subsection{Differences from STAL}
While the STAL-paper has reflection and transitivity as rules, we instead derive
them as theorems. This means that we have slightly more rules though.
